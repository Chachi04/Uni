%%%%%%%%%%%%%%%%%%%%%%%%%%%%% Define Article %%%%%%%%%%%%%%%%%%%%%%%%%%%%%%%%%%
\documentclass{article}
%%%%%%%%%%%%%%%%%%%%%%%%%%%%%%%%%%%%%%%%%%%%%%%%%%%%%%%%%%%%%%%%%%%%%%%%%%%%%%%

%%%%%%%%%%%%%%%%%%%%%%%%%%%%% Using Packages %%%%%%%%%%%%%%%%%%%%%%%%%%%%%%%%%%
\usepackage{geometry}
\usepackage{graphicx}
\usepackage{amssymb}
\usepackage{amsmath}
\usepackage{amsthm}
\usepackage{empheq}
\usepackage{mdframed}
\usepackage{booktabs}
\usepackage{lipsum}
\usepackage{graphicx}
\usepackage{color}
\usepackage{psfrag}
\usepackage{pgfplots}
\usepackage{bm}
%%%%%%%%%%%%%%%%%%%%%%%%%%%%%%%%%%%%%%%%%%%%%%%%%%%%%%%%%%%%%%%%%%%%%%%%%%%%%%%

% Other Settings
\usepackage{enumitem}

%%%%%%%%%%%%%%%%%%%%%%%%%% Page Setting %%%%%%%%%%%%%%%%%%%%%%%%%%%%%%%%%%%%%%%
\geometry{a4paper}

%%%%%%%%%%%%%%%%%%%%%%%%%% Define some useful colors %%%%%%%%%%%%%%%%%%%%%%%%%%
\definecolor{ocre}{RGB}{243,102,25}
\definecolor{mygray}{RGB}{243,243,244}
\definecolor{deepGreen}{RGB}{26,111,0}
\definecolor{shallowGreen}{RGB}{235,255,255}
\definecolor{deepBlue}{RGB}{61,124,222}
\definecolor{shallowBlue}{RGB}{235,249,255}
%%%%%%%%%%%%%%%%%%%%%%%%%%%%%%%%%%%%%%%%%%%%%%%%%%%%%%%%%%%%%%%%%%%%%%%%%%%%%%%

%%%%%%%%%%%%%%%%%%%%%%%%%% Define an orangebox command %%%%%%%%%%%%%%%%%%%%%%%%
\newcommand\orangebox[1]{\fcolorbox{ocre}{mygray}{\hspace{1em}#1\hspace{1em}}}
%%%%%%%%%%%%%%%%%%%%%%%%%%%%%%%%%%%%%%%%%%%%%%%%%%%%%%%%%%%%%%%%%%%%%%%%%%%%%%%

%%%%%%%%%%%%%%%%%%%%%%%%%%%% English Environments %%%%%%%%%%%%%%%%%%%%%%%%%%%%%
\newtheoremstyle{mytheoremstyle}{3pt}{3pt}{\normalfont}{0cm}{\rmfamily\bfseries}{}{1em}{{\color{black}\thmname{#1}~\thmnumber{#2}}\thmnote{\,--\,#3}}
\newtheoremstyle{myproblemstyle}{3pt}{3pt}{\normalfont}{0cm}{\rmfamily\bfseries}{}{1em}{{\color{black}\thmname{#1}~\thmnumber{#2}}\thmnote{\,--\,#3}}
\theoremstyle{mytheoremstyle}
\newmdtheoremenv[linewidth=1pt,backgroundcolor=shallowGreen,linecolor=deepGreen,leftmargin=0pt,innerleftmargin=20pt,innerrightmargin=20pt,]{theorem}{Theorem}[section]
\theoremstyle{mytheoremstyle}
\newmdtheoremenv[linewidth=1pt,backgroundcolor=shallowBlue,linecolor=deepBlue,leftmargin=0pt,innerleftmargin=20pt,innerrightmargin=20pt,]{definition}{Definition}[section]
\theoremstyle{myproblemstyle}
\newmdtheoremenv[linecolor=black,leftmargin=0pt,innerleftmargin=10pt,innerrightmargin=10pt,]{problem}{Problem}[section]
\theoremstyle{myproblemstyle}
\newmdtheoremenv[linecolor=black,leftmargin=0pt,innerleftmargin=10pt,innerrightmargin=10pt,]{example}{Example}[section]
%%%%%%%%%%%%%%%%%%%%%%%%%%%%%%%%%%%%%%%%%%%%%%%%%%%%%%%%%%%%%%%%%%%%%%%%%%%%%%%

%%%%%%%%%%%%%%%%%%%%%%%%%%%%%%% Plotting Settings %%%%%%%%%%%%%%%%%%%%%%%%%%%%%
\usepgfplotslibrary{colorbrewer}
\pgfplotsset{width=8cm,compat=1.9}
%%%%%%%%%%%%%%%%%%%%%%%%%%%%%%%%%%%%%%%%%%%%%%%%%%%%%%%%%%%%%%%%%%%%%%%%%%%%%%%

%%%%%%%%%%%%%%%%%%%%%%%%%%%%%%% Title & Author %%%%%%%%%%%%%%%%%%%%%%%%%%%%%%%%
\title{Linear Algebra Exercises}
\author{Jiaqi Wang}
%%%%%%%%%%%%%%%%%%%%%%%%%%%%%%%%%%%%%%%%%%%%%%%%%%%%%%%%%%%%%%%%%%%%%%%%%%%%%%%

\begin{document}
\maketitle

\section{Summary}
\subsection{Vector/Parametric descriptions of lines and planes}
\subsubsection{Lines}
$$ \textit{l}: \underbar{x} = \underbar{v} + \lambda \underbar{u} $$
where $\underbar{v}$ lands on the line and is called the \emph{position vector},
and $\underbar{u}$ is vector that is in the "direction" of the line, called the
direction vector.

\subsubsection{Plane}
$$ V: \underbar{x} = \underbar{u} + \lambda\underbar{v} + \mu\underbar{w} $$
where $\underbar{u}$ is a vector that lands on the plane, and $\underbar{v}$
and $\underbar{w}$ are two linearly independent vectors in the plane.

\subsection{Determining parametric description from equations}
\begin{example}
	Suppose $2x_1-x_2+3x_3 = 4$ is the equation of the plane $V$. To determine a parametric description we proceed as follows.
	If you assign any value, say $\lambda$ to $x_2$, and any value, say $\mu$ to $x_3$, then $x_1$ is determined: $x_1 = 2+\lambda/2 - 3\mu/2$.
	So
	$$x_1 = 2+\lambda/2 - 3\mu/2, x_2 = \lambda, x_3 = \mu$$
	In vector notation:
	$$(x_1,x_2,x_3) = (2+\lambda/2-3\mu/2,\lambda,\mu) = (2,0,0) + \lambda(1/2,1,0) + \mu(-3/2,0,1).$$
	Then the vector parametric description is
	$$V: \underbar{x} = (2,0,0) + \lambda(\frac{1}{2},1,0) + \mu(-\frac{3}{2}, 0, 1).$$
	To avoid fraction, you could also take
	$$V: \underbar{x} = (2,0,0) + \rho(1,2,0) + \sigma(-3,0,2).$$
\end{example}

\begin{example}
	To find an equation of the plane $V$ with vector parametric equation
	$\underbar{x} = (2,0,0) + \lambda(1,1,0) + \mu(0,2,1)$. We can find
	two vectors in the plane, say $\underbar{v}$ and $\underbar{w}$, and then
	use the cross product to find the normal vector to the plane. Then we can
	use the normal vector to find the equation of the plane.

	For example, we can take $\underbar{v} = (1,1,0)$ and $\underbar{w} = (0,2,1)$.
	Then the normal vector to the plane is $\underbar{n} = \underbar{v} \times \underbar{w} = (1,1,0) \times (0,2,1) = (1,-1,2)$.
	Then the equation of the plane is
	$$V: \underbar{n} \cdot (\underbar{x} - (2,0,0)) = 0$$
	\begin{align*}
		(1,-1,2) \cdot (x_1-2,x_2,x_3) & = 0 \\
		x_1-2 -x_2 + 2x_3              & = 0 \\
		x_1-x_2+3x_3                   & = 2 \\
	\end{align*}
\end{example}
\begin{definition}[Cross product]
	Let $\underbar{u} = (u_1,u_2,u_3)$ and $\underbar{v} = (v_1,v_2,v_3)$ be two vectors in $\mathbb{R}^3$. The cross product of $\underbar{u}$ and $\underbar{v}$ is the vector
	$$\underbar{u} \times \underbar{v} = (u_2v_3-u_3v_2,u_3v_1-u_1v_3,u_1v_2-u_2v_1).$$
	The cross product is orthogonal to both $\underbar{u}$ and $\underbar{v}$.
\end{definition}

% \begin{problem}
% Determine a parametric equation for each of the lines in a) and b) and for
% each of the planes in c) and d).
% \begin{enumerate}[label=\alph*.]
% 	\item The line passing through $(2, 1, 5)$ and $(5, -1, 4)$.
% 	\item The line passing through $(1, 2)$ and $(2, 4)$.
% 	\item The plane passing through $(1, 2, 2)$, $(0, 1, 1)$ and $(1, 3, 2)$.
% 	\item The plane containing the line $x = (-2, 1, 3) + \lambda(1, 2, -1)$ and the point $(4, 0, 3)$.
% \end{enumerate}
% \end{problem}

% \begin{problem}
% Determine whether $(3, 4, 0)$ is on the line with parametric description $x =
% 	(1, 2, 1) + \lambda(2, 2, -1)$. Are $x = (3, 4, 0) + \lambda(2, 2, -1)$ and $x = (1, 2, 1) + \mu(-2, -2, 1)$
% vector parametric equations of the same line?
% \end{problem}


\end{document}
