\section{Introduction}

\subsection{Introduction}

\begin{definition}[Algorithm]
    A well-defined computational procedure that takes some value, or set of
    values, as input and produces some value, or set of values, as output. An
    algorithm is thus a sequence of computational steps that transform the
    input into the output.
\end{definition}

\begin{definition}[Data Structure]
    A way to store and organize data in order to facilitate access and modifications.
\end{definition}

\begin{definition}[Abstract Data Type (ADT)]
    A data type that is defined by the operations that may be performed on it.
\end{definition}

\subsection{Describing algorithms}
A complete description of an algorithm consists of three parts:
\begin{enumerate}
    \item \textbf{The algorithm}
        \begin{itemize}
            \item A description of the algorithm in a high-level language (English, pseudocode, etc).
            \item Include clear specification of used data structures.
        \end{itemize}
    \item \textbf{Proof of correctness}
    \item \textbf{Derivation of running time}
\end{enumerate}

\subsection{Sorting}

\subsubsection*{Insertion sort}
\begin{algorithm}[H]
    \KwIn{A sequence of $n$ numbers $A = \langle a_1,\dots,a_n \rangle$}
    \KwOut{A permutation of the input such that $\langle f(a_1) \le f(a_2) \le \dots \le f(a_n) \rangle$}
    initialize: sort $A[1]$
    \For{$j = 2$ to $A.length$}{
        $key = A[j]$ \\
        $i = j - 1$ \\
        \While{$i > 0$ and $A[i] > key$}{
            $A[i + 1] = A[i]$ \\
            $i = i - 1$
        }
        $A[i + 1] = key$}
    % \caption{Insertion sort}
\end{algorithm}
\subsubsection*{Loop invariants and the correctness of insertion sort}

