\section{Sequences}

\subsection{Sequence}
\begin{definition}[Sequence]
    A sequence is a function for which the domain is $\mathbb{N}$.
    $$a: \mathbb{N} \to Y$$

    Y can be any set.
\end{definition}

\begin{example}
    Here are some functions that are sequences:
    \begin{enumerate}
        \item $a: \mathbb{N} \to \mathbb{Q}$
        \item $b: \mathbb{N} \to (\mathbb{N} \to Y)$
        \item $c: \mathbb{N} \to \mathbb{N}$
    \end{enumerate}

    And some functions that are not sequences:
    \begin{enumerate}
        \item $d: (\mathbb{N} \to \mathbb{N}) \to \mathbb{N}$
        \item $e: \mathbb{Q} \to \mathbb{N}$
    \end{enumerate}
\end{example}

\subsection{Terminology around sequences}
\subsubsection{Bounded sequences}
\begin{definition}[bouneded sequence]
    Let $(X, \dist)$ be a metric space. We say a sequence $a: \naturals \to X$ is bounded if
    \begin{myCenter}
        \tab{there exists $q \in X$, \\}
        \tab{ there exists $M > 0$, \\}
        \tab{for all $n \in \naturals$, \\}
        \tab{$\dist(a_n,q) \le M$.}
    \end{myCenter}
\end{definition}

In a normed linear space, we can use a simpler criterion to check whether a sequence is bounded. That is the content of the
following proposition.

\begin{proposition}
    Let $(V, \norm{\cdot})$ be a normed vector space. Let $a: \naturals \to V$ be a sequence. The sequence $a$ is bounded
    if and only if
    \begin{myCenter}
        \tab{there exists $M > 0$, \\}
        \tab{ for all $n \in \naturals$, \\}
        \tab{$\norm{a_n} \le M$.}
    \end{myCenter}
\end{proposition}

\subsection{Convergence of sequences}
\begin{definition}[Convergence of sequences]
    Let $(X, \dist)$ be a metric space. We say that a sequence $a: \naturals \to X$ converges to a point $p \in X$ if
    \begin{myCenter}
        \tab{for all $\epsilon > 0$, \\}
        \tab{there exists $N \in \naturals$, \\}
        \tab{for all $n \ge N$, \\}
        \tab{$\dist(a_n,p) < \epsilon$.}
    \end{myCenter}
\end{definition}

We sometimes write
$$\lim_{n\to\infty}a_n = p$$
to express that the sequence $(a_n)$ converges to $p$.

\begin{definition}[Divergence of sequences]
    Let $(X, \dist)$ be a metric space. A sequence $a: \naturals \to X$ is called \emph{divergent} is it is not convergent.
\end{definition}

\subsection{Examples and limits of simple sequences}
\begin{proposition}[The constant sequence]
    Let $(X, \dist)$ be a metric space. Let $p \in X$ and assume that the sequence $(a_n)$ is given by $a_n = p$ for every
    $n \in \naturals$. We also say that $(a_n)$ is a constant sequence. Then $\lim_{n\to\infty} = p$.
\end{proposition}

\begin{example}[A standard limit]
    Let $a: \N \to \R$ be a real-valued sequence such that $a_n = 1/n$ for $n \ge 1$. Then $a: \N \to \R$ converges to $0$.
\end{example}

\begin{proof}
    Let $\epsilon > 0$. Choose $N = \ceil{1/\epsilon} + 1$. Take $n \ge N$. Then
    $$\dist_\R(a_n,0) = |a_n - 0| = |1/n| = 1/n \le 1/N < \epsilon.$$
\end{proof}

\subsection{Uniqueness of limits}
\begin{proposition}[Uniqueness of limits]
    Let $(X, \dist)$ be a metric space and let $a: \N \to \R$ be a sequence in $X$. Assume that $p,q \in X$ and assume that
    $$\lim_{n\to\infty} = p \text{ and } \lim_{n\to\infty}a_n = q$$
    Then $p = q$.
\end{proposition}

\subsection{More properties of convergent sequences}
\begin{proposition}
    Let $(X, \dist)$ be a metric space and suppose that $a: \N \to X$ is a sequence. Let $p \in X$. Then the sequence $a: \N \to X$ converges to $p$ if and only if
    the real-valued sequence
    $$n \mapsto \dist(a_n, p)$$
    converges to 0 in $\R$.
\end{proposition}

\begin{proposition}[Convergent sequences are bounded]
    Let $(X, \dist)$ be a metric space. Let $a: \N \to X$ be a sequence in $X$ converging to $p \in X$. Then the sequence $a: \N \to X$ is bounded.
\end{proposition}

\begin{proposition}
    Let $(X, \dist)$ be a metric space and let $a: \N \to X$ and $b: \N \to X$ be two sequences. Let $p \in X$ and suppose that $\lim_{n\to\infty}a_n = p$.
    Then $\lim_{n\to\infty}b_n = p$ if and only if
    $$\lim_{n\to\infty}\dist(a_n,b_n) = 0$$
\end{proposition}

\begin{corollary}[Eventually equal sequences have the same limit]
    Let $(X, \dist)$ be a metric space and let $a: \N \to X$ and $b: \N \to X$ be two sequences such that there exists an $N \in \N$ such that for all $n \ge N$,
    $$a_n = b_n$$
    Then the sequence $a: \N \to X$ converges if and only if the sequence $b: \N \to X$ converges. If the sequences converge, they have the same limit.
\end{corollary}

\subsection{Limit theorems for sequences taking values in a normed vector space}
\begin{theorem}
    Let $(V, \norm{\cdot})$ be a normed vector space and let $a: \N \to V$ and $b: \N \to V$ be two sequences. Assume that the $\lim_{n\to\infty}a_n$ exists and is equal to $p \in V$
    and that the $\lim_{n\to\infty}b_n$ exists and is equal to $q \in V$. Let $\lambda: \N \to \R$ be a real-valued sequence. Let $\mu \in \R$.
    Assume that $\lim_{n\to\infty}\lambda_n = \mu$. Then
    \begin{enumerate}
        \item The $\lim_{n\to\infty}(a_n + b_n)$ exists and is equal to $p + q$.
        \item The $\lim_{n\to\infty}(\lambda_n a_n)$ exists and is equal to $\mu p$.
    \end{enumerate}
\end{theorem}

\subsection{Index shift}
\begin{proposition}[Index shift]
    Let $(X, \dist)$ be a metric space and let $a: \N \to X$ be a sequence. Let $k \in \N$ and $p \in X$. Then the sequence $a: \N \to X$ converges to $p$ if and only if the sequence
    $(a_{n+k})_n$ (i.e. the sequence $n \mapsto a_{n+k}$) converges to $p$.
\end{proposition}
