\section{Riemann integration in one dimension}
\subsubsection*{Main message:}
\begin{itemize}
    \item Every continuous function $f: [a,b] \to \R$ is Riemann integrable.
    \item The fundamental theorem of calculus, mainly the part that when
        $F:[a,b] \to \R$ and $f:[a,b] \to \R$ satisfy that $F^\prime = f$
        and $f$ is bounded and Riemann integrable, then
        $$\int_a^bf(x)dx = F(b)-F(a).$$
\end{itemize}

\subsection{Riemann integrable functions and the Riemann integral}
\begin{definition}[Partition]
    A \emph{partition} $P$ of an interval $[a,b]$ (with $n$ intervals) is
    a subset $\{x_0,x_1,\dots,x_n\} \subset [a,b]$ such that $a=x_0<x_1<\dots<x_n=b$
\end{definition}

\begin{definition}[Upper/Lower sum]
    Let $f:[a,b] \to \R$ be a bounded function and let $P = (x_0,x_1,\dots,x_n)$
    be a partition of $[a,b]$. Then the upper sum of $f$ with respect to $P$ is
    defined as
    $$U(P,f) := \sum_{k=1}^n M_k \Sigma x_k$$
    where $\Sigma x_k := (x_k - x_{k-1}$ and
    $$M_k := \sup_{x\in[x_{k-1},x_k]}f(x).$$
    Similarly, we define the lower sup of $f$ with respect to $P$ as
    $$L(P,f) := \sum_{k=1}^n m_k \Sigma x_k$$
    where
    $$m_k := \inf_{x\in[x_{k-1},x_k]}f(x).$$
\end{definition}

\begin{definition}[Refinement]
    Let $\tilde{P}$ be a partition of $[a,b] \subseteq \R$. A partition $P$ is called
    a \emph{refinement} of $\tilde{P}$ if $\tilde{P} \subset P$.

    If $\tilde{P}, \tilde{Q}$ are two partitions of $[a,b]$, then a partition $P$ is
    called a \emph{common refinement} of $\tilde{P}$ and $\tilde{Q}$ if $P$ is both
    a refinement of $\tilde{P}$ and $\tilde{Q}$.
\end{definition}

Note that two partitions $P,Q$ always have a common refinement, namely $P \cup Q$.

\begin{proposition}
    For every bounded $f: [a,b] \to \R$ and every partition $P$ of $[a,b]$ we have
    $$L(P,f) \leq U(P,f).$$
\end{proposition}

\begin{definition}[Darboux integral]
    Let $f:[a,b] \to \R$ be a bounded function. We define the upper Darboux integral
    of $f$ as
    $$\overline{\int_a^b}fdx := \inf\{U(P,f) \mid P \text{ partition of } [a,b] \}$$
    and the lower Darboux integral of $f$ as
    $$\underline{\int_a^b}fdx := \sup\{L(P,f) \mid P \text{ partition of } [a,b] \}.$$
\end{definition}

\begin{proposition}
    Let $f:[a,b] \to \R$ be a bounded function. Then
    $$\underline{\int_a^b}fdx \leq \overline{\int_a^b}fdx.$$
\end{proposition}

\begin{definition}[Riemann integral]
    Let $f:[a,b] \to \R$ be a bounded function. We say that $f$ is \emph{Riemann integrable} if
    $$\underline{\int_a^b}fdx = \overline{\int_a^b}fdx.$$

    In this case we say that the \emph{Riemann integral} of $f$ over $[a,b]$ is
    $$\int_a^b f(x)dx := \underline{\int_a^b}fdx = \overline{\int_a^b}fdx.$$
\end{definition}

\begin{proposition}[Alternative characterization of Riemann integrability]
    Let $f:[a,b] \to \R$ be a bounded function. Then $f$ is Riemann integrable if and only if
    \begin{myCenter}
        \tab{for every $\epsilon > 0$,}
        \tab{there exists a partition $P$ of $[a,b]$, }
        \tab{$U(P,f) - L(P,f) < \epsilon$}
    \end{myCenter}
\end{proposition}

\begin{definition}
    We denote the set of bounded, Riemann-integrable functions $f:[a,b] \to \R$ by
    $\mathcal R[a,b]$.
\end{definition}

\subsection{Sums, products of Riemann integrable functions}
\begin{proposition}[R[a,b] is a vector space]
    Let $f,g \in \mathcal R [a,b]$ and let $\lambda \in \R$. Then
    \begin{enumerate}
        \item The function $f+g \in \mathcal R[a,b]$ and
            $$\int_a^b(f(x)+g(x))dx = \int_a^bf(x)dx + \int_a^bg(x)dx$$
        \item The function $\lambda f \in \mathcal R[a,b]$ and
            $$\int_a^b\lambda f(x)dx = \lambda \int_a^bf(x)dx$$
    \end{enumerate}
\end{proposition}

\begin{definition}
    If $f:[a,b] \to \R$ is bounded and Riemann integrable on $[a,b]$,
    then we define
    $$\int_b^a f(x)dx := -\int_a^b f(x)dx.$$
\end{definition}

\begin{proposition}[Further properties]
    Let $f,g[a,b] \to \R$ be two bounded and Riemann integrable functions.
    \begin{enumerate}[label=\roman*.]
        \item We have
            $$\int_a^b1dx = b-a.$$
        \item If for all $x \in [a,b]$, we have $f(x) \le g(x)$, then \hfill (monotonicity)
            $$\int_a^bf(x)dx \le \int_a^bg(x)dx.$$
        \item If $z \in (a,b)$, then $f$ is integrable on $[a,z]$ and $[z,b]$ and \hfill (restriction)
            $$\int_a^bf(x)dx = \int_a^zf(x)dx + \int_z^bf(x)dx.$$
        \item We have\hfill \text{(triangle inequality)}
            $$\left|\int_a^bf(x)dx\right| \le \int_a^b|f(x)|dx. $$
        \item The function $f \cdot g$ is Riemann integrable on $[a,b]$.
    \end{enumerate}
\end{proposition}

\subsection{Continuous functions are Riemann integrable}
\begin{proposition}
    Let $f:[a,b] \to \R$ be continuous. Then $f$ is Riemann integrable.
\end{proposition}

\subsection{The fundamental theorem of calculus}
\begin{theorem}[Fundamental theorem of calculus]
    \begin{enumerate}[label=\roman*.]
        \item Let $f:[a,b] \to \R$ be continuous. Then the function
            $$F:[a,b] \to \R$$
            given by
            $$F(x) := \int_a^xf(s)ds$$
            is differentiable on $(a,b)$ and for all $x \in (a,b)$
            $$F^\prime(x) = f(x).$$
        \item Let $F:[a,b] \to \R$ be an anti-derivative of a function $f:[a,b] \to \R$,
            i.e. for all $x \in (a,b)$,
            $$F^\prime(x) = f(x)$$
            and suppose that $f$ is bounded and Riemann integrable on $[a,b]$.
            Then
            $$\int_a^bf(x)dx = F(b) - F(a).$$
    \end{enumerate}
\end{theorem}
