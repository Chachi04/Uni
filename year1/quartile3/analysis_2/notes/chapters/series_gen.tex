\section{Series with general terms}



\subsection{Series with real terms: the Leibniz test}
\begin{theorem}[Leibniz test, a.k.a Alternating series test]
    Let $a,b: \N \to \R$ be two real-valued sequences such that for all $k \in \N$, $b_k = (-1)^ka_k$. Assume that there exists a $K \in \N$ such that
    \begin{enumerate}
        \item $a_k \ge 0$ for every $k \ge K$,
        \item $a_k \ge a_{k+1}$ for every $k \ge K$,
        \item $\lim_{k \to \infty} a_k = 0$.
    \end{enumerate}
    Then, the series
    $$\sum_{k=K}^\infty b_k = \sum_{k=K}^\infty (-1)^k a_k$$
    is convergent. In addition, the following esitmate holds for every $N \ge K$,
    $$\left| S_N - \sum_{k=K}^\infty b_k \right| \le a_{N+1}.$$
    where for all $n \in \N$, $S_n := \sum_{k=K}^\infty b_k$.
\end{theorem}

\begin{example}
    We claim that the series
    $$\sum_{k=1}^\infty (-1)^k \frac{1}{k}$$
    converges.

    We would like to apply the Alternating series test. To do so, we need toe check its conditions.

    We define the sequence $a: \N \to \R$ by 
    $$a_k := \frac{1}{k}$$
    for $k \ge 1$ (and $a_0 = a_1 = 1$).

    We now check the conditions for the Alternating Series Test.

    \begin{enumerate}
        \item We need to show that $a_k \ge 0$ for all $k \in \N$. Let $k \in \N$. Then, 
            $$a_k = \frac{1}{k} \ge 0.$$ 
        \item We need to show that $a_k \ge a_{k+1}$ for all $k \in \N$. Let $k \in \N$. Then, 
            $$a_k = \frac{1}{k} \ge \frac{1}{k+1} = a_{k+1}.$$
        \item We need to show that
            $$\lim_{k\to\infty} a_k = 0$$.
            This follow as this is a standard limit.
    \end{enumerate}

    It follows from the Alternating Series Test that the series
    $$\sum_{k=1}^\infty (-1)^k \frac{1}{k}$$
    converges.
\end{example}

\subsection{Series charactersization of completeness in normed vector space}
\begin{definition}
    Let $(V, \norm{\cdot})$ be a normed vector space. Let $a: \N \to V$ be a sequence of vectors in $V$. We say the series
    $$\sum_{k=0}^\infty a_k$$
    converges \emph{absolutely} if
    $$\sum_{k=0}^\infty \norm{a_k}$$
    converges.
\end{definition}

\begin{definition}[Series characterization of completeness]
    We say a normed vector space $(V, \norm{\cdot})$ satisfies the \emph{series characterization of completeness} if every series in $V$ that is absolutely convergent is also convergent.
\end{definition}

\begin{proposition}
    Every finite-dimensional normed vector space satisfies the series characterization of completeness.
\end{proposition}
\begin{example}
    Consider the series 
    $$\sum_{k=1}^\infty \frac{\sin(k)}{k^2}.$$
    Since this is not an alternataing series, we cannot apply the Leibniz test. 

    However, for every $k \\in \N \setminus \{0\}$, we have
    $$\left| \frac{\sin(k)}{k^2}\right| \le \frac{1}{k^2}.$$
    The series
    $$\sum_{k=1}^\infty \frac{1}{k^2}$$
    is a standard hyperharmonic seris, of which we know that it converges. By the Cmomparison Test, we conclude that the series
    $$\sum_{k=1}^\infty \left|\frac{\sin(k)}{k^2}\right|$$
    converges as well.

    Therefore, the series 
    $$\sum_{k=1}^\infty \frac{\sin(k)}{k^2}$$
    converges absolutely. Since $(\R, | \cdot |)$ is complete, we find that
    $$\sum_{k=1}^\infty \frac{\sin(k)}{k^2}$$
    converges.
\end{example}

\begin{definition}
        Let $(V, \norm{\cdot})$ be a normed vector space. Let $a: \N \to V$ be a sequence. We say that a series
        $$\sum_{k=0}^\infty a_k$$
        converges \emph{conditionally} if it converges but does not converge absolutely.
\end{definition}

\subsection{The Cauchy product}
\begin{theorem}[Cauchy product]
    Let $a,b: \N \to \R$ be two real-valued sequences. Assume that the series
    $$\sum_{k=0}^\infty a_k$$
    and
    $$\sum_{k=0}^\infty b_k$$
    converge absolutely. Then, the series
    $$\sum_{k=0}^\infty c_k$$
    converges absolutely as well, where 
    $$c_k := \sum_{\ell = 0}^k a_\ell b_{k-\ell},$$ and
    $$\sum_{k=0}^\infty c_k = \left(\sum_{k=0}^\infty a_k\right)\left(\sum_{k=0}^\infty b_k\right)$$
\end{theorem}
