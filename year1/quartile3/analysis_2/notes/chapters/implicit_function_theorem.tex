\section{Implicit function theorem}

\subsection{The objective}
We will be considering continuously differential functions $f: \Omega \subset \R^{d+m} \to \R^m$.
It is good to think of the vector space $\R^{d+m}$ as the vector space $\R^d \bigoplus \R^m$, i.e.
as the vector space of pairs $(x,y)$ of vectors $x \in \R^d$ and $y \in \R^m$.

\subsection{Notation}
Consider a function $f: \R^{d+m} \to \R^m$.
Then for a point $(a,b) \in \R^{d+m}$
$$[Df]_{(a,b)}= \left(\begin{array}{ccc|ccc}
    \frac{\partial f_1}{\partial x_1}((a,b)) & \dots & \frac{\partial f_d}{\partial x_1}((a,b)) & \frac{\partial f_{d+1}}{\partial x_1}((a,b)) & \dots & \frac{\partial f_{d+m}}{\partial x_1}((a,b)) \\
    \vdots & \ddots & \vdots & \vdots & \ddots & \vdots \\
    \frac{\partial f_1}{\partial x_m}((a,b)) & \dots & \frac{\partial f_d}{\partial x_m}((a,b)) & \frac{\partial f_{d+1}}{\partial x_m}((a,b)) & \dots & \frac{\partial f_{d+m}}{\partial x_m}((a,b)) \\
    \end{array}\right)
$$
We will denote by
$$(D_1f)_{(a,b)}:\R^d\to\R^m$$
the restriction of the derivative $(Df)_{(a,b)}:\R^{d+m} \to \R^m$ to the subspace $\R^d \subset \R^{d+m}$.
In other words, for all $h \in \R^d$,
$$(D_1f)_{(a,b)}(h) = (Df)_{(a,b)}((h,0)).$$

Similarly, we will denote by
$$(D_2f)_{a,b}:\R^m \to \R^m$$
the restriction of the derivative $(Df)_{(a,b)}:\R^{d+m} \to \R^m$ to the subspace $\R^m \subset \R^{d+m}$.
In other words, for all $k \in \R^m$.
$$(D_2f)_{(a,b)}(k) = (Df)_{(a,b)}((0,k)).$$

By linearity of the derivative, we have that for all $h \in \R^d$ and $k \in \R^m$,
$$(Df)_{(a,b)}((h,k)) = (D_1f)_{(a,b)}(h) + (D_2f)_{(a,b)}(k).$$

We will denote the matrix representations of the maps $(D_1f)_{(a,b)}$ and $(D_2f)_{(a,b)}$ by
$[D_1f]_{(a,b)}$ and $[D_2f]_{(a,b)}$ respectively.\\
So
$$[D_1f]_{(a,b)} = \left(\begin{array}{ccc}
    \frac{\partial f_1}{\partial x_1}((a,b)) & \dots & \frac{\partial f_d}{\partial x_1}((a,b)) \\
    \vdots & \ddots & \vdots \\
    \frac{\partial f_1}{\partial x_m}((a,b)) & \dots & \frac{\partial f_d}{\partial x_m}((a,b)) \\
    \end{array}\right)
$$
and 
$$[D_2f]_{(a,b)} = \left(\begin{array}{ccc}
    \frac{\partial f_{d+1}}{\partial x_1}((a,b)) & \dots & \frac{\partial f_{d+m}}{\partial x_1}((a,b)) \\
    \vdots & \ddots & \vdots \\
    \frac{\partial f_{d+1}}{\partial x_m}((a,b)) & \dots & \frac{\partial f_{d+m}}{\partial x_m}((a,b)) \\
    \end{array}\right)
$$
Then
\begin{align*}
    (Df)_{(a,b)}((h,k)) &= (D_1f)_{(a,b)}(h) + (D_2f)_{(a,b)}(k)\\
                        &= [D_1f]_{(a,b)}\begin{pmatrix}h_1\\h_2\\h_3\end{pmatrix} + [D_2f]_{(a,b)}\begin{pmatrix}k_1\\k_2\end{pmatrix}.
\end{align*}

\subsection{The implicit function theorem}
\begin{theorem}[Implicit function theorem]
    Let $\Omega \subset \R^{d+m}$, $f:\Omega \to \R^m$ be a continuously differentiable function and let $(a,b) \in \Omega$ be a point such that
    $f(a,b) = 0$ and the matrix $[D_2f]_{(a,b)}$ is invertible. Then there exists an $r_1 > 0$ and an $r_2 > 0$, and a continuously differentiable function
    $g:B(a,r_1) \to \R^m$ such that for all $x \in B(a,r_1)$ and all $y \in B(b,r_2)$,
    $$f(x,y) = 0 \qquad \text{if and only if} \qquad y=g(x).$$

    Moreover, for all $x \in B(a,r_1),$
    $$(Dg)_x = -(D_2f)\inv_{(x,g(x))} \circ (D_1f)_{(x,g(x))}.$$
\end{theorem}

\subsection{The inverse function theorem}
\begin{theorem}[Inverse function theorem]
    Let $\Omega \subset \R^m$ be open and let $h: \Omega \to \R^m$ be a function which is continuously differentiable.
    Suppose $b \in \Omega$ and suppose that $(Dh)_b$ is non-singular. Then there exists an $r_1 > 0$ and an $r_2 >0$,
    and a continuously differentiable function $g: B(h(b),r_1) \to \R^m$ such that for all $x \in B(h(b),r_1)$
    and all $y \in B(b,r_2)$,
    $$x = h(y) \qquad \text{if and only if} \qquad y=g(x).$$

    Moreover, for all $x \in B(h(b),r_1)$,
    $$(Dg)_x=(Dh)\inv_{g(x)}.$$

    In particular, for $r_3 > 0$ small enough, the function $h$ restricted to $B(b,r_3)$ mapping to $h(B(b,r_3))$
    is invertible with continuously differentiable inverse $g$.
\end{theorem}
