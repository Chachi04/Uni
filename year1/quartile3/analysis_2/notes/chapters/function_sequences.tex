\section{Function sequences}

\subsection{Point-wise convergence}
\begin{definition}[Pointwise convergence]
    Let $(X, \dist_X)$ and $(Y,\dist_Y)$ be two metric spaces and let $D \subseteq X$.
    We say that a sequence of functions $f: \N \to (D \to Y)$ converges pointwise to
    a function $f^*:D \to Y$ if
    \begin{myCenter}
        \tab{for all $x \in D$,}
        \tab{$\displaystyle\lim_{n\to\infty}f_n(x) = f^*(x)$.}
    \end{myCenter}
\end{definition}

\begin{example}
    We consider a case in which $(X, \dist_X) = (\R, \dist_\R)$ and $D = [0,] \subset \R$.
    We consider the sequence of functions $f: \N \to ([0,1] \to \R)$ defined by
    $$f_n(x) = x^n.$$

    Then the sequence $(f_n)$ converges pointwise to the function $f^*:[0,1] \to \R$ defined by
    $$f^*(x) = \begin{cases}0 \quad\text{if }x \in [0,1) \\ 1 \quad\text{if }x = 1\end{cases}.$$

    To show this, let $x \in [0,1]$. Then we consider two cases.\\
    In case $x \in [0,1)$, then
    $$\lim_{n\to\infty}f_n(x) = \limn x^n = 0 = f^*(x) \hfill$$
    In case $x = 1$, then
    $$\limn f_n(x) = \limn 1^n = 1 = f^*(x).$$
\end{example}

\subsection{Uniform convergence}
\begin{definition}
    Let $(X, \dist_X)$ and $(Y,\dist_Y)$ be two metric spaces and let $D \subseteq X$. We say
    that a sequence of functions $f:\N \to (D \to Y)$ converges uniformly to a function $f^*:D \to Y$ if
    \begin{myCenter}
        \tab{for all $\epsilon > 0$,}
        \tab{there exists $N \in \N$ such that}
        \tab{for all $n \geq N$,}
        \tab{for all $x \in D$,}
        \tab{$\dist_Y(f_n(x), f^*(x)) < \epsilon$.}
    \end{myCenter}
\end{definition}

\begin{proposition}
    Let $(X,\dist_X)$ and $(Y,\dist_Y)$ be two metric spaces, let $D \subseteq X$, and assume
    that a sequence of functions $f:\N \to (D \to Y)$ converges uniformly to a function $f^*:D \to Y$.
    Then the sequence of functions $f$ converges pointwise to $f^*$.
\end{proposition}

\begin{corollary}
    Suppose a sequence of functions $f:\N \to (D \to Y)$ converges \emph{pointwise} to a function $f^*:D \to Y$.
    Then $(f_n)$ converges uniformly on $D$ if and only if $(f_n)$ converges uniformly to $f^*$ on $D$.
\end{corollary}

\subsection{Preservation of continuity under uniform convergence}
\begin{theorem}
    Let $(f_n)$ be a sequence of continuous functions from a domain $D$ in the metric space $(X,\dist_X)$ to the metric space $(Y,\dist_Y)$
    that converges uniformly to a function $g:D \to Y$. Then the function $g$ is also continuous on $D$.
\end{theorem}
\begin{example}
    Consider the sequence of functions $(f_n)$ from $[0,1]$ to $\R$ defined by
    $$f_n(x) = x^n.$$
    We have seen that the pointwise limit is $g:[0,1] \to \R$ given by
    $$g(x) := \begin{cases}0, \qquad\text{if } x \\in [0,1) \\1, \qquad\text{if }x=1.\end{cases}$$
    Because the function $g$ is not continuous, but for every $n \in \N$ the function $f_n$ is continuous,
    we converge that the sequence $f_n$ does not converge uniformly to $g$.
\end{example}

\subsection{Differentiability theorem}
\begin{theorem}
    Let $(f_n)$ be a sequence of functions from an open domain $\Omega$ in a vector space $V$ to $\R$ and suppose the sequence converges
    pointwise to a function $g:\Omega \to \R$. Suppose moreover that the functions $f_n$ are continuously differentiable on $\Omega$
    and suppose the sequence of functions $Df_n:\Omega \to \Lin(V,\R)$ converges uniformly to a function $\Delta:\Omega \to \Lin(V,\R)$.
    Then the function $g$ is differentiable on $\Omega$ as well and
    $$Dg = \Delta$$
\end{theorem}

\subsection{The normed vector space of bounded functions}
\begin{definition}
    Let $D$ be a set.
    The normed vector space $(\mathcal B(D),\dotnorm_\infty)$ is defined as the vector space of \emph{bounded} functions $D$ to $\R$
    with norm $\dotnorm_\infty$ given by
    $$\norm{f}_\infty = \sup_{x\in D}|f(x)|.$$
\end{definition}

\begin{proposition}
    Let $(f_n)$ be a sequence of functions from $D$ to $\R$ and let $f$ be a function. Then the sequence $(f_n)$ converges uniformly to $f$
    if and only if there exists an $N \in \N$ such that for every $n \ge N$, the function $(f_n - f)$ is bounded, and such that the
    sequence $n \mapsto (f_{N+n} - f)$ converges to 0 in $\mathcal B(D)$.
\end{proposition}

\begin{proposition}
    Let $(f_n)$ be a sequence of functions from a domain $D$ in the metric space $(X,\dist_X)$ to the metric space $(Y,\dist_Y)$
    and let $g: D \to Y$ be a function. Then $(f_n)$ converges to $g$ uniformly if and only if there exists an $N \in \N$ such
    that for every $n \ge N$, the function $h_n$ given by
    $$h_n(x) := \dist_Y(f_n(x),g(x))$$
    is bounded and the sequence $n \mapsto h_{N+n}$ converges to 0 in $\mathcal B(D)$.
\end{proposition}
