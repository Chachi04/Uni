\section{Differentiability of standard functions}

\begin{itemize}
    \item The constant function is differentiable
    \item Linear functions between finite dimensional normed vector spaces are always differentiable
    \item Sums, products, compositions and at times quotients of differentiable functions are differentiable.
\end{itemize}

\subsection{Global context}
We will consider two normed vector spaces $(V,\dotnorm_V)$ and $(W,\dotnorm_W)$ and a function $f:\Omega \to W$
where $\Omega \subseteq V$ is an open subset of $V$. We will assume that $V$ and $W$ are finite-dimensional, and
we will denote $v_1,\dots,v_d$ a basis in $V$, with corresponding coordinate map $\Phi$, and by $w_1,\dots,w_m$
a basis of $W$ with corresponding coordinate map $\Psi$.

\subsection{Polynomials and rational functions are differentiable}
\begin{proposition}[Differentiability of polynomials in one variable]
    For every $n \in \N$, it holds that the function $f: \R \to \R$ given by
    $f(x) = x^n$
    is differentiable with
    $$f^\prime(x) = nx^{n-1}.$$

    In other words, the derivative of $f$. i.e. $(Df): \R \to \Lin(\R,\R)$ is
    given by
    $$x \mapsto (h \mapsto nx^{n-1}h)$$
\end{proposition}

\begin{proposition}[Every polynomial is differentiable]
    Every polynomial on $\R^d$ is differentiable
\end{proposition}

\begin{proposition}[Every rational function is differentiable on its domain]
    Let $p:\R^d \to \R$ and $q:\R^d \to \R$ be two polynomials. Let
    $$D:=\{x \in \R^d \mid q(x) \ne 0\}.$$
    Then the function $f:D \to \R$ given by
    $$f(x) = \frac{p(x)}{q(x)}$$
    is differentiable.
\end{proposition}

\subsection{Differentiability of other standard functions}
\begin{proposition}
    The functions
    \begin{center}
        \begin{tabular}{l r}
            $\exp:\R\to\R$ & $\ln:(0,\infty) \to \R$ \\
            $\sin:\R\to\R$ & $\cos:\R\to\R$ \\
            $\tan:(-\pi/2,\pi/2)\to\R$ & $\arctan:\R\to\R$
        \end{tabular}
    \end{center}
    are all differentiable on their domain, while the functions
    \begin{center}
        \begin{tabular}{l r}
            $\arcsin:[-1,1] \to \R$ & $\arccos:[-1,1] \to \R$
        \end{tabular}
    \end{center}
    are both differentiable on the interval $(-1,1)$.

    The derivatives are given by: \newline
    \begin{minipage}{.5\linewidth}
        \begin{align*}
            \exp^\prime(t) &= \exp(t) \\
            \sin^\prime(t) &= \cos(t) \\
            \tan^\prime(t) &= \frac{1}{\cos^2(t)} \\
            \arcsin^\prime(t) &= \frac{1}{\sqrt{1-t^2}}
        \end{align*}
    \end{minipage}
    \begin{minipage}{.5\linewidth}
        \begin{align*}
            \ln^\prime(t) &= 1/t \\
            \cos^\prime(t) &= -sin(t) \\
            \arctan^\prime(t) &= \frac{1}{1+t^2} \\
            \arccos^\prime(t) &= -\frac{1}{\sqrt{1-t^2}}
        \end{align*}
    \end{minipage}
\end{proposition}

\begin{example}
    Consider the function $f: \R \to \R^2$ given by 
    $$f(t) = (t^2, \sin(t))$$
    The component functions $f_1:\R \to \R$ and $f_2:\R\to\R$ are given by
    $$f_1(t) = t^2$$ 
    and 
    $$f_2(t) = \sin(t)$$ 
    Since these components are differentiable standard functions, we find that
    $f$ is differentiable as well and 
    $$f^\prime(t) = (f^\prime_1(t),f^\prime_2(t)) = (2t,\cos(t))$$
\end{example}
