\section{Real-valued sequences}

\subsection{Terminology}
\begin{definition}[increasing, decreasing and monotone sequences]
    We say a sequence $(a_n)$ is 
    \begin{enumerate}
        \item \emph{increasing} if for every $n \in \N, a_{n+1} \ge a_n$
        \item \emph{strictly increasing} if for every $n \in \N, a_{n+1} > a_n$
        \item \emph{decreasing} if for every $n \in \N, a_{n+1} \le a_n$
        \item \emph{strictly decreasing} if for every $n \in \N, a_{n+1} < a_n$
        \item \emph{monotone} if it is either increasing or decreasing
        \item \emph{strictly monotone} if it is either strictly increasing or strictly decreasing
    \end{enumerate}
\end{definition}

\begin{definition}[upper bound and lower bound for a sequence]
    We say that a number $M \in \R$ is an \emph{upper bound} for a sequence $a: \N \to \R$ if
    $$\text{for all } n \in \N$$
    $$a_n \le M$$

    We say that a number $m \in \R$ is a \emph{lower bound} for a sequence $a: \N \to \R$ if
    $$\text{for all } n \in \N$$
    $$a_n \ge m$$
\end{definition}

\begin{definition}[bounded sequence]
    We say that a sequence $a: \N \to \R$ is \emph{bounded above} if there exists an $M \in \R$ such that $M$ is an upper bound for $a$.

    We say that a sequence $a: \N \to \R$ is \emph{bounded below} if there exists an $m \in \R$ such that $m$ is a lower bound for $a$.
\end{definition}

\begin{proposition}
    Let $a: \N \to \R$ be a sequence. Then $a: \N \to \R$ is bounded if and only if it is both bounded above and bounded below.
\end{proposition}

\subsection{Monotone, bounded sequences and convergent}
\begin{theorem}
    Let $(a_n)$ be an increasing sequence that is bounded from above. Then $(a_n)$ convergent and
    $$\lim_{n\to\infty}a_n = \sup_{n\in\N}a_n \quad \left( = \sup \{a_n \mid n\in\N\}\right)$$
\end{theorem}

\begin{theorem}
    Let $(a_n)$ be a decreasing sequence that is bounded from below. Then $(a_n)$ is convergent and
    $$\lim_{n\to\infty}a_n = \inf_{n\in\N}a_n \quad \left( = \inf \{a_n \mid n\in\N\}\right)$$
\end{theorem}

\subsection{Limit theorems}
\begin{theorem}[Limit theorems for real-valued sequences]
    Let $a: \N \to \R$ and $b: \N \to \R$ be two converging sequences, and let $c,d \in \R$ be real numbers such that
    $$\lim_{n\to\infty}a_n = c \text{ and } \lim_{n\to\infty}b_n = d.$$
    Then
    \begin{enumerate}
        \item The $\lim_{n\to\infty}(a_n + b_n)$ exists and is equal to $c + d$.
        \item The $\lim_{n\to\infty}(a_nb_n)$ exists and is equal to $c \cdot d$.
        \item If $d \ne 0$, then $\lim_{n\to\infty}(\frac{a_n}{b_n}$ exists and is equal to $\frac{c}{d}$.
        \item For every non-negative integer $m \in \N$, the  limit $\lim_{n\to\infty}(a_n)^m$ exists and is equal to $c^m$.
        \item If for every $n \in \N$, the number $a_n$ is non-negative, then for every positive integer $k \in \N \setminus \{0\}$, the limit $\lim_{n\to\infty}(a_n)^{\frac{1}{k}}$ exists and is equal to $c^{\frac{1}{k}}$.
    \end{enumerate}
\end{theorem}

\subsection{The squeeze theorem}
\begin{theorem}[The squeeze theorem]
    Let $a,b,c : \N \to \R$ be three sequences. Suppose that there exists an $N \in \N$ such that for every $n \ge N$, we have
    $$a_n \le b_n \le c_n$$
    and assume $\lim_{n\to\infty}a_n = \lim_{n\to\infty}c_n = L$ for some $L \in \R$. Then $\lim_{n\to\infty}b_n$ exists and is equal to $L$.
\end{theorem}

\subsection{Divergence to \texorpdfstring{$\infty$}{infinity} and \texorpdfstring{$-\infty$}{minus infinity}}
\begin{definition}
    We say a sequence $a: \N \to \R$ \emph{diverges to $\infty$} and write 
    $$\lim_{n\to\infty} = \infty$$
    if
    \begin{myCenter}
        \tab{for all $M \in \R$, \\}
        \tab{there exists $N \in \N$, \\}
        \tab{for all $n \ge N$, \\}
        \tab{$a_n > M$.}
    \end{myCenter}

    Similarly, we say a sequence $(a_n)$ \emph{diverges} to $-\infty$ and write
    $$\lim_{n\to\infty}a_n = -\infty$$
    if
    \begin{myCenter}
        \tab{for all $M \in \R$, \\}
        \tab{there exists $N \in \N$, \\}
        \tab{for all $n \ge N$, \\}
        \tab{$a_n < M$.}
    \end{myCenter}
\end{definition}

\begin{proposition}
    Let $a: \N \to \R$ be a sequence such that 
    $$\lim_{n\to\infty}a_n = \infty.$$
    Then the sequence $(a_n)$ is bounded from below.

    Similarly, let $b: \N \to \R$ be a sequence such that 
    $$\lim_{n\to\infty}b_n = -\infty.$$
    Then the sequence $(b_n)$ is bounded from above.
\end{proposition}

\subsection{Limit theorems for improper limits}
\begin{theorem}[Limit theorems for improper limits]
    Let $a,b,c,d: \N \to \R$ be four sequences such that
    $$\lim_{n\to\infty}a_n = \infty \text{ and } \lim_{n\to\infty}c_n = -\infty$$
    the sequence $(b_n)$ is bounded from below and the sequence $(d_n)$ is bounded from above. Let $\lambda: \N \to \R$ be a sequence bounded below by some $\mu > 0$. Then
    \begin{enumerate}[label=\roman*.]
        \item $\lim_{n\to\infty}(a_n + b_n) = \infty$
        \item $\lim_{n\to\infty}(c_n + d_n) = -\infty$
        % \item $\lim_{n\to\infty}(a_n + d_n) = \infty$
        % \item $\lim_{n\to\infty}(c_n + b_n) = -\infty$
        \item $\lim_{n\to\infty}(\lambda_n a_n) = \infty$
        \item $\lim_{n\to\infty}(\lambda_n c_n) = -\infty$
    \end{enumerate}
\end{theorem}

\begin{proposition}
    Let $a: \N \to \R$ and $b: \N \to (0,\infty)$ be two sequences. Then
    \begin{enumerate}
        \item $\lim_{n\to\infty}a_n = \infty$ if and only if $\lim_{n\to\infty}(-a_n) = -\infty$.
        \item $\lim_{n\to\infty}b_n = \infty$ if and only if $\lim_{n\to\infty}\frac{1}{b_n} = 0$.
    \end{enumerate}
\end{proposition}

\subsection{Standard sequences}

\subsubsection{Geometric sequence}
\begin{proposition}[Standard limit of of geometric sequence]
    Let $q \in \R$. The sequence $(a_n)$ defined by $a_n := q^n$ for $n \in \N$
    \begin{itemize}
        \item converges to 0 if $q \in (-1,1)$
        \item converges to 1 if $q = 1$
        \item diverges to $\infty$ if $q > 1$
        \item diverges, but not to $\infty$ or $-\infty$ if $q \le -1$
    \end{itemize}
\end{proposition}

\subsubsection{The $n^{\text{th}}$ root of $n$}
\begin{proposition}[Standard limit of the $n^{\text{th}}$ root of $n$]
    The sequence $(a_n)$ defined by $a_n := \sqrt[n]{n}$ for $n \in \N$ converges to 1.
\end{proposition}

\begin{corollary}
    Let $a > 0$. Then the sequence $(b_n)$ defined by $b_n := \sqrt[n]{a}$ converges to 1.
\end{corollary}

\subsubsection{The number $e$}
First let's define the sequence $(a_n)$ by
$$a_n := \left(1 + \frac{1}{n}\right)^n.$$
We show that $(a_n)$ is increasing and bounded from above by $3$. Hence $(a_n)$ converges to some $e \in \R$ by the monotone convergence theorem.
\begin{lemma}
    The sequence $(a_n)$ defined by $a_n := \left(1 + \frac{1}{n}\right)^n$ for $n \in \N \setminus \{0\}$ and $a_0 = 1$ is increasing.
\end{lemma}

\begin{lemma}
    The sequence $(a_n)$ defined by $a_n := \left(1 + \frac{1}{n}\right)^n$ for $n \in \N \setminus \{0\}$ and $a_0 = 1$ is bounded from above by 3.
\end{lemma}

By these two lemmas, the sequence
$$n \mapsto \left(1 + \frac{1}{n}\right)^n$$
converges.

\begin{definition}[(Standard limit of $e$)]
    We define the number $e$ by
    $$e := \lim_{n\to\infty}\left(1 + \frac{1}{n}\right)^n.$$
\end{definition}

\subsubsection{Exponentials beat powers}
\begin{proposition}
    Let $a \in (1,\infty)$ and let $p \in (0,\infty)$. Then
    $$\limn\frac{n^p}{a^n} = 0.$$
\end{proposition}

\subsection{Sequences with values in \texorpdfstring{$\R^d$}{Rd}}
\begin{proposition}
    Consider the metric space $(\R^d, \norm{\cdot}_2)$. Let $z \in \R^d$ and let $x: \N \to \R^d$ be a sequence. Denote by $y_i$ the $i$th component of a vector $y \in \R^d$. Then the sequence $(x^{(n)})$ converges to $z$ if and only if for
    all $i \in \{1,\dots, d\}$, the sequence $(x_i^{(n)})$ converges to $z_i$.
\end{proposition}

