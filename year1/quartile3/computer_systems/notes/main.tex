\documentclass{classnotes}

\author{Jiaqi Wang}
\title{Computer Systems}

\begin{document}
\maketitle

\section{Lecture 1: History (chapter 1.1-1.4)}

\subsection{Did the ancient Greeks have computers?}
- Antikythera 1 century BC - Astronomical device (more than 30 gears).
\subsection{History of computers}
\begin{itemize}
    \item 1642: Pascal's calculator (could add and subtract numbers)
    \item 1671: Leibniz's calculator (could multiply and divide)
    \item 1822: Babbage's difference engine (calculate values of polynomials automatically)
    \item 1831: Babbage's analytical engine (full-fledged computer with storage, arithmetic units, programmable by punch cards)
    \item 1890: Hollerith's tabulating machine (used for the 1890 US census)
    % \item 1936: Turing's universal machine
    % \item 1943: Colossus
    % \item 1945: ENIAC
\end{itemize}


\include{chapters/optimization}

\include{chapters/computer_architecture}

\include{chapters/registers}

\include{chapters/state_automata}

\include{chapters/hardware_description_languages}

\include{chapters/processor_structure}

\include{chapters/instruction_optimization}

\include{chapters/assembly_language}

\include{chapters/rpi}

\include{chapters/interrupt_routines}

\include{chapters/compilers}

\include{chapters/computer_organization}

\section{History}
\subsection{Did the ancient Greeks have computers?}
- Antikythera 1 century BC - Astronomical device (more than 30 gears).
\subsection{History of computers}
\begin{itemize}
    \item 1642: Pascal's calculator (could add and subtract numbers)
    \item 1671: Leibniz's calculator (could multiply and divide)
    \item 1822: Babbage's difference engine (calculate values of polynomials automatically)
    \item 1831: Babbage's analytical engine (full-fledged computer with storage, arithmetic units, programmable by punch cards)
    \item 1890: Hollerith's tabulating machine (used for the 1890 US census)
    % \item 1936: Turing's universal machine
    % \item 1943: Colossus
    % \item 1945: ENIAC
\end{itemize}

% \subsection
\end{document}
