
\section{From Linear Algebra 1}

\subsection{Field}
\begin{definition}[Field]
    Sets in which we can compute like in $\R$.
\end{definition}

\subsection{Vector Space and Subspace}
\begin{definition}[Vector Space]
    A set $V$ with two operations, addition and scalar multiplication, such that
    \begin{enumerate}
        \item $V$ is an abelian group under addition.
        \item $\forall \alpha, \beta \in F, \forall \vec{u}, \vec{v} \in V$, we have
            \begin{enumerate}
                \item $\alpha(\vec{u} + \vec{v}) = \alpha\vec{u} + \alpha\vec{v}$.
                \item $(\alpha + \beta)\vec{u} = \alpha\vec{u} + \beta\vec{u}$.
                \item $(\alpha\beta)\vec{u} = \alpha(\beta\vec{u})$.
                \item $1\vec{u} = \vec{u}$.
            \end{enumerate}
    \end{enumerate}
\end{definition}

\begin{definition}[Subspace]
    A subset $W$ of a vector space $V$ is a subspace of $V$ if $W$ is a vector space under the same operations as $V$.
\end{definition}

\begin{definition}[Bases]
    A set of vectors $\{\vec{v}_1, \vec{v}_2, \cdots, \vec{v}_n\}$ is a basis of a vector space $V$ if
    \begin{enumerate}
        \item $\{\vec{v}_1, \vec{v}_2, \cdots, \vec{v}_n\}$ is linearly independent.
        \item $\{\vec{v}_1, \vec{v}_2, \cdots, \vec{v}_n\}$ spans $V$.
    \end{enumerate}
\end{definition}

\subsection{Linear Map}
\begin{definition}[Linear Map]
    A map $\A: V \to W$ ($V, W$ vector spaces) is linear if
    \begin{enumerate}
        \item $\A(\vec{u} + \vec{v}) = \A(\vec{u}) + \A(\vec{v})$.
        \item $\A(\lambda\vec{u}) = \lambda \A(\vec{u})$.
    \end{enumerate}
    Combined we have $\mathcal{A}(\lambda \vec{u} + \mu \vec{v}) = \lambda\mathcal{A}\vec{u} + \mu\mathcal{A}\vec{v}$.
\end{definition}

\begin{example}
Reflections, rotations, projections, identity map, zero map, etc.
\end{example}

\subsection{Multiplication with Matrices}
$ \mathcal{A}(\vec{v})) = A \cdot \vec{v} $, where $A$ is a matrix.

\subsection{Orthogonal projection}
\begin{definition}[Orthogonal projection]
    Let $V$ be a vector space with inner product $\langle \cdot, \cdot \rangle$. Let $W$ be a subspace of $V$. The orthogonal projection of $V$ onto $W$ is the linear map $\mathcal{P}_W: V \to W$ such that
    \begin{enumerate}
        \item $\mathcal{P}_W(\vec{v}) \in W$.
        \item $\vec{v} - \mathcal{P}_W(\vec{v}) \in W^\perp$.
    \end{enumerate}
\end{definition}

\begin{theorem}[Addition]
    For $\mathcal{A}, \mathcal{B}$ linear maps, we define $(\mathcal{A} + \mathcal{B})(\vec{v}) = \mathcal{A}(\vec{v}) + \mathcal{B}(\vec{v})$.
    % Let $V$ be a vector space with inner product $\langle \cdot, \cdot \rangle$. Let $W$ be a subspace of $V$. Then $\forall \vec{v} \in V$, we have
    % \begin{equation}
    %     \vec{v} = \mathcal{P}_W(\vec{v}) + \vec{v} - \mathcal{P}_W(\vec{v}).
    % \end{equation}
\end{theorem}

\begin{theorem}[Scalar multiplication]
    For $\mathcal{A}$ linear map, we define $(\lambda \mathcal{A})(\vec{v}) = \lambda \mathcal{A}(\vec{v})$.
\end{theorem}

\begin{theorem}[Composition]
    For $\mathcal{A}, \mathcal{B}$ linear maps, we define $(\mathcal{A} \circ \mathcal{B})(\vec{v}) = \mathcal{A}(\mathcal{B}(\vec{v}))$.
\end{theorem}

\begin{theorem}[Inverse]
    For $\mathcal{A}$ linear map, we define $\mathcal{A}^{-1}$ such that $\mathcal{A}^{-1} \circ \mathcal{A} = \mathcal{A} \circ \mathcal{A}^{-1} = \mathcal{I}$.
\end{theorem}

\subsubsection{Powers of maps}
\begin{enumerate}
    % \item $\mathcal{A}(\vec{0}) = \vec{0}$.
    \item $\mathcal{A}^2 = \mathcal{A}\mathcal{A}$
    \item $\mathcal{A}^n = \mathcal{A}\mathcal{A}^{n-1}$
    \item $\mathcal{A}^{-n} = (\mathcal{A}^{-1})^n$
\end{enumerate}

\subsubsection{Null space and range}
\begin{definition}[Null space]
    The null space of a linear map $\mathcal{A}: V \to W$ is the set of vectors $\vec{v} \in V$ such that $\mathcal{A}(\vec{v}) = \vec{0}$.
\end{definition}

\begin{definition}[Range]
    The range of a linear map $\mathcal{A}: V \to W$ is the set of vectors $\vec{w} \in W$ such that $\exists \vec{v} \in V$ such that $\mathcal{A}(\vec{v}) = \vec{w}$.
\end{definition}

\begin{theorem}
    Let $\mathcal{A}: V \to W$ be a linear map. Then $\mathcal{A}$ is injective if and only if $\mathcal{N}(\mathcal{A}) = \{\vec{0}\}$.
\end{theorem}

\begin{theorem}
    Let $\mathcal{A}: V \to W$ be a linear map. Then $\mathcal{A}$ is surjective if and only if $\mathcal{R}(\mathcal{A}) = W$.
\end{theorem}

\begin{theorem}
    Let $\mathcal{A}: V \to W$ be a linear map. Then $\mathcal{A}$ is bijective if and only if $\mathcal{N}(\mathcal{A}) = \{\vec{0}\}$ and $\mathcal{R}(\mathcal{A}) = W$.
\end{theorem}

\subsubsection{Null space / Range for matrix multiplication}
\begin{theorem}
    Let $A$ be an $m \times n$ matrix. Then $\mathcal{N}(A) = \{\vec{v} \in V \mid A\vec{v} = \vec{0}\}$ and $\mathcal{R}(A) = \{\vec{w} \in V \mid \exists \vec{v} \in V \text{ such that } A\vec{v} = \vec{w}\}$.
\end{theorem}

\subsubsection{Quotient spaces}
\begin{definition}[Quotient space]
    Let $V$ be a vector space and $W$ a subspace of $V$. The quotient space $V/W$ is the set of cosets of $W$ in $V$.
    I.e. $V/W = \{\vec{v} + W \mid \vec{v} \in V\}$.
\end{definition}

\begin{theorem}[Noether's fundamental theorem on homomorphisms]
    For any linear map $\mathcal{A}: V \to W$, there exists a linear bijection between its range $\mathcal{R}$ and the quottient space $V/\mathcal{N}$.
\end{theorem}

\subsubsection{Example}
Take $P: \mathbb{R}^3 \to \mathbb{R}^3, (x,a,b) \mapsto (0,a,b)$ and $\mathcal{R}(P) = <(0,a,0),(0,0,b)>$

\begin{proof}
    ... $\bar{\mathcal{A}}: v/\mathcal{N}(\mathcal{A}) \to W, \vec{v} \mapsto \mathcal{A}\vec{v}$
    Restrict target space of $\bar{\mathcal{A}}$ to $\mathcal{R}(A)$. Homework : show that $\bar{\mathcal{A}}$ is linear and injective.
    So $\bar{\mathcal{A}}$ is a linear bijection.
\end{proof}

$\bar{\mathcal{A}}^{-1}:$

