\section{Point-set topology of metric spaces}
Here we introduce three properties for subsets of a metric space: \emph{closedness}, \emph{completeness}, and \emph{compactness}.
For those three properties we known that every compact set is complete, and every complete set is closed. However, not every closed set is complete,
and not every complete set is compact.

\subsection{Open sets}
\begin{definition}[Open set]
    Let $(X, \dist)$ be a metric space. We say that a subset $O \subseteq X$ is \emph{open} if every $x \in O$ is an interior point of $O$.
\end{definition}

Now we need to say what it means to be an interior point.
\begin{definition}[Interior point]
    Let $(X, \dist)$ be a metric space and let $A$ be subset of $X$. A point $a \in A$ is called an \emph{interior point} of $A$ if
    \begin{myCenter}
        \tabline{there exists $r > 0$}
        \tabline{$B(a,r) \subseteq A$ }
    \end{myCenter}
    where $B(a,r)$ is an (open) ball around point $a$ with radius $r$ (\cref{ball}).
\end{definition}

\begin{proposition}
    Let $(X, \dist)$ be a metric space. The ball
    $$B(p,r) := \{x \in X | \dist(x,p) < r \}$$
    is indeed open.
\end{proposition}

\begin{proposition}['Open' intervals are open]
    Let $a,b \in \R$ with $a<b$. Then the intervals $(a,b), (-\infty,b), (a,\infty)$ are all open subsets of $\R$.
\end{proposition}

\begin{proposition}
    Let $(X, \dist)$ be a metric space. Then both the empty set $\emptyset$ and the set $X$ itself (both of these are subsets of $X$) are open.
\end{proposition}
\begin{proof}
    We first show that the empty set is open. We argue by contradiction. Suppose there exists a point $x \in \emptyset$ such that
    $x$ is not an interior point of $X$. Then we have a contradiction, because the empty set has no elements.

    We will now show that $X$ is open. Let $x \in X$. We will show that $x$ is an interior point, i.e. we will show that there exists an $r > 0$
    such that $B(x,r) \subseteq X$.\\
    Choose $r:=1$. Then $B(x,r) = B(x,1) \subseteq X$.
\end{proof}

The set of all interior points of a subset $A \subseteq X$ is called the \emph{interior} of the set $A$.
\begin{definition}[The interior of a set]
    Let $(X,\dist)$ be a metric space and let $A \subseteq X$ be a subset of $X$. Then the \emph{interior} of the set $A$, denoted
    by $\inter\ A$ is the set of all interior points of $A$, i.e $\inter\ A$ is defined as
    $$\inter\ A := \{x \in A \mid x \text{ is an interior point of } A \}.$$
\end{definition}
\begin{example}
    The interior of the interval $[2,5)$ (viewed as subset of $(\R, |\cdot|)$) is the interval $(2,5)$.
\end{example}

The interior of a set is always open.
\begin{proposition}
    Let $(X, \dist)$ be a metric space and let $A \subseteq X$. Then $\inter\ A$ is open.
\end{proposition}

\subsubsection*{The union of open sets is always open}
Unions of open sets are always open. You may recall that if $\mathcal{I}$ is some set and if for every $\alpha \in \mathcal{I}$
we have a subset $A_\alpha \subseteq X$, then the union
$$\bigcup_{\alpha \in \mathcal{I}}A_\alpha$$
is defined as 
$$\bigcup_{\alpha \in \mathcal{I}}A_\alpha := \{x \in X \mid \text{ there exists } \alpha \in \mathcal{I} \text{ such that } x \in A_\alpha\}$$

\begin{proposition}
    Let $(X, \dist)$

\subsection{Closed sets}

\subsection{Cauchy sequences}

\subsection{Completeness}

\subsection{Series characterization of completeness in normed vector spaces}
