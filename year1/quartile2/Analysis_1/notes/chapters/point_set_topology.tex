\section{Point-set topology of metric spaces}
Here we introduce three properties for subsets of a metric space: \emph{closedness}, \emph{completeness}, and \emph{compactness}.
For those three properties we known that every compact set is complete, and every complete set is closed. However, not every closed set is complete,
and not every complete set is compact.

\subsection{Open sets}
\begin{definition}[Open set]
    Let $(X, \dist)$ be a metric space. We say that a subset $O \subseteq X$ is \emph{open} if every $x \in O$ is an interior point of $O$.
\end{definition}

Now we need to say what it means to be an interior point.
\begin{definition}[Interior point]
    Let $(X, \dist)$ be a metric space and let $A$ be subset of $X$. A point $a \in A$ is called an \emph{interior point} of $A$ if
    \begin{myCenter}
        \tabline{there exists $r > 0$}
        \tabline{$B(a,r) \subseteq A$ }
    \end{myCenter}
    where $B(a,r)$ is an (open) ball around point $a$ with radius $r$ (\cref{ball}).
\end{definition}

\begin{proposition}
    Let $(X, \dist)$ be a metric space. The ball
    $$B(p,r) := \{x \in X | \dist(x,p) < r \}$$
    is indeed open.
\end{proposition}

\begin{proposition}['Open' intervals are open]
    Let $a,b \in \R$ with $a<b$. Then the intervals $(a,b), (-\infty,b), (a,\infty)$ are all open subsets of $\R$.
\end{proposition}

\begin{proposition}
    Let $(X, \dist)$ be a metric space. Then both the empty set $\emptyset$ and the set $X$ itself (both of these are subsets of $X$) are open.
\end{proposition}
\begin{proof}
    We first show that the empty set is open. We argue by contradiction. Suppose there exists a point $x \in \emptyset$ such that
    $x$ is not an interior point of $X$. Then we have a contradiction, because the empty set has no elements.

    We will now show that $X$ is open. Let $x \in X$. We will show that $x$ is an interior point, i.e. we will show that there exists an $r > 0$
    such that $B(x,r) \subseteq X$.\\
    Choose $r:=1$. Then $B(x,r) = B(x,1) \subseteq X$.
\end{proof}

The set of all interior points of a subset $A \subseteq X$ is called the \emph{interior} of the set $A$.
\begin{definition}[The interior of a set]
    Let $(X,\dist)$ be a metric space and let $A \subseteq X$ be a subset of $X$. Then the \emph{interior} of the set $A$, denoted
    by $\inter\ A$ is the set of all interior points of $A$, i.e $\inter\ A$ is defined as
    $$\inter\ A := \{x \in A \mid x \text{ is an interior point of } A \}.$$
\end{definition}
\begin{example}
    The interior of the interval $[2,5)$ (viewed as subset of $(\R, |\cdot|)$) is the interval $(2,5)$.
\end{example}

The interior of a set is always open.
\begin{proposition}
    Let $(X, \dist)$ be a metric space and let $A \subseteq X$. Then $\inter\ A$ is open.
\end{proposition}

\subsection*{The union of open sets is always open}
Unions of open sets are always open. You may recall that if $\mathcal{I}$ is some set and if for every $\alpha \in \mathcal{I}$
we have a subset $A_\alpha \subseteq X$, then the union
$$\bigcup_{\alpha \in \mathcal{I}}A_\alpha$$
is defined as 
$$\bigcup_{\alpha \in \mathcal{I}}A_\alpha := \{x \in X \mid \text{ there exists } \alpha \in \mathcal{I} \text{ such that } x \in A_\alpha\}$$

\begin{proposition}
    Let $(X, \dist)$ be a metric space, let $\mathcal{I}$ be some set and assume that for every $\alpha \in \mathcal{I}$, we have a subset
    $O_\alpha \subseteq X$. Suppose that for all $\alpha \in \mathcal{I}$ the set $O_\alpha$ is open. Then also the union
    $$\bigcup_{\alpha \in \mathcal{I}}O_\alpha$$
    is open.
\end{proposition}
\begin{example}
    We already know that for every $n \in \N$, the interval $(2n, 2n+1)$ is an open subset of $(\R, |\cdot|)$. Therefore,
    (choosing $\mathcal{I} = \N$ and $O_\alpha = (2\alpha, 2\alpha+1)$ in the previous proposition,) we also know that the set
    $$\bigcup_{n\in\N}(2n,2n+1)$$
    is an open set of $(\R, |\cdot|)$ as well.
\end{example}

\subsection*{Finite intersections of open sets are open}
\begin{proposition}
    Let $(X, \dist)$ be a metric space and let $O_1,\dots,O_N$ be open subsets of $X$. Then the intersection
    $$O_1 \cap \dots \cap O_N$$
    is also open.
\end{proposition}

\subsection*{Cartesian products of open sets}
\begin{proposition}
    Let $O_1,\dots,O_d$ be open subsets of $\R$. Then
    $$O_1 \times \dots \times O_d (=\{(o_1, \dots, o_d) \mid o_i \in O_i\})$$
    is an open subset of $(\R^d, \norm{\cdot}_2)$.
\end{proposition} 

\subsection{Closed sets}
\begin{definition}
    Let $(X, \dist)$ be a metric space. We say that a subset $C \subseteq X$ is \emph{closed} if its complement
    $X \setminus C$ is open.
\end{definition}

\begin{proposition}
    Let $(X, \dist)$ be a metric space. Then both the empty set $\emptyset$ and the set $X$ itself (both of these are subsets of $X$) are closed.
\end{proposition}
\begin{warning}
    If you want to show that a set is closed \emph{it is not enough} to show that the set is not open.
\end{warning}

\begin{proposition}[Sequence characterization of closedness]
    A set $C \subseteq X$ is closed if and only if for every sequence $(c_n)$ in $C$ converging to some $x \in X$,
    it holds that $x \in C$.
\end{proposition}
\begin{example}
    Consider the subset $A$ of the metric space $(\R^2, \norm{\cdot})$ defined by
    $$A:=\{(x_1,x_2) \in \R^2 \mid x_1 \le (x_2)^2\}$$
\end{example}
\begin{proof}[Proof]
    By the sequence characterization of closedness, it suffices to show that for all sequences $y: \N \to A$, if the sequence $y$
    converges to some point $z \in \R^2$, then actually $z \in A$.

    \noindent Let $y: \N \to A$ be a sequence in $A$. \\
    Assume that the sequence $(y)$ converges to some point $z \in \R^2$. \\
    We need to show that $z \in A$. \\
    Since $y$ converges to $z$, we know that the components sequences $y_1$ and $y_2$ of $y$ converge to the components $z_1$ and $z_2$ of $z$, namely
    $$\limn y_1^{(n)}=z_1 \quad \text{and} \quad \limn y_2^{(n)}=z_2.$$
    By limit theorems, we know that
    $$\limn \left(y_2^{(n)}\right)^2 = (z_2)^2.$$
    Since for all $n \in \N, y^{(n)}\in A$, we also know that for all $n \in \N, y_1(n) \le (y_2(n))^2$. Therefore,
    $$z_1 = \limn y_1^{(n)} \le \limn \left(y_2^{(n)}\right)^2 = (z_2)^2.$$
    We conclude that indeed $z \in A$.
\end{proof}

\begin{proposition}
    Let $a,b \in \R$ with $a<b$. Then the intervals $[a,b], (-\infty,b]$ and $[a, \infty)$ are all closed.
\end{proposition}

We now provide a few ways to create new closed sets out of sets about which you already know that they are closed.

\subsection*{Intersections of closed sets are always closed}
Let $(X, \dist)$ be a metric space. If $\mathcal{I}$ is a set, and for every $\alpha \in \mathcal{I}$, we have a subset
$A_\alpha$ of $X$, then the intersection
$$\bigcap_{\alpha \in \mathcal{I}}A_\alpha$$
is defined as
$$\bigcap_{\alpha \in \mathcal{I}}A_\alpha := \{x \in X \mid \text{ for all } \alpha \in \mathcal{I}, x \in A_\alpha\}.$$

\begin{proposition}
    Let $(X, \dist)$ be a metric space. Let $\mathcal{I}$ be a set and suppose for every $\alpha \in \mathcal{I}$ we have a subset $C_\alpha \subseteq X$.
    Assume that for every $\alpha \in \mathcal{I}$ the set $C_\alpha$ is closed. Then the intersection
    $$\bigcap_{\alpha\in\mathcal{I}}C_\alpha$$
    is closed as well.
\end{proposition}

\subsection*{Finite unions of closed sets are closed}
\begin{proposition}
    Let $(X, \dist)$ be a metric space. Let $C_1, \dots, C_N$ be closed subsets of $X$. Then the finite union
    $$C_1 \cup \dots \cup C_N$$
    is also closed.
\end{proposition}

\subsection*{Products of closed sets}
\begin{proposition}
    Let $C_1,\dots,C_d$ be closed subsets of $\R$. Then
    the Cartesian product
    $$C_1 \times \dots \times C_d (=\{(c_1,\dots,c_d) \mid c_i \in C_i\})$$
    is a closed subset of $(\R^d, |\cdot|)$
\end{proposition}

\subsection*{The topological boundary of a set}
\begin{definition}[The topological boundary]
    Let $(X, \dist)$ be a metric space and let $A \subseteq X$. The \emph{topological boundary} of a set $A$ is denoted by
    $\partial A$ and defined as
    $$\partial A := X \setminus ((\inter A)\cup(\inter (X \setminus A)))$$
\end{definition}

\begin{example}
    The topological boundary of the interval $[2,5)$ is the set $\{2,5\}$ that consists of exactly the points 2 and 5.
\end{example}

\subsection{Cauchy sequences}
\begin{definition}[Cauchy sequence]
    Let $(X, \dist)$ be a metric space. We say that a sequence $a: \N \to X$ is a Cauchy sequence if
    \begin{myCenter}
        \tabline{for all $\epsilon > 0$,}
        \tabline{there exists $N \in \N$,}
        \tabline{for all $m,n \ge N$,}
        \tabline{$\dist(a_m, a_n) < \epsilon$}
    \end{myCenter}
\end{definition}

\begin{proposition}
    Every Cauchy sequence is bounded
\end{proposition}

\begin{proposition}
    \label{prop:8.3.3}
    Let $a: \N \to X$ be a Cauchy sequence and assume that $a$ has a subsequence converging to $p \in X$. Then the sequence
    $a$ itself converge to $p$.
\end{proposition}

\begin{proposition}
    Let $(X, \dist)$ be a metric space. Let $(x_n)$ be a converging sequence in $X$. Then $(x_n)$ is a Cauchy sequence.
\end{proposition}

\subsection{Completeness}
\begin{definition}
    Let $(X, \dist)$ be a metric space. We say that a subset $A \subseteq X$ is \emph{complete} (in $(X, \dist)$) if every
    Cauchy sequence in $A$ is convergent, with limit in $A$.

    We also say the metric space $(X, \dist)$ itself is complete if $X$ is a complete subset of $X$ in $(X,\dist)$.
\end{definition}

\begin{theorem}
    The metric space $(\R, \dist_\R)$ is complete.
\end{theorem}
\begin{proof}
    Let $a:\N\to\R$ be a Cauchy sequence. Because $a$ is a Cauchy sequence, it is in particular bounded. as a consequence, by
    \cref{thm:8.4.3}, there is a subsequence $a \circ n$ such that $a \circ n$ converges to 
    $$\limsup_{k\to\infty}a_k$$
    Finally, we know from \cref{prop:8.3.3} that if a subsequence of a Cauchy sequence converges, that then the whole
    sequence converges. Therefore, the sequence $a: \N \to \R$ converges.
\end{proof}

\begin{proposition}
    The metric space $(\R^d, \dist_{\norm{\cdot}_2})$ is complete, where $\norm{\cdot}_2$ is the Euclidean norm.
\end{proposition}

\begin{proposition}
    Let $(X, \dist)$ be a metric space. Suppose $A \subseteq X$ is complete. Then $A$ is closed
\end{proposition}

\begin{proposition}
    Let $(X, \dist)$ be a metric space and let $C \subseteq X$ be a complete subset. Let $A \subseteq C$ be a subset of
    $C$. Then, $A$ is complete if and only if $A$ is closed.
\end{proposition}

\subsection{Series characterization of completeness in normed vector spaces}
\begin{theorem}
    Let $(V, \norm{\cdot})$ be a normed vector space. Then $(V, \norm{\cdot})$ is complete if and only if every
    absolutely converging series is convergent.
\end{theorem}

\begin{corollary}
    Let $a: \N \to \R$ be a real-valued sequence. Suppose the series
    $$\sum_{n=0}^\infty a_n$$
    converges absolutely, i.e. the series
    $$\sum_{n=0}^\infty |a_n|$$
    converges. Then also the series
    $$\sum_{n=0}^\infty a_n$$
    converges.
\end{corollary}
\begin{example}
    The series
    $$\sum_{k=0}^\infty (-1)^k\frac{1}{k^2}$$
    converges, because it converges absolutely.
\end{example}
