\section{Compactness}

\subsection{Definition of (sequential) compactness}
\begin{definition}[(Sequential) compactness]
    Let $(X, \dist)$ be a metric space. We say a subset $K \subseteq X$ is \emph{(sequentially) compact} if every sequence
    $x: \N \to K$ has a converging subsequence $x \circ n$, converging to a point $z \in K$.
\end{definition}

\subsection{Boundedness and total boundedness}
\begin{definition}[Bounded sets]
    Let $(X, \dist)$ be a metric space. We say that a subset $A \subseteq X$ is \emph{bounded} if
    \begin{myCenter}
        \tabline{there exists $q \in X$,}
        \tabline{there exists $M > 0$,}
        \tabline{for all $p \in A$,}
        \tabline{$\dist(p, q) \leq M$.}
    \end{myCenter}
\end{definition}

Just as with the concept of boundedness for sequences, in normed vector spaces boundedness has a somewhat easier alternative characterization.
\begin{proposition}
    Let $(V, \norm{\cdot})$ be a normed linear space. A subset $A \subseteq V$ is bounded if and only if
    \begin{myCenter}
        \tabline{there exists $M > 0$,}
        \tabline{for all $v \in A$,}
        \tabline{$\norm{v} \leq M$.}
    \end{myCenter}
\end{proposition}

\begin{definition}[Totally bounded sets]
    Let $(X, \dist)$ be a metric space. We say that a subset $A \subseteq X$ is \emph{totally bounded} if
    \begin{myCenter}
        \tabline{for all $r > 0$,}
        \tabline{there exists $N \in \N$,}
        \tabline{there exists $p_1,\dots,p_N \in X$,}
        \tabline{$A \subseteq \bigcup_{i=1}^N B(p_i,r)$.}
    \end{myCenter}
\end{definition}

In the next proposition we will say that "total boundedness" is a stronger property than just "boundedness".

\begin{proposition}
    Let $(X, \dist)$ be a metric space and let $A$ be a subset of $X$. If $A$ is totally bounded, it is bounded.
\end{proposition}

In the special case of the normed vector space $(\R^d, \norm{\cdot}_2)$, however, a subset is totally bounded if and only
if it is bounded.

\begin{proposition}
    Consider now the normed vector space $(\R^d, \norm{\cdot}_2)$. A subset $A \subseteq \R^d$ is bounded in $(\R^d, \norm{\cdot}_2)$
    if and only if it is totally bounded.
\end{proposition}

\subsection{Alternative characterization of compactness}
\begin{theorem}
    A subset $K \subseteq X$ is compact if and only if it is complete and totally bounded.
\end{theorem}

In the special case of $(\R^d, \norm{\cdot})$ we have an easier alternative characterization of compactness.
\begin{theorem}[Heine-Borel Theorem]
    \label{thm:heine-borel}
    A subset of $(\R^d, \norm{\cdot}_2)$ is compact if and only if it is closed and bounded.
\end{theorem}
