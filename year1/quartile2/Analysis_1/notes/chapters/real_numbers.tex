\section{Real Numbers}

\subsection{What are the real numbers?}
\begin{definition}[Real numbers]
    The real numbers are a complete totally ordered field.
\end{definition}

\subsection{The completeness axiom}

\begin{definition}[Upper and Lower bound]
    We say a number $M \in \mathbb{R}$ is an \emph{upper bound} for a set $A \subseteq \mathbb{R}$ if
    $$\forall a \in A [a \le M].$$

    We say a number $m \in \mathbb{R}$ is a \emph{lower bound} for a set $A \subseteq \mathbb{R}$ if
    $$\forall a \in A [a \ge M].$$
\end{definition}

Given the definition of upper and lower bounds, we define what it means
for a set to be bounded from above, bounded from below and just bounded.

\begin{definition}[bounded from above, bounded from below, bounded]
    A set $A \subseteq \mathbb{R}$ is \emph{bounded from above} if there exists an upper bound for $A$.

    A set $A \subseteq \mathbb{R}$ is \emph{bounded from below} if there exists a lower bound for $A$.

    A set $A \subseteq \mathbb{R}$ is \emph{bounded} if it is bounded from above and bounded from below.
\end{definition}

\begin{definition}[Least upper bound (supremum)]
    Precisely, $M$ is a \emph{least upper bound} of a subset $A$ if both
    \begin{enumerate}
        \item $M$ is an upper bound of $A$.
        \item For every upper bound $L \in \mathbb{R}$ of $A$, it holds that $M \le L$.
    \end{enumerate}
\end{definition}

\begin{proposition}
    Suppose both $M$ and $W$ are a least upper bound of a subset $A \subseteq \mathbb{R}$. Then $M = W$.
\end{proposition}

\begin{axiom}[Completeness axiom]
    We say that a totally ordered field \textbf{R} satisfies the \emph{completeness axiom} if every nonempty subset of \textbf{R} that is bounded from above has a least upper bound.
\end{axiom}

\begin{lemma}
    Every non-empty subset of the real line that is bounded from below has a \emph{largest lower bound}.
\end{lemma}

\begin{definition}[infimum]
    We usually call the largest lower bound of a non-empty set $A \subseteq \R$ that is bounded from below the \emph{infimum} of $A$, and we denote it by $\inf A$.
\end{definition}

\subsection{Alternative characterizations of suprema and infima}
\begin{proposition}[alternative characterizationa of supremum]
    Let $A \subseteq \R$ be non-empty and bounded from above. Let $M \in \R$. Then $M$ is the supremum of $A$ if and only if
    \begin{enumerate}
        \item $M$ is an upper bound for $A$,
        \item and 
            \begin{myCenter}
                \tabline{for all $\epsilon > 0$,}
                \tabline{there exists $a \in A$,}
                \tabline{$a > M - \epsilon$.}
            \end{myCenter}
    \end{enumerate}
\end{proposition}

\begin{proposition}[alternative characterizationa of infimum]
    Let $A \subseteq \R$ be non-empty and bounded from below. Let $m \in \R$. Then $m$ is the infimum of $A$ if and only if
    \begin{enumerate}
        \item $m$ is a lower bound for $A$,
        \item and 
            \begin{myCenter}
                \tab{for all $\epsilon > 0$,}
                \tab{there exists $a \in A$,}
                \tab{$a < m + \epsilon$.}
            \end{myCenter}
    \end{enumerate}
\end{proposition}

These alternative characterizations of the supremum and infimum really provide a standard way to determining the supremum and infimum of subsets of the real line.

\subsection{Maxima and minima}
\begin{definition}[maximum and minimum]
    Let $A \subseteq \R$ be a subset of the real numbers. We say that $y \in A$ is the \emph{maximum} of $A$, and write $y = \max A$, if 
    \begin{center}
        for all $a \in A$, \\
        $a \le y$.
    \end{center}

    We say that $x \in A$ is the \emph{minimum} of $A$, and write $x = \min A$, if
    \begin{center}
        for all $a \in A$, \\
        $a \ge x$.
    \end{center}
\end{definition}

\begin{remark}
    Even if a set $A \subseteq \R$ is non-empty and bounded, it may not have a maximum or minimum. For example, the set $(0, 1)$ has no maximum or minimum.
\end{remark}

\begin{proposition}
    Let $A$ be a subset of $\R$. If $A$ has a maximum, then $A$ is non-empty and bounded from above, and $\sup A = \max A$. If $A$ has a minimum, then $A$ is non-empty and bounded from below, and $\inf A = \min A$.
\end{proposition}

\begin{proposition}
    Let $A$ be a subset of $\R$. Assume that $A$ is non-empty and bounded from above. If $\sup A \in A$ then $A$ has a maximum and $\max A = \sup A$.
\end{proposition}

\begin{proposition}
    Let $A$ be a subset of $\R$. Assume that $A$ is non-empty and bounded from below. If $\inf A \in A$ then $A$ has a minimum and $\min A = \inf A$.
\end{proposition}

\subsection{The Archimedean property}
\begin{proposition}[Archimedeean property]
    For every real number $x \in \R$ there exists a natural number $n \in \N$ such that $x < n$.
\end{proposition}

Given this proposition, we can define the ceiling function.

\begin{definition}[ceiling function]
    The \emph{ceiling function} $\ceil{\cdot}: \R \to \Z$ is defined as follows. For $x \in \R$, $\ceil{x}$ denotes the smallest integer $z \in \Z$ such that $x \le z$.
\end{definition}

\begin{proposition}
    For every two real numbers $a,b \in \R$ with $a < b$ there exists a $q \in \Q$ with $a < q < b$.
\end{proposition}

\subsection{Computation rules for suprema}
In the proposition below, we use the definitions
$$ A + B = \{a + b \mid a \in A, b \in B\}$$
and 
$$\lambda A = \{\lambda a \mid a \in A\}$$
for subsets $A,B \subseteq \R$ and a scalar $\lambda \in \R$.

\begin{proposition}
    Let $A,B,C,D$ be non-empty subsets of $\R$. Assume that $A$ and $B$ are bounded from above and $C$ and $D$ are bounded from below. Then
    \begin{enumerate}
        \item $\sup(A+B) = \sup A + \sup B$.
        \item $\inf(C+D) = \inf C + \inf D$.
        \item For all $\lambda \ge 0$, $\sup(\lambda A) = \lambda \sup A$.
        \item For all $\lambda \le 0$, $\sup(\lambda A) = \lambda \inf A$.
        \item $\sup(-C) = -\inf C$.
        \item $\inf(-C) = -\sup C$.
    \end{enumerate}
\end{proposition}

\subsection{Bernoulli's inequality}
\begin{proposition}[Bernoulli's inequality]
    Let $x \in \R$ and $n \in \N$. Then
    \begin{enumerate}
        \item If $x \ge -1$, then $(1 + x)^n \ge 1 + nx$.
        \item If $x \ge 0$ and $n \ge 2$, then $(1 + x)^n \ge 1 + nx$.
    \end{enumerate}
\end{proposition}
