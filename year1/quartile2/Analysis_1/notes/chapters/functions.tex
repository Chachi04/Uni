\section{Real-valued functions}

\subsection{More limit laws}
\begin{theorem}[Limit laws for real-valued functinos]
    Let $(X, \dist)$ be a metric space, let $D$ be a subset of $X$ and assume that $a \in D^\prime$. Let 
    $f: D \to \R$ and $g: D \to \R$ be two real-valued functions and assume that $\lim_{x\to a}f(x)$ exists and
    equals $M \in \R$ and that $\lim_{x\to a}g(x)$ exists and equals $L \in \R$. Then
    \begin{enumerate}
        \item For every $m \in \N$, the limit $\lim_{x\to a}(f(x))^m$ exists and equals $M^m$.
        \item The limit $\lim_{x \to a}(f(x)g(x))$ exists and equals $ML$.
        \item If $L \ne 0$, then the limit $\lim_{x \to a}\frac{f(x)}{g(x)}$ exists and equals $\frac{M}{L}$.
        \item If for all $x \in D, f(x) \ge 0$, then for every $k \in \N \setminus \{0\}$,
            $$\lim_{x\to a}\sqrt[k]{f(x)}=\sqrt[k]{M}$$
    \end{enumerate}
\end{theorem}

\subsection{Building new continuous functions}
The following theorem translates the limit laws from the previous section into statements
about continuity.
\begin{theorem}
    Let $(X,\dist)$ be a metric space, let $D$ be a subset of $X$ and assume $a \in D$.
    Let $f:D \to \R$ and $g:D \to \R$ be two real-valued functions that are continuous in $a$.
    Then 
    \begin{enumerate}
        \item For every $m \in \N$, the function $f^m$ is continuous in $a$.
        \item The function $f + g$ is continuous in $a$.
        \item The function $fg$ is continuous in $a$. 
        \item If $g(a) \ne 0$, then the function $\frac{f}{g}$ is continuous in $a$.
        \item If for all $x \in D, f(x) \ge 0$, then for every $k \in \N \setminus \{0\}$,
            the function $\sqrt[k]{f}$ is continuous in $a$.
    \end{enumerate}
\end{theorem}

\subsection{Continuity of standard functions}
\begin{proposition}[Polynomials are continuous]
    Every (possibly multivariate) polynomial is continuous as a function from $(\R^d, \norm{\cdot}_2)$ to $(\R,|\cdot|)$.
\end{proposition}
\begin{proposition}[Rational functions are continuous]
    Every (possibly multivariate) rational function is continuous as a function from $(\R^d, \norm{\cdot}_2)$ to $(\R,|\cdot|)$.
\end{proposition}

\begin{proposition}[Continuity of some standard functions]
    The functions
    \begin{center}
        \begin{tabular}{l r}
            $\exp:\R \to \R$ & $\ln: (0,\infty)\to\R$ \\
            $\sin:\R \to \R$ & $\arcsin:[-1,1] \to \R$ \\
            $\cos:\R \to \R$ & $\arccos:[-1,1] \to \R$ \\
            $\tan:(-\pi/2,\pi/2) \to \R$ & $\arctan:\R \to \R$
        \end{tabular}
    \end{center}
    are all continuous.
\end{proposition}

\subsection{Limits from the left and from the right}
\begin{definition}[Limit from the left]
    Let $(Y,\dist_Y)$ be a metric space, and let $D \subseteq \R$ be a subset of $\R$. Let $f: D \to Y$ be a function.
    Let $a \in \R$ be such that $a \in ((-\infty,a)\cap D)^\prime$, i.e. such that $a$ is an accumulation point in the set
    $(-\infty,a) \cap D$ in the metric space $(\R, \dist_\R)$. Let $q \in Y$. We say that $f(x)$ \emph{converges to q as
    $x$ approaches $a$ from the left (or from below)}, and write
    $$\lim_{x\uparrow a}f(x) = q (\lim_{x\to a^-}f(x)=q)$$
    if
    \begin{myCenter}
        \tab{for all $\varepsilon > 0$,}
        \tab{there exists $\delta > 0$,}
        \tab{for all $x \in D \cap (-\infty, a)$,}
        \tab{$0 < \dist_\R(x,a) < \delta \implies \dist_Y(f(x),q) < \varepsilon$}
    \end{myCenter}
\end{definition}
\begin{definition}[Limit from the right]
    Let $(Y,\dist_Y)$ be a metric space, and let $D \subseteq \R$ be a subset of $\R$. Let $f: D \to Y$ be a function.
    Let $a \in \R$ be such that $a \in ((a,\infty)\cap D)^\prime$, i.e. such that $a$ is an accumulation point in the set
    $(a,\infty) \cap D$ in the metric space $(\R, \dist_\R)$. Let $q \in Y$. We say that $f(x)$ \emph{converges to q as
    $x$ approaches $a$ from the right (or from above)}, and write
    $$\lim_{x\downarrow a}f(x) = q (\lim_{x\to a^+}f(x)=q)$$
    if
    \begin{myCenter}
        \tab{for all $\varepsilon > 0$,}
        \tab{there exists $\delta > 0$,}
        \tab{for all $x \in D \cap (a, \infty)$,}
        \tab{$0 < \dist_\R(x,a) < \delta \implies \dist_Y(f(x),q) < \varepsilon$}
    \end{myCenter}
\end{definition}

\subsection{The extended real line}
\begin{definition}[The extended real line]
    The extended real line $\R_\text{ext}$ is the union -f the set $\R$ and two symbols, "$\infty$" and "$-\infty$".
    That is $\R_\text{ext} = \R \cup \{\infty\} \cup \{-\infty\}$.
\end{definition}

To turn $\R_\text{ext}$ into a metric space, we need to define a distance function. First, we define a map $\iota: \R_\text{ext} \to [-1,1]$ by
$$\iota(x) = \begin{cases}
    -1 &\text{if } x = -\infty \\
    \frac{x}{1+x} &\text{if } x \in \R \land x \ge 0 \\
    \frac{x}{1-x} &\text{if } x \in \R \land x < 0 \\
    1 &\text{if } x = \infty
\end{cases}$$
Because this function is injective, we can now build a distance on $\R_\text{ext}$.

\begin{definition}[Distance on extended real line]
    Given the definition of the injective function $\iota: \R_\text{ext} \to [-1,1]$ above, we define the distance on
    $\R_\text{ext}$ by
    $$\dist_{\R_\text{ext}}(x,y) := dist_\R(\iota(x),\iota(y)) \qquad \text{for }x,y \in \R_\text{ext}$$
\end{definition}

\subsection{Limits to \texorpdfstring{$\infty$}{infinity} or \texorpdfstring{$-\infty$}{minus infinity}}
\begin{definition}[Divergence to $\infty$]
    Let $(X,\dist_X)$ be a metric space and $D \subseteq X$ and assume $a \in D^\prime$. Let $f: D \to \R$.
    We say that $f$ diverges to $\infty$ in $a$ if
    \begin{myCenter}
        \tab{for all $M \in \R$,}
        \tab{there exists $\delta > 0$,}
        \tab{for all $x \in D$,}
        \tab{$0 < \dist_X(x,a) < \delta \implies f(x) > M$}
    \end{myCenter}
\end{definition}
\begin{definition}[Divergence to $-\infty$]
    Let $(X,\dist_X)$ be a metric space and $D \subseteq X$ and assume $a \in D^\prime$. Let $f: D \to \R$.
    We say that $f$ diverges to $-\infty$ in $a$ if
    \begin{myCenter}
        \tab{for all $M \in \R$,}
        \tab{there exists $\delta > 0$,}
        \tab{for all $x \in D$,}
        \tab{$0 < \dist_X(x,a) < \delta \implies f(x) < M$}
    \end{myCenter}
\end{definition}
\begin{proposition}[Alternative characterization of divergence to $\infty$]
    Let $(X,\dist_X)$ be a metric space and $D \subseteq X$ and assume $a \in D^\prime$. Let $f: D \to \R$.
    Then $f$ diverges to $\infty$ in $a$ if and only if $f$ \emph{converges} in $a$ to the element $\infty \in \Rext$
    when viewed as a function mapping from $D$ as a subset of $(X,\dist_X)$ to the extended real line $(\Rext, \dist_{\R_\text{ext}})$.
\end{proposition}

\subsection{Limits at \texorpdfstring{$\infty$}{infinity} and \texorpdfstring{$-\infty$}{minus infinity}}
\begin{definition}[Limit at $\infty$]
    Let $(Y,\dist_Y)$ be a metric space and let $D$ be a subset of $\R$ that is unbounded from above. Let $q \in Y$ and
    $f: D \to Y$ be a function. We say that $f(x)$ converges to $q$ as $x \to \infty$, and write
    $$\lim_{x\to\infty}f(x) = q$$
    if
    \begin{myCenter}
        \tab{for all $\epsilon > 0$,}
        \tab{there exists $z \in \R$,}
        \tab{for all $x \in D$,}
        \tab{$x > z \implies \dist_Y(f(x),q) < \epsilon$}
    \end{myCenter}
\end{definition}
\begin{definition}[Limit at $-\infty$]
    Let $(Y,\dist_Y)$ be a metric space and let $D$ be a subset of $\R$ that is unbounded from below. Let $q \in Y$ and
    $f: D \to Y$ be a function. We say that $f(x)$ converges to $q$ as $x \to -\infty$, and write
    $$\lim_{x\to-\infty}f(x) = q$$
    if
    \begin{myCenter}
        \tab{for all $\epsilon > 0$,}
        \tab{there exists $z \in \R$,}
        \tab{for all $x \in D$,}
        \tab{$x < z \implies \dist_Y(f(x),q) < \epsilon$}
    \end{myCenter}
\end{definition}

We can also combine divergence to and at infinity.
\begin{definition}[Divergence to $\infty$ at $\infty$]
    Let $D \subseteq \R$ be unbounded from above. Let $f: D \to \R$ be a function. We say that $f$ diverges to $\infty$
    as $x \to \infty$, and write
    $$\lim_{x\to\infty}f(x) = \infty$$
    if
    \begin{myCenter}
        \tab{for all $M \in \R$,}
        \tab{there exists $z \in \R$,}
        \tab{for all $x \in D$,}
        \tab{$x > z \implies f(x) > M$}
    \end{myCenter}
\end{definition}
\begin{definition}[Divergence to $-\infty$ at $\infty$]
    Let $D \subseteq \R$ be unbounded from above. Let $f: D \to \R$ be a function. We say that $f$ diverges to $-\infty$
    as $x \to \infty$, and write
    $$\lim_{x\to\infty}f(x) = -\infty$$
    if
    \begin{myCenter}
        \tab{for all $M \in \R$,}
        \tab{there exists $z \in \R$,}
        \tab{for all $x \in D$,}
        \tab{$x > z \implies f(x) < M$}
    \end{myCenter}
\end{definition}

\subsection{The Intermediate Value Theorem}
\begin{theorem}[Intermediate Value Theorem]
    Let $f: [a,b] \to \R$ be a continuous function and let $c \in \R$ be a value between $f(a)$ and $f(b)$. Then,
    there exists an $x \in [a,b]$ such that $f(x) = c$.
\end{theorem}

\subsection{The Extreme Value Theorem}
The \emph{Extreme Value Theorem} states that a continuous, real-valued function defined on a non-empty, compact domain
$K$ always attains both a maximum and a minimum on $K$. 
\begin{theorem}[Extreme Value Theorem]
    Let $(X,\dist_X)$ be a metric space, $K \subseteq X$ be a non-empty compact subset and $f: K \to \R$ be continuous.
    Then $f$ attains a maximum and a minimum on $K$.
\end{theorem}

\subsection{Equivalence of norms}
\begin{definition}[Equivalent norms]
    Let $V$ be a vector space and let $\norm{\cdot}_A$ and $\norm{\cdot}_B$ be two different norms on $V$. We say
    that the norms $\norm{\cdot}_A$ and $\norm{\cdot}\_B$ are \emph{equivalent} if there exists a constant $c_1 > 0$ and
    $c_2 > 0$ such that for all $v \in V$
    $$c_1\norm{x}_A \le \norm{x}_B \le c_2\norm{x}_A.$$
\end{definition}
\begin{theorem}[Equivalence of norms on finite-dimensional vector spaces]
    Let $V$ be a finite-dimensional vector space and let $\norm{\cdot}_A$ and $\norm{\cdot}_B$ be two norms on $V$.
    Then the norms $\norm{\cdot}_A$ and $\norm{\cdot}_B$ are equivalent.
\end{theorem}

\begin{theorem}
    Let $(V,\norm{\cdot})$ be a finite-dimensional normed vector space. Then $(V, \norm{\cdot})$ is complete.
\end{theorem}

\begin{theorem}[Heine-Borel Theorem for finite-dimensional normed vector spaces]
    Let $(V,\norm{\cdot})$ be a finite-dimensional normed vector space. Then a subset $A \subseteq V$ is compact
    if and only if $A$ is closed and bounded.
\end{theorem}

\subsection{Bounded linear maps and operator norms}
\begin{definition}[Linear map]
    Let $V$ and $W$ be two vector spaces. A function $L: V \to W$ is called a \emph{linear map} if both
    \begin{enumerate}
        \item for all $a,b \in V$,
            $$L(a+b) = L(a) + L(b)$$

        \item for all $\lambda \in \R$ and $a \in V$,
            $$L(\lambda a) = \lambda L(a)$$
    \end{enumerate}
\end{definition}

\begin{definition}[Bounded linear map]
    Let $(V, \norm{\cdot}_V)$ and $(W, \norm{\cdot}_W)$ be two normed vector spaces. We say that a linear map
    $L: V \to W$ is \emph{bounded} if the image under $L$ of the closed unit ball
    $$\bar{B}_V(0,1) = \{v \in V \mid \norm{v}_V \le 1\}$$
    is a bounded subset of $(W, \norm{\cdot}_W)$, i.e. if
    $$L(\bar{B}_V(0,1))$$
    is a bounded subset of $(W, \norm{\cdot}_W)$.
\end{definition}

\begin{proposition}[Alternative characterization of bounded linear maps]
    Let $(V, \norm{\cdot}_V)$ and $(W, \norm{\cdot}_W)$ be two normed vector spaces. A linear map $L: V \to W$ is
    bounded if and only if there exists an $M > 0$ such that for all $v \in V$,
    $$\norm{L(v)}_W \le M\norm{v}_V$$
\end{proposition}

\begin{proposition}
    The space of bounded linear maps between one normed vector space to another is itself again a vector space, that 
    we denote by $\BLin(V,W)$. Addition and scalar multiplication are defined pointwise, that means that if $L: V \to W$
    and $K: V \to W$ are two linear maps and $\lambda \in \R$ is a scalar, then the linear map $L+K: V \to W$ is defined by
    $$(L+K)(v) = L(v) + K(v)$$
    and the map 
    $$(\lambda L)(v) = \lambda(L(v)).$$
    The zero-element in this vector space $\BLin(V,W)$ is the map that maps every vector to the zero-element of $W$.
\end{proposition}

We now define the operator norm on the space of bounded linear maps.
\begin{proposition}
    Let $(V,\norm{\cdot}_V)$ and $(W,\norm{\cdot}_W)$ be two normed vector spaces. Consider the vector space $\BLin(V,W)$
    of bounded linear maps $L: V \to W$. Then the function $\norm{\cdot}_{V\to W}: \BLin(V,W) \to \R$ defined by
    $$\norm{L}_{V \to W} := \sup_{x \in \bar{B}_V(0,1)} \norm{L(x)}_W$$
    is a norm on $\BLin(V,W)$.
\end{proposition}
\begin{definition}[Operator norm]
    The norm $\norm{\cdot}_{V \to W}$ on the vectors space $\BLin(V,W)$ is called the \emph{operator norm}.
\end{definition}
\begin{proposition}
    Let $(V,\norm{\cdot}_V)$ and $(W,\norm{\cdot}_W)$ be two normed vector spaces. Let $L: V \to W$ be a bounded linear map.
    Then for all $v \in V$,
    $$\norm{L(v)}_W \le \norm{L}_{V \to W}\norm{v}_V$$
    and in fact
    $$\norm{L}_{L \to W} = \min\{C \ge 0 \mid \forall v \in V, \norm{L(v)}_W \le C\norm{v}_V\}$$
\end{proposition}

\begin{theorem}
    Let $(V,\norm{\cdot}_V)$ and $(W,\norm{\cdot}_W)$ be two normed vector spaces and assume that $V$ is finite-dimensional. Let $L: V \to W$ be a linear map.
    Then $L$ is bounded.
\end{theorem}

\begin{theorem}
    Let $(V,\norm{\cdot}_V)$ and $(W,\norm{\cdot}_W)$ be two normed vector spaces. Let $L: V \to W$ be a linear map.
    The function $L$ is continuous if and only if it is bounded.
\end{theorem}
