\section{Subsequences, $\lim\sup$ and $\lim\inf$}

\subsection{Index sequences and subsequences}
\begin{definition}[Index sequence]
    We say a sequence $n: \N \to \N$ is an \emph{index sequence} if it is strictly increasing.
\end{definition}
\begin{example}
    The sequence $n: \N \to \N$ defined by
    $$n_k := 2k$$
    is a strictly increasing sequence of natural numbers. In other words, it is an index sequence.
\end{example}

\begin{definition}[Subsequence]
    Let $a: \N \to \R$ be a sequence. A sequence $b: \N \to \R$ is called a \emph{subsequence} of $a$ if there exists an index sequence $n: \N \to \N$ such that
    $b = a \circ n$
\end{definition}
Just as we often write $(a_n)_{n \in \N}$ for a sequence called $a$, we often write $(a_{n_k})_{k\in\N}$ for the subsequence $a \circ n$.

\subsection{(Sequential) accumulation points}
\begin{definition}[(Sequential) accumulation points]
    Let $(X, \dist)$ be a metric space. A point $p \in X$ is called an \emph{accumulation point} of a sequence $a: \N \to X$ if there is
    a subsequence $a \circ n$ of $a$ such that $a \circ n$ converges to $p$.
\end{definition}

\subsection{Subsequences of a converging sequence}
\begin{proposition}
    Let $(X, \dist)$ be a metric space. Let $(a_n)$ be a sequence in $X$ converging to $p \in X$. Then every subsequence of $(a_n)$ is convergent to $p$.
\end{proposition}

\subsection{$\limsup$}
Consider a real-valued sequence $(a_n)$ that is bounded from above and does not diverge to $-\infty$. We can then define a new sequence
$$k \mapsto \sup_{n \ge k} a_n.$$

Note that this sequence is decreasing, beacuse for larger $k$ the supremum is taken over a smaller set. 
\begin{lemma}
    Let $a: \N \to \R$ be a sequence that is bounded from above and does not diverge to $-\infty$. Then, the sequence $k \mapsto \sup_{n \ge k}a_n$ is bounded
    from below.
\end{lemma}

Since the sequence $k \mapsto \sup_{n \ge k}a_n$ is decreasing and bounded from below, it has a limit, and the limit is in fact equal to the infumum of the sequence. This limit is called the $\limsup$
\begin{align*}
    \limsup_{n\to\infty}a_n &:= \inf_{k\in\N}\sup_{n \ge k}a_n \\
                            &=\lim_{k\to\infty}\left(\sup_{n\ge k}a_n\right)
\end{align*}

\begin{proposition}[Alternative characterization of $\limsup$]
    Let $(a_n)$ be a real-valued sequence. Let $M \in \R$. Then, $M = \limsup_{n\to\infty}a_n$ if and only if
    \begin{enumerate}[label=\roman*.]
        \item 
            \begin{center}
                \parbox{\linewidth}{%
                    \leftskip=0.4\linewidth
                    For every $\epsilon > 0$, \\
                    \hspace*{1em} there exists $N \in \N$, \\
                    \hspace*{2em} for all $\ell \ge N$, \\
                    \hspace*{3em} $a_\ell < M + \epsilon$
                }
            \end{center}
        \item 
            \begin{center}
                \parbox{\linewidth}{
                    \leftskip=0.4\linewidth
                    For every $\epsilon > 0$, \\
                    \hspace*{1em} for all $k \in \N$, \\
                    \hspace*{2em} there exists $m \ge k$, \\
                    \hspace*{3em} $a_m > M - \epsilon$ \\
                }
            \end{center}
    \end{enumerate}
\end{proposition}

\begin{theorem}
    Let $a: \N \to \R$ be a real-valued sequence that is bounded from above and does not diverge to $-\infty$. Then
    $\limsup_{\ell \to \infty}a_\ell$ is a (sequential) accumulation point of $a$, i.e. there exists a subsequences of $a$ that converges to $\limsup_{\ell \to \infty}a_\ell$.
\end{theorem}

\begin{corollary}[Bolzano-Weierstrass]
    Every bounded, real-valued sequence has a subsequence that converges in $(\R, \dist_\R)$.
\end{corollary}

\begin{theorem}
    Suppose a sequence $a: \N \to \R$ is bounded from above and does not diverge to $-\infty$. Then
    $$\limsup_{\ell\to\infty}a_\ell$$
    is the maximum of the set of sequential accumulation points.
\end{theorem}

\subsection{$\liminf$}
Similarly to the $\limsup$, we can define the $\liminf$. In some sense,
$$\liminf_{\ell\to\infty}a_\ell = -\limsup_{\ell\to\infty}(-a_\ell)$$
More precisely, 
\begin{align*}
    \liminf_{\ell\to\infty}a_\ell &:= \sup_{\ell\in\N}\inf_{k \ge \ell}a_k\\
                                  &=\lim_{\ell\to\infty}\left(\inf_{k\ge\ell}a_k\right)
\end{align*}

\begin{proposition}[Alternative characterization of $\liminf$]
    Let $a: \N \to \R$ and $M \in \R$. Then $M$ equals $\liminf_{\ell\to\infty}a_\ell$ if and only if
    \begin{enumerate}
        \item 
            \begin{center}
                \parbox{\linewidth}{
                    \leftskip=0.4\linewidth
                    For every $\epsilon > 0$, \\
                    \hspace*{1em} there exists $N \in \N$, \\
                    \hspace*{2em} for all $\ell \ge N$, \\
                    \hspace*{3em} $a_\ell > M - \epsilon$
                }
            \end{center}
        \item 
            \begin{center}
                \parbox{\linewidth}{
                    \leftskip=0.4\linewidth
                    For every $\epsilon > 0$, \\
                    \hspace*{1em} for all $K \in \N$, \\
                    \hspace*{2em} there exists $m \ge K$, \\
                    \hspace*{3em} $a_m < M + \epsilon$ \\
                }
            \end{center}
    \end{enumerate}
\end{proposition}

\begin{theorem}
    Let $a: \N \to \R$ be a real-valued sequence that is bounded below and does not diverge to $\infty$. Then
    $\liminf_{\ell\to\infty}a_\ell$ is a sequential accumulation point of the sequence $a$, i.e. there is a
    subsequence of $a$ that converges to $\liminf_{\ell\to\infty}a_\ell$.
\end{theorem}

\begin{theorem}
    Let $a: \N \to \R$ be a real-valued sequence that is bounded below and does not diverge to $\infty$. Then
    $\liminf_{\ell\to\infty}a_\ell$ is the minimum of the set of sequential accumulation points.
\end{theorem}

\subsection{Relations between $\lim$, $\lim\sup$ and $\lim\inf$}
\begin{proposition}
    Let $a: \N \to \R$ be a real-valued sequence and let $L \in \R$. Then $a: \N \to \R$ converges to $L$ if and only if
    $$\liminf_{\ell\to\infty}a_\ell = \limsup_{\ell\to\infty} = L$$
\end{proposition}

\begin{proposition}
    Let $a,b: \N \to \R$ be two real-valued sequences, such that there exists an $N \in \N$ such that for all $\ell \ge N$,
    $a_\ell \le b_\ell.$ Then
    $$\limsup_{\ell\to\infty}a_\ell \le \limsup_{\ell\to\infty}b_\ell$$ and 
    $$\liminf_{\ell\to\infty}a_\ell \le \liminf_{\ell\to\infty}b_\ell.$$
\end{proposition}
