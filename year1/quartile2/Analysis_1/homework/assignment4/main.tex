%%%%%%%%%%%%%%%%%%%%%%%%%%%%% Define Article %%%%%%%%%%%%%%%%%%%%%%%%%%%%%%%%%%
\documentclass{article}
%%%%%%%%%%%%%%%%%%%%%%%%%%%%%%%%%%%%%%%%%%%%%%%%%%%%%%%%%%%%%%%%%%%%%%%%%%%%%%%

%%%%%%%%%%%%%%%%%%%%%%%%%%%%% Using Packages %%%%%%%%%%%%%%%%%%%%%%%%%%%%%%%%%%
\usepackage{geometry}
\usepackage{graphicx}
\usepackage{amssymb}
\usepackage{amsmath}
\usepackage{amsthm}
\usepackage{empheq}
\usepackage{mdframed}
\usepackage{booktabs}
\usepackage{lipsum}
\usepackage{graphicx}
\usepackage{color}
\usepackage{psfrag}
\usepackage{pgfplots}
\usepackage{bm}
%%%%%%%%%%%%%%%%%%%%%%%%%%%%%%%%%%%%%%%%%%%%%%%%%%%%%%%%%%%%%%%%%%%%%%%%%%%%%%%

% Other Settings
\newcommand{\N}{\mathbb{N}}
\newcommand{\R}{\mathbb{R}}
\newcommand{\Q}{\mathbb{Q}}
\newcommand{\Z}{\mathbb{Z}}
\newcommand{\C}{\mathbb{C}}

\newcommand{\e}{\epsilon}
\newcommand{\de}{\delta}

\newcommand{\limn}{\lim_{n\to\infty}}


%%%%%%%%%%%%%%%%%%%%%%%%%% Page Setting %%%%%%%%%%%%%%%%%%%%%%%%%%%%%%%%%%%%%%%
\geometry{a4paper}

%%%%%%%%%%%%%%%%%%%%%%%%%% Define some useful colors %%%%%%%%%%%%%%%%%%%%%%%%%%
\definecolor{ocre}{RGB}{243,102,25}
\definecolor{mygray}{RGB}{243,243,244}
\definecolor{deepGreen}{RGB}{26,111,0}
\definecolor{shallowGreen}{RGB}{235,255,255}
\definecolor{deepBlue}{RGB}{61,124,222}
\definecolor{shallowBlue}{RGB}{235,249,255}
%%%%%%%%%%%%%%%%%%%%%%%%%%%%%%%%%%%%%%%%%%%%%%%%%%%%%%%%%%%%%%%%%%%%%%%%%%%%%%%

%%%%%%%%%%%%%%%%%%%%%%%%%% Define an orangebox command %%%%%%%%%%%%%%%%%%%%%%%%
\newcommand\orangebox[1]{\fcolorbox{ocre}{mygray}{\hspace{1em}#1\hspace{1em}}}
%%%%%%%%%%%%%%%%%%%%%%%%%%%%%%%%%%%%%%%%%%%%%%%%%%%%%%%%%%%%%%%%%%%%%%%%%%%%%%%

%%%%%%%%%%%%%%%%%%%%%%%%%%%% English Environments %%%%%%%%%%%%%%%%%%%%%%%%%%%%%
\newtheoremstyle{mytheoremstyle}{3pt}{3pt}{\normalfont}{0cm}{\rmfamily\bfseries}{}{1em}{{\color{black}\thmname{#1}~\thmnumber{#2}}\thmnote{\,--\,#3}}
\newtheoremstyle{myproblemstyle}{3pt}{3pt}{\normalfont}{0cm}{\rmfamily\bfseries}{}{1em}{{\color{black}\thmname{#1}~\thmnumber{#2}}\thmnote{\,--\,#3}}
\theoremstyle{mytheoremstyle}
\newmdtheoremenv[linewidth=1pt,backgroundcolor=shallowGreen,linecolor=deepGreen,leftmargin=0pt,innerleftmargin=20pt,innerrightmargin=20pt,]{theorem}{Theorem}[section]
\theoremstyle{mytheoremstyle}
\newmdtheoremenv[linewidth=1pt,backgroundcolor=shallowBlue,linecolor=deepBlue,leftmargin=0pt,innerleftmargin=20pt,innerrightmargin=20pt,]{definition}{Definition}[section]
\theoremstyle{myproblemstyle}
\newmdtheoremenv[linecolor=black,leftmargin=0pt,innerleftmargin=10pt,innerrightmargin=10pt,]{problem}{Problem}[section]
%%%%%%%%%%%%%%%%%%%%%%%%%%%%%%%%%%%%%%%%%%%%%%%%%%%%%%%%%%%%%%%%%%%%%%%%%%%%%%%

%%%%%%%%%%%%%%%%%%%%%%%%%%%%%%% Plotting Settings %%%%%%%%%%%%%%%%%%%%%%%%%%%%%
\usepgfplotslibrary{colorbrewer}
\pgfplotsset{width=8cm,compat=1.9}
%%%%%%%%%%%%%%%%%%%%%%%%%%%%%%%%%%%%%%%%%%%%%%%%%%%%%%%%%%%%%%%%%%%%%%%%%%%%%%%

%%%%%%%%%%%%%%%%%%%%%%%%%%%%%%% Title & Author %%%%%%%%%%%%%%%%%%%%%%%%%%%%%%%%
\title{Homework: Week 2}
\author{Jiaqi Wang}
%%%%%%%%%%%%%%%%%%%%%%%%%%%%%%%%%%%%%%%%%%%%%%%%%%%%%%%%%%%%%%%%%%%%%%%%%%%%%%%

\begin{document}
    \maketitle

    \section{Exercise 6.8.1}
    \begin{problem}
        Let $a: \N \to \R$ be a sequence. Then $a: \N \to \R$ is bounded if and only if it is both bounded above and bounded below. Prove it.
    \end{problem}
    \begin{proof}
        We prove both directions of the if and only if statement.
        $a: \N \to \R$ is bounded if and only if it is both bounded above and bounded below.

        1. We prove the forward direction. \\
        Suppose $a: \N \to \R$ is bounded. \\
        Then
        \begin{center}
            \parbox{\linewidth}{
                \leftskip=0.4\linewidth
                there exists $M_0 > 0$, \\
                \hspace*{1em} for all $n \in \N$, \\
                \hspace*{2em} $|a_n| \leq M_0$.
            }
        \end{center}
        Obtain such $M_0$. \\
        It holds that
        \begin{center}
            \parbox{\linewidth}{
                \leftskip=0.4\linewidth
                for all $n \in \N$, \\
                \hspace*{1em} $-M_0 \leq a_n \leq M_0$.
            }
        \end{center}
        Choose $m = -M_0$, then $m \in \R$. \\
        It holds that
        \begin{center}
            \parbox{\linewidth}{
                \leftskip=0.4\linewidth
                for all $n \in \N$, \\
                \hspace*{1em} $m \leq a_n$.
            }
        \end{center}
        We conclude that $a:\N \to \R$ is bounded from below.

        Choose $M = M_0$, then $M \in \R$. \\
        It holds that
        \begin{center}
            \parbox{\linewidth}{
                \leftskip=0.4\linewidth
                for all $n \in \N$, \\
                \hspace*{1em} $a_n \leq M$.
            }
        \end{center}
        We conclude that $a:\N \to \R$ is bounded from above.

        2. Now we prove the reverse direction. \\
        Assume $a:\N \to \R$ is bounded from above and bounded from below. \\
        By definition of bounded from above, we have
        \begin{center}
            \parbox{\linewidth}{
                \leftskip=0.4\linewidth
                there exists $M_1 \in \R$, \\
                \hspace*{1em} for all $n \in \N$, \\
                \hspace*{2em} $a_n \leq M_1$.
            }
        \end{center}

        By definition of bounded from below, we have
        \begin{center}
            \parbox{\linewidth}{
                \leftskip=0.4\linewidth
                there exists $M_2 \in \R$, \\
                \hspace*{1em} for all $n \in \N$, \\
                \hspace*{2em} $M_2 \leq a_n$.
            }
        \end{center}

        We need to show that
        \begin{center}
            \parbox{\linewidth}{
                \leftskip=0.4\linewidth
                there exists $M_0 > 0$, \\
                \hspace*{1em} for all $n \in \N$, \\
                \hspace*{2em} $|a_n| \leq M_0$.
            }
        \end{center}

        Choose $M_0 = \max\{|M_1|, |M_2|\}$, then $M_0 \in \R$. \\
        It holds that
        \begin{center}
            \parbox{\linewidth}{
                \leftskip=0.4\linewidth
                for all $n \in \N$, \\
                \hspace*{1em} $-M_0 \leq a_n$ and $a_n \leq M_0$.
            }
        \end{center}
        Then it holds that $|a_n| \leq M_0$.\\
        We conclude that $a: \N \to \R$ is bounded.

        Since both directions hold, we conclude that $a: \N \to \R$ is bounded if and only if it is both bounded above and bounded below.
    \end{proof}

    \section{Exercise 6.8.2}
    \begin{problem}
        Let $a: \N \to \R$ and $b: \N \to (0,\infty)$ be real-valued sequences. Prove that
        $$\lim_{n\to\infty}a_n = \infty \iff \lim_{n\to\infty}(-a_n) = -\infty$$
    \end{problem}
    \begin{proof}
        We show both directions of the if and only if statement.
        $$\lim_{n\to\infty}a_n = \infty \iff \lim_{n\to\infty}(-a_n) = -\infty$$

        1. First we prove the forward direction. \\
        Suppose $\lim_{n\to\infty}a_n = \infty$. \\
        Then
        \begin{center}
            \parbox{\linewidth}{
                \leftskip=0.4\linewidth
                for all $M \in \R$, \\
                \hspace*{1em} there exists $N_0 \in \N$, \\
                \hspace*{2em} for all $n \geq N_0$, \\
                \hspace*{3em} $a_n > M$.
            }
        \end{center}
        Obtain such $N_0$. \\
        We need to show that
        \begin{center}
            \parbox{\linewidth}{
                \leftskip=0.4\linewidth
                for all $M \in \R$, \\
                \hspace*{1em} there exists $N \in \N$, \\
                \hspace*{2em} for all $n \geq N$, \\
                \hspace*{3em} $-a_n < M$.
            }
        \end{center}
        Take $M \in \R$
        Choose $N = N_0$, then $N \in \N$. \\
        It holds that
        \begin{center}
            \parbox{\linewidth}{
                \leftskip=0.4\linewidth
                for all $M \in \R$, \\
                \hspace*{1em} for all $n \geq N$, \\
                \hspace*{2em} $a_n > M$. \\
                \hspace*{3em} $-a_n < -M$.
            }
        \end{center}
        Because $M$ is arbitrary, we conclude that
        \begin{center}
            \parbox{\linewidth}{
                \leftskip=0.4\linewidth
                for all $M \in \R$, \\
                \hspace*{1em} there exists $N \in \N$, \\
                \hspace*{2em} for all $n \geq N$, \\
                \hspace*{3em} $-a_n < M$.
            }
        \end{center}
    \end{proof}

    \section{Exercise 6.8.3}
    \begin{problem}
        Let $a: \N \to \R$ and $b: \N \to (0,\infty)$ be real-valued sequences. Prove that
        $$\lim_{n\to\infty}b_n = \infty \iff \lim_{n\to\infty}\frac{1}{b_n} = 0$$
    \end{problem}
    \begin{proof}
        We need to show both directions of the if and only if statement.
        $$\lim_{n\to\infty}b_n = \infty \iff \lim_{n\to\infty}\frac{1}{b_n} = 0$$

        1. First we prove the forward direction. \\
        We need to show that $\lim_{n\to\infty}b_n = \infty \implies \lim_{n\to\infty}\frac{1}{b_n} = 0$. \\
        I.e.
        \begin{center}
            \parbox{\linewidth}{
                \leftskip=0.4\linewidth
                for all $\e > 0$, \\
                \hspace*{1em} there exists $N \in \N$, \\
                \hspace*{2em} for all $n \geq N$, \\
                \hspace*{3em} $|\frac{1}{b_n} - 0| < \e$.
            }
        \end{center}
        Take $\e > 0$. \\
        Suppose $\lim_{n\to\infty}b_n = \infty$. \\
        Then
        \begin{center}
            \parbox{\linewidth}{
                \leftskip=0.4\linewidth
                for all $M \in \R$, \\
                \hspace*{1em} there exists $N_0 \in \N$, \\
                \hspace*{2em} for all $n \geq N_0$, \\
                \hspace*{3em} $b_n > M$.
            }
        \end{center}
        Choose $M = \frac{1}{\e}$, then $M \in \R$. \\
        Obtain such $N_0$. \\
        Take $\e > 0$. \\
        Choose $N = N_0$, then $N \in \N$. \\
        It holds that
        \begin{center}
            \parbox{\linewidth}{
                \leftskip=0.4\linewidth
                for all $\e > 0$, \\
                \hspace*{1em} for all $n \geq N$, \\
                \hspace*{2em} $b_n > M$. \\
                \hspace*{3em} $\frac{1}{b_n} < \frac{1}{M}$. \\
                \hspace*{4em} $|\frac{1}{b_n} - 0| < \frac{1}{M}$. \\
                \hspace*{5em} $|\frac{1}{b_n} - 0| < \e$.
            }
        \end{center}

        2. Now we prove the reverse direction. \\
        We need to show that $\lim_{n\to\infty}\frac{1}{b_n} = 0 \implies \lim_{n\to\infty}b_n = \infty$. \\
        I.e.
        \begin{center}
            \parbox{\linewidth}{
                \leftskip=0.4\linewidth
                for all $M \in (0,\infty)$, \\
                \hspace*{1em} there exists $N \in \N$, \\
                \hspace*{2em} for all $n \geq N$, \\
                \hspace*{3em} $b_n > M$.
            }
        \end{center}
        Take $M \in \R$. \\
        Assume $\lim_{n\to\infty}\frac{1}{b_n} = 0$. \\
        Then
        \begin{center}
            \parbox{\linewidth}{
                \leftskip=0.4\linewidth
                for all $\e > 0$, \\
                \hspace*{1em} there exists $N_0 \in \N$, \\
                \hspace*{2em} for all $n \geq N_0$, \\
                \hspace*{3em} $|\frac{1}{b_n} - 0| < \e$.
            }
        \end{center}
        Choose $\e = \frac{1}{M}$, then $\e > 0$. \\
        Obtain such $N_0$. \\
        Choose $N = N_0$, then $N \in \N$. \\
        It holds that
        \begin{center}
            \parbox{\linewidth}{
                \leftskip=0.4\linewidth
                \hspace*{1em} for all $n \geq N$, \\
                \hspace*{2em} $|\frac{1}{b_n} - 0| < \e$. \\
                \hspace*{3em} $\frac{1}{b_n} < \e$. \\
                \hspace*{4em} $\frac{1}{b_n} < \frac{1}{M}$. \\
                \hspace*{5em} $b_n > M$.
            }
        \end{center}
    \end{proof}

    \section{Exercise 6.8.5}
    \begin{problem}
        Define the sequence $x: \N \to \R$ recursively by
        $$x_{n+1} := \frac{2+x_n^2}{2x_n}$$
        for $n \in \N$ while $x_0 = 2$. Prove that the sequence $x: \N \to \R$ converges and determine its limit.
    \end{problem}
    \begin{proof}
        $$x_{n + 1} := \frac{2 + x_n ^2}{2x_n} = \frac{1}{x_n} + \frac{x_n}{2}$$ \\
        Since we are starting with $x_0 = 2$, and never substracting to get the following $x$, we know $x_n \geq 0$.\\
        This gives us a lower bound for the sequence.\\
        Notice, that for $x_n > 1$ we have $x_n = \frac{x_n}{2} + \frac{x_n}{2} > \frac{1}{x_n} + \frac{x_n}{2}$.\\
        And $\frac{1}{x_n} + \frac{x_n}{2} > 1 \iff \frac{2 + x_n ^2}{2x_n} > 1 \iff x_n ^2 - 2x_n + 1 > 0$, which is always true for every $x > 0$.\\
        Thus since we start at $x_0 = 2$, our sequence decreases. \\
        Hence $x_n > x_{n + 1} \iff x_n > \frac{2 + x_n ^2}{2x_n} \iff 2x_n ^2 - x_n ^2 > 2 \iff x_n ^2 > 2$
        This gives us that $x_n$ is either larger than $\sqrt{2}$ or smaller than $-\sqrt{2}$. The latter is impossible
        since $x_n$ is always positive, hence our new lower bound is $\sqrt{2}$.
        This lower bound is in fact our infimum. \\
        Since our infimum is $\sqrt{2}$ and the function is decreasing,
        the sequence converges to $\sqrt{2}$.
    \end{proof}

    \section{Exercise 6.8.6}
    \begin{problem}
        Determine whether the following sequences converge, diverge to $+\infty$, diverge to $-\infty$, or diverge in a different way.
        In case the sequence converges, determine the limit.
        \begin{itemize}
            \item $a_n := \frac{1}{n^3} - 3$
            \item $b_n := \frac{5n^5 + 2n^2}{3n^5+7n^3+4}$
            \item $c_n := n - \sqrt{n}$
            \item $d_n := \frac{2^n}{n^{100}}$
            \item $e_n := \sqrt{n^2+n}-n$
            \item $f_n := \sqrt[n]{3n^2}$
            \item $g_n := \frac{2^n + 5n^200}{3^n+n^10}$
            \item $h_n := (-1)^n3^n$
            \item $i_n := \sqrt[n]{5^n + n^2}$
        \end{itemize}
    \end{problem}
    \subsection{a)}
    $$a_n := \frac{1}{n^3} - 3$$
        By limit laws and standard limits, we have
        \begin{align*}
            \limn a_n &= \limn\left(\frac{1}{n^3} - 3\right) \\
                    &= \limn\frac{1}{n^3} - \limn 3 \\
                    &= \left(\limn\frac{1}{n}\right)^3 - 3 \\
                    &= 0^3 - 3 \\
                    &= -3 \\
        \end{align*}
        So the sequence converges to $-3$.
    \subsection{b)}
    $$b_n := \frac{5n^5 + 2n^2}{3n^5+7n^3+4}$$
    By limit laws and standard limits, we have
    \begin{align*}
        \limn b_n &= \limn\left(\frac{5n^5 + 2n^2}{3n^5+7n^3+4}\right) \\
        &= \limn\left(\frac{n^5(5 + \frac{2}{n^3})}{n^5(3+\frac{7}{n^2}+\frac{4}{n^5})}\right) \\
        &= \limn\left(\frac{5 + \frac{2}{n^3}}{3+\frac{7}{n^2}+\frac{4}{n^5}}\right) \\
        &= \frac{\limn(5 + \frac{2}{n^3})}{\limn(3+\frac{7}{n^2}+\frac{4}{n^5})} \\
        &= \frac{\limn 5 + \limn\frac{2}{n^3}}{\limn 3+\limn\frac{7}{n^2}+\limn\frac{4}{n^5}} \\
        &= \frac{5 + 0}{3+0+0} \\
        &= \frac{5}{3}
    \end{align*}
    So the sequence converges to $\frac{5}{3}$.
    \subsection{c)}
    $$c_n := n - \sqrt{n}$$
    By limit laws and standard limits, we have
    \begin{align*}
        \limn c_n &= \limn(n - \sqrt{n}) \\
                    &= \limn\left(\frac{(n - \sqrt{n})(n + \sqrt{n})}{n + \sqrt{n}}\right) \\
                    &= \limn\left(\frac{n^2 - n}{n + \sqrt{n}}\right) \\
                    &= \limn\left(\frac{n(n - 1)}{n(1 + \frac{1}{\sqrt{n}})}\right) \\
                    &= \limn\left(\frac{n - 1}{1 + \frac{1}{\sqrt{n}}}\right) \\
                    &= \frac{\limn(n - 1)}{\limn(1 + \frac{1}{\sqrt{n}})} \\
                    &= \frac{\limn n - \limn 1}{\limn 1 + \limn\frac{1}{\sqrt{n}}} \\
                    &= \frac{\limn n - 1}{1 + \limn\frac{1}{\sqrt{n}}} \\
                    &= \frac{\limn n - 1}{1 + 0} \\
                    &= \limn n - 1 \\
                    &= \infty \\
    \end{align*}
    So the sequence diverges to $\infty$.
    \subsection{d)}
    $$d_n := \frac{2^n}{n^{100}}$$
        By limit laws and standard limits, we have
        \begin{align*}
            \limn d_n &= \limn(\frac{2^n}{n^{100}}) \\
                      &= \limn\left(\frac{2^n}{n^{100}}\right) \\
                      &= \limn{\left(\frac{1}{\frac{n^{100}}{2^n}}\right)} \\
                      &= \frac{\limn 1}{\limn\frac{n^{100}}{2^n}} \\
                      &= \frac{1}{\limn\frac{n^{100}}{2^n}} \\
                      &= \infty
        \end{align*}
    So the sequence diverges to $\infty$.
    \subsection{e)}
    $$e_n := \sqrt{n^2+n}-n$$
        By limit laws and standard limits, we have
        \begin{align*}
            \limn{e_n} &= \limn{\sqrt{n^2+n} - n} \\
            &= \limn{\frac{(\sqrt{n^2+n} - n)(\sqrt{n^2+n} + n)}{\sqrt{n^2+n} + n}} \\
            &= \limn{\frac{n^2+n - n^2}{\sqrt{n^2+n} + n}} \\
            &= \limn{\frac{n}{\sqrt{n^2+n} + n}} \\
            &= \limn{\frac{1}{\frac{\sqrt{n^2+n}}{n} + 1}} \\
            &= \frac{\limn 1}{\limn\frac{\sqrt{n^2+n}}{n} + \limn 1} \\
            &= \frac{1}{\limn\frac{\sqrt{n^2+n}}{n} + 1} \\
            &= \frac{1}{\limn\sqrt{\frac{n^2+n}{n^2}} + 1} \\
            &= \frac{1}{\limn\sqrt{1 + \frac{1}{n}} + 1} \\
            &= \frac{1}{\sqrt{\limn 1 + \limn\frac{1}{n}} + 1} \\
            &= \frac{1}{\sqrt{1 + 0} + 1} \\
            &= \frac{1}{2}
        \end{align*}
    So the sequence converges to $\frac{1}{2}$.
    \subsection{f)}
    $$f_n := \sqrt[n]{3n^2}$$
    By limit laws and standard limits, we have
        \begin{align*}
            \limn{f_n} &= \limn{\sqrt[n]{3n^2}} \\
            &= \limn{\sqrt[n]{3}\sqrt[n]{n^2}} \\
            &= \limn{\sqrt[n]{3}(\sqrt[n]{n})^2} \\
            &= \limn{\sqrt[n]{3}}(\limn{\sqrt[n]{n}})^2 \\
            &= 1 \cdot 1^2 \\
            &= 1
        \end{align*}
        So the sequence converges to $1$.
    \subsection{g)}
    $$g_n := \frac{2^n + 5n^200}{3^n+n^10}$$
    By limit laws and standard limits, we have
        \begin{align*}
            \limn{g_n} &= \limn{\frac{2^n + 5n^{200}}{3^n+n^{10}}} \\
            &= \limn{\frac{\frac{2^n}{3^n} + \frac{5n^{200}}{3^n}}{1+\frac{n^{10}}{3^n}}} \\
            &= \frac{\limn{(\frac{2^n}{3^n} + \frac{5n^{200}}{3^n})}}{\limn{(1+\frac{n^{10}}{3^n})}} \\
            &= \frac{\limn{(\frac{2}{3})^n + 5\limn{\frac{n^{200}}{3^n}}}}{\limn{1}+\limn{\frac{n^{10}}{3^n}}} \\
            &= \frac{0 + 5\cdot 0}{1+0} \\
            &= 0
        \end{align*}
        So the sequence converges to $0$.
    \subsection{h)}
    $$h_n := (-1)^n3^n$$
        Since the sequence $3^n$ diverges to $\infty$ and $(-1)^n$ osilates between $-1$ and $1$, the sequence $h_n$ doesn't converge, but neither diverges to $\infty$ or to $-\infty$.
    \subsection{i)}
    $$i_n := \sqrt[n]{5^n + n^2}$$
    By limit laws and standard limits, we have
        \begin{align*}
            \limn{i_n} &= \limn{\sqrt[n]{5^n + n^2}} \\
            &= \limn{\left(5^n+\left(1+\frac{n^2}{5^n}\right)\right)^{\frac{1}{n}}} \\
            &= \limn{\exp{\left(\ln{\left(5^n+\left(1+\frac{n^2}{5^n}\right)\right)^{\frac{1}{n}}}\right)}} \\
            &= \limn{\exp\left(\frac{1}{n}\ln{\left(5^n+\frac{n^2}{5^n}\right)}\right)} \\
            &= \exp\left(\limn{\frac{\ln{\left(5^n+\frac{n^2}{5^n}\right)}}{n}}\right) \\
            &= \exp\left(\limn{\frac{\ln{5^n} + \ln{\left(1+\frac{n^2}{5^n}\right)}}{n}}\right) \\
            &= \exp\left(\limn{\frac{n\ln{5}}{n} + \limn{\frac{\ln{\left(1+\frac{n^2}{5^n}\right)}}{n}}}\right) \\
            &= \exp\left(\ln{5} + \frac{\limn{\ln\left(1+\frac{n^2}{5^n}\right)}}{\limn{n}}\right) \\
            &= \exp\left(\ln{5} + \frac{\ln(\limn\left(1+\frac{n^2}{5^n}\right))}{\limn{n}}\right) \\
            &= \exp\left(\ln{5} + \frac{\ln\left(\limn 1 + \limn\left(\frac{n^2}{5^n}\right)\right)}{\limn{n}}\right) \\
            &= \exp\left(\ln{5} + \frac{\ln{1}}{\limn n}\right) \\
            &= \exp\left(\ln{5} + 0\right) \\
            &= 5
        \end{align*}
        So the sequence converges to $5$.
\end{document}
