%%%%%%%%%%%%%%%%%%%%%%%%%%%%% Define Article %%%%%%%%%%%%%%%%%%%%%%%%%%%%%%%%%%
\documentclass{article}
%%%%%%%%%%%%%%%%%%%%%%%%%%%%%%%%%%%%%%%%%%%%%%%%%%%%%%%%%%%%%%%%%%%%%%%%%%%%%%%

%%%%%%%%%%%%%%%%%%%%%%%%%%%%% Using Packages %%%%%%%%%%%%%%%%%%%%%%%%%%%%%%%%%%
\usepackage{geometry}
\usepackage{graphicx}
\usepackage{amssymb}
\usepackage{amsmath}
\usepackage{amsthm}
\usepackage{empheq}
\usepackage{mdframed}
\usepackage{booktabs}
\usepackage{lipsum}
\usepackage{graphicx}
\usepackage{color}
\usepackage{psfrag}
\usepackage{pgfplots}
\usepackage{bm}
%%%%%%%%%%%%%%%%%%%%%%%%%%%%%%%%%%%%%%%%%%%%%%%%%%%%%%%%%%%%%%%%%%%%%%%%%%%%%%%

\usepackage[most]{tcolorbox}
% \usepackage[mathscr]{euscript}
\usepackage{enumitem}
\usepackage{physics}
\usepackage{microtype}
\usepackage[T1]{fontenc}
\usepackage[utf8]{inputenc}
\usepackage{hyperref}
\hypersetup{
    colorlinks=true,
    linkcolor=blue,
    filecolor=magenta,
    urlcolor=cyan,
    pdftitle={Homework: Week 1},
    % pdfpagemode=FullScreen,
    }

\urlstyle{same}

\setcounter{tocdepth}{2}
\usepackage{cleveref}


%%%%%%%%%%%%%%%%%%%%%%%%%% Page Setting %%%%%%%%%%%%%%%%%%%%%%%%%%%%%%%%%%%%%%%
\geometry{a4paper}

%%%%%%%%%%%%%%%%%%%%%%%%%% Define some useful colors %%%%%%%%%%%%%%%%%%%%%%%%%%
\definecolor{ocre}{RGB}{243,102,25}
\definecolor{mygray}{RGB}{243,243,244}
\definecolor{deepGreen}{RGB}{26,111,0}
\definecolor{shallowGreen}{RGB}{235,255,255}
\definecolor{deepBlue}{RGB}{61,124,222}
\definecolor{shallowBlue}{RGB}{235,249,255}
%%%%%%%%%%%%%%%%%%%%%%%%%%%%%%%%%%%%%%%%%%%%%%%%%%%%%%%%%%%%%%%%%%%%%%%%%%%%%%%

%%%%%%%%%%%%%%%%%%%%%%%%%% Define an orangebox command %%%%%%%%%%%%%%%%%%%%%%%%
\newcommand\orangebox[1]{\fcolorbox{ocre}{mygray}{\hspace{1em}#1\hspace{1em}}}
\newcommand\tab[1][1cm]{\hspace*{#1}}
%%%%%%%%%%%%%%%%%%%%%%%%%%%%%%%%%%%%%%%%%%%%%%%%%%%%%%%%%%%%%%%%%%%%%%%%%%%%%%%

%%%%%%%%%%%%%%%%%%%%%%%%%%%% English Environments %%%%%%%%%%%%%%%%%%%%%%%%%%%%%
\newtheoremstyle{mytheoremstyle}{3pt}{3pt}{\normalfont}{0cm}{\rmfamily\bfseries}{}{1em}{{\color{black}\thmname{#1}~\thmnumber{#2}}\thmnote{\,--\,#3}}
\newtheoremstyle{myproblemstyle}{3pt}{3pt}{\normalfont}{0cm}{\rmfamily\bfseries}{}{1em}{{\color{black}\thmname{#1}~\thmnumber{#2}}\thmnote{\,--\,#3}\newline}
\theoremstyle{mytheoremstyle}
\newmdtheoremenv[linewidth=1pt,backgroundcolor=shallowGreen,linecolor=deepGreen,leftmargin=0pt,innerleftmargin=20pt,innerrightmargin=20pt,]{theorem}{Theorem}[section]
\theoremstyle{mytheoremstyle}
\newmdtheoremenv[linewidth=1pt,backgroundcolor=shallowBlue,linecolor=deepBlue,leftmargin=0pt,innerleftmargin=20pt,innerrightmargin=20pt,]{definition}{Definition}[section]
\theoremstyle{myproblemstyle}
\newmdtheoremenv[linecolor=black,leftmargin=0pt,innerleftmargin=10pt,innerrightmargin=10pt,]{problem}{Problem}[section]
%%%%%%%%%%%%%%%%%%%%%%%%%%%%%%%%%%%%%%%%%%%%%%%%%%%%%%%%%%%%%%%%%%%%%%%%%%%%%%%

%%%%%%%%%%%%%%%%%%%%%%%%%%%%%%% Plotting Settings %%%%%%%%%%%%%%%%%%%%%%%%%%%%%
\usepgfplotslibrary{colorbrewer}
\pgfplotsset{width=8cm,compat=1.9}
%%%%%%%%%%%%%%%%%%%%%%%%%%%%%%%%%%%%%%%%%%%%%%%%%%%%%%%%%%%%%%%%%%%%%%%%%%%%%%%

%%%%%%%%%%%%%%%%%%%%%%%%%%%%%%% Title & Author %%%%%%%%%%%%%%%%%%%%%%%%%%%%%%%%
\title{Homework: Week 1}
\author{Jiaqi Wang}
%%%%%%%%%%%%%%%%%%%%%%%%%%%%%%%%%%%%%%%%%%%%%%%%%%%%%%%%%%%%%%%%%%%%%%%%%%%%%%%

\begin{document}
    \maketitle

    \section{2.7.1}
    \begin{problem}
        Let $(Y, \text{dist}_Y)$ be a metric space. Let $X$ be a set and $f: X \to Y$ be injective. Define $d: X \times X \to \mathbb{R}$ by
        $$d(x,y) := \text{dist}_Y(f(x),f(y)), \text{ for all } x,y \in X.$$

        Show that the function $d$ is a distance on $X$.
    \end{problem}

    \begin{proof}
        To show that $d$ is a distance on $X$, we need to show that $d$ satisfies the following three properties:
        \begin{enumerate}
            \item Positivity: $d(x,y) \geq 0$ for all $x,y \in X$.
            \item Non-degeneracy: $d(x,y) = 0$ if and only if $x = y$.
            \item Symmetry: $d(x,y) = d(y,x)$ for all $x,y \in X$.
            \item Triangle inequality: $d(x,z) \leq d(x,y) + d(y,z)$ for all $x,y,z \in X$.
            \item Reflexivity: $d(x,x) = 0$ for all $x \in X$.
        \end{enumerate}

        We show each of these properties in turn.

        1. Positivity: \\
        Take $x \in X$. \\
        Take $y \in X$. \\
        It holds that $f(x) \in Y$. \\
        It holds that $f(y) \in Y$. \\
        By positivity of dist$_Y$ it holds that $\text{dist}_Y(f(x),f(y)) \geq 0$. \\
        It holds that $d(x,y) = \text{dist}_Y(f(x),f(y)) \geq 0$. \\
        We conclude that $d(x,y) \geq 0$ for all $x,y \in X$. \\

        2. Non-degeneracy: \\
        We need to show both directions. \\
        2.1 Forward direction: \\
        Take $x \in X$. \\
        Take $y \in X$. \\
        Assume $d(x,y) = 0$. \\
        It holds that $d(x,y) = \text{dist}_Y(f(x),f(y)) = 0$. \\
        By non-degeneracy of dist$_Y$ it holds that $f(x) = f(y)$. \\
        By injectivity of $f$ it holds that $x = y$. \\
        We conclude that $d(x,y) = 0 \implies x = y$. \\
        2.2 Backward direction: \\
        Take $x \in X$. \\
        Take $y \in X$. \\
        Assume $x = y$. \\
        It holds that $f(x) = f(y)$. \\
        By non-degeneracy of dist$_Y$ it holds that $\text{dist}_Y(f(x),f(y)) = 0$. \\
        It holds that $d(x,y) = \text{dist}_Y(f(x),f(y)) = 0$. \\
        We conclude that $x = y \implies d(x,y) = 0$. \\
        We conclude that $d(x,y) = 0 \iff x = y$. \\

        3. Symmetry: \\
        Take $x \in X$. \\
        Take $y \in X$. \\
        It holds that $d(x,y) = \text{dist}_Y(f(x),f(y))$. \\
        By symmetry of dist$_Y$ it holds that $\text{dist}_Y(f(x),f(y)) = \text{dist}_Y(f(y),f(x))$. \\
        It holds that $d(x,y) = \text{dist}_Y(f(x),f(y)) = \text{dist}_Y(f(y),f(x)) = d(y,x)$. \\
        We conclude that $d(x,y) = d(y,x)$ for all $x,y \in X$. \\

        4. Triangle inequality: \\
        Take $x \in X$. \\
        Take $y \in X$. \\
        Take $z \in X$. \\
        It holds that $d(x,y) = \text{dist}_Y(f(x),f(y))$. \\
        It holds that $d(x,z) = \text{dist}_Y(f(x),f(z))$. \\
        It holds that $d(z,y) = \text{dist}_Y(f(z),f(y))$. \\
        By triangle inequality of dist$_Y$ it holds that $\text{dist}_Y(f(x),f(z)) \leq \text{dist}_Y(f(x),f(y)) + \text{dist}_Y(f(y),f(z))$. \\
        It holds that $d(x,z) = \text{dist}_Y(f(x),f(z)) \leq \text{dist}_Y(f(x),f(y)) + \text{dist}_Y(f(y),f(y)) = d(x,y) + d(y,z)$. \\
        We conclude that $d(x,z) \leq d(x,y) + d(y,z)$ for all $x,y,z \in X$. \\

        5. Reflexivity: \\
        Take $x \in X$. \\
        It holds that $d(x,x) = \text{dist}_Y(f(x),f(x))$. \\
        By reflexivity of dist$_Y$ it holds that $\text{dist}_Y(f(x),f(x)) = 0$. \\
        It holds that $d(x,x) = \text{dist}_Y(f(x),f(x)) = 0$. \\
        We conclude that $d(x,x) = 0$ for all $x \in X$. \\

        We conclude that $d$ is a distance on $X$.
    \end{proof}

    \newpage

    \section{2.7.3}
    \begin{problem}
        Consider the function $d: \mathbb{Z} \times \mathbb{Z} \to \mathbb{R}$ defined by
        $$d(a,b) = 
        \begin{cases} 0, &\text{ if } a = b \\
                      3, &\text{ if } a \ne b
        \end{cases}$$

        Show that $d$ is a distance function on $\mathbb{Z}$.
    \end{problem}

    \begin{proof}
        To show that $d$ is a distance on $\mathbb{Z}$, we need to show that $d$ satisfies the following 5 properties:
        \begin{enumerate}
            \item Positivity: $d(a,b) \geq 0$ for all $a,b \in \mathbb{Z}$.
            \item Non-degeneracy: $d(a,b) = 0$ if and only if $a = b$.
            \item Symmetry: $d(a,b) = d(b,a)$ for all $a,b \in \mathbb{Z}$.
            \item Triangle inequality: $d(a,c) \leq d(a,b) + d(b,c)$ for all $a,b,c \in \mathbb{Z}$.
            \item Reflexivity: $d(a,a) = 0$ for all $a \in \mathbb{Z}$.
        \end{enumerate}

        We show each of these properties in turn.

        1. Positivity: \\
        Take $a \in \mathbb{Z}$. \\
        Take $b \in \mathbb{Z}$. \\
        Either $a = b$ or $a \ne b$. \\
        - Case 1: $a = b$. \\
        \tab It holds that $d(a,b) = 0$. \\
        \tab We conclude that $d(a,b) = 0 \geq 0$. \\
        - Case 2: $a \ne b$. \\
        \tab It holds that $d(a,b) = 3$. \\
        \tab We conclude that $d(a,b) = 3 \geq 0$. \\
        We conclude that $d(a,b) \geq 0$ for all $a,b \in \mathbb{Z}$. \\

        2. Non-degeneracy: \\
        We need to show both directions. \\
        2.1 Forward direction: \\
        Take $a \in \mathbb{Z}$. \\
        Take $b \in \mathbb{Z}$. \\
        Assume $d(a,b) = 0$. \\
        It holds that $a = b$. \\
        We conclude that $d(a,b) = 0 \implies a = b$. \\
        2.2 Backward direction: \\
        Take $a \in \mathbb{Z}$. \\
        Take $b \in \mathbb{Z}$. \\
        Assume $a = b$. \\
        It holds that $d(a,b) = 0$. \\
        We conclude that $a = b \implies d(a,b) = 0$. \\
        We conclude that $d(a,b) = 0 \iff a = b$. \\

        3. Symmetry: \\
        Take $a \in \mathbb{Z}$. \\
        Take $b \in \mathbb{Z}$. \\
        Either $a = b$ or $a \ne b$. \\
        - Case 1: $a = b$. \\
        \tab It holds that $d(a,b) = 0$. \\
        \tab We conclude that $d(a,b) = 0 = 0 = d(b,a)$. \\
        - Case 2: $a \ne b$. \\
        \tab It holds that $d(a,b) = 3$. \\
        \tab We conclude that $d(a,b) = 3 = 3 = d(b,a)$. \\
        We conclude that $d(a,b) = d(b,a)$ for all $a,b \in \mathbb{Z}$. \\

        4. Triangle inequality: \\
        Take $a \in \mathbb{Z}$. \\
        Take $b \in \mathbb{Z}$. \\
        Take $c \in \mathbb{Z}$. \\
        Either $a = b$ or $a \ne b$. \\
        Either $a = c$ or $a \ne c$. \\
        Either $b = c$ or $b \ne c$. \\
        - Case 1: $a = b$. \\
        \tab It holds that $d(a,b) = 0$. \\
        \tab Either $a = c$ or $a \ne c$. \\
        \tab + Case 1.1: $a = c$. \\
        \tab \tab It holds that $d(a,c) = 0$. \\
        \tab \tab It holds that $b = c$ \\
        \tab \tab It holds that $d(b,c) = 0$. \\
        \tab \tab It holds that $d(a,c) = 0 \le 0 = d(a,b) + d(b,c)$. \\
        \tab \tab We conclude that $d(a,c) \leq d(a,b) + d(b,c)$. \\
        \tab + Case 1.2: $a \ne c$. \\
        \tab \tab It holds that $d(a,c) = 3$. \\
        \tab \tab Either $b = c$ or $b \ne c$. \\
        \tab \tab \tab ++ Case 1.2.1: $b = c$. \\
        \tab \tab \tab \tab It holds that $d(b,c) = 0$. \\
        \tab \tab \tab \tab It holds that $d(a,c) = 3 \leq 3 = d(a,b) + d(b,c)$. \\
        \tab \tab \tab \tab We conclude that $d(a,c) \leq d(a,b) + d(b,c)$. \\
        \tab \tab \tab ++ Case 1.2.2: $b \ne c$. \\
        \tab \tab \tab \tab It holds that $d(b,c) = 3$. \\
        \tab \tab \tab \tab It holds that $d(a,c) = 3 \leq 6 = d(a,b) + d(b,c)$. \\
        \tab \tab \tab \tab We conclude that $d(a,c) \leq d(a,b) + d(b,c)$. \\
        % \tab \tab We conclude that $d(a,c) \leq d(a,b) + d(b,c)$. \\
        - Case 2: $a \ne b$. \\
        \tab It holds that $d(a,b) = 3$. \\
        \tab Either $a = c$ or $a \ne c$. \\
        \tab + Case 2.1: $a = c$. \\
        \tab \tab It holds that $d(a,c) = 0$. \\
        \tab \tab Either $b = c$ or $b \ne c$. \\
        \tab \tab \tab ++ Case 2.1.1: $b = c$. \\
        \tab \tab \tab \tab It holds that $d(b,c) = 0$. \\
        \tab \tab \tab \tab It holds that $d(a,c) = 0 \leq 3 = d(a,b) + d(b,c)$. \\
        \tab \tab \tab \tab We conclude that $d(a,c) \leq d(a,b) + d(b,c)$. \\
        \tab \tab \tab ++ Case 2.1.2: $b \ne c$. \\
        \tab \tab \tab \tab It holds that $d(b,c) = 3$. \\
        \tab \tab \tab \tab It holds that $d(a,c) = 0 \leq 6 = d(a,b) + d(b,c)$. \\
        \tab \tab \tab \tab We conclude that $d(a,c) \leq d(a,b) + d(b,c)$. \\
        \tab \tab We conclude that $d(a,c) \leq d(a,b) + d(b,c)$. \\
        \tab + Case 2.2: $a \ne c$. \\
        \tab \tab It holds that $d(a,c) = 3$. \\
        \tab \tab Either $b = c$ or $b \ne c$. \\
        \tab \tab \tab ++ Case 2.2.1: $b = c$. \\
        \tab \tab \tab \tab It holds that $d(b,c) = 0$. \\
        \tab \tab \tab \tab It holds that $d(a,c) = 3 \leq 3 = d(a,b) + d(b,c)$. \\
        \tab \tab \tab \tab We conclude that $d(a,c) \leq d(a,b) + d(b,c)$. \\
        \tab \tab \tab ++ Case 2.2.2: $b \ne c$. \\
        \tab \tab \tab \tab It holds that $d(b,c) = 3$. \\
        \tab \tab \tab \tab It holds that $d(a,c) = 3 \leq 6 = d(a,b) + d(b,c)$. \\
        \tab \tab \tab \tab We conclude that $d(a,c) \leq d(a,b) + d(b,c)$. \\
        \tab \tab We conclude that $d(a,c) \leq d(a,b) + d(b,c)$. \\
        We conclude that $d(a,c) \leq d(a,b) + d(b,c)$ for all $a,b,c \in \mathbb{Z}$. \\

        5. Reflexivity: \\
        Take $a \in \mathbb{Z}$. \\
        It holds that $d(a,a) = 0$. \\
        We conclude that $d(a,a) = 0$ for all $a \in \mathbb{Z}$. \\
    \end{proof}
    
    \newpage

    \section{2.7.4}
    \begin{problem}
        Let $(X, \text{dist})$ be a metric space. Define $d: X \times X \to \mathbb{R}$ by
        $$d(x,y) = \sqrt{\text{dist}(x,y)}$$

        Show that $d$ is a distance function on $X$.
    \end{problem}

    \begin{proof}
        To show that $d$ is a distance on $X$, we need to show that $d$ satisfies the following 5 properties:
        \begin{enumerate}
            \item Positivity: $d(x,y) \geq 0$ for all $x,y \in X$.
            \item Non-degeneracy: $d(x,y) = 0$ if and only if $x = y$.
            \item Symmetry: $d(x,y) = d(y,x)$ for all $x,y \in X$.
            \item Triangle inequality: $d(x,z) \leq d(x,y) + d(y,z)$ for all $x,y,z \in X$.
            \item Reflexivity: $d(x,x) = 0$ for all $x \in X$.
        \end{enumerate}

        We show each of these properties in turn.

        1. Positivity: \\
        Take $x \in X$. \\
        Take $y \in X$. \\
        It holds that $d(x,y) = \sqrt{\text{dist}(x,y)}$. \\
        By positivity of dist it holds that $\text{dist}(x,y) \geq 0$. \\
        By positivity of square root it holds that $\sqrt{\text{dist}(x,y)} \geq 0$. \\
        It holds that $d(x,y) = \sqrt{\text{dist}(x,y)} \geq 0$. \\
        We conclude that $d(x,y) \geq 0$ for all $x,y \in X$. \\

        2. Non-degeneracy: \\
        We need to show both directions. \\
        2.1 Forward direction: \\
        Take $x \in X$. \\
        Take $y \in X$. \\
        Assume $d(x,y) = 0$. \\
        It holds that $d(x,y) = \sqrt{\text{dist}(x,y)} = 0$. \\
        It holds that $\text{dist}(x,y) = 0$. \\
        By non-degeneracy of dist it holds that $x = y$. \\
        We conclude that $d(x,y) = 0 \implies x = y$. \\
        2.2 Backward direction: \\
        Take $x \in X$. \\
        Take $y \in X$. \\
        Assume $x = y$. \\
        By non-degeneracy of dist it holds that $\text{dist}(x,y) = 0$. \\
        It holds that $d(x,y) = \sqrt{\text{dist}(x,y)} = 0$. \\
        We conclude that $x = y \implies d(x,y) = 0$. \\
        We conclude that $d(x,y) = 0 \iff x = y$. \\

        3. Symmetry: \\
        Take $x \in X$. \\
        Take $y \in X$. \\
        It holds that $d(x,y) = \sqrt{\text{dist}(x,y)}$. \\
        It holds that $d(y,x) = \sqrt{\text{dist}(y,x)}$. \\
        By symmetry of dist it holds that $\text{dist}(x,y) = \text{dist}(y,x)$. \\
        It holds that $d(x,y) = \sqrt{\text{dist}(x,y)} = \sqrt{\text{dist}(y,x)} = d(y,x)$. \\
        We conclude that $d(x,y) = d(y,x)$ for all $x,y \in X$. \\

        4. Triangle inequality: \\
        Take $x \in X$. \\
        Take $y \in X$. \\
        Take $z \in X$. \\
        It holds that $d(x,y) = \sqrt{\text{dist}(x,y)}$. \\
        It holds that $d(x,z) = \sqrt{\text{dist}(x,z)}$. \\
        It holds that $d(z,y) = \sqrt{\text{dist}(z,y)}$. \\
        By triangle inequality of dist it holds that $\text{dist}(x,z) \leq \text{dist}(x,y) + \text{dist}(y,z)$. \\
        It holds that $d(x,z) = \sqrt{\text{dist}(x,z)} \leq \sqrt{\text{dist}(x,y) + \text{dist}(y,z)}$. \\
        We need to show that $\sqrt{\text{dist}(x,y) + \text{dist}(y,z)} \leq \sqrt{\text{dist}(x,y)} + \sqrt{\text{dist}(y,z)}$. \\
        By positivity of square root it holds that $\sqrt{\text{dist}(x,y) + \text{dist}(y,z)} \le \sqrt{\text{dist}(x,y) + 2\sqrt{\text{dist(x,y)}}\sqrt{\text{dist}(y,z)} + \text{dist}(y,z)}$ \\
        It holds that $\sqrt{\text{dist}(x,y) + \text{dist}(y,z)} \le \sqrt{\left(\sqrt{\text{dist}(x,y)}\right)^2 + 2\sqrt{\text{dist}(x,y)}\sqrt{\text{dist}(y,z)} + \left(\sqrt{\text{dist}(y,z)}\right)^2}$ \\
        It holds that $\sqrt{\text{dist}(x,y) + \text{dist}(y,z)} \le \sqrt{\left( \sqrt{\text{dist}(x,y)} + \sqrt{\text{dist}(y,z)} \right)^2} $ \\
        It holds that $\sqrt{\text{dist}(x,y) + \text{dist}(y,z)} \le  \sqrt{\text{dist}(x,y)} + \sqrt{\text{dist}(y,z)} $ \\
        It holds that $\sqrt{\text{dist}(x,z)} \le \sqrt{\text{dist}(x,y) + \text{dist}(y,z)} \le \sqrt{\text{dist}(x,y)} + \sqrt{\text{dist}(y,z)} $ \\
        It holds that $d(x,z) \le d(x,y) + d(y,z)$ \\
        We conclude that $d(x,z) \le d(x,y) + d(y,z)$ for all $x,y,z \in X$.\\

        5. Reflexivity: \\
        Take $x \in X$. \\
        It holds that $d(x,x) = \sqrt{\text{dist}(x,x)}$. \\
        By reflexivity of dist it holds that $\text{dist}(x,x) = 0$. \\
        It holds that $d(x,x) = \sqrt{\text{dist}(x,x)} = \sqrt{0} = 0$. \\
        We conclude that $d(x,x) = 0$ for all $x \in X$. \\

    \end{proof}

    \newpage

    \section{2.7.5}
    \begin{problem}
        Let $(V, \norm{\cdot})$ be a normed vector space. We say a subset $U \subseteq V$ is convex if
        \begin{align*}
            \text{for all } x,y &\in U \\
            \text{for all } \lambda &\in (0,1) \\
            \lambda x + (1 - \lambda)y &\in U
        \end{align*}

        Let $z \in V$ and $r > 0$. Define $B(z,r) := \{x \in V: \norm{x - z} < r\}$. Show that $B(z,r)$ is convex.
    \end{problem}

    \begin{proof}
        To show that $B(z,r)$ is convex, we need to show that $B(z,r)$ satisfies the following property:
        \begin{center}
            For all $x,y \in B(z,r)$ and for all $\lambda \in (0,1)$, $\lambda x + (1 - \lambda)y \in B(z,r)$. \\
            i.e. $\norm{\lambda x + (1 - \lambda)y - z} < r$
        \end{center}
        Take $x \in B(z,r)$. \\
        Take $y \in B(z,r)$. \\
        Take $\lambda \in (0,1)$. \\
        It holds that $\norm{x - z} < r$. \\
        It holds that $\norm{y - z} < r$. \\
        It holds that $\lambda x + (1 - \lambda)y = \lambda (x - z) + (1 - \lambda)(y - z) + z$. \\
        It suffices to show that $\norm{\lambda (x - z) + (1 - \lambda)(y - z) + z - z} < r$. \\
        By the triangle inequality it holds that $\norm{\lambda(x-z) + (1-\lambda)(y-z)} \le \norm{\lambda(x-z)} + \norm{(1-\lambda)(y-z)}$ \\
        Since $\lambda \in (0,1)$ it holds that $\norm{\lambda(x-z) + (1-\lambda)(y-z)} \le \lambda\norm{x-z} + (1-\lambda)\norm{y-z}$. \\
        It holds that $\norm{\lambda(x-z) + (1-\lambda)(y-z)} \le \lambda r + (1-\lambda)r = r$ \\
        We conclude that $\norm{\lambda (x - z) + (1 - \lambda)(y - z) + z - z} < r$. \\
        We conclude that $B(z,r)$ is convex. \\
    \end{proof}

    \newpage

    \section{3.11.1}
    \begin{problem}
        Show that 
        \begin{align*}
            \exists M &\in \mathbb{R}, \\
            \forall x &\in [0,5], \\
            x &\leq M
        \end{align*}
    \end{problem}

    \begin{proof}
        Choose $M = 5$. \\
        Let $x \in [0,5]$. \\
        It holds that $x \leq 5$. \\
        We conclude that $\exists M \in \mathbb{R}, \forall x \in [0,5], x \leq M$. \\
    \end{proof}

    \section{3.11.2}
    \begin{problem}
        Show that 
        \begin{align*}
            \forall x &\in \mathbb{R}, \\
            \exists y &\in \mathbb{R}, \\
            \forall u &\in \mathbb{R}, \\
            u > 0 &\implies \exists v \in \mathbb{R}, \\
            v > 0 &\land x + u < y + v.
        \end{align*}
    \end{problem}

    \begin{proof}
        Let $x \in \mathbb{R}$. \\
        Choose $y = x$ \\
        Let $u \in \mathbb{R}$. \\
        Assume $u > 0$. \\
        Choose $v = u + 1$. \\
        It holds that $v > 0$. \\
        It holds that $x + u < x + (u + 1) = y + v$. \\
        We conclude that $\forall x \in \mathbb{R}, \exists y \in \mathbb{R}, \forall u \in \mathbb{R}, [u > 0 \implies \exists v \in \mathbb{R} [v > 0 \land x + u < y + v]]$. \\
    \end{proof}
\end{document}
