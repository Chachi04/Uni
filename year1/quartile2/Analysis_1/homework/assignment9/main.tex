\documentclass{assignment}

\title{Assignment 9}
\author{Group 1}

\begin{document}

\maketitle

\section{Exercise 12.4.4}
\begin{problem}
    Let $(X, \dist)$ be a metric space and let $K \subseteq X$ be a compact subset. Let $a: \N \to X$ be a sequence with values in $X$, such that
    \begin{myCenter}
        \tab{for all $N \in \N$,}
        \tab{there exists $\ell \ge N$,}
        \tab{$a_\ell \in K$ } 
    \end{myCenter}
    \vspace*{-\dimexpr\parskip+2\baselineskip}\hfill$(*)$
    \begin{enumerate}
        \item Use $(*)$ to inductively define an index sequence $n: \N \to \N$ such that for every $k \in \N$, $a_{n_k} \in K$.
        \item Use the fact that $K$ is compact to show that there is a point $p \in K$ and a subsequence of $a: \N \to X$ converging to $p$.
    \end{enumerate}
\end{problem}
\begin{proof}
    1. We define an index sequence $n: \N \to \N$ inductively as follows:
    \textbf{Base:} \\
    Choose $N = 0$ in $(*)$, then there exists $\ell \ge 0$, such that $a_\ell \in K$.\\
    Obtain such $\ell$ \\
    Set $n_0 = \ell$

    \textbf{Inductive step:} \\
    Suppose we have defined $n_0, n_1, \dots, n_k$ for some $k \in \N$, such that for all $0 \le i \le k$, $a_{n_i} \in K$.\\
    Choose $N = n_k + 1$ in $(*)$, then there exists $\ell \ge N$, such that $a_\ell \in K$.\\
    Obtain such $\ell$ \\
    Set $n_{k+1} = \ell$ \\
    Then it holds that $n_k < n_{k+1}$ and $a_{n_{k+1}} \in K$.

    2. Since $K$ is compact, it holds that 
    \begin{myCenter}
        \tab{for all sequences $b: \N \to \K$,}
        \tab{there exists a subsequence $b': \N \to \K$,}
        \tab{such that $b'$ converges to a point $p \in K$}
    \end{myCenter}
    \vspace*{-\dimexpr\parskip+1\baselineskip}\hfill$(**)$ \\
    Choose $b = a \circ n$ in $(**)$, then there exists a subsequence $b': \N \to \K$, such that $b'$ converges to a point $p \in K$.\\
    I.e. there exists an index sequence $m: \N \to \N$, such that $b \circ m$ converges to $p$.\\
    The the subsequence $a \circ n \circ m$ converges to $p \in K$.\\
\end{proof}

\section{Exercise 12.4.5}
\begin{problem}
    Consider the sets
    $$A := \{(x_1,x_2) \in \R^2 \mid x_1-x_2 = 1\}$$
    and 
    $$B := \{(x_1,x_2) \in \R^2 \mid (x_1)^2 + (x_2)^2 \le 1\}$$
    Prove that the set $A \cap B$ is compact (as a subset of the normed vector space $(\R^2, \norm{\cdot}_2)$).
\end{problem}
\begin{proof}
    Note: $B$ is the closed unit ball in $(\R^2, \norm{\cdot}_2)$. \\
    % Since $\norm[(x_1,x_2)] = \sqrt{x_1^2 + x_2^2} \le x_1^2+x_2^2 \le 1$
    Since $A \cap B \subseteq (\R^2, \norm{\cdot}_2)$ by the Heine-Borel Theorem, it suffices to show that $A \cap B$ is closed and bounded.\\
    Note: $A \cap B = \{(x_1,x_2) \mid x_1-x_2=1 \land x_1^2+x_2^2\le1\}$ which is the closed line segment from $(0,-1)$ to $(1,0)$.\\

    \textbf{Closed:} \\
    By the sequence characterization of closedness, it suffices to show that 
    \begin{myCenter}
        \tab{for all sequences $a: \N \to A \cap B$,}
        \tab{if $a$ converges to a point $p \in \R^2$,}
        \tab{then $p \in A \cap B$}
    \end{myCenter}
    Let $a: \N \to A \cap B$ be a sequence, such that $a$ converges to a point $p \in \R^2$. \\
    We need to show that $p \in A \cap B$. \\
    Since $a_n \in A \cap B$ for all $n \in \N$, it holds that $p = \lim_{n\to\infty}a_n \in A \cap B$. \\
    Hence $A \cap B$ is closed.

    \textbf{Bounded:} \\
    Need to show that $A \cap B$ is bounded, i.e. 
    \begin{myCenter}
        \tab{there exists $q \in A \cap B$,}
        \tab{there exists $M > 0$,}
        \tab{for all $p \in A \cap B$,}
        \tab{$\norm{p-q} \le M$}
    \end{myCenter}
    Choose $q = (1,0)$, then $q \in A \cap B$, \\
    Choose $M = 2$, then $M > 0$, \\
    Let $p = (p_1, p_2) \in A \cap B$, \\
    Need to show that $\norm{p-q} \le M$ \\
    $\norm{p-q} = \sqrt{(p_1-1)^2 + p_2^2} = \sqrt{p_1^2+p_2^2-2p_1+1} \le \sqrt{2 - 2p_1} \le \sqrt{2} < 2 = M$

    Since $A \cap B$ is closed and bounded, by the Heine-Borel Theorem, $A \cap B$ is compact.
\end{proof}

\newpage

\section{Exercise 13.11.1}
\begin{problem}
    Let $(X, \dist_X) := (\R^2, \dist_{\norm{\cdot}_2})$ and $(Y, \dist_Y):=(\R,\dist_\R)$. Let $D = B(0,1) \subseteq \R^2$.
    Let $f: D \to \R$ be defined as
    $$f(x) := \begin{cases}
        x_1^2+x_2^2 & \text{if } x \ne (0,0) \\
        185 & \text{if } x = (0,0).
    \end{cases}$$
    Show that
    $$\lim_{x\to(0,0)}f(x)=0$$
\end{problem}
\begin{proof}
    \textbf{Method 1: ($\epsilon-\delta$ proof)} \\
    We need to show that 
    \begin{myCenter}
        \tab{for all $\epsilon > 0$,}
        \tab{there exists $\delta > 0$,}
        \tab{for all $x \in D$,}
        \tab{$0 < \norm{x - (0,0)} < \delta \implies |f(x) - 0| < \epsilon$}
    \end{myCenter}
    Let $\epsilon > 0$, \\
    Choose $\delta = \sqrt{\epsilon}$, \\
    Let $x \in D$, \\
    Assume $0 < \norm{x - (0,0)} < \delta$, i.e. $0 < \sqrt{x_1^2 + x_2^2} < \delta$ \\
    Then $x \ne (0,0)$ and $f(x) = x_1^2 + x_2^2$ \\
    Need to show that $|f(x) - 0| < \epsilon$ \\
    Indeed $|f(x) - 0| = |x_1^2+x_2^2| < \delta^2 = \epsilon$\\
    Therefore,
    $$\lim_{x \to (0, 0)} f(x) = 0$$

    \textbf{Method 2: (Sequence characterization proof)} \\
    By the sequence characterization of limits, it suffices to show that, 
    \begin{myCenter}
        \tab{for all sequences $(x^n)$ in $D \setminus \{(0, 0)\}$ converging to $(0, 0)$,}
        \tab{$\limn f(x^n) = 0$}
    \end{myCenter}
    Let $(x^n)$ be a sequence in $D \setminus \{(0, 0)\}$ converging to $(0, 0)$. \\
    It holds that $\limn x^n = (0,0)$. \\
    Since $x^n \ne (0,0)$ for all $n \in \N$, we know $f(x^n) = (x^n)_1^2 + (x^n)_2^2$ for all $n \in \N$. \\
    Hence $\limn f(x^n) = \limn (x^n)_1^2 + (x^n)_2^2 = 0^2 + 0^2 = 0$. \\
    Since $\limn f(x^n) = 0$ for all $(x^n)$ in $D \setminus \{(0, 0)\}$ converging to $(0, 0)$, $$\lim_{x \to (0, 0)} f(x) = 0$$

    %     Let $(x_n)$ be a sequence in $D \setminus \{(0, 0)\}$ converging to $(0, 0)$. \\
    % It holds that $\limn x_n = (0,0)$. \\
    % Since $x_n \ne (0,0)$ for all $n \in \N$, we know $f(x_n) = (x_n)_1^2 + (x_n)_2^2$ for all $n \in \N$. \\
    % Hence $\limn f(x_n) = \limn (x_n)_1^2 + (x_n)_2^2 = 0^2 + 0^2 = 0$. \\
    % Therefore,
    % $$\lim_{x \to (0, 0)} f(x) = 0$$

    % For all sequences $x_n$ in $D \setminus \{(0, 0)\}$ converging to $(0, 0)$,
    % we know $\limn x_n = (0, 0)$ Since $x_n$ is never $(0, 0)$ no matter the
    % $n$, we know $f(x_n) = (x_n)_1^2 + (x_n)_2^2$ for all $n \in \N$. Hence
    % $\limn f(x_n) = \limn (x_n)_1^2 + (x_n)_2^2 = 0^2 + 0^2 = 0$. Since $\limn
    % f(x_n) = 0$ for all $x_n$ in $D \setminus \{(0, 0)\}$ converging to $(0,
    % 0)$, $$\lim_{x \to (0, 0)} f(x) = 0$$

\end{proof}

\section{Exercise 13.11.2}
\begin{problem}
    Consider the function $f: D \to \R$ defined by
    $$f(x) = x \quad \text{for } x \in \R$$
    where $D = \R$. 

    Prove that for every $a \in D$, the function $f$ is continuous at $a$.
\end{problem}
\begin{proof}
    We need to show that for every $a \in D$, $f$ is continuous at $a$. \\ 
    Take $a \in D$. \\
    By the sequence chracterization of continuity, it suffices to show that
    \begin{myCenter}
        \tab{for all sequences $x_n$ in $D$ converging to $a \in D$,}
        \tab{$\limn f(x_n) = f(a)$}
    \end{myCenter}
    Let $x_n: \N \to D$ be a sequence converging to $a \in D$. \\
    It holds that $f(x_n) = x_n$ for all $n \in \N$. \\
    Therefore, $\limn f(x_n) = \limn x_n = a = f(a)$. \\
    Thus, $f$ is continuous at $a$.

    %    Take $a \in D$, then for every sequence $x_n : \N \to D$ converging to $a$, we know $f(x_n) = x_n \forall n \in \N$.
    % Thus for all sequences $x_n$ in $D$ converging to $a$ we have $\limn f(x_n) = \limn x_n = a = f(a)$ which proves continuity of $f$ on $D$.
\end{proof}

\end{document}
