%%%%%%%%%%%%%%%%%%%%%%%%%%%%% Define Article %%%%%%%%%%%%%%%%%%%%%%%%%%%%%%%%%%
\documentclass[problem]{classnotes}
%%%%%%%%%%%%%%%%%%%%%%%%%%%%%%%%%%%%%%%%%%%%%%%%%%%%%%%%%%%%%%%%%%%%%%%%%%%%%%%


% \usepackage{assignment}
% \usepackage{environments}

\newcommand{\blue}{\text{\color{blue}{blue}}}
\newcommand{\orange}{\text{\color{orange}{orange}}}

%%%%%%%%%%%%%%%%%%%%%%%%%%%%%%% Title & Author %%%%%%%%%%%%%%%%%%%%%%%%%%%%%%%%
\title{Assignment 7}
\author{Jiaqi Wang}
%%%%%%%%%%%%%%%%%%%%%%%%%%%%%%%%%%%%%%%%%%%%%%%%%%%%%%%%%%%%%%%%%%%%%%%%%%%%%%%

\begin{document}
    \maketitle
    \section{Exercise 10.7.6}
    \begin{problem}
        Let $P: \N \to \{\blue, \orange\}$ be a sequence taking values in the set with exactly the two elements \blue\ and \orange.
        Assume that
        \begin{center}
            \parbox{\linewidth}{
                \leftskip=0.4\linewidth
                for all $k \in \N$, \\
                \hspace*{1em}there exists $m \ge k$, \\
                \hspace*{2em}$P_m = \blue$. \hfill (*)

            }
        \end{center}

        Show that there is a subsequence of $P: \N \to \{\blue, \orange\}$ for which every term equal \blue.%$by going through the following 
        % \begin{enumerate}
        %     \item Inductively define an index sequence $n: \N \to \N$ such that for all $k \in \N$, $P_{n_k} = \blue$.
        %         \begin{enumerate}
        %             \item First define $n_0 \in \N$ appropriately and prove that $CP_{n_0$
        %         \end{enumerate}
        % \end{enumerate}
    \end{problem}
    \begin{proof}
        We construct a index sequence $n: \N \to \N$ inductively such that for all $\ell \in \N$, $P_{n_\ell} = \blue$ and $n_\ell < n_{\ell+1}$.\\

        \noindent\textbf{Base step:} \\
        Choose $k = 0$ in (*), then there exists $m \ge 0$, such that $P_m = \blue$. \\
        Obtain such $m$. \\
        Set $n_0 = m$. \\

        \noindent\textbf{Inductive step:} \\
        Suppose we have defined $n_0,\dots,n_\ell$ for some $\ell \in \N$ \\
        such that $P_{n_0} = \blue,\dots,P_{n_\ell} = \blue$ and $n_0 < \dots < n_\ell$. \\
        Choose $k = n_\ell + 1$ in (*), then there exists $m \ge n_\ell + 1 > n_\ell$ such that $P_m = \blue$. \\
        Obtain such $m$. \\
        Choose $n_{\ell+1} = m$. \\
        Then $P_{n_{\ell+1}} = \blue$ and $n_{\ell+1} > n_\ell$. \\

        By induction, we have defined $n: \N \to \N$ such that for all $\ell \in \N$, $P_{n_\ell} = \blue$ and $n_\ell < n_{\ell+1}$. \\
        Then $P_{n_\ell} = \blue$ for all $\ell \in \N$. \\
    \end{proof}

    \newpage

    \section{Exercise 11.6.1}
    \begin{problem}
        Let $(V, \norm{\cdot})$ be a normed linear space and let $A$ be the closed ball of radius $1$ around the origin, i.e.
        $$A:=\{v \in V \mid \norm{v} \le 1\}.$$
        Show that the set $A$ is closed
    \end{problem}
    \begin{proof}
        Need to show that $A$ is closed, i.e. $V \setminus A$ is open. \\
        I.e. $\overline{A} = V \setminus A$, defined by
        $$\overline{A} := \{v \in V \mid \norm{v} > 1\}$$
        is open. \\
        We need to show that for all $q \in \overline{A}$, $q$ is an interior point of $\overline{A}$. \\
        I.e. for all $q \in \overline{A}$, there exists $r > 0$ such that $B(q,r) \subseteq \overline{A}$. \\
        Let $q \in \overline{A}$. \\
        Then $\norm{q} > 1$. \\
        Choose $r = \frac{\norm{q}-1}{2} > 0$. \\
        Need to show that $B(q,r) \subseteq \overline{A}$, i.e. for all $p \in B(q,r)$, $p \in \overline{A}$. \\
        Let $p \in B(q,r)$. \\
        % Then $\norm{p-q} < r = \frac{\norm{q}-1}{2}$. \\
        Need to show that $\norm{p} > 1$ \\
        Note: $\norm{p} > \inf \{\norm{v} \mid v \in B(q,r)\}= \norm{q}-r = \norm{q}-\frac{\norm{q}-1}{2} = \frac{\norm{q}+1}{2} > 1$\\
        Since $\norm{p} > 1$ \\
        Then $p \in \overline{A}$. \\
        So $B(q,r) \subseteq \overline{A}$. \\
        And $\overline{A}$ is open. \\
        Therefore $A$ is closed. \\


        % By the sequence characterization of closedness it suffices to show that for all sequences $a: \N \to A$ converging to
        % some $q \in V$, we have $q \in A$. \\
        % Let $a: \N \to A$ be a sequence converging to some $q \in V$. \\
        % Then 
        % \begin{myCenter}
        %     \tabline{for all $\epsilon_0 > 0$}
        %     \tabline{there exists $N_0 \in \N$}
        %     \tabline{for all $n \ge N$}
        %     \tabline{$\norm{a_n - q} < \epsilon_0$ \hfill (*)}
        % \end{myCenter}
        % Choose $\epsilon_0 = $ in (*), then there exists $N_0 \in \N$ such that for all $n \ge N$, we have $\norm{a_n - q} < \epsilon_0$. \\
        % Obtain such $N_0$

    \end{proof}

    \section{Exercise 11.6.2}
    \begin{problem}
        Show that the interval $[0,1)$ is neither open nor closed (seen as a subset of the normed linear space $(\R, |\cdot|)$).
    \end{problem}

    \begin{proof}
        We first show that $[0,1)$ is not open. \\
        i.e. there exists $q \in [0,1)$ such that for all $r > 0$, $B(q,r) \not\subseteq [0,1)$. \\
        Choose $q = 0$, then $q \in [0,1)$ \\
        Let $r > 0$, \\
        We need to show that $B(q,r) \not\subseteq [0,1)$, i.e. there exists $p \in B(q,r)$ such that $p \not\in [0,1)$. \\
        Let $p = -\frac{r}{2}$. \\
        Then $p \in B(q,r)$. \\
        But $p \not\in [0,1)$. \\
        So $B(q,r) \not\subseteq [0,1)$. \\
        Therefore $[0,1)$ is not open. \\

        Now we show that $[0,1)$ is not closed. \\
        i.e. there exists $q \in \R \setminus [0,1)$ such that for all $r > 0$, $B(q,r) \not\subseteq \R \setminus [0,1)$. \\
        Choose $q = 1$, then $q \in \R \setminus [0,1)$ \\
        Let $r > 0$, \\
        Need to show that $B(q,r) \not\subseteq \R \setminus [0,1)$, i.e. there exists $p \in B(q,r)$ such that $p \not\in \R \setminus [0,1)$. \\
        Let $p = 1 + \frac{r}{2}$. \\
        Then $p \in B(q,r)$. \\
        But $p \not\in \R \setminus [0,1)$. \\
        So $B(q,r) \not\subseteq \R \setminus [0,1)$. \\
        Therefore $[0,1)$ is not closed. \\
    \end{proof}

    \section{Exercise 11.6.4}
    \begin{problem}
        Consider the following line $\R^2$
        $$L:=\{(x,y) \in \R^2 \mid x+2y = 1\}.$$
        Show that $L$ is a closed subset of $\R^2$ and that $L$ is complete.
    \end{problem}

    \begin{proof}
        First we show that $L$ is closed. \\
        I.e. $\overline{L} := \R^2 \setminus L = \{(x,y) \in \R^2 \mid x+2y \ne 1\}$ is open. \\
        We need to show that for all $q \in \overline{L}$, $q$ is an interior point of $\overline{L}$. \\
        I.e. for all $q \in \overline{L}$, there exists $r > 0$ such that $B(q,r) \subseteq \overline{L}$. \\
        Let $q \in \overline{L}$. \\
        Choose $r = |\mathcal{P}(q) - q|$, where $\mathcal{P}(q)$ is the orthogonal projection of $q$ onto $L$. \\
        Then $r > 0$. \\
        and $B(q,r) \subseteq \overline{L}$, i.e. for all $p \in B(q,r)$, $p \in \overline{L}$. \\
        Thus $L$ is closed. \\
        
        Now we show that $L$ is complete. \\
        By proposition 11.4.3, we have that $\R^d$ is complete, in particular $\R^2$ \\
        Since $L$ is a closed subset of $\R^2$, by proposition 11.4.5, we have that $L$ is complete. \\
        % Then $q \not\in L$. \\
        % Then $q \not\in L$, i.e. $q \not\in \{(x,y) \in \R^2 \mid x+2y = 1\}$. \\
        % Then $q \not\in \{(x,y) \in \R^2 \mid x+2y = 1\}$, i.e. $q \not\in \{(x,y) \in \R^2 \mid x+2y = 1\}$. \\
        % Then $q \not\in \{(x,y) \in \R^2 \mid x+2y = 1\}$, i.e. $q \not\in \{(x,y) \in \R^2 \mid x+2y = 1\}$. \\
        % Then $q \not\in \{(x,y) \in \R^2 \mid x+2y = 1\}$, i.e. $q \not\in \{(x,y) \in \R^2 \mid x+2y = 1\}$. \\
        % Then $q \not\in \{(x,y) \in \R^2 \mid x+2y = 1\}$, i.e. $q \not\in \{(x,y) \in \R^2 \mid x+2y = 1\}$. \\
        % Then $q \not\in \{(x,y) \in \R^2 \mid x+2y = 1\}$, i.e. $q \not\in \{(x,y) \in \R^2 \mid x+2y = 1\}$. \\
        % Then $q \not\in \{(x,y) \in \R^2 \mid x+2y = 1\}$, i.e. $q \not\in \{(x,y) \in \R^2 \mid x+2y = 1\}$. \\
        % Then $q \not\in \{(x,y) \in \R^2 \mid x+2y = 1\}$, i.e. $q \not\in \{(x,y) \in \

        % By the sequence characterization of closedness, it suffices to show that for all sequences $a: \N \to L$ converging to some $q \in \R^2$, we have $q \in L$. \\
        % Let $a:\N \to L$ be a sequence converging to some $q \in \R^2$. \\
        % Then
        % \begin{myCenter}
        %     \tabline{for all $\epsilon_0 > 0$}
        %     \tabline{there exists $N_0 \in \N$}
        %     \tabline{for all $n \ge N_0$}
        %     \tabline{$\norm{a_n - q} < \epsilon_0$ \hfill (*)}
        % \end{myCenter}
        % and we know that the sequence $a: \N \to L$ is convergent if and only if the component sequences of $a$ converge to the corresponding components of $q$. \\
        % so we have
        % $$\limn a_1^{(n)} = q_1 \quad \text{and} \quad \limn a_2^{(n)} = q_2.$$
        

        % Choose $\epsilon_0 = \frac{1}{2}$ in (*), then there exists $N_0 \in \N$ such that for all $n \ge N_0$, we have $\norm{a_n - q} < \frac{1}{2}$. \\
        % Obtain such $N_0$. \\
        % Let $n \ge N_0$. \\
        % Then $\norm{a_n - q} < \frac{1}{2}$. \\
        % Then $a_n \in B(q,\frac{1}{2})$. \\
        % Since $a_n \in L$, we have $a_n \in L \cap B(q,\frac{1}{2})$. \\
        % Since $L \cap B(q,\frac{1}{2})$ is open, there exists $r > 0$ such that $B(a_n,r) \subseteq L \cap B(q,\frac{1}{2})$. \\
        % Choose $r = \frac{1}{2}$. \\
        % Then $B(a_n,r) \subseteq L \cap B(q,\frac{1}{2})$. \\
        % Then $B(a_n,r) \subseteq L$. \\
        % Then $a_n$ is an interior point of $L$. \\
        % Then $a_n \in \overline{L}$. \\
        % Since $a: \N \to L$, we have $a_n \in L$ for all $n \in \N$. \\
        % Then $a_n \in L \cap \overline{L}$. \\
        % Then $a_n \in L \cap \overline{L} \cap B(q,\frac{1}{2})$. \\
        % Then $a_n \in L \cap \overline{L} \cap B(q,\frac{1}{2}) \subseteq L \cap B(q,\frac{1}{2})$. \\
        % Then $a_n \in L \cap B(q,\frac{1}{2})$. \\
        % Then $a_n \in B(q,\frac{1}{2})$. \\
        % Then $\norm{a_n - q} < \frac{1}{2}$. \\
        % Then $\norm{a_n - q} < \epsilon_0$. \\
        % Then $q \in \overline{L}$. \\
        % Then $q \in L \cap \overline{L}$. \\
        % Then $q \in L$. \\
        % Therefore $L$ is closed. \\
    \end{proof}

\end{document}
