\section{Recursion and Induction}

\subsection{Recursion}
A \emph{recursive definition} tells us how to build objects by using ones we have
already built. Let us start with some examples of some common functions from
$ \mathbb{N} $ to $ \mathbb{N} $ which can be defined recursively.

\begin{example}[Factorial]
    The function $ f(n) = n! $
\end{example}

\begin{example}[Sum]
    The sum $ 1 + 2 + 3 + \dots + n $, also written as $ \displaystyle\sum_{i=1}^{n}i $
\end{example}

\begin{example}[Fibonachi sequence]
    \begin{align}
    F(1) &= 1 \\
    F(2) &= 1 \\
    F(n+2) &= F(n+1) + F(n)
    \end{align}
\end{example}

In the examples above we see that for a recursively defined function $f$ we need
two ingredients:
\begin{itemize}
    \item a \emph{base} part, where we define the function value $f(n)$ for some
    small values of $n$ like 0 or 1.
    \item a \emph{recursive} part in which we explain how to compute the function
    in $n$ with the help of the values for integers smaller than $n$.
\end{itemize}

Of course, we do not have to restrict our attention to functions with domain
$\mathbb{N}$. Recursion can be used at several places.

\begin{example}
Let $S$ be the subset of $\mathbb{Z}$ defined by:

$3 \in S$;

if $x,y \in S$ then also $-x$ and $x+y \in S$.

Then $S$ consists of all the multiples of 3. Indeed, if $n = 3m$ for some
$m \in N$, then $n = (\dots(3 + 3) + 3) + \dots + 3$, and hence is in $S$. But
then also $-3m \in S$. Thus $S$ contains all multiples of 3. On the other hand,
if $S$ contains only multiples of 3, then in the next step of the recursion,
only multiples of 3 are added to $S$. So, since initially $S$ contains only 3,
$S$ contains only multiples of 3.
\end{example}

\subsection{Natural induction}
\begin{principle}[Principle of Natural Induction]
    Suppose $ P(n) $ is a predicate for $n \in \mathbb{Z}$. Let $ b \in \mathbb{Z} $.
    If the following holds:
    \begin{itemize}
        \item P(b) is true:
        \item for all $ k \in \mathbb{Z} $, $ k \ge b $ we have that $P(k)$ implies
        $ P(k+1) $
    \end{itemize}
    Then $ P(n) $ is true for all $k \ge b$
\end{principle}

\subsection{Strong induction and Minimal counter examples}

\begin{principle}[Principle of Strong Induction]
    Suppose $ P(n) $ is a predicate for $n \in \mathbb{Z}$. Let $ b \in \mathbb{Z} $.
    If the following holds:
    \begin{itemize}
        \item P(b) is true:
        \item for all $ k \in \mathbb{Z} $, $ k \ge b $ we have that
        $P(b), P(b+1), \dots, P(k)$ together imply $P(k+1)$.
    \end{itemize}
    Then $ P(n) $ is true for all $k \ge b$
\end{principle}

\begin{principle}[Minimal counter example]
    Let $P(n)$ be a predicate for all $n \in \mathbb{Z}$. Let $b \in \mathbb{Z}$.
    If the statement that $P(n)$ is true for all $n \in \mathbb{Z},n \ge b$, is
    not true, then there is a minimal counter example. That means, there is an
    $m \in \mathbb{Z}$, $m \ge b$ with $P(m)$ false and $P(n)$ true for all
    $n \in \mathbb{N}$ with $b \le n < m$.

\end{principle}


\subsection{Structural induction}

\begin{principle}[Structural Induction]
    If a structure of data types is defined recursively, then we can use this
    recursive definition to derive properties by induction.

    In particular,
    \begin{itemize}
        \item if all basic elements of a recursively defined structure satisfy some
        property $P$
        \item and if newly constructed elements satisfy $P$, assuming the elements
        used in the construction already satisfy $P$,
    \end{itemize}
    then all elements in the structure satisfy $P$.
\end{principle}

\begin{principle}[The Principle of Induction on a well founded order]
    Let $ (P, \sqsubseteq) $ be a well founded order. Suppose $ Q(x) $ is a
    predicate for all $ x \in P $ satisfying:
    \begin{itemize}
        \item $ Q(x) $ is true for all minimal elements $b \in P$.
        \item If $ x \in P $ and $ Q(y) $ is true for all $ y \in P $ with
        $ y \sqsubseteq x $, but $ u \ne x $, then $P(x)$ holds.
    \end{itemize}
    Then $ Q(x) $ holds for all $ x \in P $.
\end{principle}