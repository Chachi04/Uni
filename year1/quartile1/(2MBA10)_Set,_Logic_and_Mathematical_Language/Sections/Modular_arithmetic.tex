\section{Modular Arithmetic}

\subsection{Arthimetic modulo n}
\begin{definition}
    Let $n$ be an integer. On the set $ \mathbb{Z} $ of integers we define
    the relation \emph{congruence modulo n} as follows: $a$ and $b$ are
    \emph{congruent modulo n} if and only if $ n \mid a - b $.

    We write $ a \equiv b \pmod{n} $ to denote that $a$ and $b$ are congruent
    modulo $n$.
\end{definition}

\begin{example}
    If $a =342, b=241$, and $n=17$, then $a$ is not congruent to $b$ modulo $n$.
\end{example}

\begin{proposition}
    Let $n$ be an integer. The relation congruence modulo $n$ is an equivalence
    relation.

    For nonzero $n$, there are exactly $n$ distinct equivalence classes

    The set of equivalence classes of $ \mathbb{Z} $ modulo $n$ is denoted
    by $ \mathbb{Z}/n\mathbb{Z} $.
\end{proposition}

\begin{example}
    The relation modulo 2 partitions the integers into two clases, the even
    numbers and the odd numbers.
\end{example}

\begin{theorem}[Addition and Multiplication]
    On $ \modset{n} $ we define two so-called binary operations, an
    \emph{addition} and a \emph{multiplication}, by:
    \begin{itemize}
        \item Addition: $ x \pmod{n} + y \pmod{n} = x + y \pmod{n} $
        \item Multiplication: $ x \pmod{n} \cdot y \pmod{n} = x \cdot y \pmod{n} $
    \end{itemize}

    Both operations are well defined.
\end{theorem}

\begin{proposition}[Propoerties of Modular Arithmetic]
    Let $n$ be an integer bnigger than 1. For all integers $a,b,c$ we have
    the following equalities.
    \begin{itemize}
        \item Commutativity of addition:
        $$ a + b = b + a \pmod{n} $$
        \item Commutativity of multiplication:
        $$ a \cdot b = b \cdot a \pmod{n} $$
        \item Associativity of additiono:
        $$ (a + b) + c = a + (b + c) \pmod{n} $$
        \item Associativity of multiplication:
        $$ (a \cdot b) \cdot c = a \cdot (b \cdot c) \pmod{n} $$
        \item Distributivity of multiplication over addition:
        $$ a \cdot (b + c) = a \cdot b + a \cdot c \pmod{n} $$
    \end{itemize}
\end{proposition}

\subsection{Invertible elements and zero divisors}
\begin{definition}
    An element $ a \in \modset{n} $ is called \emph{invertible} if there is
    an element $b$, called \emph{inverse} of $a$, such that $ a \cdot b = 1 $.

    Of $a$ is invertible, its inverse will be denoted by $ a\inv $.

    The set of all invertible elements in $\modset{n}$ will be denoted by
    $ \modset{n}^\times $. This set is also called the \emph{multiplicative
    group} of $ \modset{n} $.
\end{definition}

\begin{proposition}[Uniqueness of the Inverse]
    Let $ n > 1 $. If an element $a \in \modset{n}$ is invertible, then its
    inverse is unique.
\end{proposition}

In $ \mathbb{Z} $ division is not always possible. Some nonzero elemetns do
have an inverse, others don't. The following theorem tells us precisely
which elements of $ \modset{n} $ have an inverse.

\begin{theorem}[Characterization of Modular Invertibility]
    Let $ n>1 $ and $ a \in \mathbb{Z} $
    \begin{enumerate}[label=(\alph*)]
        \item The class $a \pmod{n}$ in $ \mathbb{Z}/n\mathbb{Z} $ has a multiplicative inverse if and only if $ \gcd(a,n) = 1 $
        \item If $a$ and $n$ are relatively prime, then the inverse of $ a \pmod{n} $ is the class $ \egcd(a,n)_2 \pmod{n} $
        \item In $ \mathbb{Z}/n\mathbb{Z} $, every class distinct from 0 has an inverse if and only if $n$ is prime.
    \end{enumerate}
\end{theorem}

\begin{example}
    The invertible elements in $ \modset{2^n} $ are the classes $ x \pmod{2^n} $
    for which $x$ is an ood integer.
\end{example}

An arithmetical system such as $ \modset{n} $ with $p$ prime, in which every
element not equal to 0 has a multiplicative inverse, is called a \emph{field},
just like $ \mathbb{Q}, \mathbb{R}, \mathbb{C} $.

Besides invertible elements in $ \modset{n} $, which can be viewed as divisors
of 1, one can also consider the divisors of 0.

\begin{definition}[Zero Divisor]
    An element $ a \in \modset{n} $ not equal to 0 is called a \emph{zero
    divisor} if there is a nonzero element $b$ such that $a \cdot b = 0$.
\end{definition}

The following theorem shows which elements of $ \modset{n} $ are zero divisors.
They turn out to be those nonzero elements that are not invertible.

\begin{theorem}[Zero Divisor Characterization]
    Let $ n>1 $ and $ n \in \mathbb{Z} $
    \begin{enumerate}
        \item The class $ a \pmod{n} $ in $ \mathbb{Z}/n\mathbb{Z} $ is a zero divisor if and only if $ \gcd(a,n) > 1 $ and $ a \pmod{n} $ is nonzero.
        \item The residue ring $ \mathbb{Z}/n\mathbb{Z} $ has no zero divisors if and only if $n$ is prime.
    \end{enumerate}
\end{theorem}

Let $n$ be an integer. Inside $ \modset{n} $, we can distinguish the set of
invertible elements and the set of zero divisors. The set of invertible
elements is closed under multiplication, the set of zero divisors together
with 0 is even closed under multiplication by arbitrary elements.

\begin{lemma}
    Let $n$ be an integer with $n > 1$.
    \begin{enumerate}
        \item If $a$ and $b$ are elements in $\modset{n}^\times$, then their
        product $a \cdot b$ is invertible and therefore also in
        $ \modset{n}^\times $. The inverse of $a\cdot b$ is given by
        $ b\inv\cdot a\inv$.
        \item If $a$ is a zero divisor in $ \modset{n} $ and $b$ is an item
        arbitrary element, then $a \cdot b$ is either 0 or a zero divisor.
    \end{enumerate}
\end{lemma}

\subsection{Linear congruence}

\begin{algorithm}[Linear Congruence]
    % \KwIn{integers}
    % \KwOut{the set }
    % \begin{itemize}
    %     \item Input:
    %     \item Output:
    % \end{itemize}
\end{algorithm}

\begin{remark}
    There are exactly $ \gcd(a,n) $ distict solutions.
\end{remark}

\begin{example}
In order to find all solutions to the congruence $ 24x \equiv 12 \pmod{15} $
we first compute the $ \gcd(24,15) $.Using the Extended Euclidean Algorithm
we find
$$ \gcd(24,15) = 3 = 2 \cdot 24 - 3 \cdot 15 $$

Now 3 divide 12, so the solution set is
$$ \{2 \cdot 12 + k \cdot 15 \mid k \in \mathbb{Z}\} $$
\end{example}

Instead of using the algorithm, we can also use the expression of the gcd as
a linear combination of 24 and 15 to argue what the solution is. To this end,
multiply both sides of the equality $ 3 = 2 \cdot 24 - 3 \cdot 15 $ by 4. This
gives $ 12 = 8 \cdot 24 - 12 \cdot 15 $.

So, a solution of the confgurence is $ x = 8 \pmod{15} $.

We extend the study of a single congruence to a method for solvin special
systems of congruences.

\begin{theorem}[Chinese Remainder Theorem]
    Suppose that $ n_1,\dots,n_k $ are pairwise coprime integers. Then for
    all integers $ a_1,\dots,a_k $ the system of linear congruences
    $$ x \equiv a_1 \pmod{n_i} $$
    with $ i \in  \{1,\dots,k\} $ has solution.

    Indeed, the integer
    $$ x = \sum_{i=1}^{k}a_i \cdot y_i \cdot \frac{n}{n_i} $$
    where
    $$ n = \prod_{i=1}^kn_i $$
    and for each $i$ we have
    $$ y_i = \egcd\left(\frac{n}{n_i},n_i\right)_3 $$
    satisfies all congruences.

    Any two solutions to the system of congruences are congruent modulo the
    product $ \displaystyle\prod_{i=1}^{k}n_i $.
\end{theorem}


\subsection{The theorems of Fermat and Euler}
Let $p$ be a prime. Consider $ \modset{p} $, the set of equivalence classes
of $ \mathbb{Z} $ modulo $p$. In $ \modset{p} $ we can add, subtract, multiply
and divide by elemnts which are not 0. Moreover, it contains no zero divisors.

\begin{theorem}[Fermat's Little Theorem]
    Let $p$ be a prime. For every integer $a$ we have
    $$ a^p \equiv a \pmod{p} $$
    In particular, if $a$ is not in $ 0 \pmod{p} $ then
    $$ a^{p-1} \equiv 1 \pmod{p} $$
\end{theorem}

\begin{example}
    The integer $ 1234^1234 - 2 $ is divisible by 7.

    Indeed, if we compute modulo 7, then we find that $ 1234 \equiv 2 \pmod{7} $.
    Moreover, by Fermat's Little Theorem we have $ 2^6 \equiv 1 \pmod{7} $, so
    $$ 1324^1234 = 2^1234 = 2^{6 \cdot 205 + 4} = 2^4 = 2 \pmod{7} $$
\end{example}

Fermat's Little Theorem states that the multiplicative group
$ \modset{p}^\times $, where $p$ is a prime, contains precisely $p-1$
elements. For arbitrary positive $n$, the number of elements in the multiplicative
group $ \modset{n}^\times $ is given by the so-called \emph{Euler totient function}.

\begin{definition}[Euler totient function]
    The Euler totient function $ \Phi: \mathbb{N} \to \mathbb{N} $ is defined
    by $$ \Phi(n) = \left\vert\modset{n}^\times\right\vert $$
    for all $ n \in \mathbb{N} $ with $ n > 1 $, and by $ \Phi(1) = 1 $.
\end{definition}

\begin{theorem}[Euler Totient]
    The Euler totient function satisfies the following properties.
    \begin{enumerate}
        \item Suppose that $n$ and $m$ are positive integers. If $ \gcd(n,m) = 1 $,
        then $$ \Phi(n \cdot m) = \Phi(n) \cdot \Phi(m) $$

        \item If $p$ is a prime and $n$ is a positive integer, then
        $$ \Phi(p^n) = p^n - p^n-1 $$
    \end{enumerate}
\end{theorem}

\begin{theorem}[Euler's Theorem]
    Suppose $n$ is an integer with $n \ge 2$. Let $a$ be an element of
    $ \modset{n}^\times $. Then $$ a^{\Phi(n)} = 1 $$
\end{theorem}

Let $n$ be an integer. The \emph{order} of an element $a$ in $\modset{n}^\times$
is the smallest positive integer $m$ such that $a^m = 1$. By Euler's Theorem
the order of $a$ exists and is at most $ \Phi(n) $. More precise statements
on the order of elements in $ \modset{n}^\times $ can be found in the
following result.

\begin{theorem}[Orders]
    Let $n$ be an integer greater than 1.
    \begin{enumerate}
        \item If $a \in \modset{n}$ satisfies $a^m=1$ for some positive integer
        $m$, then $a$ is invertible and its order divides $m$.

        \item For all elements $a \in \modset{n}^\times$ the order of $a$ is
        a divisor of $ \Phi(n) $

        \item If $ \modset{n} $ contains an element $a$ of order $n-1$, then
        $n$ is prime.
    \end{enumerate}
\end{theorem}

\begin{definition}
    An element $a$ from $ \modset{p} $ is called a \emph{primitive element}
    of $ \modset{p} $ if every element of $ \modset{p}^\times $ is a power
    of $a$.
\end{definition}

\begin{theorem}
    For each prime $p$ there exists a primitive element in $ \modset{p} $.
\end{theorem}

\subsection{The RSA cryptosystem}
