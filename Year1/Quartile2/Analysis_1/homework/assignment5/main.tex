%%%%%%%%%%%%%%%%%%%%%%%%%%%%% Define Article %%%%%%%%%%%%%%%%%%%%%%%%%%%%%%%%%%
\documentclass{article}
%%%%%%%%%%%%%%%%%%%%%%%%%%%%%%%%%%%%%%%%%%%%%%%%%%%%%%%%%%%%%%%%%%%%%%%%%%%%%%%

%%%%%%%%%%%%%%%%%%%%%%%%%%%%% Using Packages %%%%%%%%%%%%%%%%%%%%%%%%%%%%%%%%%%
\usepackage{geometry}
\usepackage{graphicx}
\usepackage{amssymb}
\usepackage{amsmath}
\usepackage{amsthm}
\usepackage{empheq}
\usepackage{mdframed}
\usepackage{booktabs}
\usepackage{lipsum}
\usepackage{graphicx}
\usepackage{color}
\usepackage{psfrag}
\usepackage{pgfplots}
\usepackage{bm}
%%%%%%%%%%%%%%%%%%%%%%%%%%%%%%%%%%%%%%%%%%%%%%%%%%%%%%%%%%%%%%%%%%%%%%%%%%%%%%%

% Other Settings
\newcommand{\N}{\mathbb{N}}
\newcommand{\R}{\mathbb{R}}
\newcommand{\Q}{\mathbb{Q}}
\newcommand{\Z}{\mathbb{Z}}
\newcommand{\C}{\mathbb{C}}

\newcommand{\e}{\epsilon}
\newcommand{\de}{\delta}

\newcommand{\limn}{\lim_{n\to\infty}}


%%%%%%%%%%%%%%%%%%%%%%%%%% Page Setting %%%%%%%%%%%%%%%%%%%%%%%%%%%%%%%%%%%%%%%
\geometry{a4paper}

%%%%%%%%%%%%%%%%%%%%%%%%%% Define some useful colors %%%%%%%%%%%%%%%%%%%%%%%%%%
\definecolor{ocre}{RGB}{243,102,25}
\definecolor{mygray}{RGB}{243,243,244}
\definecolor{deepGreen}{RGB}{26,111,0}
\definecolor{shallowGreen}{RGB}{235,255,255}
\definecolor{deepBlue}{RGB}{61,124,222}
\definecolor{shallowBlue}{RGB}{235,249,255}
%%%%%%%%%%%%%%%%%%%%%%%%%%%%%%%%%%%%%%%%%%%%%%%%%%%%%%%%%%%%%%%%%%%%%%%%%%%%%%%

%%%%%%%%%%%%%%%%%%%%%%%%%% Define an orangebox command %%%%%%%%%%%%%%%%%%%%%%%%
\newcommand\orangebox[1]{\fcolorbox{ocre}{mygray}{\hspace{1em}#1\hspace{1em}}}
%%%%%%%%%%%%%%%%%%%%%%%%%%%%%%%%%%%%%%%%%%%%%%%%%%%%%%%%%%%%%%%%%%%%%%%%%%%%%%%

%%%%%%%%%%%%%%%%%%%%%%%%%%%% English Environments %%%%%%%%%%%%%%%%%%%%%%%%%%%%%
\newtheoremstyle{mytheoremstyle}{3pt}{3pt}{\normalfont}{0cm}{\rmfamily\bfseries}{}{1em}{{\color{black}\thmname{#1}~\thmnumber{#2}}\thmnote{\,--\,#3}}
\newtheoremstyle{myproblemstyle}{3pt}{3pt}{\normalfont}{0cm}{\rmfamily\bfseries}{}{1em}{{\color{black}\thmname{#1}~\thmnumber{#2}}\thmnote{\,--\,#3}}
\theoremstyle{mytheoremstyle}
\newmdtheoremenv[linewidth=1pt,backgroundcolor=shallowGreen,linecolor=deepGreen,leftmargin=0pt,innerleftmargin=20pt,innerrightmargin=20pt,]{theorem}{Theorem}[section]
\theoremstyle{mytheoremstyle}
\newmdtheoremenv[linewidth=1pt,backgroundcolor=shallowBlue,linecolor=deepBlue,leftmargin=0pt,innerleftmargin=20pt,innerrightmargin=20pt,]{definition}{Definition}[section]
\theoremstyle{myproblemstyle}
\newmdtheoremenv[linecolor=black,leftmargin=0pt,innerleftmargin=10pt,innerrightmargin=10pt,]{problem}{Problem}[section]
%%%%%%%%%%%%%%%%%%%%%%%%%%%%%%%%%%%%%%%%%%%%%%%%%%%%%%%%%%%%%%%%%%%%%%%%%%%%%%%

%%%%%%%%%%%%%%%%%%%%%%%%%%%%%%% Plotting Settings %%%%%%%%%%%%%%%%%%%%%%%%%%%%%
\usepgfplotslibrary{colorbrewer}
\pgfplotsset{width=8cm,compat=1.9}
%%%%%%%%%%%%%%%%%%%%%%%%%%%%%%%%%%%%%%%%%%%%%%%%%%%%%%%%%%%%%%%%%%%%%%%%%%%%%%%

%%%%%%%%%%%%%%%%%%%%%%%%%%%%%%% Title & Author %%%%%%%%%%%%%%%%%%%%%%%%%%%%%%%%
\title{Assignment 5 Exercise 7.8.1}
\author{Jiaqi Wang}
%%%%%%%%%%%%%%%%%%%%%%%%%%%%%%%%%%%%%%%%%%%%%%%%%%%%%%%%%%%%%%%%%%%%%%%%%%%%%%%

\begin{document}
    \maketitle

    \section{7.8.1}
    \begin{problem}
        Let $a: \N \to \R$ be a real-valued sequence. Define the sequence $b: \N \to \R$ by
        $$b_n := a_{n+1}-a_n, \text{ for } n \in \N$$
        \begin{enumerate}
            \item Show that the series
            $$\sum_{n=0}^\infty b_n$$
            converges if and only if $a$ converges.
            \item Show that if the sequence $a:\N \to \R$ converges, then
            $$\limn a_n = a_0 + \sum_{n=0}^\infty b_n$$
        \end{enumerate}
    \end{problem}
    \begin{proof}
        \begin{enumerate}
            \item We need to show both directions of the biimplication
            $$a \text{ converges } \iff \sum_{n=0}^\infty b_n \text{ converges}$$

            1. First we show the foward direction. \\
            Suppose $a$ converges to $p \in \R$
            We need to show that $\limn \sum_{k=0}^n b_k$ converges. \\
            Consider the partial sum
            $$S_n := \sum_{k=0}^n b_k = \sum_{k=0}^n (a_{k+1}-a_k) = a_{n+1}-a_0$$
            Then we have
            $$\limn S_n = \limn (a_{n+1}-a_0) = \limn a_{n+1} - \limn a_0 = p - a_0$$

            2. Now we show the backward direction. \\
            Suppose $\sum_{n=0}^\infty b_n$ converges to $q \in \R$. \\
            We need to show that $a$ converges. \\
            Since the series $\sum_{n=0}^\infty b_n$ converges, we have
            \begin{align*}
                \limn \sum_{k=0}^n b_k &= q \\
                \limn \sum_{k=0}^n (a_{k+1}-a_k) &= q \\
                \limn (a_{n+1}-a_0) &= q \\
                \limn a_{n+1} - \limn a_0 &= q \\
                \limn a_{n+1} - a_0 &= q \\
                \limn a_{n+1} &= q + a_0 \text{ by index shift we get} \\
                \limn a_n &= q + a_0
            \end{align*}

            Thus we have shown both directions of the biimplication, and we conclude that
            $$a \text{ converges } \iff \sum_{n=0}^\infty b_n \text{ converges}$$
            Further more if $a$ converges, then
            $$\limn a_n = a_0 + q$$
            $$\limn a_n = a_0 + \sum_{n=0}^\infty b_n$$
        \end{enumerate}
    \end{proof}

    \section{Exercise 7.8.4}
    \subsection{e)}
    \begin{problem}
        $$\sum_{k=0}^\infty \frac{2k+3}{(k+1)^2(k+2)^2}$$
    \end{problem}

    \subsection{f)}
    \begin{problem}
        $$\sum_{k=1}^\infty \sqrt[k]{2^{-k}+3}$$
    \end{problem}
    Claim: the series diverges.
    \begin{proof}[Proof]
        Consider the sequence $a_k = \sqrt[k]{2^{-k}}.$
        Note that $\sqrt[k]{2^{-k}} < \sqrt[k]{2^{-k}+3}$ for all $k \in \N$. \\
        Note that $$\sum_{k=1}^\infty \sqrt[k]{2^{-k}} = \sum_{k=1}^\infty \frac{1}{2}$$ diverges.
        By the comparison test, we conclude that
        $$\sum_{k=1}^\infty \sqrt[k]{2^{-k}+3}$$ diverges.
    \end{proof}
\end{document}
