%%%%%%%%%%%%%%%%%%%%%%%%%%%%% Define Article %%%%%%%%%%%%%%%%%%%%%%%%%%%%%%%%%%
\documentclass{article}
%%%%%%%%%%%%%%%%%%%%%%%%%%%%%%%%%%%%%%%%%%%%%%%%%%%%%%%%%%%%%%%%%%%%%%%%%%%%%%%

%%%%%%%%%%%%%%%%%%%%%%%%%%%%% Using Packages %%%%%%%%%%%%%%%%%%%%%%%%%%%%%%%%%%
\usepackage{geometry}
\usepackage{graphicx}
\usepackage{amssymb}
\usepackage{amsmath}
\usepackage{amsthm}
\usepackage{empheq}
\usepackage{mdframed}
\usepackage{booktabs}
\usepackage{lipsum}
\usepackage{graphicx}
\usepackage{color}
\usepackage{psfrag}
\usepackage{pgfplots}
\usepackage{bm}
%%%%%%%%%%%%%%%%%%%%%%%%%%%%%%%%%%%%%%%%%%%%%%%%%%%%%%%%%%%%%%%%%%%%%%%%%%%%%%%

\usepackage{physics}
\newcommand{\dist}{\text{dist}}

%%%%%%%%%%%%%%%%%%%%%%%%%% Page Setting %%%%%%%%%%%%%%%%%%%%%%%%%%%%%%%%%%%%%%%
\geometry{a4paper}

%%%%%%%%%%%%%%%%%%%%%%%%%% Define some useful colors %%%%%%%%%%%%%%%%%%%%%%%%%%
\definecolor{ocre}{RGB}{243,102,25}
\definecolor{mygray}{RGB}{243,243,244}
\definecolor{deepGreen}{RGB}{26,111,0}
\definecolor{shallowGreen}{RGB}{235,255,255}
\definecolor{deepBlue}{RGB}{61,124,222}
\definecolor{shallowBlue}{RGB}{235,249,255}
%%%%%%%%%%%%%%%%%%%%%%%%%%%%%%%%%%%%%%%%%%%%%%%%%%%%%%%%%%%%%%%%%%%%%%%%%%%%%%%

%%%%%%%%%%%%%%%%%%%%%%%%%% Define an orangebox command %%%%%%%%%%%%%%%%%%%%%%%%
\newcommand\orangebox[1]{\fcolorbox{ocre}{mygray}{\hspace{1em}#1\hspace{1em}}}
%%%%%%%%%%%%%%%%%%%%%%%%%%%%%%%%%%%%%%%%%%%%%%%%%%%%%%%%%%%%%%%%%%%%%%%%%%%%%%%

%%%%%%%%%%%%%%%%%%%%%%%%%%%% English Environments %%%%%%%%%%%%%%%%%%%%%%%%%%%%%
\newtheoremstyle{mytheoremstyle}{3pt}{3pt}{\normalfont}{0cm}{\rmfamily\bfseries}{}{1em}{{\color{black}\thmname{#1}~\thmnumber{#2}}\thmnote{\,--\,#3}}
\newtheoremstyle{myproblemstyle}{3pt}{3pt}{\normalfont}{0cm}{\rmfamily\bfseries}{}{1em}{{\color{black}\thmname{#1}~\thmnumber{#2}}\thmnote{\,--\,#3}}
\theoremstyle{mytheoremstyle}
\newmdtheoremenv[linewidth=1pt,backgroundcolor=shallowGreen,linecolor=deepGreen,leftmargin=0pt,innerleftmargin=20pt,innerrightmargin=20pt,]{theorem}{Theorem}[section]
\theoremstyle{mytheoremstyle}
\newmdtheoremenv[linewidth=1pt,backgroundcolor=shallowBlue,linecolor=deepBlue,leftmargin=0pt,innerleftmargin=20pt,innerrightmargin=20pt,]{definition}{Definition}[section]
\theoremstyle{myproblemstyle}
\newmdtheoremenv[linecolor=black,leftmargin=0pt,innerleftmargin=10pt,innerrightmargin=10pt,]{problem}{Problem}[section]
%%%%%%%%%%%%%%%%%%%%%%%%%%%%%%%%%%%%%%%%%%%%%%%%%%%%%%%%%%%%%%%%%%%%%%%%%%%%%%%

%%%%%%%%%%%%%%%%%%%%%%%%%%%%%%% Plotting Settings %%%%%%%%%%%%%%%%%%%%%%%%%%%%%
\usepgfplotslibrary{colorbrewer}
\pgfplotsset{width=8cm,compat=1.9}
%%%%%%%%%%%%%%%%%%%%%%%%%%%%%%%%%%%%%%%%%%%%%%%%%%%%%%%%%%%%%%%%%%%%%%%%%%%%%%%

%%%%%%%%%%%%%%%%%%%%%%%%%%%%%%% Title & Author %%%%%%%%%%%%%%%%%%%%%%%%%%%%%%%%
\title{Assignment 3}
\author{Jiaqi Wang}
%%%%%%%%%%%%%%%%%%%%%%%%%%%%%%%%%%%%%%%%%%%%%%%%%%%%%%%%%%%%%%%%%%%%%%%%%%%%%%%

\begin{document}
    \maketitle

    \section{5.9.1}
    \begin{problem}
        Let $(X, \text{dist})$ be a metric space. Let $p \in X$ and assume that the sequence $(a_n)$ is given by $a_n = p$ 
        for every $n \in \mathbb{N}$. Prove that $\displaystyle\lim_{n\to\infty}a_n = p$.
    \end{problem}
    \begin{proof}
        To show that the sequence $(a_n)$ converges to $p$, we need to show that 
        \begin{center}
            for all $\epsilon > 0$, \\
            there exists $N > 0$, \\
            for all $n > N$, \\
            $\dist(a_n, p) < \epsilon$.
        \end{center}

        \noindent 
        Let $\epsilon > 0$. \\
        Choose $N = 1$. \\
        Take $n > N$, \\
        It holds that $\text{dist}(a_n, p) = \text{dist}(p, p) = 0 < \epsilon$. \\
        We conclude that $\displaystyle\lim_{n\to\infty}a_n = p$.
    \end{proof}

    \section{5.9.2}
    \begin{problem}
        Let $(x, \text{dist})$ be a metric space and let $a: \mathbb{N} \to X$ be a sequence in $X$. Let $k \in \mathbb{N}$ and $p \in X$.
        Then the sequence $(a_n)$ converges to $p$ if and only if the sequence $(a_{n+k})$ converges to $p$.
    \end{problem}
    \begin{proof}
        To show that the sequence $(a_n)$ converges to $p$ if and only if the sequence $(a_{n+k})$ converges to $p$,
        we need to show both directions. \\
        1. Forward direction: \\
        Assume that the sequence $(a_n)$ converges to $p$. \\
        I.e. 
        \begin{center}
            for all $\epsilon > 0$, \\
            there exists $N_1 > 0$, \\
            for all $n > N_1$, \\
            $\dist(a_n, p) < \epsilon$.
        \end{center}
        We need to show that 
        \begin{center}
            for all $\epsilon > 0$, \\
            there exists $N > 0$, \\
            for all $n > N$, \\
            $\dist(a_{n+k}, p) < \epsilon$.
        \end{center}
        Let $\epsilon > 0$. \\
        Choose $N = N_1$. \\
        Take $n > N$, \\
        It holds that $n > N_1$. \\



        It holds that (for all $\epsilon > 0$, there exists $N > 0$ for which $n > N$ implies $\text{dist}(a_n, p) < \epsilon$). \\
        We need to show that (for all $\epsilon > 0$, there exists $N > 0$ for which $n > N$ implies $\text{dist}(a_{n+k}, p) < \epsilon$). \\
    \end{proof}

    \section{5.9.3}
    \begin{problem}
        Let $(V, \norm{\cdot})$ be a normed vector space. Let $a: \mathbb{N} \to V$ and $b: \mathbb{N} \to V$ be two sequences.
        Assume that the limit $\displaystyle\lim_{n\to\infty}a_n$ exists and is equal to $p \in V$ and that the limit $\displaystyle\lim_{n\to\infty}b_n$
        exists and is equal to $q \in V$. Let $\lambda: \mathbb{N} \to \mathbb{R}$ be a real-valued sequence. Let $\mu \in \mathbb{R}$.
        Assume that $\displaystyle\lim_{n\to\infty}\lambda_n = \mu$. Prove that the limit $\displaystyle\lim_{n\to\infty}(a_n + b_n)$ exists
        and is equal to $p + q$.
    \end{problem}
    \begin{proof}
    \end{proof}

    \section{5.9.4}
    \begin{problem}
        Let $(X, \text{dist})$ be a metric space and let $a: \mathbb{N} \to X$ be a bounded sequence in $X$. Let $p \in X$.
        Define also the sequence $s: \mathbb{N} \to \mathbb{R}$ by
        $$s_k = \sup \{\text{dist}(a_l,p) \mid l \in \mathbb{N}, l \ge k\}.$$
        Show that $\displaystyle\lim{n\to\infty}a_n = p$ if and only if
        $$\inf_{k\in\mathbb{N}}s_k = 0$$
    \end{problem}
    \begin{proof}
    \end{proof}
    
\end{document}
