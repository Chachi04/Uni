%%%%%%%%%%%%%%%%%%%%%%%%%%%%% Define Article %%%%%%%%%%%%%%%%%%%%%%%%%%%%%%%%%%
\documentclass{article}
%%%%%%%%%%%%%%%%%%%%%%%%%%%%%%%%%%%%%%%%%%%%%%%%%%%%%%%%%%%%%%%%%%%%%%%%%%%%%%%

%%%%%%%%%%%%%%%%%%%%%%%%%%%%% Using Packages %%%%%%%%%%%%%%%%%%%%%%%%%%%%%%%%%%
\usepackage{geometry}
\usepackage{graphicx}
\usepackage{amssymb}
\usepackage{amsmath}
\usepackage{amsthm}
\usepackage{empheq}
\usepackage{mdframed}
\usepackage{booktabs}
\usepackage{lipsum}
\usepackage{graphicx}
\usepackage{color}
\usepackage{psfrag}
\usepackage{pgfplots}
\usepackage{bm}
%%%%%%%%%%%%%%%%%%%%%%%%%%%%%%%%%%%%%%%%%%%%%%%%%%%%%%%%%%%%%%%%%%%%%%%%%%%%%%%

% Other Settings
\newcommand{\N}{\mathbb{N}}
\newcommand{\R}{\mathbb{R}}
\newcommand{\Q}{\mathbb{Q}}
\newcommand{\Z}{\mathbb{Z}}
\newcommand{\C}{\mathbb{C}}

\newcommand{\e}{\epsilon}
\newcommand{\de}{\delta}

\newcommand{\limn}{\lim_{n\to\infty}}


%%%%%%%%%%%%%%%%%%%%%%%%%% Page Setting %%%%%%%%%%%%%%%%%%%%%%%%%%%%%%%%%%%%%%%
\geometry{a4paper}

%%%%%%%%%%%%%%%%%%%%%%%%%% Define some useful colors %%%%%%%%%%%%%%%%%%%%%%%%%%
\definecolor{ocre}{RGB}{243,102,25}
\definecolor{mygray}{RGB}{243,243,244}
\definecolor{deepGreen}{RGB}{26,111,0}
\definecolor{shallowGreen}{RGB}{235,255,255}
\definecolor{deepBlue}{RGB}{61,124,222}
\definecolor{shallowBlue}{RGB}{235,249,255}
%%%%%%%%%%%%%%%%%%%%%%%%%%%%%%%%%%%%%%%%%%%%%%%%%%%%%%%%%%%%%%%%%%%%%%%%%%%%%%%

%%%%%%%%%%%%%%%%%%%%%%%%%% Define an orangebox command %%%%%%%%%%%%%%%%%%%%%%%%
\newcommand\orangebox[1]{\fcolorbox{ocre}{mygray}{\hspace{1em}#1\hspace{1em}}}
%%%%%%%%%%%%%%%%%%%%%%%%%%%%%%%%%%%%%%%%%%%%%%%%%%%%%%%%%%%%%%%%%%%%%%%%%%%%%%%

%%%%%%%%%%%%%%%%%%%%%%%%%%%% English Environments %%%%%%%%%%%%%%%%%%%%%%%%%%%%%
\newtheoremstyle{mytheoremstyle}{3pt}{3pt}{\normalfont}{0cm}{\rmfamily\bfseries}{}{1em}{{\color{black}\thmname{#1}~\thmnumber{#2}}\thmnote{\,--\,#3}}
\newtheoremstyle{myproblemstyle}{3pt}{3pt}{\normalfont}{0cm}{\rmfamily\bfseries}{}{1em}{{\color{black}\thmname{#1}~\thmnumber{#2}}\thmnote{\,--\,#3}}
\theoremstyle{mytheoremstyle}
\newmdtheoremenv[linewidth=1pt,backgroundcolor=shallowGreen,linecolor=deepGreen,leftmargin=0pt,innerleftmargin=20pt,innerrightmargin=20pt,]{theorem}{Theorem}[section]
\theoremstyle{mytheoremstyle}
\newmdtheoremenv[linewidth=1pt,backgroundcolor=shallowBlue,linecolor=deepBlue,leftmargin=0pt,innerleftmargin=20pt,innerrightmargin=20pt,]{definition}{Definition}[section]
\theoremstyle{myproblemstyle}
\newmdtheoremenv[linecolor=black,leftmargin=0pt,innerleftmargin=10pt,innerrightmargin=10pt,]{problem}{Problem}[section]
%%%%%%%%%%%%%%%%%%%%%%%%%%%%%%%%%%%%%%%%%%%%%%%%%%%%%%%%%%%%%%%%%%%%%%%%%%%%%%%

%%%%%%%%%%%%%%%%%%%%%%%%%%%%%%% Plotting Settings %%%%%%%%%%%%%%%%%%%%%%%%%%%%%
\usepgfplotslibrary{colorbrewer}
\pgfplotsset{width=8cm,compat=1.9}
%%%%%%%%%%%%%%%%%%%%%%%%%%%%%%%%%%%%%%%%%%%%%%%%%%%%%%%%%%%%%%%%%%%%%%%%%%%%%%%

%%%%%%%%%%%%%%%%%%%%%%%%%%%%%%% Title & Author %%%%%%%%%%%%%%%%%%%%%%%%%%%%%%%%
\title{Homework: Week 2}
\author{Jiaqi Wang}
%%%%%%%%%%%%%%%%%%%%%%%%%%%%%%%%%%%%%%%%%%%%%%%%%%%%%%%%%%%%%%%%%%%%%%%%%%%%%%%

\begin{document}
    \maketitle

    \section{Exercise 6.8.1}
    \begin{problem}
        Let $a: \N \to \R$ be a sequence. Then $a: \N \to \R$ is bounded if and only if it is both bounded above and bounded below. Prove it.
    \end{problem}
    \begin{proof}
        We prove both directions of the if and only if statement.
        $a: \N \to \R$ is bounded if and only if it is both bounded above and bounded below.
        
        1. We prove the forward direction. \\
        Suppose $a: \N \to \R$ is bounded. \\
        Then 
        \begin{center}
            \parbox{\linewidth}{
                \leftskip=0.4\linewidth
                there exists $M_0 > 0$, \\
                \hspace*{1em} for all $n \in \N$, \\
                \hspace*{2em} $|a_n| \leq M_0$.
            }
        \end{center}
        Obtain such $M_0$. \\ 
        It holds that
        \begin{center}
            \parbox{\linewidth}{
                \leftskip=0.4\linewidth
                for all $n \in \N$, \\
                \hspace*{1em} $-M_0 \leq a_n \leq M_0$.
            }
        \end{center}
        Choose $m = -M_0$, then $m \in \R$. \\
        It holds that
        \begin{center}
            \parbox{\linewidth}{
                \leftskip=0.4\linewidth
                for all $n \in \N$, \\
                \hspace*{1em} $m \leq a_n$.
            }
        \end{center}
        We conclude that $a:\N \to \R$ is bounded from below.

        Choose $M = M_0$, then $M \in \R$. \\
        It holds that
        \begin{center}
            \parbox{\linewidth}{
                \leftskip=0.4\linewidth
                for all $n \in \N$, \\
                \hspace*{1em} $a_n \leq M$.
            }
        \end{center}
        We conclude that $a:\N \to \R$ is bounded from above.

        2. Now we prove the reverse direction. \\
        Assume $a:\N \to \R$ is bounded from above and bounded from below. \\
        By definition of bounded from above, we have
        \begin{center}
            \parbox{\linewidth}{
                \leftskip=0.4\linewidth
                there exists $M_1 \in \R$, \\
                \hspace*{1em} for all $n \in \N$, \\
                \hspace*{2em} $a_n \leq M_1$.
            }
        \end{center}

        By definition of bounded from below, we have
        \begin{center}
            \parbox{\linewidth}{
                \leftskip=0.4\linewidth
                there exists $M_2 \in \R$, \\
                \hspace*{1em} for all $n \in \N$, \\
                \hspace*{2em} $M_2 \leq a_n$.
            }
        \end{center}

        We need to show that
        \begin{center}
            \parbox{\linewidth}{
                \leftskip=0.4\linewidth
                there exists $M_0 > 0$, \\
                \hspace*{1em} for all $n \in \N$, \\
                \hspace*{2em} $|a_n| \leq M_0$.
            }
        \end{center}

        Choose $M_0 = \max\{|M_1|, |M_2|\}$, then $M_0 \in \R$. \\
        It holds that
        \begin{center}
            \parbox{\linewidth}{
                \leftskip=0.4\linewidth
                for all $n \in \N$, \\
                \hspace*{1em} $-M_0 \leq a_n$ and $a_n \leq M_0$.
            }
        \end{center}
        Then it holds that $|a_n| \leq M_0$.\\
        We conclude that $a: \N \to \R$ is bounded.

        Since both directions hold, we conclude that $a: \N \to \R$ is bounded if and only if it is both bounded above and bounded below.
    \end{proof}

    \section{Exercise 6.8.2}
    \begin{problem}
        Let $a: \N \to \R$ and $b: \N \to (0,\infty)$ be real-valued sequences. Prove that
        $$\lim_{n\to\infty}a_n = \infty \iff \lim_{n\to\infty}(-a_n) = -\infty$$
    \end{problem}
    \begin{proof}
        We show both directions of the if and only if statement.
        $$\lim_{n\to\infty}a_n = \infty \iff \lim_{n\to\infty}(-a_n) = -\infty$$

        1. First we prove the forward direction. \\
        Suppose $\lim_{n\to\infty}a_n = \infty$. \\
        Then
        \begin{center}
            \parbox{\linewidth}{
                \leftskip=0.4\linewidth
                for all $M \in \R$, \\
                \hspace*{1em} there exists $N_0 \in \N$, \\
                \hspace*{2em} for all $n \geq N_0$, \\
                \hspace*{3em} $a_n > M$.
            }
        \end{center}
        We need to show that
        \begin{center}
            \parbox{\linewidth}{
                \leftskip=0.4\linewidth
                for all $M \in \R$, \\
                \hspace*{1em} there exists $N \in \N$, \\
                \hspace*{2em} for all $n \geq N$, \\
                \hspace*{3em} $-a_n \leq M$.
            }
        \end{center}
        Take $M \in \R$
        Choose $N = N_0$, then $N \in \N$. \\
        It holds that
        \begin{center}
            \parbox{\linewidth}{
                \leftskip=0.4\linewidth
                for all $n \geq N$, \\
                \hspace*{1em} $a_n > M$.
            }
        \end{center}
    \end{proof}

    \section{Exercise 6.8.3}
    \begin{problem}
        Let $a: \N \to \R$ and $b: \N \to (0,\infty)$ be real-valued sequences. Prove that
        $$\lim_{n\to\infty}b_n = \infty \iff \lim_{n\to\infty}\frac{1}{b_n} = 0$$
    \end{problem}
    \begin{proof}
    \end{proof}

    \section{Exercise 6.8.5}
    \begin{problem}
        Define the sequence $x: \N \to \R$ recursively by 
        $$x_{n+1} := \frac{2+x_n^2}{2x_n}$$
        for $n \in \N$ while $x_0 = 2$. Prove that the sequence $x: \N \to \R$ converges and determine its limit.
    \end{problem}
    \begin{proof}
    \end{proof}

    \section{Exercise 6.8.6}
    \begin{problem}
        Determine whether the following sequences converge, diverge to $+\infty$, diverge to $-\infty$, or diverge in a different way.
        In case the sequence converges, determine the limit.
        \begin{itemize}
            \item $a_n := \frac{1}{n^3} - 3$
            \item $b_n := \frac{5n^5 + 2n^2}{3n^5+7n^3+4}$
            \item $c_n := n - \sqrt{n}$
            \item $d_n := \frac{2^n}{n^100}$
            \item $e_n := \sqrt{n^2+n}-n$
            \item $f_n := \sqrt[n]{3n^2}$
            \item $g_n := \frac{2^n + 5n^200}{3^n+n^10}$
            \item $h_n := (-1)^n3^n$
            \item $i_n := \sqrt[n]{5^n + n^2}$
        \end{itemize}
    \end{problem}
    \begin{proof}
        By limit laws and standard limits, we have
        \begin{align*}
            \limn a_n &= \limn\left(\frac{1}{n^3} - 3\right) \\
                      &= \limn\frac{1}{n^3} - \limn 3 \\
                      &= \left(\limn\frac{1}{n}\right)^3 - 3 \\
                      &= 0^3 - 3 \\
                      &= -3 \\
        \end{align*}
        \begin{align*}
            \limn b_n &= \limn\left(\frac{5n^5 + 2n^2}{3n^5+7n^3+4}\right) \\
                      &= \limn\frac{5 + \frac{2}{n^3}}{3+\frac{7}{n^2}+\frac{4}{n^5}} \\
                      &= \frac{\limn(5 + \frac{2}{n^3})}{\limn(3+\frac{7}{n^2}+\frac{4}{n^5})} \\
                      &= \frac{\limn 5 + \limn\frac{2}{n^3}}{\limn 3+\limn\frac{7}{n^2}+\limn\frac{4}{n^5}} \\
                      &= \frac{5}{3} \\
        \end{align*}
        \begin{align*}
            \limn c_n &= \limn(n - \sqrt{n}) \\
                      &= \limn\left(\frac{(n - \sqrt{n})(n + \sqrt{n})}{n + \sqrt{n}}\right) \\
                      &= \limn\left(\frac{n^2 - n}{n + \sqrt{n}}\right) \\
                      &= \limn\left(\frac{n(n - 1)}{n(1 + \frac{1}{\sqrt{n}})}\right) \\
                      &= \limn\left(\frac{n - 1}{1 + \frac{1}{\sqrt{n}}}\right) \\
                        &= \frac{\limn(n - 1)}{\limn(1 + \frac{1}{\sqrt{n}})} \\
                        &= \frac{\limn n - \limn 1}{\limn 1 + \limn\frac{1}{\sqrt{n}}} \\
                        &= \frac{\limn n - 1}{1 + \limn\frac{1}{\sqrt{n}}} \\
                        &= \frac{\limn n - 1}{1 + 0} \\
                        &= \limn n - 1 \\
                        &= \infty \\
        \end{align*}
        \begin{align*}
            \limn d_n &= \limn(\frac{2^n}{n^100}) \\
                      &= \limn\left(\frac{2^n}{n^{100}}\right) \\
        \end{align*}
    \end{proof}
\end{document}
