%%%%%%%%%%%%%%%%%%%%%%%%%%%%% Define Article %%%%%%%%%%%%%%%%%%%%%%%%%%%%%%%%%%
\documentclass{article}
%%%%%%%%%%%%%%%%%%%%%%%%%%%%%%%%%%%%%%%%%%%%%%%%%%%%%%%%%%%%%%%%%%%%%%%%%%%%%%%

%%%%%%%%%%%%%%%%%%%%%%%%%%%%% Using Packages %%%%%%%%%%%%%%%%%%%%%%%%%%%%%%%%%%
\usepackage{geometry}
\usepackage{graphicx}
\usepackage{amssymb}
\usepackage{amsmath}
\usepackage{amsthm}
\usepackage{empheq}
\usepackage{mdframed}
\usepackage{booktabs}
\usepackage{lipsum}
\usepackage{graphicx}
\usepackage{color}
\usepackage{psfrag}
\usepackage{pgfplots}
\usepackage{bm}
%%%%%%%%%%%%%%%%%%%%%%%%%%%%%%%%%%%%%%%%%%%%%%%%%%%%%%%%%%%%%%%%%%%%%%%%%%%%%%%

% Other Settings

%%%%%%%%%%%%%%%%%%%%%%%%%% Page Setting %%%%%%%%%%%%%%%%%%%%%%%%%%%%%%%%%%%%%%%
\geometry{a4paper}

%%%%%%%%%%%%%%%%%%%%%%%%%% Define some useful colors %%%%%%%%%%%%%%%%%%%%%%%%%%
\definecolor{ocre}{RGB}{243,102,25}
\definecolor{mygray}{RGB}{243,243,244}
\definecolor{deepGreen}{RGB}{26,111,0}
\definecolor{shallowGreen}{RGB}{235,255,255}
\definecolor{deepBlue}{RGB}{61,124,222}
\definecolor{shallowBlue}{RGB}{235,249,255}
%%%%%%%%%%%%%%%%%%%%%%%%%%%%%%%%%%%%%%%%%%%%%%%%%%%%%%%%%%%%%%%%%%%%%%%%%%%%%%%

%%%%%%%%%%%%%%%%%%%%%%%%%% Define an orangebox command %%%%%%%%%%%%%%%%%%%%%%%%
\newcommand\orangebox[1]{\fcolorbox{ocre}{mygray}{\hspace{1em}#1\hspace{1em}}}
%%%%%%%%%%%%%%%%%%%%%%%%%%%%%%%%%%%%%%%%%%%%%%%%%%%%%%%%%%%%%%%%%%%%%%%%%%%%%%%

%%%%%%%%%%%%%%%%%%%%%%%%%%%% English Environments %%%%%%%%%%%%%%%%%%%%%%%%%%%%%
\newtheoremstyle{mytheoremstyle}{3pt}{3pt}{\normalfont}{0cm}{\rmfamily\bfseries}{}{1em}{{\color{black}\thmname{#1}~\thmnumber{#2}}\thmnote{\,--\,#3}}
\newtheoremstyle{myproblemstyle}{3pt}{3pt}{\normalfont}{0cm}{\rmfamily\bfseries}{}{1em}{{\color{black}\thmname{#1}~\thmnumber{#2}}\thmnote{\,--\,#3}}
\theoremstyle{mytheoremstyle}
\newmdtheoremenv[linewidth=1pt,backgroundcolor=shallowGreen,linecolor=deepGreen,leftmargin=0pt,innerleftmargin=20pt,innerrightmargin=20pt,]{theorem}{Theorem}[section]
\theoremstyle{mytheoremstyle}
\newmdtheoremenv[linewidth=1pt,backgroundcolor=shallowBlue,linecolor=deepBlue,leftmargin=0pt,innerleftmargin=20pt,innerrightmargin=20pt,]{definition}{Definition}[section]
\theoremstyle{myproblemstyle}
\newmdtheoremenv[linecolor=black,leftmargin=0pt,innerleftmargin=10pt,innerrightmargin=10pt,]{problem}{Problem}[section]
%%%%%%%%%%%%%%%%%%%%%%%%%%%%%%%%%%%%%%%%%%%%%%%%%%%%%%%%%%%%%%%%%%%%%%%%%%%%%%%

%%%%%%%%%%%%%%%%%%%%%%%%%%%%%%% Plotting Settings %%%%%%%%%%%%%%%%%%%%%%%%%%%%%
\usepgfplotslibrary{colorbrewer}
\pgfplotsset{width=8cm,compat=1.9}
%%%%%%%%%%%%%%%%%%%%%%%%%%%%%%%%%%%%%%%%%%%%%%%%%%%%%%%%%%%%%%%%%%%%%%%%%%%%%%%

%%%%%%%%%%%%%%%%%%%%%%%%%%%%%%% Title & Author %%%%%%%%%%%%%%%%%%%%%%%%%%%%%%%%
\title{Homework: Week 2}
\author{Jiaqi Wang}
%%%%%%%%%%%%%%%%%%%%%%%%%%%%%%%%%%%%%%%%%%%%%%%%%%%%%%%%%%%%%%%%%%%%%%%%%%%%%%%

\begin{document}
    \maketitle

    \section{Exercise 4.9.1}
    \begin{problem}
        Show that for all $a,b \in \mathbb{R}$, if $a < b$ then
        $$\inf[a,b) = a$$
    \end{problem}
    \begin{proof}
        To show that $a$ is an infima we need to show that $a$ is a lower bound and that
        \begin{center}
            for all $\epsilon > 0$, \\
            there exists $x \in [a,b)$, \\
            $x < a + \epsilon$.
        \end{center}

        First we show that $a$ is a lower bound. \\
        Take $x \in [a,b)$. \\
        It holds that $a \le x$. \\
        We conclude that $a$ is a lower bound. \\

        Now we show that for all $\epsilon > 0$, there exists $x \in [a,b)$, $x < a + \epsilon$. \\
        Take $\epsilon > 0$. \\
        Choose $x = a$. Then $x \in [a,b)$ \\
        It holds that $x = a < a + \epsilon$. \\
        We conclude that $\inf [a,b) = a$.
    \end{proof}

    \section{Exercise 4.9.2}
    \begin{problem}
        Let $A$ be a subset of $\mathbb{R}$. Prove that if $\sup A \in A$ then $A$ has a maximum and $\max A = \sup A$.
        % Let $A$ be a subset of $\mathbb{R}$. Assume that $A$ is non-empty and
        % bounded from below. If $\inf A \in A$ then $A$ has a minimum and $\min
        % A = \inf A$. Prove this.
    \end{problem}
    \begin{proof}
        We need to show that $\max A = \sup A$.
        
        \noindent Assume $\sup A \in A$. \\
        Choose $x = \sup A$. \\
        It holds that $x \in A$. \\
        Take $a \in A$. \\
        It holds that $a \le x$. \\
        We conclude that $a = \sup A = \max A$.
    \end{proof}

    \section{Exercise 4.9.3}
    \begin{problem}
        Show that
        $$\sup[0,4) = 4$$
    \end{problem}
    \begin{proof}
        We need to show that $4$ is an upper bound and that
        \begin{center}
            for all $\epsilon > 0$, \\
            there exists $x \in [0,4)$, \\
            $x > 4 - \epsilon$.
        \end{center}

        First we show that $4$ is an upper bound. \\
        Take $x \in [0,4)$. \\
        It holds that $x \le 4$. \\
        We conclude that $4$ is an upper bound. \\

        Now we show that for all $\epsilon > 0$, there exists $x \in [0,4)$, $x > 4 - \epsilon$. \\
        Take $\epsilon > 0$. \\
        Choose $x = \max(0, 4 - \frac{\epsilon}{2})$. Then $x \in [0,4)$ \\
        It holds that $x > 4 - \epsilon$. \\
        We conclude that $\sup [0,4) = 4$.
    \end{proof}

    \newpage

    \section{Exercise 4.9.5}
    \begin{problem}
        Let $A$ be a non-empty subset of $\mathbb{R}$. Assume that $A$ is bounded from above.
        Prove that for all $\lambda \ge 0, \sup(\lambda A) = \lambda \sup A$.
    \end{problem}
    \begin{proof}
        Take $A \subseteq \mathbb{R}$.
        Suppose $A$ is bounded from above. \\
        It holds that $\sup A$ exists. \\
        Take $\lambda \ge 0$. \\
        By the alternative characterization of the supremum we need to show that $\lambda \sup A$ is an upper bound for $\lambda A$ and that
        \begin{center}
            for all $\epsilon > 0$, \\
            there exists $a \in \lambda A$, \\
            $a > \lambda \sup A - \epsilon$.
        \end{center}

        First we show that $\lambda \sup A$ is an upper bound for $\lambda A$. \\
        Since $\sup A$ is a supremum for $A$, it is an upper bound for $A$. \\
        It holds that $\sup A \ge a$ for all $a \in A$. \\
        It holds that $\lambda \sup A \ge \lambda a$ for all $a \in A$ and $\lambda a \in \lambda A$ \\
        We conclude that $\lambda \sup A$ is an upper bound for $\lambda A$. \\

        Now we show that for all $\epsilon > 0$, there exists $a \in \lambda A$, $a > \lambda \sup A - \epsilon$. \\
        Take $\epsilon > 0$. \\
        Since $\sup A$ is a supremum for $A$, it is an upper bound for $A$. \\
        By the alternative characterization of the supremum, 
        \begin{center}
            for all $\epsilon_1 > 0$, \\
            there exists $a_1 \in A$, \\
            $a_1 > \sup A - \epsilon_1$. \\
        \end{center}
        Either $\lambda = 0$ or $\lambda > 0$. \\

        - Case $\lambda = 0$, then $\lambda \sup A - \epsilon = -\epsilon < 0$. \\
        Choose $a = 0$ and $a \in \lambda A$. \\
        It holds that $a = 0 > \lambda \sup A - \epsilon = -\epsilon$. \\

        - Case $\lambda > 0$ \\
        Choose $\epsilon_1 = \displaystyle\frac{\epsilon}{\lambda}$. Then $\epsilon_1 > 0$ and there exists $a_1 \in A$, such that $a_1 > \sup A - \epsilon_1 = \sup A - \frac{\epsilon}{\lambda}$ \\
        Obtain such $a_1$. \\
        Choose $a = \lambda a_1$. \\
        It holds that $a = \lambda a_1 > \lambda \sup A - \epsilon$. \\
        We have shown that for all $\epsilon > 0$, there exists $a = \lambda a_1 \in \lambda A$, $a > \lambda \sup A - \epsilon$. \\
        We conclude that $\sup (\lambda A) = \lambda \sup A$.
    \end{proof}

    \newpage

    \section{Exercise 4.9.6}
    \begin{problem}
        Let $A$ be a non-empty subset of $\mathbb{R}$. Assume that $A$ is bounded from below.
        Prove that for all $\lambda \ge 0, \inf(\lambda A) = \lambda \inf A$.
    \end{problem}
    \begin{proof}
        We need to show that $\lambda \inf A$ is a lower bound for $\lambda A$ and that
        \begin{center}
            for all $\epsilon > 0$, \\
            there exists $a \in \lambda A$, \\
            $a < \lambda \inf A + \epsilon$.
        \end{center}

        First we show that $\lambda \inf A$ is a lower bound for $\lambda A$. \\
        Since $\inf A$ is a infimum for $A$, it is a lower bound for $A$. \\
        It holds that $\inf A \le a$ for all $a \in A$. \\
        It holds that $\lambda \inf A \le \lambda a$ for all $a \in A$ and $\lambda a \in \lambda A$ \\
        We conclude that $\lambda \inf A$ is a lower bound for $\lambda A$. \\

        Now we show that for all $\epsilon > 0$, there exists $x \in \lambda A$, $x < \lambda \inf A + \epsilon$. \\
        Take $\epsilon > 0$. \\
        Since $\inf A$ is a infimum for $A$, by the alternative characterization of the infimum it holds that
        \begin{center}
            for all $\epsilon_1 > 0$, \\
            there exists $a_1 \in A$, \\
            $a_1 < \inf A + \epsilon_1$. \\
        \end{center}
        Either $\lambda = 0$ or $\lambda > 0$. \\

        - Case $\lambda = 0$, then $\lambda \inf A + \epsilon = \epsilon > 0$. \\
        Choose $a = 0$ and $a \in \lambda A$. \\
        It holds that $a = 0 < \lambda \inf A + \epsilon = \epsilon$. \\

        - Case $\lambda > 0$ \\
        Choose $\epsilon_1 = \frac{\epsilon}{\lambda}$. Then $\epsilon_1 > 0$ and there exists $a_1 \in A$, such that $a_1 < \inf A + \epsilon_1 = \inf A + \frac{\epsilon}{\lambda}$ \\
        Obtain such $a_1$. \\
        It holds that $a_1 < \inf A + \frac{\epsilon}{\lambda}$. \\
        It holds that $a = \lambda a_1 < \lambda \inf A + \epsilon$. \\
        We conclude that $\inf(\lambda A) = \lambda \inf A$
    \end{proof}
\end{document}
