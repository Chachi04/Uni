%%%%%%%%%%%%%%%%%%%%%%%%%%%%% Using Packages %%%%%%%%%%%%%%%%%%%%%%%%%%%%%%%%%%
\usepackage{geometry}
\usepackage{graphicx}
\usepackage{amssymb}
\usepackage{amsmath}
\usepackage{amsthm}
\usepackage{empheq}
% \usepackage{mdframed}
\usepackage{booktabs}
\usepackage{lipsum}
\usepackage{graphicx}
\usepackage{color}
\usepackage{psfrag}
\usepackage{pgfplots}
\usepackage{bm}
%%%%%%%%%%%%%%%%%%%%%%%%%%%%%%%%%%%%%%%%%%%%%%%%%%%%%%%%%%%%%%%%%%%%%%%%%%%%%%%

% Other Settings
% \usepackage{mathptmx}
\usepackage{mathrsfs}
\usepackage{thmtools}
% \usepackage[dvipsnames]{xcolor}
\usepackage[most]{tcolorbox}
% \usepackage[mathscr]{euscript}
\usepackage{enumitem}
% \usepackage{physics}
% \usepackage{microtype}
% \usepackage[T1]{fontenc}
% \usepackage[utf8]{inputenc}
\usepackage{hyperref}
\hypersetup{
    colorlinks=true,
    linkcolor=blue,
    filecolor=magenta,
    urlcolor=cyan,
    pdftitle={Linear Algebra 2},
    % pdfpagemode=FullScreen,
    }

\urlstyle{same}

\setcounter{tocdepth}{2}
\usepackage{cleveref}
% \usepackage{algorithm2e}

%%%%%%%%%%%%%%%%%%%%%%%%%% Page Setting %%%%%%%%%%%%%%%%%%%%%%%%%%%%%%%%%%%%%%%
\geometry{a4paper}

%%%%%%%%%%%%%%%%%%%%%%%%%% Custom commands %%%%%%%%%%%%%%%%%%%%%%%%%%%%%%%%%%%%%%%
\newcommand{\inv}{^{-1}}
\newcommand{\sym}{\text{Sym}}
\newcommand{\alt}{\text{Alt}}
\newcommand{\fix}{\text{fix}}
\newcommand{\support}{\text{support}}
\newcommand{\sign}{\text{sign}}
\newcommand{\spec}[1]{\text{spec}(#1)}
\renewcommand{\vec}[1]{\underline{#1}}

\newcommand{\K}{\mathbb{K}}
\newcommand{\N}{\mathbb{N}}
\newcommand{\Z}{\mathbb{Z}}
\newcommand{\R}{\mathbb{R}}
\newcommand{\C}{\mathbb{C}}
\newcommand{\Q}{\mathbb{Q}}
\newcommand{\Id}{\mathcal{I}}
\newcommand{\A}{\mathcal{A}}

\newcommand{\range}{\mathcal{R}}
\newcommand{\ns}{\mathcal{N}}
\newcommand{\tr}{\text{tr}}
% \newcommand{\and}{\text{ and }}


%%%%%%%%%%%%%%%%%%%%%%%%%% Define some useful colors %%%%%%%%%%%%%%%%%%%%%%%%%%
\definecolor{ocre}{RGB}{243,102,25}
\definecolor{mygrey}{RGB}{243,243,244}
\definecolor{deepGreen}{RGB}{178,254,179}
\definecolor{shallowGreen}{RGB}{243,255,243}
\definecolor{deepBlue}{RGB}{61,124,222}
\definecolor{shallowBlue}{RGB}{235,249,255}
\definecolor{deepOrange}{RGB}{254,129,0}
\definecolor{shallowOrange}{RGB}{255,243,231}
\definecolor{deepPurple}{RGB}{128,129,254}
\definecolor{shallowPurple}{RGB}{243, 243, 255}
\definecolor{deepRed}{RGB}{254,76,77}
\definecolor{shallowRed}{RGB}{254,242,243}
\definecolor{theoremTitle}{RGB}{78,93,247}
%%%%%%%%%%%%%%%%%%%%%%%%%%%%%%%%%%%%%%%%%%%%%%%%%%%%%%%%%%%%%%%%%%%%%%%%%%%%%%%

%%%%%%%%%%%%%%%%%%%%%%%%%% Define an orangebox command %%%%%%%%%%%%%%%%%%%%%%%%
\newcommand\orangebox[1]{\fcolorbox{ocre}{mygrey}{\hspace{1em}#1\hspace{1em}}}
\newcommand\egcd{\text{Extgcd}}
\newcommand{\modset}[1]{\mathbb{Z}/#1\mathbb{Z}}
%%%%%%%%%%%%%%%%%%%%%%%%%%%%%%%%%%%%%%%%%%%%%%%%%%%%%%%%%%%%%%%%%%%%%%%%%%%%%%%

\newlist{theoremlist}{description}{1}
\setlist[theoremlist]{labelsep=1em, left=0pt, labelwidth=2cm, style=multiline, font=\normalfont}

%%%%%%%%%%%%%%%%%%%%%%%%%%%% English Environments %%%%%%%%%%%%%%%%%%%%%%%%%%%%%
\declaretheoremstyle[
    spaceabove=2pt,
    spacebelow=2pt,
    headfont=\normalfont\bfseries\color{theoremTitle},
    bodyfont=\normalfont,
    headpunct={},
    postheadspace=3pt,
    headformat={\color{theoremTitle}\NAME~\NUMBER\color{black}\,--\,\NOTE},
    notefont=\normalfont\bfseries,
    notebraces={}{},
]{theoremStyle}

\declaretheoremstyle[
    spaceabove=2pt,
    spacebelow=2pt,
    headfont=\normalfont\bfseries,
    bodyfont=\normalfont,
    headpunct={},
    postheadspace=\newline,
    headformat={\NAME~\NUMBER~\NOTE},
    notefont=\normalfont\bfseries,
    notebraces={(}{)},
]{exampleStyle}

\declaretheoremstyle[
  spaceabove=3pt,
  spacebelow=3pt,
  headfont=\normalfont\bfseries,
  bodyfont=\normalfont,
  headpunct={},
  % postheadspace=1em,
  postheadspace=\newline,
%   headformat={\color{black}\thmname{#1}~\thmnumber{#2}\,--\,#3},
]{myproblemstyle}

\declaretheoremstyle[
    spaceabove=3pt,
    spacebelow=3pt,
    headfont=\normalfont\bfseries,
    bodyfont=\normalfont,
    headpunct={},
    postheadspace=\newline,
]{remark}

% \declaretheoremstyle[
%   spaceabove=6pt,
%   spacebelow=6pt,
%   headfont=\bfseries,
%   notefont=\bfseries,
%   notebraces={(}{)},
%   bodyfont=\normalfont,
%   headpunct={},
%   postheadspace=1em,
%   numbered=no
% ]{mystyle-no-number}

\declaretheorem[
    style=theoremStyle,
    name=Theorem,
    numberlike=subsubsection
]{theorem}
\tcolorboxenvironment{theorem}{
    enhanced,
    colback=shallowOrange,
    boxrule=0pt,
    frame hidden,
    breakable,
    left=2mm,
    top=1mm,
    bottom=1mm,
    right=0mm,
    borderline west={1mm}{0pt}{deepOrange},
    % before skip=10pt,
    % after skip=10pt,
}
\declaretheorem[
    style=theoremStyle,
    name=Definition,
    numberlike=subsubsection,
    % shaded={
    %     bgcolor=shallowGreen,
    % },
]{definition}
\tcolorboxenvironment{definition}{
    enhanced,
    colback=shallowPurple,
    boxrule=0pt,
    frame hidden,
    breakable,
    left=2mm,
    top=0mm,
    bottom=0mm,
    right=0mm,
    borderline west={1mm}{0pt}{deepPurple},
}
\declaretheorem[
    style=myproblemstyle,
    name=Problem,
    numberlike=subsubsection,
]{problem}
\declaretheorem[
    style=theoremStyle,
    name=Proposition,
    numberlike=subsubsection,
]{proposition}
\tcolorboxenvironment{proposition}{
    enhanced,
    colback=shallowOrange,
    boxrule=0pt,
    frame hidden,
    breakable,
    left=2mm,
    top=0mm,
    bottom=0mm,
    right=0mm,
    borderline west={1mm}{0pt}{deepOrange},
    % before skip=10pt,
    % after skip=10pt,
}
\declaretheorem[
    style=theoremStyle,
    name=Lemma,
    numberlike=subsubsection,
]{lemma}
\tcolorboxenvironment{lemma}{
    enhanced,
    colback=shallowOrange,
    boxrule=0pt,
    frame hidden,
    breakable,
    left=2mm,
    top=0mm,
    bottom=0mm,
    right=0mm,
    borderline west={1mm}{0pt}{deepOrange},
    % before skip=10pt,
    % after skip=10pt,
}

\declaretheorem[
    style=theoremStyle,
    name=Corollary,
    numberlike=subsubsection,
]{corollary}
\tcolorboxenvironment{corollary}{
    enhanced,
    colback=shallowOrange,
    boxrule=0pt,
    frame hidden,
    breakable,
    left=2mm,
    top=0mm,
    bottom=0mm,
    right=0mm,
    borderline west={1mm}{0pt}{deepOrange},
    % before skip=10pt,
    % after skip=10pt,
}

\declaretheorem[
    style=theoremStyle,
    name=Algorithm,
    numberlike=subsubsection,
]{algorithm}
\tcolorboxenvironment{algorithm}{
    enhanced,
    colback=shallowRed,
    boxrule=0pt,
    frame hidden,
    breakable,
    left=2mm,
    top=0mm,
    bottom=0mm,
    right=0mm,
    borderline west={1mm}{0pt}{deepRed},
}

\declaretheorem[
    style=theoremStyle,
    name=Principle,
    numberlike=subsubsection,
]{principle}
\tcolorboxenvironment{principle}{
    enhanced,
    colback=shallowGreen,
    boxrule=0pt,
    frame hidden,
    breakable,
    left=2mm,
    top=0mm,
    bottom=0mm,
    right=0mm,
    borderline west={1mm}{0pt}{deepGreen}
}
\declaretheorem[
    style=remark,
    name=Remark,
    numberlike=subsubsection,
]{remark}

\declaretheorem[
    style=exampleStyle,
    name=Example,
    numberlike=subsubsection,
]{example}

\declaretheorem[
    name=Exercise
    numberlike=subsubsection,
]{exercise}
% \newtheorem{example}[subsubsection]{Example}
% \newtheorem{exercise}[subsubsection]{Exercise}
% \newtheorem{remark}[subsubsection]{Remark}


%%%%%%%%%%%%%%%%%%%%%%%%%%%%%%%%%%%%%%%%%%%%%%%%%%%%%%%%%%%%%%%%%%%%%%%%%%%%%%%

%%%%%%%%%%%%%%%%%%%%%%%%%%%%%%% Plotting Settings %%%%%%%%%%%%%%%%%%%%%%%%%%%%%
\usepgfplotslibrary{colorbrewer}
\pgfplotsset{width=8cm,compat=1.9}
%%%%%%%%%%%%%%%%%%%%%%%%%%%%%%%%%%%%%%%%%%%%%%%%%%%%%%%%%%%%%%%%%%%%%%%%%%%%%%%

%%%%%%%%%%%%%%%%%%%%%%%%%%%%%%% Title & Author %%%%%%%%%%%%%%%%%%%%%%%%%%%%%%%%
\title{Linear Algebre 2}
\author{Jiaqi Wang}
%%%%%%%%%%%%%%%%%%%%%%%%%%%%%%%%%%%%%%%%%%%%%%%%%%%%%%%%%%%%%%%%%%%%%%%%%%%%%%%
