\chapter{Exploratory Data Analysis}

\section{Data Types and Representations}
\begin{definition}[Data and Information]
    In data analysis we are interested in the following two concepts:
    \begin{itemize}
        \item Data - raw numbers, facts, etc.
        \item Information - structured, meaningful and useful numbers, facts
    \end{itemize}
\end{definition}

\begin{definition}[Data types]
    We can classify data into two types:
    \begin{itemize}
        \item Categorical - data that has no intrinsic numerical value
            \begin{itemize}
                \item Nominal - two or more outcomes that have no natural order, e.g. movie genre, hair color, etc.
                \item Ordinal - two or more outcomes that have a natural order, e.g. movie ratings, level of education, etc.
            \end{itemize}

        \item Numerical - data that has an intrinsic numerical value
            \begin{itemize}
                \item Discrete - data that can attain certain values (typically integers). E.g. the number of days
                    with sunshine in a certain year, the number of traffic incidents.
                \item Continuous - data that can attain any value on a given measurement scale.
                    \begin{itemize}
                        \item Interval data - equal intervals represent equal differences, there is \emph{no fixed "zero point"}.
                            e.g. temperature in Celsius, clock time, birth year, etc.
                        \item Ratio data - both differences and ratios make sense; there is a \emph{fixed "zero point"}.
                            e.g. temperature in Kelvin, height, weight, movie budget, distance, time duration, etc.
                    \end{itemize}
            \end{itemize}
    \end{itemize}
\end{definition}

\begin{remark}[Key points]
    Some key points to remember:
    \begin{itemize}
        \item The difference between continuous and discrete data: \\
            Discrete data has "gaps".
        \item Why are distances ratio data? \\
            Because 0 km is an absolute minimum (and 10 km is twice as long as 5 km).
        \item Why is temperature in degrees Celsius interval data? \\
            Because 0 degrees Celsius is not an absolute minimum (and 10 degrees Celsius is not twice as hot as 5 
            degrees Celsius).
        \item Is temperature difference in Celsius not ratio data either? \\
            It is ratio data, since the difference of $0^\circ$C represents a fixed minimum.
        \item Can data represented by numbers be categorical? \\
            Yes, e.g. sectors of a stadium (nominal), or movie ratings (ordinal).
            No arithmetic operations can be performed on these numbers.
    \end{itemize}
\end{remark}

\begin{remark}[Logarithmic scale data]
    Logarithmic scale data is numerical, but it is neither ratio nor truly interval.
    \begin{itemize}
        \item Logarithmic scale data is a special case of ratio data.
        \item It is used to represent data that spans a large range of values.
        \item It is used to represent data that is highly skewed.
    \end{itemize}
\end{remark}

\section{Tables}
\begin{definition}[Tables]
    Tables are a common way to represent data.
    \begin{itemize}
        \item Each row represents an observation.
        \item Each column represents a variable.
    \end{itemize}

    They are good for reading off values and drawing attention to actual values.

    There are two types of tables
    \begin{itemize}
        \item Reference tables - stores "all" raw data in a table so that it can be looked up easily.
        \item Demonstration tables - stores a subset of the data in a table to demonstrate a point.
    \end{itemize}
\end{definition}

\section{Investigating data in tables}

\subsection{Questions you should ask}
\begin{itemize}
    \item What is the \emph{unit of observation}?
    \item What is the \emph{unit of measurement}?
    \item Do the values make sense when you compare columns or rows?
    \item What is the \emph{data type}?
    \item Which column/row has the largest values?
    \item Which column/row has the smallest values?
    \item What is the \emph{population}?
    \item What is the \emph{sample}?
\end{itemize}



% \begin{definition}[Frequency table]
%     A frequency table is a table that shows the number of observations in each category of a categorical variable.
% \end{definition}
%
% \begin{definition}[Relative frequency table]
%     A relative frequency table is a table that shows the proportion of observations in each category of a categorical variable.
% \end{definition}

\section{Going beyond plots}
Plots help us to explore and give clues.

Numerical summaries like averages help us to document essential features of
data sets.

One should use both plots and numerical summaries. They complement each
other.

Numerical summaries are often called statistics or summary statistics (note the
double meaning of the word: both a scientific field and computed numbers).

\section{Summary Statistics}
There are different types of summary statistics:
\begin{itemize}
    \item Level - location summary statistics \textrightarrow what are "typical" values?
    \item Spread - scale summary statistics \textrightarrow how much do values vary?
    \item Relation - association summary statistics \textrightarrow how do values of different quantities vary simultaneously?
\end{itemize}


\subsection{Location Summary Statistics}
\begin{definition}[Location summary statistics]
    Location summary statistics are statistics that describe the location of the data.
    \begin{itemize}
        \item Mean (average) - $ \bar{x} = \displaystyle\frac{1}{n}\sum_{i=1}^{n}x_i $
        \item Median - the value separating the higher half from the lower half of a data set.
        \item Mode - the most frequently occurring value in a data set (it may not be unique).
    \end{itemize}
\end{definition}

\begin{remark}
    The mean is sensitive to outliers, while the median and the mode are not.

    The mean can be misleading / difficult to interpret for non-symmetric data sets.
\end{remark}

\begin{definition}[Percentiles]
    Percentiles are a way to describe the location of a data point in a data set.
    \begin{itemize}
        \item The $p$-th percentile is the value such that $p$ percent of the data is below it.
        \item The 0-th percentile is the minimum.
        \item The 100-th percentile is the maximum.
        \item For a dataset with $n$ observations, the interval between neighboring percentiles is $\displaystyle\frac{100}{n-1}\%$.
        \item The median is the 50-th percentile.
        \item The first quartile is the 25-th percentile.
        \item The third quartile is the 75-th percentile.
    \end{itemize}
\end{definition}

\begin{definition}[Quartiles]
    Quartiles are a way to describe the location of a data point in a data set.
    \begin{itemize}
        \item The first quartile is the 25-th percentile.
        \item The second quartile = 20-th percentile = median.
        \item The third quartile is the 75-th percentile.
        \item The interquartile range (IQR) is the difference between the third and the first quartile.
    \end{itemize}
\end{definition}

\subsection{Scale Summary Statistics}
\begin{definition}[Scale summary statistics]
    Scale summary statistics are statistics that describe the scale of the data.
    \begin{itemize}
        \item Range - the difference between the maximum and the minimum.
        \item Interquartile Range (IQR) - the difference between the third and the first quartile.
        \item Sample Variance - $\sigma^2 = \displaystyle\frac{1}{n-1}\sum_{i=1}^{n}(x_i - \bar{x})^2$ (sometime denoted as $s^2$).
        \item Sample Standard Deviation - $\sigma = \sqrt{\displaystyle\frac{1}{n-1}\sum_{i=1}^{n}(x_i-\bar{x})^2}$
        \item Median Absolute Deviation (MAD) - the median of the absolute deviations from the median.
    \end{itemize}
    
    The higher these statistics, the more spread/variability in the data.
\end{definition}

\begin{remark}
    \begin{itemize}
        \item The standard deviation has right (physical) unit.
        \item The variance is more convenient for calculations.
        \item The range, variance and standard deviation are sensitive to outliers, while IQR and MAD are not. 
        \item The standard deviation can be used as a general unit to describe variability.
    \end{itemize}
\end{remark}

\subsubsection*{Standardization}
\begin{definition}[z-score]
    The z-score $z$ (=normalized value) of values $x$ shows is a standardized value of a data point.
    \begin{itemize}
        \item It is the number of standard deviations that a data point is above or below the mean.
        \item It is a dimensionless quantity.
        \item And it is calculated as $z = \displaystyle\frac{x-\bar{x}}{\sigma}$.
    \end{itemize}
\end{definition}

\begin{remark}[Rule of thumb]
    Observations with a z-score larger than 2.5 or smaller than -2.5 are considered extreme (\emph{outliers}).
\end{remark}

\subsection{Association Summary Statistics}
Association statistics try to quantify the strength of the relation between two variables (attributes).

\begin{definition}[Sample Covariance]
    The covariance of two variables $x$ and $y$ is a measure of how much they vary together.
    \begin{itemize}
        \item It is calculated as $s_{xy} = \displaystyle\frac{1}{n-1}\sum_{i=1}^{n}(x_i-\bar{x})(y_i-\bar{y})$.
        \item It is sensitive to the scale of the variables.
        \item The sample covariance is not bounded.
    \end{itemize}
\end{definition}

In order to be useful, the smaple covariance needs to be scaled / standardized.

\begin{definition}[Sample Correlation]
    The correlation of two variables $x$ and $y$ is a standardized measure of how much they vary together.
    \begin{itemize}
        \item It is calculated as $r = \displaystyle\frac{cov(x,y)}{\sigma_x\sigma_y}$.
        \item It is bounded between -1 and 1.
    \end{itemize}
\end{definition}

\begin{remark}[Some remarks about correlation]
    
    \begin{itemize}
        \item The correlation is a measure of linear association.
        \item The correlation is not a measure of causation.
        \item The correlation is sensitive to outliers.
    \end{itemize}

\end{remark}

\begin{remark}[Beware of spurious correlations!]
    \emph{Spurious correlations} are statistical phenomenon where two variables
    \emph{appear to be related or correlated}, but \emph{their relation is coincidental}, due to chance
    or the influence of an unaccounted-for variable.
\end{remark}

\section{Elementary Statistical Plots}

Typical data analysis questions are:
\begin{itemize}
    \item Are values as you expect?
    \item What are typical sizes?
    \item How much do values vary?
    \item How are values distributed?
    \item Are there exceptional values?
\end{itemize}

Graphs/Plots/Charts are useful for answering these questions.

\subsection{Dot plots / Strip plots}
\begin{definition}[Dot plot]
    A dot plot is a type of plot that displays data values as dots on a number line.

    \begin{itemize}
        \item One-dimensional numerical data
        \item Plotting actual values of observations
        \item The \emph{jitter} option is used to avoid over-plotting.
        \item Tasks: find clusters, outliers, gaps, etc.
        \item Not suitable for large data sets.
    \end{itemize}
\end{definition}

\subsection{Scatter plot}
\begin{definition}[Scatter plot]
    A scatter plot is a type of plot that displays values of two numerical variables as points in a plane.

    \begin{itemize}
        \item Two-dimensional numerical data
        \item Plotting actual values of observations
        \item Tasks: investigate relations; find clusters, outliers, gaps, etc.
        \item Not suitable for large data sets.
    \end{itemize}

\end{definition}

\subsection{Histogram}
\begin{definition}[Histogram]
    A histogram is a type of plot that displays the distribution of numerical data.

    \begin{itemize}
        \item One-dimensional numerical data
        \item Plotting counts of observations in bins (interval of values)
        \item Tasks: investigate distribution; compare the counts/percentages of bins
        \item Suitable for large data sets.
    \end{itemize}
\end{definition}

\begin{remark}[Bin width sensitivity]
    Histograms are sensitive to the choice of bin width.
    \begin{itemize}
        \item Bin width too small \textrightarrow too many bins \textrightarrow too much detail (wiggly)
        \item Bin width too large \textrightarrow too few bins \textrightarrow too little detail (smooth)
    \end{itemize}
\end{remark}

\begin{remark}[Rule of thumb]
    The number of bins should be $\approx \sqrt{n}$, where $n$ is the number of observations.
\end{remark}

\subsection{Cumulative histogram}
\begin{definition}[Cumulative histogram]
    A cumulative histogram is a type of plot that displays the distribution of numerical data.

    \begin{itemize}
        \item One-dimensional numerical data
        \item Plotting counts of observations in bins (interval of values)
        \item Tasks: investigate distribution; explore unexpected "jumps"; to lookup percentiles or thresholds
    \end{itemize}

\end{definition}

\subsection{Bar chart}
\begin{definition}[Bar chart]
    A bar chart is a type of plot that displays the distribution of categorical data.

    \begin{itemize}
        \item One-dimensional categorical data
        \item Also used for two-dimensional data with one categorical variable and one numerical variable.
        \item Plotting counts of observations in categories
        \item Tasks: lookup and compare values.
    \end{itemize}
\end{definition}

\section{Advanced Statistical Plots}
\subsection{Box plot}
\begin{definition}[Box and whisker plot]
    Convenient way to display summary statistics.

    \begin{itemize}
        \item Box: IQR (interquartile range)
        \item Outliers: dots/crosses/diamonds... for all values above/below $1.5 \times IQR$
        \item Endpoints of whiskers: min and max (excluding outliers)
        \item Median: line in the box
        \item Mean: dot in the box (Optional)
    \end{itemize}

    Insights are:
    \begin{itemize}
        \item Is the plot symmetric with the median in the middle?
        \item Are there outliers?
    \end{itemize}

\end{definition}

\subsection{Kernel density plot}
\begin{definition}[Kernel density plot]
    The kernel density function shows the likelihood of finding a data point at a specific value.

    \begin{itemize}
        \item The total area under the curve is 1.
        \item The area under the curve between two values is the probability of finding a data point between these values.
        \item It is impossible to determine the minimal and maximal values or to find the mode of a dataset based on its kernel density plot!
    \end{itemize}

\end{definition}

\subsubsection{How to generate a kernel density plot?}
\begin{itemize}
    \item Choose a kernel function and a bandwidth to be taken around each data point.
    \item Generate a kernel with the chose bandwidth for every data point in the data set
    \item For a data set with $n$ points, the area under the curve of the kernel of a point is $\displaystyle\frac{1}{n}$.
    \item The kernel density plot is the sum of the kernels of the data points.
    \item Bandwidth choice is important!
\end{itemize}

\subsection{Typical distribution shapes}
\begin{itemize}
    \item Symmetric
    \item Skewed (right or left skewed)
    \item Types of peaks: kurtosis 
    \item Unimodal (1 peak)
    \item Bimodal (2 peaks)
    % \item Uniform
    % \item U-shaped
\end{itemize}

\subsection{Violin plot}
\begin{definition}[Violin plot]
    A violin plot is a combination of a box plot and a kernel density plot.

    \begin{itemize}
        \item Global shape of box-and-whisker plot
        \item Local details of kernel density plot
    \end{itemize}

\end{definition}

\subsection{Empirical cumulative distribution function (ECDF)}
\begin{definition}[Empirical cumulative distribution function (ECDF)]
    The empirical cumulative distribution function (ECDF) is a function that maps a value to the fraction of values that are smaller or equal to it.

    \begin{itemize}
        \item It is a step function.
        \item It is a non-parametric estimator of the cumulative distribution function (CDF).
        \item It is a good way to visualize the distribution of a data set.
        \item It is a good way to compare the distribution of two data sets.
    \end{itemize}

\end{definition}

\subsubsection{Interpreting ECDF}
\begin{itemize}
    \item For all $x$ \emph{smaller} than the \emph{minimal} value of the data set, the ECDF is 0.
    \item For all $x$ \emph{larger or equal} to the \emph{maximal} value of the data set, the ECDF is 1.
\end{itemize}
