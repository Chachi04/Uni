%!LW recipe=withCite
%%%%%%%%%%%%%%%%%%%%%%%%%%%%% Define Article %%%%%%%%%%%%%%%%%%%%%%%%%%%%%%%%%%
\documentclass{article}
%%%%%%%%%%%%%%%%%%%%%%%%%%%%%%%%%%%%%%%%%%%%%%%%%%%%%%%%%%%%%%%%%%%%%%%%%%%%%%%

%%%%%%%%%%%%%%%%%%%%%%%%%%%%% Using Packages %%%%%%%%%%%%%%%%%%%%%%%%%%%%%%%%%%
\usepackage{geometry}
\usepackage{amssymb}
\usepackage{amsmath}
\usepackage{amsthm}
\usepackage{pgfplots}
\usepackage{hyperref}
\usepackage{cleveref}
\usepackage{indentfirst}
%%%%%%%%%%%%%%%%%%%%%%%%%%%%%%%%%%%%%%%%%%%%%%%%%%%%%%%%%%%%%%%%%%%%%%%%%%%%%%%

% Other Settings
\usetikzlibrary{calc}

%%%%%%%%%%%%%%%%%%%%%%%%%% Page Setting %%%%%%%%%%%%%%%%%%%%%%%%%%%%%%%%%%%%%%%
\geometry{a4paper}


%%%%%%%%%%%%%%%%%%%%%%%%%% Define vector command %%%%%%%%%%%%%%%%%%%%%%%%%%%%%%%%%%%%%%%
\renewcommand{\v}[1]{\underline{#1}}

\begin{document}
    % \maketitle
    \include{titlepage}
    \section{Problem Description}
    In $ \triangle ABC$ (the points $A,B,C$ are non co-linear),
    $P$ is the midpoint of the segment $BC$ and
    $R$ is the point on the line $AB$ such that $A$
    is the midpoint of the segment $BR$. Use vectors to determine the point of intersection
    $Q$ of lines $PR$ and $AC$, and show that $AQ:QC = 1:2$.

    \begin{figure}[h]
        \begin{center}
            \begin{tikzpicture}
                \def\A{$A$}
                \def\B{$B$}
                \def\C{$C$}
                \def\P{$P$}
                \def\Q{$Q$}
                \def\R{$R$}
                \coordinate[label=below:\A] (A) at (0,0) ;
                \coordinate[label=right:\B] (B) at (5,0) ;
                \coordinate[label=above:\C] (C) at (1,4) ;
                \coordinate[label=right:\P] (P) at ($(B)!0.5!(C)$);
                \coordinate[label=left:\R] (R) at (-5,0);
                \coordinate[label=above:\Q] (Q) at ($(A)!1/3!(C)$);
                \draw (A) -- (B) -- (C) -- cycle;
                \draw[dashed] (R) -- (A) ;
                \draw[dashed] (R) -- (P) ;
                \node[circle, fill=blue, inner sep=1pt] at (A) {};
                \node[circle, fill=blue, inner sep=1pt] at (B) {};
                \node[circle, fill=blue, inner sep=1pt] at (C) {};
                \node[circle, fill=blue, inner sep=1pt] at (P) {};
                \node[circle, fill=blue, inner sep=1pt] at (R) {};
                \node[circle, fill=blue, inner sep=1pt] at (Q) {};
            \end{tikzpicture}
        \end{center}
        \caption{Example Construction}
    \end{figure}

    \section{Solution}
    In solving this problem, we will mainly use vector techniques.
    In geometric exercises that involve vectors, we first need to choose where to put the origin
    if it is not explicitly specified. Without loss of generality, we can choose the point $A$ as our origin and assign the vectors
    $\v{b}, \v{c}, \v{p}, \v{q}, \v{r}$ to the points $B, C, P, Q, R$, respectively. \\

    The way we solve the exercise is first we find vector representations of the lines $RP$ and $AC$, then by finding their point of intersection we find $Q$. \\

    Since $A$ is the origin and the midpoint of the segment $BR$, we get $\v{r} = - \v{b}$. Additionally, since $P$ is the
    midpoint of the segment $BC$, we get:
    $$\v{p} = \v{b} + \frac{1}{2}\left(\v{c} - \v{b}\right) = \frac{1}{2} \v{b} + \frac{1}{2} \v{c}.$$ \\
    The line $RP$ can be defined as $RP: \v{x} = \v{r} + \lambda \left(\v{p} - \v{r}\right)$, where $\lambda \in \mathbb{R}$ \cite{vecLineRep}, using our above defined substitutions, we thus get:
    \begin{equation}\label{RP}
    RP: \v{x} = - \v{b} + \lambda \left(\frac{3}{2} \v{b} + \frac{1}{2} \v{c}\right)
    \end{equation}
    And the parametric equation for $AC$ is
    \begin{equation}\label{AC}
    AC: \v{x} = \mu\v{c}, \text{ where } \mu \in \mathbb{R}
    \end{equation}
    since $AC$ passes through the origin ($A$) and has direction $\v{c}$.
    Intersecting \cref{RP} and \cref{AC} we get:
    \begin{align*}
        - \v{b} + \lambda \left(\frac{3}{2} \v{b} + \frac{1}{2} \v{c}\right) &= \mu \v{c} \\
        - \v{b} + \lambda \left(\frac{3}{2} \v{b} + \frac{1}{2} \v{c}\right) - \mu \v{c} &= \v{0} \\
        \left(-1 + \frac{3}{2} \lambda\right) \v{b} + \left(\frac{1}{2}\lambda - \mu\right)\v{c} &= \v{0} \\
    \end{align*}
    The points $B$ and $C$ are not co-linear, therefore, $\v{b}$ and $\v{c}$ are linearly independent. Thus from \cite{linIndepVecs} we get:
    $$
        \begin{cases}
            -1 + \frac{3}{2} \lambda = 0 \\
            \frac{1}{2}\lambda - \mu = 0 \\
        \end{cases}
        \iff
        \begin{cases}
            \frac{3}{2} \lambda = 1 \\
            \mu = \frac{1}{2} \lambda \\
        \end{cases}
        \iff
        \begin{cases}
            \lambda = \frac{2}{3} \\
            \mu = \frac{1}{3} \\
        \end{cases}
    $$
    Since $\v{q} \in AC$, we can write $$\v{q} = \mu \v{c} = \frac{1}{3} \v{c}.$$
    From this, we get:
    $$AQ = \lVert\frac{1}{3} \v{c}\rVert = \frac{1}{3}\lVert\v{c}\rVert$$
    and $$QC = AC - AQ = \lVert\v{c}\rVert - \frac{1}{3}\lVert\v{c}\rVert =  \frac{2}{3}\lVert\v{c}\rVert.$$
    Having calculated the lengths of $AQ$ and $QC$ we see that
    \begin{align*}
    \frac{AQ}{QC} &= \frac{\frac{1}{3}\lVert\v{c}\rVert}{\frac{2}{3}\lVert\v{c}\rVert} \\
    \frac{AQ}{QC} &=\frac{1}{2} \\
    \therefore AQ : QC &= 1 : 2
    \end{align*}

    \section{Conclusion}
    We were able able to prove a geometric property using vector arithmetic and, indeed, we found that the intersection $Q$ of lines $RP$ and $AC$ divides the segment $AC$
    into two segments $AQ$ and $QC$, where $QC$ is twice the length of $AQ$ (i.e. $AQ:QC = 1:2$).

    Certainly, there are also other ways in which we could have solved this exercise. For instance, instead of choosing $A$ as origin,
    we could have chosen any other given point or even an arbitrary point on the plane. This, however, may lead
    to more complicated expressions for the lines $RP$ and $AC$, thus making it more prone to computational errors.

     Additionally, it may also be possible to solve the given problem using only geometric properties of triangles, however, its investigation is beyond the scope of this article and this course.

\newpage
    \section{Roles of Group Members}

    \begin{itemize}
        \item Jiaqi Wang - document organization, visual organization and final editing
        \item Mil Majerus - solution writing
        \item Jean Nguyen - proof reading
        \item Long Pham - proof reading, visual organization and final editing
    \end{itemize}

    \bibliographystyle{plain}
    \bibliography{references}

\end{document}