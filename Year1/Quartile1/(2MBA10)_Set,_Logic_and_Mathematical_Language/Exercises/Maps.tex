\subsection{Exercises}

\subsubsection{Which of the following relations are maps from $A = \{1, 2, 3, 4\}$ to $A$?}
\begin{enumerate}[label=(\alph*)]
    \item $\{(1,3),(2,4),(3,1),(4,2)\}$: As for all $a \in A$ there is one and only $b \in A$, the relation is a map.
    \item $\{(1,3),(2,4)\}$: As 3 and 4 are not mapped to any element, this relation is not a map from $A$ to $A$.
    \item $\{(1,1),(2,2),(3,3),(4,4),(1,3),(2,4),(3,1),(4,2)\}$: As elements from A are not mapped uniquely
    to another element, it is not a map.
    \item $\{(1, 1), (2, 2), (3, 3), (4, 4)\}$: As for all $a \in A$ there is one and only $b \in A$, the relation is a map.
\end{enumerate}

\subsubsection{Suppose f and g are maps from R to R defined by $f (x) = x^2$ and $g(x) = x + 1$ for all $x \in R$.
What is $g \circ f$ and what is $f \circ g$?}

$$g \circ f = g(f(x)) = x^2+1$$
$$f \circ g = f(g(x)) = (x+1)^2$$

\subsubsection{Which of the following maps is injective, surjective or bijective?}
\begin{enumerate}[label=(\alph*)]
    \item $f: \mathbb{R} \rightarrow \mathbb{R}, f(x) = x^2$ for all $x \in \mathbb{R}$\\
    Take $b = -4$, then there is no $a \in A$ such that $f(a) = b$. Therefore it is not surjective.\\
    Take $c = -2, d=2, e=4$, then $f(c) = f(d) = e$. Therefore it is not injective.\\
    Consequently, it is not bijective.

    \item $f: \mathbb{R} \rightarrow \mathbb{R}_{\ge 0}, f(x) x^2$ for all $x \in \mathbb{R}$\\
    It is surjective, since $\forall b \in \mathbb{R}_{\ge 0}[\exists a \in \mathbb{R}: f(a) = b]$ \\
    Take $c = -2, d=2, e=4$, then $f(c) = f(d) = e$. Therefore it is not injective.\\
    \item $f: \mathbb{R}_{\ge 0} \rightarrow \mathbb{R}_{\ge 0}, f(x) = x^2$ for all $x \in \mathbb{R}$
    It is bijective, since: \\
    \begin{enumerate}[label=\arabic*.]
        \item $\forall b \in \mathbb{R}_{\ge 0}[\exists a \in \mathbb{R}: f(a) = b]$
        \item there is a one-to-one relation.:we
    \end{enumerate}
\end{enumerate}

\subsubsection{Suppose $R_1$ and $R_2$ are relations on a set $S$ with $R_1;R_2 = I$ and $R_2;R_1 = I$. Prove that both $R_1$ and $R_2$ are bijective maps.}

\subsubsection{Let R be a relation from a finite set S to a finite set T with adjacency matrix A. Prove the
following statements:}

\subsubsection{}

\subsubsection{}

\subsubsection{}

\subsubsection{}

\subsubsection{Let $g$ be a permutation in $Sym_n$.Show that if $i\in \text{support}(g)$, then $g(i) \in \text{support}(g)$.}
If $i \in \text{support}(g)$, then $g(i) \ne i \iff g(g(i)) \ne g(i)$

\subsubsection{How many elements of $Sym_5$ have the cycle structure 2, 3?}
First we choose two elements two permute and get the 2-cycle. The number of that
is 5 choose 2 $5 \choose 2$. Then we permute the other 3 remaing elements which gives us
$3!$ permutations, however, we need to subtract the 2-cycles that we get from those permutations,
which are $3 \choose 2$. And also the case where we have 3 1-cycles. \newline
So for the number of permutations with cycle structure 2 3 we get:
$$ {5 \choose 2} \cdot \left(3! - {3 \choose 2} - 1\right) = 20$$

\subsubsection{Let g be the permutation}
$$ (1,2,3)\cdot(2,3,4)\cdot(3,4,5)\cdot(4,5,6)\cdot(5,6,7)\cdot(6,7,8)\cdot(7,8,9) $$
\textbf{in $Sym_6$}

\begin{enumerate}[label=(\alph*)]
    \item Write g as a product of disjoint cycles. \\
    $(9 \ 8) \cdot (7) \cdot (6) \cdot (5) \cdot (4) \cdot (3) \cdot (2 \ 1)$
    \item Calculate the fixed points of g  \\
    fixed(g) = $\{3,4,5,6,7\}$
    \item Write $g^{-1}$ as a product of disjoint cycles \\
    $g^{-1} = g$, as applying a 2-cycle twice give us the identity.
    \item is g even? \\
    g is composed two odd cycles, thus their product is even
\end{enumerate}

\subsubsection{}
\begin{enumerate}[label=(\alph*)]
    \item \textbf{If the permutations $g$ and $h$ in Sym${}_n$ have disjoint supports, then $g$ and $h$ commute, i.e $g \cdot h = h \cdot g$. Prove this.} \newline
    Since they are $g$ and $h$ are disjoint, so support($g$) $\cap$ support($h$) $= \emptyset$. Then it would not matter in which order
    we take $g$ and $h$ as they will not permute the same element, thus they commute.

    \begin{proof}[Proof]
        Let $g$ and $h$ be disjoint permutations \newline
        Let $i \in Fix(g)$. Then:
        $$ hg(i) = h(i) $$
        % whereas:
        % $$ gh(i) = g(h(i)) $$
        Assume $$ h(i) \notin Fix(g) $$
        So $$ h^2(i) = h(i) $$
        Then
        $$ h^{-1}h^2(i) = h^{-1}h(i) $$
        $$ h(i) = i $$
        Hence $$ i \in Fix(h) $$
        However, this contradicts our assumption $ i \in Fix(g) $ \newline
        Therefore:
        $$ h(i) \in Fix(g) $$
        So
        $$ gh(i) = h(i) = hg(i) $$
    \end{proof}
\end{enumerate}


