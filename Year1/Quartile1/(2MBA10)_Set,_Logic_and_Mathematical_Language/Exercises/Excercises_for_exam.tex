\section{Exercises for exam}

\subsection{Logic}

\begin{exercise}
    The statements $P$ and $Q$ can be true or false.  When is the statement
    \par when is the statement $$ R = (P \land Q) \lor ((\lnot P \lor Q) \land (P \lor \lnot Q)) $$true?
\end{exercise}
\begin{enumerate}
    \item If $P$ and $Q$ are true, then $ P \land Q $ hence $R$ is true
    \item If $P$ is true and $Q$ is false, then $ P \land Q $ and $ \lnot P \lor Q $ are false, hence $R$ is false
    \item If $P$ is false and $Q$ is true, then $ P \land Q $ and $ P \lor \lnot Q $ are false, hence $R$ is false
    \item If $P$ and $Q$ are false, then $ \lnot P \lor Q $ and $ P \lor \lnot Q $ are true, hence $R$ is true
\end{enumerate}

\begin{exercise}
    Prove or disprove the following statement: \par
    For all statements $ p,q,r $ we have $ ((p \lor q) \land r) \iff ((p \land r) \lor (q \land r)) $
\end{exercise}
Using the distribive property of the $ \land \text{ over } \lor $ we get:
$$ (p \lor q) \land r = (p \land r) \lor (q \land r) $$
Thus we see that $ ((p \lor q) \land r) \iff ((p \land r) \lor (q \land r)) $

\begin{exercise}
    Prove or disprove the following statement:
    for all statements $P,Q$ and $R$ it holds that:
    $$ \left[(P \implies R) \lor (P \implies Q)\right] \iff \left[P \implies (Q \lor R) \right] $$
\end{exercise}
When $P$ is true we get $ \text{true} \iff \text{true} $ which is true.
When $P$ is false we get $ R \lor Q \iff Q \lor R $ which is also true.
Hence for all $P,Q,R$ the statement is true.

\subsubsection{Sets}
\begin{exercise}
    Prove or disprove
    $$ \forall x \in U \left[x \in (A \cap B) \implies (x \in A \lor x \in B)\right] \iff A=B $$
\end{exercise}
The statement is false and a counter example is $A = \left\{1,2\right\}, B = \left\{1\right\}$

\begin{exercise}
    Prove or disprove:
    For all sets $A,B$ and $C$ we have: $A$
\end{exercise}


\subsection{Final Exam 2022/2023}
\begin{exercise}
    Which statement is true
    \begin{enumerate}
        \item For all sets $A$ and $B$ we have if $ \mathscr{P}(A) = \mathscr{P}(B) $, then $ A = B $
        \item For all sets $A,B,C$ we have $ (A \cup B = A \cup C \land B \subseteq C) \implies A = B $
    \end{enumerate}
\end{exercise}

