\subsection{Exercises}

\subsection{Homework}
\subsubsection*{Ex 5}
Let's prove that if $x$ is an element of order $\Phi(n)$ in $\mathbb{Z}/n\mathbb{Z}$ (where $\Phi(n)$ is Euler's totient function), then every invertible element in $\mathbb{Z}/n\mathbb{Z}$ is a power of $x$.

We'll use a few key concepts:

\begin{enumerate}
    \item The order of an element in a group is the smallest positive integer $k$ such that $x^k$ is the identity element of the group.
    \item Euler's totient function $\Phi(n)$ is the number of positive integers less than or equal to $n$ that are coprime to $n$.
    \item In $\mathbb{Z}/n\mathbb{Z}$, the invertible elements are precisely those that are coprime to $n$ (i.e., gcd$(a, n) = 1$).
\end{enumerate}

Now, let $y$ be an invertible element in $\mathbb{Z}/n\mathbb{Z}$. We want to show that $y$ is a power of $x$. We'll use the properties of Euler's totient function and group theory to prove this.

Since $y$ is invertible, gcd$(y, n) = 1$. Now, consider the group generated by $x$ in $\mathbb{Z}/n\mathbb{Z}$, denoted as $\langle x \rangle$. By definition, the order of $x$ is $\Phi(n)$, which means that all the elements in $\langle x \rangle$ have orders that divide $\Phi(n)$.

We know that $y$ is invertible, so gcd$(y, n) = 1$. This means that $y$ is coprime to $n$ and, therefore, belongs to the group of invertible elements modulo $n$. This group is isomorphic to the group $\langle x \rangle$, so $y$ must also have an order that divides $\Phi(n)$.

Let $k$ be the order of $y$, where $k$ divides $\Phi(n)$. By Lagrange's theorem, in any group, the order of an element divides the order of the group. Since the order of $y$ divides $\Phi(n)$, it also divides $\Phi(n)$. This means that $k$ divides $\Phi(n)$, and since $\Phi(n)$ is the order of $x$, $k$ must be less than or equal to $\Phi(n)$.

Since $x$ has the smallest positive integer order in $\langle x \rangle$ (which is $\Phi(n)$), and $k$ divides $\Phi(n)$, we conclude that $k$ must be $\Phi(n)$. This implies that $y$ has the same order as $x$, so $y = x^t$ for some positive integer $t$.

Therefore, we have shown that every invertible element in $\mathbb{Z}/n\mathbb{Z}$ is a power of $x$, as desired.