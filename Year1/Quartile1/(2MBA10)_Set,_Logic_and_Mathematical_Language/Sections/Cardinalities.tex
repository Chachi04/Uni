\section{Cardinalities}

\subsection{Cardinality}
\begin{definition}[Cardinality]
    Two sets $A$ and $B$ have the same \emph{cardinality} if there exists a
    bijection from $A$ to $B$.
\end{definition}

\begin{example}
    Two finite sets have the same cardinality if and only if thery have the same
    number of elements.
\end{example}

\begin{example}
    The sets $\mathbb{N}$ and $ \mathbb{Z} $ have the same cardinality. Indeed,
    consider the map $ f:\mathbb{N} \rightarrow \mathbb{Z} $ defined by
    $ f(2n) = n $ and $ f(2n+1) = -n $ where $ n \in \mathbb{N} $.
    This map is clearly a bijection
\end{example}

\begin{theorem}[Cardinality as equivalence relation]
    Having the same cardinality is an equivalence relation.
\end{theorem}

\subsection{Countable sets}
\begin{definition}[Finite/Inifinite sets]
    A set is called \emph{finite} if it is empty or has the same cardinality as the set $ \mathbb{N}_n := \left\{1,2,\dots,n\right\} $ and \emph{infinite} otherwise.
\end{definition}

\begin{definition}[Countable/Uncountable sets]
    A set is called \emph{countable} if it is finite or has the same cardinality as the set $ \mathbb{N} $.

    An infinite set that is not countable is called \emph{uncountable}.
\end{definition}

\begin{theorem}[Countable sets in infinite sets]
    Every infinite set contains an infinite countable subset.
\end{theorem}
\begin{proof}[Proof]
    Suppose $A$ is an infinite set. Since $A$ is infinite, we can start enumerating the elements $ a_1,a_2,\dots $ such that all the elements are distinct. This yields a sequence of elements in $A$.
    The set of all the elements in this sequence form a countable subset of $A$.
\end{proof}

\begin{theorem}
    Let $A$ be a set. If there is a surjective map from $ \mathbb{N} $ to $A$, then $A$ is countable.
\end{theorem}
\begin{proof}[Proof]
    Let $ f:\mathbb{N} \rightarrow A $ be a surjection. Then consider the sequence $ f(1), f(2), \dots $. Remove from this sequence (going from left to right) each element that you have seen before. The result is either a finite sequence,
    or an infinite sequence $ f(n_1), f(n_2), \dots $ of which all elements are distinct. In the latter case, consider the map
    $ g: \mathbb{N} \rightarrow A $ with $ g(i) = f(n_i) $. This map is a bijection, which proves $A$ to be countable.
\end{proof}
\begin{corollary}
    Let $A$ be countable and $ f : A \rightarrow B $ surjective, then B is countable.
\end{corollary}

\begin{proof}[Proof]
    Suppose A is a countable set and $ f: A \to B $ a surjective map. If $A$ is finite, then so is B.
    Thus assume that $A$ has infintely many elements. Since $A$ is countable, there is a bijection $ g: \mathbb{N} \to A $.
    But then $ f \circ g $ is a surjection from $ \mathbb{N} $ to B. Hence we can apply the previous result and find a bijection from $ \mathbb{N} $ to $B$. This proves B to be countable.
\end{proof}

\begin{theorem}
    Any subset of a countable set is countable.
\end{theorem}
\begin{proof}[Proof]
    Suppose $A$ is an infinite subset of a countable set $B$. Let $ f : \mathbb{N} \to B $ be bijective and fix an element $ a \in A $.
    Now consider the map $ g : \mathbb{N} \to A $ defined by $ g(x) = f(x) $ if $ f(x) \in A $ and $ g(x) = a $ if $ f(x) \in B \setminus A $.
    Then $g$ is surjective, as $f$ is surjective. Thus A is countable.
\end{proof}

\begin{proposition}
    $ \mathbb{N} \times \mathbb{N} $ is countable.
\end{proposition}
\begin{proof}[Proof]
    Let $ n \in \mathbb{N} $. Let $m$ be maximal with $ \sum_{i=0}^m i < n $. Now let $k= n - \sum_{i=0}^m i$So, $ 1 \leq k \leq m+1 $.
    \par We define $ f : \mathbb{N} \to \mathbb{N} $ in the following way:
    $$ f(n) = (k, m+2-k). $$
    So, in a table this looks as follows:
    \begin{center}
        \begin{tabular}{| c | c | c | c | c |}
            \hline
            $ f(1) = (1,1) $ & $ f(2) = (1,2) $ & $ f(4) = (1,3) $ & $ f(7) = (1,4) $ \\
            \hline
            $ f(3) = (2,1) $ & $ f(5) = (2,2) $ & $ f(8) = (2,3) $ & \dots & \\
            \hline
            $ f(6) = (3,1) $ & $ f(9) = (3,2) $ & \dots & & \\
            \hline
            \vdots & \vdots & & & \\
            \hline
        \end{tabular}
    \end{center}
    By construction, $f$ is injective. Indeed, the $m$ and $k$ are uniquely defined by n. \par
    So it only remains to prove surjectivity. Let $ (k,l) \in \mathbb{N} \times \mathbb{N} $. Set $ m=k+l-2 $. Hence $ (k,l) = (k,m+2-k) $ and $ (k,l) = f(n) $ for $n$ equal to $ \sum_{i=0}^m i+k $.
\end{proof}

\begin{theorem}
    Let $A$ and $B$ be countable sets. Then $ A \times B $ is countable.
\end{theorem}
\begin{proof}[Proof]
    Suppose $ f: \mathbb{N} \to A $ and $ g: \mathbb{N} \to B $ are surjections. The map $ h: \mathbb{N} \times \mathbb{N} \to A \times B $ defined by $ h(i,j) = (f(i), h(i)) $ is surjective.
    So, since $ \mathbb{N} \times \mathbb{N} $ is countable, also $ A \times B $ is countable.
\end{proof}

\begin{proposition}
    The sets $ \mathbb{Z} $ and $ \mathbb{Q} $ are countable.
\end{proposition}

\begin{proof}[Proof]
    The map $ g: \{-1,1\} \times \mathbb{N} \to \mathbb{Z} $ given by $ g(x,y) = xy $ is surjective. Since $ \{-1,1\} \times \mathbb{N} $ is countable, hence $\mathbb{Z}$ is also countable.

    Now let $ f: \mathbb{Z} \times \mathbb{N} \to \mathbb{Q} $ be defined by $ f(i,j) = \frac{i}{j}$ for $ (i,j) \in \mathbb{Z} \times \mathbb{N} $. This is clearly a surjective map.
    Since $\mathbb{Z}$ and $\mathbb{N}$ are countable so is $\mathbb{Z} \times \mathbb{N}$. Hence $\mathbb{Q}$ is also countable.
\end{proof}

\begin{theorem}
    Let $\mathscr{C}$ be a countable collection of countable sets. Then
    $ \bigcup_{A \in \mathscr{C}}A $ is countable.
\end{theorem}
\begin{proof}[Proof]
    For each $ A \in \mathscr{C} $ there exists a bijection
    $ f_A : \mathbb{N} \to A $. Moreover, as $\mathscr{C}$ is countable, there
    exists also a bijection $ g: \mathbb{N} \to \mathscr{C} $. We write
    $ A_i = g(i) $.

    Now consider the map $ f: \mathbb{N} \times \mathbb{N} \to \bigcup_{A\in\mathscr{C}}A $
    defined by $ f(i,j) = f_{A_i}(j) $. This is a surjection. Thus
    $ \bigcup_{A\in\mathscr{C}}A $ is countable.
\end{proof}
\begin{example}
Let $S$ be the set of all finite subsets of $\mathbb{N}$. Then $S = \bigcup_{i\in\mathbb{N}S_i}$, where $S_i$ is the set of subsets of size at most $i$ of $\mathbb{N}$. \par
We already showed that $ \mathbb{N}^i $ is countable. But the map $ (a_1,\dots,a_i) \in \mathbb{N}^i \mapsto \{a_1,\dots, a_i\} \in S_i $ is clearly surjective.
Thus $S_i$ is also countable. Hence $ S = \bigcup_{i\in\mathbb{N}S_i} $ is also countable.
\end{example}

\begin{proposition}
    If $A$ is infinite and $B$ is finite, then $A$ and $A \cup B$ have the same cardinality.
\end{proposition}

\begin{proof}[Proof]
    Assume that $A$ is infinite and, withouth loss of generality, that $A$ and
    $B$ are disjunct. Let $A_0$ be a countable subset of $A$. Then $A_0 \cup B$
    is also countable. Then there exists a bijection $ g: A_0 \cup B \to A_0 $.
    Now define $ f: A \cup B \to A $ by
    $$ f(x) \begin{cases}
        g(x) &\text{if } x \in A_0 \cup B \\
        x &\text{if } x \notin A_0 \cup B,
    \end{cases} $$

    Then clearly $f$ is a bijection between $A \cup B$ and $A$.
\end{proof}

\subsection{Some uncountable sets}
\begin{proposition}
    The set $ \{0,1\}^\mathbb{N} $ is uncountable.
\end{proposition}
\begin{proof}[Proof]
    Let $ F: \mathbb{N} \to \{0,1\}^\mathbb{N} $. By $ f_i $ we denote the
    function $F(i)$ from $\mathbb{N}$ to $ \{0,1\} $.

    We will show that $F$ is not surjective by constructing a function
    $ f \in \{0,1\}^\mathbb{N} $ which is different from all the function
    $f_i$ with $i \in \mathbb{N}$.

    For each $ i \in \mathbb{N} $ let
    $$ f(i) = 0 \text{ if } f_i(i) = 1 \text{ and}$$
    $$ f(i) = 1 \text{ if } f_i(i) = 0 $$

    Clearly, for all $ i \in \mathbb{N} $ we have $ f(i) \ne f_i(i) $ and
    hence $ f \ne f_i $. So $F$ is not surjective. This shows that there is
    no surjection from $\mathbb{N}$ to $\{0,1\}^\mathbb{N}$. In particular,
    $ \{0,1\}^\mathbb{N} $ is not countable.
\end{proof}

\begin{remark}[Cantor's diagonal argument]

\end{remark}

If $A$ is a set, then for each subset $B$ of $A$ we define the
\emph{characteristic function} $ \chi_B: A \to \{0,1\}^\mathbb{N} $ to be the
function that takes the value 1 on all elements in $B$ and the value 0 on all
elements in $A \setminus B$.

Clearly, every element $ f \in \{0,1\}^\mathbb{N} $ is the characteristic
fucntion of the set $ \{a \in A \mid f(a) = 1\} $. So, we find the map
$ B \in \mathscr{A} \mapsto \chi_B $ to be a bijection between
$ \mathscr{P}(A) $ to $ \{0,1\}^\mathbb{N} $.

\begin{corollary}
    The set $ \mathscr{P}(A) $ has the same cardinality as $ \{0,1\}^\mathbb{N} $
    and hence is uncountable.
\end{corollary}

\begin{proposition}
    The interval $ [0,1) $ is uncountable.
\end{proposition}
\begin{proof}[Proof]
    Consider the map $ f \in \{0,1\}^\mathbb{N} \mapsto
    \displaystyle\sum_{i=1}^{\infty}\frac{f(i)}{10^i} $. This map is injective.
    So, if $ [0,1) $ is countable, then so is $ \{0,1\}^\mathbb{N} $, which is
    a contradiction.

    This proves that $ [0,1) $ is uncountable.
\end{proof}

\begin{corollary}
    $ \mathbb{R} $ is uncountable.
\end{corollary}
\begin{proof}[Proof]
    As $ \mathbb{R} $ contains the uncountable subset $ [0,1) $, it is also
    uncountable.
\end{proof}

\begin{theorem}
    If $A$ and $B$ are sets with the same cardinality, then $ \mathscr{P}(A) $
    and $ \mathscr{P}(B) $ also have the same cardinality.
\end{theorem}
\begin{proof}[Proof]
    Suppose $A$ and $B$ have the same cardinality. Let $f : A \to B$ be a
    bijection. Consider the map $\hat{f}:P(A) \to P(B)$ given by $\hat{f}(S)
    = \{f(s) | s \in S\}$. This map is a bijection.
\end{proof}

\begin{corollary}
    If $A$ is an infinite set, then $ \mathscr{P}(A) $ is an uncountable set.
\end{corollary}

\begin{theorem}
    Let $X$ be a set, then $ \mathscr{P}(X) $ does not have the same
    cardinality as $X$.
\end{theorem}

\begin{remark}
    The above theorem shows us that we can get bigger and bigger sets in the following way:
    \begin{align}
        X_1 &:= \mathbb{N} \\
        \text{for } n > 1, X_n &:= \mathscr{P}(X_{n-1})
    \end{align}
\end{remark}

\subsection{Cantor-Schr\"oder-Bernstein Theorem}
\begin{theorem}[Contor-Schr\"oder-Bernstein Tehorem]
    Let $A$ and $B$ be sets and assume that there are two maps $ f:A \to B $ and $ g:B \to A $ which are injective.
    Then there exists a bijection $ h:A \to B $.

    In particular, $A$ and $B$ have the same cardinality.
\end{theorem}

\begin{corollary}
    Let $A$ be a set and assume $B \subseteq A$ has the same cardinality as $A$.
    Then each subset $C$ of $A$ with $B \subseteq C \subseteq A$ has the same cardinality as $A$.
\end{corollary}

\begin{proposition}
    The sets $ \{0,1\}^\mathbb{N} $ and $ [0,1) $ have the same cardinality.
\end{proposition}

\begin{theorem}
    The sets $ \mathbb{R}, \{0,1\}^\mathbb{N}, \mathscr{P}(N) $ have the same cardinality.
\end{theorem}

\begin{theorem}
    The sets $ \mathbb{R}^n $ with $n>0$, and $ \mathbb{R} $ have the same cardinality.
\end{theorem}

\subsection{Additional axioms of set theory}
\begin{principle}[Axiom of Choice]
    Let $ \mathscr{C} $ be a collection of nonempty sets. Then there exists a map
    $$ f: \mathscr{C} \to \bigcup_{A \in \mathscr{C}}A $$
    with $ f(A) \in A $.

    The image of $f$ is a subset of $ \displaystyle\bigcup_{A \in \mathbb{C}}A $.

    The function $f$ is called a \emph{choice function}.
\end{principle}

\begin{principle}
    The following statements are equivalent to the Axiom of Choice.
    \begin{itemize}
        \item For any two sets $A$ and $B$ ther edoes exist a surjective map from $A$ to $B$ or from $B$ to $A$.
        \item The cardinality of an infinte set $A$ is equal to the cardinality of $A \times A$.
        \item Every vector space has a basis.
        \item For every surjective map $ f:A \to B $ there is a map $ g:B \to A $ with $ f(g(b)) = b $ for all $b \in B$.
    \end{itemize}
\end{principle}

\begin{principle}[Axiom of Regularity]
    Let $X$ be a nonempty set of sets. Then $X$ contains an element $Y$ with $ X \cap Y = \emptyset $.
\end{principle}