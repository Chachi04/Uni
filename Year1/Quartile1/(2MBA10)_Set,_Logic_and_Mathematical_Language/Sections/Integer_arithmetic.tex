\section{Integer Arithmetic}



\subsection{Divisors and multiples}
\begin{definition}
    Let $ a,b \in \mathbb{Z} $.
    \begin{itemize}
        \item We call $b$ a divisor of $a$, if there is an integer $q$ such that $a=q\cdot b$
        \item If $b$ is a non-zero divisor of $a$ then the (unique) integer $q$ with $ a=q\cdot b $ is called the \emph{quotient} of $a$ by $b$ and denoted by $ \frac{a}{b}$, $a/b$ or quot$(a,b)$.
    \end{itemize}
    If $b$ is a divisor of $a$, we also say that $b$ \emph{divides} $a$, or $a$ is a \emph{multiple} of $b$, or $a$ is \emph{divisible} by $b$. We write this as $ b|a $
\end{definition}

\begin{example}
    If $ a = 13 $ and $ b = 5 $ then $b$ does not divide $a$. However, if
    $ a = 15 $ and $ b = 5 $, then $b$ does divide $a$.
\end{example}

\begin{example}
    For all integers $n$ we find $ n-1 $ to be a divisor of $ n^2 - 1 $.

    More generally, for all $m \ge 2$ we h:w
    ave $ n^m - 1 =(n-1)(n^{m-1}+n^{m-2}+\dots+1) $.
    So, $ n-1 $ is a divisor of $ n^m -1 $.
\end{example}

\begin{lemma}
    Suppose that $a,b$ and $c$ are integers.
    \begin{enumerate}
        \item If $a$ divides $b$ and $b$ divides $c$, then $a$ divide $c$.
        \item If $a$ divides $b$ and $c$, then $a$ divides $x\cdot a + y\cdot b$
        for all integers $ x,y $
        \item If $b$ is non-zero and $a$ divide $b$, then $ |a| \le |b| $
    \end{enumerate}
\end{lemma}

\begin{theorem}[Division with Remainder]
    If $ a \in \mathbb{Z} $ and $ b \in \mathbb{Z} \setminus \{0\} $, then
    there are unique integers $q,r$ such that $a = q \cdot b + r, |r|<|b|$
    and $ b \cdot r \ge 0 $.
\end{theorem}



\subsection{Euclid's algorithm}

\subsection{Linear diaphantine equtions}

\subsection{Prime numbers}

\subsection{Factorization}

\subsection{Number systems}