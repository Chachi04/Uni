\section{Sets}
\subsection{Sets and subsets}
\begin{definition}[Set]
    A is set any collection of "things" or "objects
\end{definition}
\begin{definition}[subset]
    Suppose $A$ and $B$ are sets. The $A$ is called a \emph{subset} of $B$,
    if for every element $ a \in A $ we also have that $ a \in B $. \par
    If $A$ is a subset of $B$,then we write $ A \subset B $ or $ A \subseteq B $. We also say that $B$ conatins $A$. \par
    By $ B \supset A $ or $ B \supseteq A $ we mean $ A \subset B $ or $ A \subseteq B $.
\end{definition}
\begin{example}
    It is true that $ 1 \in \{1,2,3\} $ and $ \{1\} \subseteq \{1,2,3\} $, but \emph{not} that $ 1\subseteq \{1\} \in \{1,2,3\} $ or $ \{1\} \in \{1,2,3\} $
\end{example}
\begin{example}
    Notice that $ \emptyset \in \{\emptyset\} $ and $ \emptyset \subseteq \{\emptyset\} $
\end{example}
\begin{example}
    To following inclusions are proper $$ \mathbb{N} \subsetneq \mathbb{Z} \subsetneq \mathbb{Q} \subsetneq \mathbb{R} \subsetneq \mathbb{C} $$
\end{example}

\begin{definition}[Power set]
    If $B$ is a set, then by $ \mathcal{P}(B) $ we denote the set of all subsets $A$ of $B$. The set $ \mathcal{P}(B) $ is called the \emph{power set} of B.
    \par
    {\color{red} !} The power set of a set is never empty.
\end{definition}

\begin{example}
    Suppose $ A = \{x,y,z\} $m then $ \mathcal{P}(A) $ consists of 8 subsets of $A$.
\end{example}

\begin{proposition}
    Let $ A $ be a set with $n$ elements. Then its power set $ \mathcal{P}(A) $ contains $ 2^n $ elements.
\end{proposition}

\begin{proposition}
    Suppose $ A,B\text{ and } C $ are sets. Then the following holds:
    \begin{enumerate}
        \item If $ A \subseteq B $ and $ B \subseteq C $, then $ A \subseteq C $.
        \item If $ A \subseteq B $ and $ B \subseteq A $, then $ A = B $
    \end{enumerate}
\end{proposition}
\begin{proof}[Proof: Statement 1]
    Suppose $ A \subseteq B $ and $ B \subseteq C $. Let $ a \in A $. Since $ A \subseteq B $, $ a \in B $. Now since $ B \subseteq C $, $ a \in C $.
    Since for every $ a \in A : a \in C $, $ A \subseteq C $
\end{proof}

\subsection{How to describe a set}
\begin{definition}[Set description]
    Let $ P $ be a predicate with reference set $ X $, then $$ \{x \in X \mid P(x)\} $$
    denotes the subset of $ X $ consisting of all elements $ x \in X $ for which the statement $ P(x) $ is true.
\end{definition}
\begin{example}
    The set $ \{x \in \mathbb{R} \mid x > 0\} $ consists of all posistive real numbers.
\end{example}

\subsection{Operations on sets}
\begin{definition}
    Let $A, B$ be sets.
    \begin{enumerate}
        \item \emph{intersection}: $ A \cap B $ - the set of all elements contained in both $A$ and $B$.
        \item \emph{union}: $ A \cup B $ - the set of elements that are in at least on of $A$ or $B$.
        \item Two sets $A$ and $B$ are called \emph{disjoint}, if their intersection $ A \cap B $ is the empty set.
    \end{enumerate}
\end{definition}

\begin{proposition}
    Let $A$,$B$ and $C$ be sets. Then the following holds:
    \begin{enumerate}[label=(\alph*)]
        \item $ A \cup B = B \cup A $
        \item $ A \cup \emptyset = A $
        \item $ A \subseteq (A \cup B) $
        \item If $ A \subseteq B $, then $ A \cup B = B $
        \item $ (A \cup B) \cup C = A \cup (B \cup C)$
        \item $ A \cap B = B \cap A $
        \item $ A \cap \emptyset = \emptyset $
        \item $ A \cap B \subseteq A $
        \item If $ A \subseteq B $, then $ A \cap B = A $
        \item $ (A \cap B) \cap B = A \cap (B \cap C) $
    \end{enumerate}
\end{proposition}

\begin{definition}[Big Unions and Intersections of sets]
    Suppose $I$ is a set and for each element $i$ there exists a set $A_i$, then
    $$ \bigcup_{i\in I}A_i := \{x \mid \text{there is an } i \in I \text{ with } x \in A_i\} $$
    and
    $$ \bigcap_{i \in I} A_i := \{x \mid \text{for all } i \in I \text{ we have } x \in A_i\} $$
    (the set $I$ is called the index set) \par
    If $ \mathscr{C} $ is a set/collection of sets, then we can define
    $$ \bigcup_{A \in \mathscr{C}}A := \{x \mid \text{there is an } A \in \mathscr{C}\} $$
    and
    $$ \bigcap_{A \in \mathscr{C}}A := \{x \mid \text{for all } A \in \mathscr{C} \text{ we have } x \in A\} $$
\end{definition}

\begin{example}
    Suppose for each $ i \in \mathbb{N} $ the set $A_i$ is defined as $\{x \in \mathbb{R} \mid 0 \leq x \leq i\}$. Then
    $$ \bigcap_{i \in \mathbb{N}}A_i = \{0\} $$
    (here we assume that $ 0 \in \mathbb{N} $) and
    $$ \bigcup_{i \in \mathbb{N}}A_i = \mathbb{R}_{\geq 0} = \{x \in \mathbb{R} \mid x \geq 0\} $$
\end{example}

\begin{definition}[Setminus and symmetric difference]
    Let $A$ and $B$ be sets. The \emph{difference} of $A$ and $B$, notation $A \setminus B$, is the set of all elements from $A$ that are \emph{not} in $B$. \par
    The \emph{symmetric difference} of $A$ and $B$, notation $A \triangle B$, is the set of all elements in \emph{exactly one} of $A$ or $B$.
\end{definition}

\begin{proposition}
    Let $A$,$B$ and $C$ be sets. Then the following holds:
    \begin{enumerate}
        \item $ A \setminus B \subseteq A $
        \item If $ A \subseteq B $, then $ A \setminus B = 0 $
        \item $ A = (A \setminus B) \cup (A \cap B) $
        \item $ A \triangle B = (A \setminus B) \cup (B \setminus A) $
        \item $ A \triangle B = B \triangle A $
        \item If $ A \subseteq B $, then $ A \triangle B = B \setminus A $
        \item $ A \triangle (B \triangle C) = (A \triangle B) \triangle C $
    \end{enumerate}
\end{proposition}

\begin{proposition}
    Let $A$,$B$ and $C$ be sets. Then the following hold:
    \begin{enumerate}
        \item $ (A \cup B) \cap C = (A \cap C) \cup (B \cap C) $
        \item $ (A \cap B) \cup C = (A \cup C) \cap (B \cup C) $
        \item $ A \setminus (B \cup C) = (A \setminus B) \cap (A \setminus C) $
        \item $ A \setminus (B \cap C) = (A \setminus B) \cup (A \setminus C) $
    \end{enumerate}
\end{proposition}

\begin{definition}[Set Complement]
    If one is working inside a fixed set $U$ and only cnsidering subsets of $U$, then the difference $U \setminus A$ is also called the \emph{complement} of $A$ in $U$.
    We write $A^*$ or $A^c$ for the complement of $A$ in $U$. In this case the set $U$ is also called the \emph{universe}.
\end{definition}

\begin{proposition}
    For subsets $A$,$B$ and $C$ of the universe $U$ we have:
    \begin{enumerate}
        \item $ A \cup A^* = U $
        \item $ B \setminus C = B \cap C^* $
        \item $ (A^*)^* = A $
        \item If $ A \subseteq B $ then $ B^* \subseteq A^* $
        \item $ (A \cup B)^* = A^* \cap B^* $
        \item $ (A \cap B)^* = A^* \cup B^* $
    \end{enumerate}
\end{proposition}

\subsection{Cartesian product}
\begin{definition}[Cartesian Product]
    The Cartesian product $ A_1 \times A_2 \times \dots \times A_k $ of sets $ A_1,\dots,A_k $ is the set of all ordered k-tuples $ (a_1,a_2,\dots,a_k) $ where $ a_i \in A_i $ for $ 1 \leq i \leq k $. \par
    In particular, if $A$ and $B$ are sets, then
    $$ A \times B = {(a,b) \mid a \in A \text{ and } b \in B} $$
\end{definition}

\subsection{Partitions}
\begin{definition}[Partition]
    Let $S$ be a none-empty set. A collection $ \Pi $ of subsets of $S$ is called a \emph{partition} if and only if
    \begin{enumerate}
        \item $ \emptyset \notin \Pi $
        \item $ \bigcup_{X \in \Pi}X = S $
        \item for all $ X \ne Y \in \Pi $ we have $ X \cap Y = \emptyset $
    \end{enumerate}
\end{definition}

\begin{example}
    The set $ \{1,2,\dots,10\} $A can be partiioned into the sets $ \{1,2,3\}, \{4,5\}, \{6,7,8,9,10\} $
\end{example}
\begin{example}
    Suppose $ \mathcal{L} $ is the set of all lines in $\mathbb{R}^2$ parallel to a fixed line $\ell$. Then $ \mathcal{L} $ partitions $\mathbb{R}^2$
\end{example}
\begin{example}
    Let $ n > 1 $ be an integer. Then the set $ \mathbb{Z} $ can be partitioned into the following subsets:
    \begin{align*}
        \{z \in \mathbb{Z} &\mid z = 0 + nx \text{ for some } x \in \mathbb{Z}\} \\
        \{z \in \mathbb{Z} &\mid z = 1 + nx \text{ for some } x \in \mathbb{Z}\} \\
                           &\vdots \\
        \{z \in \mathbb{Z} &\mid z = (n-1) + nx \text{ for some } x \in \mathbb{Z}\} \\
    \end{align*}
\end{example}

\subsection{Quantifiers}
\begin{definition}[Quantifiers]
    Let $ P $ be a predicate on a reference set $X$. Then by
    $$ \forall x \in X \left[P(x)\right] $$
    we denote the assertion "For all $ x \in X $ the assertion $ P(x) $ is true".

    $ \forall $ is called the \emph{for all-}quantifier or \emph{universal quantifier}.

    By
    $$ \exists x \in X \left[P(x)\right] $$
    we denote the assertion "There exists an $ x \in X $ with $ P(x) $ true".

    $ \exists $ is called the \emph{existential quantifier}.
\end{definition}

\begin{example}
    The following statements are true:
    $$ \forall x \in \mathbb{R} \left[x \ge 0 \implies |x| = x\right], $$
    $$ \exists x \in \mathbb{R} \left[|x| = x\right] $$
    $$ \forall x \in \mathbb{Q} \left[-1 < \sin(x) < 1\right] $$

    Here a few statements that are false:
    $$ \forall x \in \mathbb{R} \left[|x| = x\right] $$
    $$ \forall x \in \mathbb{R} \left[-1 < \sin(x) < 1\right] $$
\end{example}

\begin{example}
    We can make combinations of quantifiers to create various assertions. For example
    $$ \forall x \in \mathbb{Z} \exists y \in \mathbb{Z} \left[x + y = 0\right] $$
\end{example}

\begin{proposition}[DeMorgan's rule]
    $$ \lnot (\forall x \in X \left[P(x)\right]) \iff \exists x \in X \left[\lnot P(x)\right]$$
    $$ \lnot (\exists x \in X \left[P(x)\right]) \iff \forall x \in x \left[\lnot P(x)\right] $$
\end{proposition}

\begin{example}
    Let $ X = \left\{1,2,\dots,9\right\} $ and consider the following statements.
    $$ P = \forall x \in X \exists y \in X \left[x+y = 10\right] $$
    $$ Q = \exists x \in X \forall y \in X \left[x+y = 10\right] $$

    The assertion $P$ is clearly true.

    The assertion $Q$ is false. We prove $ \lnot Q $. By DeMorgan's rule the assertion $ \lnot Q $ is equivalent with
    $$ R = \forall x \in X \exists y \in X \left[x+y \ne 10\right] $$
\end{example}