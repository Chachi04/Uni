\section{Logic}
\subsection{Statements}
\begin{definition}[Statement]
    A \emph{statement} is a sentence that is either true or false but never both.
    A \emph{proposition}, \emph{logical statement} or \emph{assertion} can also be used to refer to a statement.
\end{definition}

\subsection{Logical operations}
\begin{itemize}
    \item Logical and: $\lor$
    \item Logical or: $\land$
    \item Logical not: $\lnot$
\end{itemize}
\begin{definition}[Implication]
    If A and B are assertions, then the assertion if A then B ($A \Rightarrow B$)
    is true if and only if one of the following occurs:
    \begin{itemize}
        \item A is true and B is true
        \item A is false and B is true
        \item A is false and B is false
    \end{itemize}
\end{definition}

\begin{definition}[Biimplication (if and only if)]
    $A \Leftrightarrow B \equiv (A \Rightarrow B) \land (B \Rightarrow A)$
\end{definition}

\subsection{Proposition Calculus}
Using logical operators and assertions $P_1,P_2,...,P_k$ to form new assertions and analyze them.

\begin{theorem}[Some true assertions]
    Suppose P,Q, and R are assertions. Then the following assertions are true:
    \begin{enumerate}[label=(\alph*)]
        \item $P \lor \lnot P$
        \item $P \Leftrightarrow \lnot (\lnot P)$
        \item $\lnot (P \land \lnot P)$
        \item $P \Rightarrow Q \Leftrightarrow \lnot P \lor Q$
        \item $\lnot (P \lor Q) \Leftrightarrow \lnot P \land \lnot Q$
        \item $\lnot (P \land Q) \Leftrightarrow \lnot P \lor \lnot Q$
        \item $P \Rightarrow Q \Leftrightarrow \lnot Q \Rightarrow \lnot P$
        \item $(P \lor Q) \land R \Leftrightarrow (P \land R) \lor (Q \land R)$
        \item $(P \land Q) \lor R \Leftrightarrow (P \lor R) \land (Q \lor R)$
        \item $(P \lor Q) \Rightarrow R \Leftrightarrow (P \Rightarrow R) \land (Q \Rightarrow R)$
    \end{enumerate}
\end{theorem}

\subsection{Methods of proof}
If the statement is of the form
\begin{center}
    If P then Q.
\end{center}
\subsubsection{Direct proof}
We only need to consider the case where P is true and deduce the truth of Q. \\
A direct proof of $P \Rightarrow Q$ looks like:
\begin{center}
    Assume that P is true.
\end{center}
Then we use arguements that imply that Q is also true and end the proof with:
\begin{center}
    Hence Q is true.
\end{center}

\subsubsection{Proof by contraposition}
In instead of proving the statement $P \Rightarrow Q$ we prove its contrapositive
($\lnot Q \Rightarrow \lnot P$).
\subsubsection{Proof by contradiction}
In order to prove P we assume the opposite $\lnot P$ to be true and deduce a
condradiction with some obviously true statement Q.
\par Thus, we prove that $\lnot Q \Rightarrow \lnot P$. But then the contrapositive
$Q \Rightarrow P$ must also be true. And the obvious truth of Q implies P to be true.