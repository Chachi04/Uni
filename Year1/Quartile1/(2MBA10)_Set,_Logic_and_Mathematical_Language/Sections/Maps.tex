\section{Maps}
\subsection{Definition}
\begin{definition}
    A relation $F$ from a set $A$ to a set $B$ is called a map or function from $A$ to $B$
    if for each $a \in A$ there is one and only one $b \in B$ with $aFb$
    \par If $F$ is a map from $A$ to $B$, we write $F: A \rightarrow B$
    \par The set of all maps from $A$ to $B$ if denoted by $B^A$
    \par A \emph{partial map} $F$ from $A$ to $B$ is a relation with the property
    that for each $a \in A$ there is at most one $b \in B$ with $aFb$.
\end{definition}

\begin{example}
    \begin{enumerate}
        \item polynomial functions like $ f:\mathbb{R} \to \mathbb{R} $,
         with $ f(x) = x^3 $ for all $x$
        \item functions like $ \cos, \sin, \tan $
        \item $ \sqrt{}:\mathbb{R^+} \to \mathbb{R} $, taking square roots
        \item $ \ln:\mathbb{R^+} \to \mathbb{R} $, the natural logarithm
    \end{enumerate}
\end{example}

\begin{proposition}
    Let $ f: A \to B $ and $ g: B \to C $ be maps, then the composition
    $ g \circ f = f;g $ is a map from $A$ to $C$.
\end{proposition}

Let $A$ and $B$ be two sets and $ f:A \to B $. The set $A$ is called the
\emph{domain} of $f$, the set $B$ the \emph{codomain}. If $ a \in A $, then
the element $ b = f(a) $ is called the \emph{image} of $a$ under $f$.
The subset of $B$ consisting of the images of the elements of $A$ under $f$
is called the \emph{image} or \emph{range} of $f$ and is denoted by $\text{Im}(f)$. So
$$ \text{Im}(f) = \{b \in B \mid \text{ there is an } a \in A \text{ with } b = f(a)\} $$
If $ A^\prime $ is a subset of $A$, then the image of $A^\prime$ under $f$ is the set
$ f(A^\prime) =\{f(a) \mid a \in A^\prime\}$
If $ A^\prime $ is a subset of $A$, then the image of $ A^\prime $ under the set
$ f(A^\prime) =\{f(a) \mid a \in A^\prime\} $. So, $ \text{Im}(f) = f(A) $.

If $ a \in A $ and $ b = f(a) $, then the element $a$ is called a \emph{pre-image}
of $b$. Notice that $b$ can have more than one pre-image. The set of all pre-images
of $b$ is denoted by $ f^{-1}(b) $. So
$$ f^{-1}(b) =  \{a \in A \mid f(a) = b\}$$
If $B^\prime$ is a subset of $B$, then the pre-image of $B^\prime$, denoted by
$ f^{-1}(B^\prime) $ is the set of elements $a$ from $A$ tjhat are mapped to
an element $b$ of $B^\prime$. In particular
$$ f^{-1}(B^\prime) = \{a \in A \mid f(a) \in B^\prime\} $$

\begin{example}
    \begin{enumerate}
        \item Let $ f:\mathbb{R} \to \mathbb{R} $ with $ f(x) = x^2 $
        for all $ x \in \mathbb{R} $. Then $ f^{-1}(\left[0,4\right]) = \left[-2,2\right] $
        \item Consider the map from $ \mathbb{Z} $ to $ \mathbb{Z} $, which
        maps an integer $a$ to the unique element $b$ in {0,$\dots$,7} with
        $ a=b \pmod{8} $. The inverse image of 3 is the set $\left\{\dots, -5,3, 11, \dots\right\} $.
        The inverse image of 11, however, is the emptyset.
    \end{enumerate}
\end{example}

\newpage
\begin{theorem}
    Let $ f: A \to B $ be a map.
    \begin{itemize}
        \item If $ A^\prime \subseteq A $, then $ f^{-1}(f(A^\prime)) \supseteq A^\prime $
        \item If $ B^\prime \subseteq B $, then $ f(f^{-1}(B^\prime)) \subseteq B^\prime $
    \end{itemize}
\end{theorem}

\begin{proof}[Proof]
    Let $ a^\prime \in A^\prime $, then $ f(a^\prime) \in f(A^\prime) $ and hence
    $ a^\prime \in f^{-1}(f(A^\prime)) $. Thus $ A^\prime \subseteq f^{-1}(f(A^\prime)) $

    Let $ a \in f^{-1}(B^\prime) $, then $ f(a) \in B^\prime $. Thus
    $ f(f^\prime(B^\prime)) \subseteq B^\prime $
\end{proof}

\begin{theorem}
    Let $ f: A \to B $ and $ g: B \to C $ be maps. Then
    $ \text{Im}(g \circ f) = g(f(A)) \subseteq \text{Im}(g) $
\end{theorem}

\subsection{Special maps}
\begin{definition}[Surjective, injective and bijective maps]
    A map $ f: A \to B $ is called \emph{surjective}, if for every $b \in B$ there is
    an $ a \in A $ with $ b = f(a) $. In other words if $ \Im(f) = B $.

    The map $f$ is called \emph{injective}, if for each $b \in B$, there is at most
    one $a$ with $ f(a) = b $. So the pre-image of $b$ is either empty or consists
    of a unique element. In other words, $f$ is injective if for any elements $a$
    and $b$ from $A$ we find that $ f(a) = f(b) $ implies $ a = b $.

    The map $f$ is \emph{bijective} if it is both surjective and injective. So, if
    for each $b \in B$ there is a unique $a \in A$ with $f(a) = b$.
\end{definition}

\begin{example}
    \begin{enumerate}[label=(\alph*)]
        \item The map $ \sin:\mathbb{R} \to \mathbb{R} $ is not surjective,
        nor injective
        \item The map $ \sin:\left[-\pi/2,\pi/2\right] \to \mathbb{R} $ is
        injective, but not surjective
        \item The map $ \sin:\mathbb{R} \to \left[-1,1\right] $ is a surjective,
        but not injective map
        \item The map $ \sin:\left[-\pi/2,\pi/2\right] \to \left[-1,1\right] $
        is a bijective map
    \end{enumerate}
\end{example}

\begin{theorem}[Pigeonhole Principle]
    Let $A$ be a set of size $n$ and $B$ be a set of size $m$. Let $ f:A \to B $
    be a map between sets $A$ and $B$.
    \begin{enumerate}[label=(\alph*)]
        \item If $ n < m $, then $f$ cannot be surjective.
        \item If $ n > m $, then $f$ cannot be injective.
        \item If $ n = m $, then $f$ is injective if and only if $f$ is surjective.
    \end{enumerate}
\end{theorem}

\begin{remark}
    The above result is called the pigeonhole principle because of the following.
    If one has $n$ pigeons (the set $A$) and the same number of holes (the set $B$),
    then one pigeonhole is empty if and only if one of the other holes contains at
    least two pigeons.
\end{remark}

\begin{example}
    Suppose you have to pick seven distinct numbers of the set
    $\left\{1,2,\dots,11\right\}$. Then among these seven numbers there is a pair
    that adds up to 12.

    Suppose $S$ is the set of 7 numbers picked. Now consider the following six
    subsets $$ \{1, 11\},\{2, 10\},\{3, 9\},\{4, 8\},\{5, 7\},\{6\} $$ partitioning
    $ \{1,\dots,11\} $. The map that assigns to each of the seven elements of $S$
    the unique part of this partition to which it belongs can not be injective.
    So, there is a pair of this partition that is contained in S providing us
    with two numbers in S adding up to 12.
\end{example}

\begin{proposition}
    Let $ f:A \to B $ be a bijection. Then for all $ a \in A $ and $ b \in B $
    we have $ f\inv(f(a)) = a $ and $ f(f\inv(b)) = b $. In particular, $f$ is the
    inverse of $ f\inv $.
\end{proposition}

\begin{theorem}
    Let $ f: A \to B $ and $ g: B \to C $ be two maps.
    \begin{enumerate}[label=(\alph*)]
        \item If $f$ and $g$ are surjective, then so is $ g \circ f $
        \item If $f$ and $g$ are injective, then so is $ g \circ f $
        \item If $f$ and $g$ are bijective, then so is $ g \circ f $
    \end{enumerate}
\end{theorem}

\begin{proposition}
    If $ f:A \to B $ and $ g:B \to A $ are maps with $ f \circ g = I_B $ and
    $ g \circ f = I_A $ where $I_A$ and $I_B$ denote the identity maps on $A$
    and $B$, respectively. Then $f$ and $g$ are bijections. Moreover, $ f\inv = g $
    and $ g\inv = f $.
\end{proposition}

\begin{lemma}
    Suppose $ f:A \to B $ and $ g:B \to C $ are bijective maps. Then the inverse
    of the map $ g \circ f $ equals $ f\inv \circ g\inv $.
\end{lemma}

\subsection{Permutations and Symmetric groups}
\begin{definition}[Permutations and Symmetric groups]
Let $X$ be a set.
\begin{itemize}
    \item A bijection on $X$ to itself is also called a \emph{permutation} of X.
    The set of all permutations of $X$ is denoted by $ \sym(X) $. It is called the
    \emph{symmetric group} on $X$.

    \item The product $ g \cdot h $ of two permutations $ g,h \in \sym(X) $ is
    defined as the composition $ g \circ h $ of $g$ and $h$. Thus for all $x \in X$
    we have $ g \cdot h(x) = g(h(x)) $.

    \item If $ X = \{1,\dots, n\} $, we also write $ \sym_n $ instead of $ \sym(X) $.
    Furthermore, a permutation $f$ of $X$ is often given by $ \left[f(1),f(2),\dots,f(n)\right] $.
\end{itemize}
\end{definition}

\begin{theorem}
    $ \sym_n $ has exactly $ n! $ elements.
\end{theorem}

\begin{definition}[Order of a permutation]
    The order of a permutation $g$ is the smallest positive integer $m$ such that
    $ g^m = e $.
\end{definition}

\subsection{Cycles}
\begin{definition}[Fix points and Support]
    The \emph{fixed points} of $g$ in $X$ are the elements of $x$ in $X$ for which
    $ g(x) = x $ holds. The set of all fix points is $ \fix(g) = \{x \in X \mid g(x) = x\} $.

    The \emph{support} of $g$ is the complement in $X$ of $ \fix(g) $. It is denoted
    by $ \support(x) $
\end{definition}

\begin{example}
    Consider the permutation $g = [1, 3, 2, 5, 4, 6] \in \sym_6$.
    The fixed points of $g$ are 1 and 6. So $\fix(g) = \{1, 6\}$.
    Thus the points moved by $g$ form the set $\support(g) = \{2, 3, 4, 5\}$.
\end{example}

    Cycles are elements in $Sym_n$ of special importance.

\begin{definition}[Cycles]
    Let $ g \in \sym_n $ be a permutation with $ \support(g) = \{a_1,\dots,a_m\} $,
    where the $ a_i $ are pairwise distict. We say $g$ is an $m$-cycle if
    $ g(a_i) = g(a_{i+1}) $ for all $ i \in \{1,\dots,m-1\} $ and $ g(a_m) = a_1 $.
    For such a cycle $g$ we also use the cycle notation $ \left(a_1,\dots,a_m\right) $.

    2-cycles are called \emph{transpositions}.
\end{definition}

\begin{theorem}
    Every permutation in $ \sym_n $ is a product of disjoint cycles. This product
    is unique up to rearrangement of the factors.
\end{theorem}

\begin{definition}[Cycle structure]
    The cycle structure of a permutation is the unordered sequence of the cycle
    lenghts in an expression of $g$ as a product of disjoint cycles.
\end{definition}

\subsection{Alternating groups}
\begin{theorem}
    If a permutation is written in two ways as a product of transpositions, then
    both products have even length or both have odd length.
\end{theorem}

\begin{definition}
    Let $g$ be an element of $ \sym_n $. the sign of $g$, denoted by $ \sign(g) $,
    is defined as
    \begin{itemize}
        \item 1 if $g$ can be written as a product of an even number of 2-cycles, and
        \item -1 if $g$ can be writeen as a product of an odd number of 2-cycles.
    \end{itemize}
    We say that $g$ is even $ \sign(g) = 1 $ and odd if $ \sign(g) = -1 $.
\end{definition}

\begin{theorem}[Multiplicative property of $ \sign $]
    For all permutations $ g,h $ in $ \sym_n $, we have
    $$ \sign(g \cdot h) = \sign(g) \cdot \sign(h) $$
\end{theorem}

\begin{corollary}
    If a permutation $g$ is written as a product of cycles, then $ \sign(g) = (-1)^w $,
    where $w$ is the number of cycles of even length.
\end{corollary}

\begin{definition}[Alternating group]
    By $ \alt_n $ we denote the set of even permutations in $ \sym_n $. We call
    $ \alt_n $ the \emph{alternating group} on $n$ letters.

    The alternating group is closed with respect to taking products and inverse
    elements.
\end{definition}

There are exactly as many even as odd permutations in $ Sym_n $.

\begin{theorem}[Size of $ \alt_n $]
    For $ n > 1 $, the alternating group $ \alt_n $ contains precisely
    $ \frac{n!}{2} $ elements.
\end{theorem}

\begin{theorem}
    Every even permutation is a product of 3-cycles.
\end{theorem}