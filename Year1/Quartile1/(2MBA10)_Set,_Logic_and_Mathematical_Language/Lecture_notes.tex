%%%%%%%%%%%%%%%%%%%%%%%%%%%%% Define Article %%%%%%%%%%%%%%%%%%%%%%%%%%%%%%%%%%
\documentclass{article}
%%%%%%%%%%%%%%%%%%%%%%%%%%%%%%%%%%%%%%%%%%%%%%%%%%%%%%%%%%%%%%%%%%%%%%%%%%%%%%%

%%%%%%%%%%%%%%%%%%%%%%%%%%%%% Using Packages %%%%%%%%%%%%%%%%%%%%%%%%%%%%%%%%%%
\usepackage{geometry}
\usepackage{graphicx}
\usepackage{amssymb}
\usepackage{amsmath}
\usepackage{mathptmx}
\usepackage{mathrsfs}
\usepackage{amsthm}
\usepackage{thmtools}
\usepackage{empheq}
\usepackage{booktabs}
\usepackage{lipsum}
\usepackage{graphicx}
\usepackage{color}
\usepackage{psfrag}
\usepackage{pgfplots}
\usepackage{bm}
% \usepackage{mdframed}
%%%%%%%%%%%%%%%%%%%%%%%%%%%%%%%%%%%%%%%%%%%%%%%%%%%%%%%%%%%%%%%%%%%%%%%%%%%%%%%

% Other Settings
% \usepackage[dvipsnames]{xcolor}
\usepackage[most]{tcolorbox}
% \usepackage[mathscr]{euscript}
\usepackage{enumitem}
\usepackage{physics}
\usepackage{microtype}
\usepackage[T1]{fontenc}
\usepackage[utf8]{inputenc}
\usepackage{hyperref}
\hypersetup{
    colorlinks=true,
    linkcolor=blue,
    filecolor=magenta,
    urlcolor=cyan,
    pdftitle={Overleaf Example},
    pdfpagemode=FullScreen,
    }

\urlstyle{same}

\setcounter{tocdepth}{2}
\usepackage{cleveref}
% \usepackage{algorithm2e}

%%%%%%%%%%%%%%%%%%%%%%%%%% Page Setting %%%%%%%%%%%%%%%%%%%%%%%%%%%%%%%%%%%%%%%
\geometry{a4paper}

%%%%%%%%%%%%%%%%%%%%%%%%%% Custom commands %%%%%%%%%%%%%%%%%%%%%%%%%%%%%%%%%%%%%%%
\newcommand{\inv}{^{-1}}
\newcommand{\sym}{\text{Sym}}
\newcommand{\alt}{\text{Alt}}
\newcommand{\fix}{\text{fix}}
\newcommand{\support}{\text{support}}
\newcommand{\sign}{\text{sign}}

%%%%%%%%%%%%%%%%%%%%%%%%%% Define some useful colors %%%%%%%%%%%%%%%%%%%%%%%%%%
\definecolor{ocre}{RGB}{243,102,25}
\definecolor{mygrey}{RGB}{243,243,244}
\definecolor{deepGreen}{RGB}{178,254,179}
\definecolor{shallowGreen}{RGB}{243,255,243}
\definecolor{deepBlue}{RGB}{61,124,222}
\definecolor{shallowBlue}{RGB}{235,249,255}
\definecolor{deepOrange}{RGB}{254,129,0}
\definecolor{shallowOrange}{RGB}{255,243,231}
\definecolor{deepPurple}{RGB}{128,129,254}
\definecolor{shallowPurple}{RGB}{243, 243, 255}
\definecolor{deepRed}{RGB}{254,76,77}
\definecolor{shallowRed}{RGB}{254,242,243}
\definecolor{theoremTitle}{RGB}{78,93,247}
%%%%%%%%%%%%%%%%%%%%%%%%%%%%%%%%%%%%%%%%%%%%%%%%%%%%%%%%%%%%%%%%%%%%%%%%%%%%%%%

%%%%%%%%%%%%%%%%%%%%%%%%%% Define an orangebox command %%%%%%%%%%%%%%%%%%%%%%%%
\newcommand\orangebox[1]{\fcolorbox{ocre}{mygrey}{\hspace{1em}#1\hspace{1em}}}
\newcommand\egcd{\text{Extgcd}}
\newcommand{\modset}[1]{\mathbb{Z}/#1\mathbb{Z}}
%%%%%%%%%%%%%%%%%%%%%%%%%%%%%%%%%%%%%%%%%%%%%%%%%%%%%%%%%%%%%%%%%%%%%%%%%%%%%%%

\newlist{theoremlist}{description}{1}
\setlist[theoremlist]{labelsep=1em, left=0pt, labelwidth=2cm, style=multiline, font=\normalfont}

%%%%%%%%%%%%%%%%%%%%%%%%%%%% English Environments %%%%%%%%%%%%%%%%%%%%%%%%%%%%%
\declaretheoremstyle[
    spaceabove=2pt,
    spacebelow=2pt,
    headfont=\normalfont\bfseries\color{theoremTitle},
    bodyfont=\normalfont,
    headpunct={},
    postheadspace=3pt,
    headformat={\color{theoremTitle}\NAME~\NUMBER\color{black}\,--\,\NOTE},
    notefont=\normalfont\bfseries,
    notebraces={}{},
]{theoremStyle}

\declaretheoremstyle[
    spaceabove=2pt,
    spacebelow=2pt,
    headfont=\normalfont\bfseries,
    bodyfont=\normalfont,
    headpunct={},
    postheadspace=2pt,
    headformat={\NAME~\NUMBER~\NOTE},
    notefont=\normalfont\bfseries,
    notebraces={}{},
]{exampleStyle}

% \declaretheoremstyle[
%   spaceabove=3pt,
%   spacebelow=3pt,
%   headfont=\normalfont\bfseries,
%   bodyfont=\normalfont,
%   headpunct={},
%   postheadspace=1em,
% %   headformat={\color{black}\thmname{#1}~\thmnumber{#2}\,--\,#3},
% ]{myproblemstyle}

% \declaretheoremstyle[
%   spaceabove=6pt,
%   spacebelow=6pt,
%   headfont=\bfseries,
%   notefont=\bfseries,
%   notebraces={(}{)},
%   bodyfont=\normalfont,
%   headpunct={},
%   postheadspace=1em,
%   numbered=no
% ]{mystyle-no-number}

\declaretheorem[
    style=theoremStyle,
    name=Theorem,
    numberlike=subsubsection
]{theorem}
\tcolorboxenvironment{theorem}{
    enhanced,
    colback=shallowOrange,
    boxrule=0pt,
    frame hidden,
    breakable,
    left=2mm,
    top=1mm,
    bottom=1mm,
    right=0mm,
    borderline west={1mm}{0pt}{deepOrange},
    % before skip=10pt,
    % after skip=10pt,
}
\declaretheorem[
    style=theoremStyle,
    name=Definition,
    numberlike=subsubsection,
    % shaded={
    %     bgcolor=shallowGreen,
    % },
]{definition}
\tcolorboxenvironment{definition}{
    enhanced,
    colback=shallowPurple,
    boxrule=0pt,
    frame hidden,
    breakable,
    left=2mm,
    top=0mm,
    bottom=0mm,
    right=0mm,
    borderline west={1mm}{0pt}{deepPurple},
}
\declaretheorem[
    style=myproblemstyle,
    name=Problem,
    numberlike=subsubsection,
]{problem}
\declaretheorem[
    style=theoremStyle,
    name=Proposition,
    numberlike=subsubsection,
]{proposition}
\tcolorboxenvironment{proposition}{
    enhanced,
    colback=shallowOrange,
    boxrule=0pt,
    frame hidden,
    breakable,
    left=2mm,
    top=0mm,
    bottom=0mm,
    right=0mm,
    borderline west={1mm}{0pt}{deepOrange},
    % before skip=10pt,
    % after skip=10pt,
}
\declaretheorem[
    style=theoremStyle,
    name=Lemma,
    numberlike=subsubsection,
]{lemma}
\tcolorboxenvironment{lemma}{
    enhanced,
    colback=shallowOrange,
    boxrule=0pt,
    frame hidden,
    breakable,
    left=2mm,
    top=0mm,
    bottom=0mm,
    right=0mm,
    borderline west={1mm}{0pt}{deepOrange},
    % before skip=10pt,
    % after skip=10pt,
}

\declaretheorem[
    style=theoremStyle,
    name=Corollary,
    numberlike=subsubsection,
]{corollary}
\tcolorboxenvironment{corollary}{
    enhanced,
    colback=shallowOrange,
    boxrule=0pt,
    frame hidden,
    breakable,
    left=2mm,
    top=0mm,
    bottom=0mm,
    right=0mm,
    borderline west={1mm}{0pt}{deepOrange},
    % before skip=10pt,
    % after skip=10pt,
}

\declaretheorem[
    style=theoremStyle,
    name=Algorithm,
    numberlike=subsubsection,
]{algorithm}
\tcolorboxenvironment{algorithm}{
    enhanced,
    colback=shallowRed,
    boxrule=0pt,
    frame hidden,
    breakable,
    left=2mm,
    top=0mm,
    bottom=0mm,
    right=0mm,
    borderline west={1mm}{0pt}{deepRed},
}

\declaretheorem[
    style=theoremStyle,
    name=Principle,
    numberlike=subsubsection,
]{principle}
\tcolorboxenvironment{principle}{
    enhanced,
    colback=shallowGreen,
    boxrule=0pt,
    frame hidden,
    breakable,
    left=2mm,
    top=0mm,
    bottom=0mm,
    right=0mm,
    borderline west={1mm}{0pt}{deepGreen}
}
\declaretheorem[
    style=remark,
    name=Remark,
    numberlike=subsubsection,
]{remark}

\declaretheorem[
    style=exampleStyle,
    name=Example,
    numberlike=subsubsection,
]{example}

\declaretheorem[
    name=Exercise
    numberlike=subsubsection,
]{exercise}
% \newtheorem{example}[subsubsection]{Example}
% \newtheorem{exercise}[subsubsection]{Exercise}
% \newtheorem{remark}[subsubsection]{Remark}


%%%%%%%%%%%%%%%%%%%%%%%%%%%%%%%%%%%%%%%%%%%%%%%%%%%%%%%%%%%%%%%%%%%%%%%%%%%%%%%

%%%%%%%%%%%%%%%%%%%%%%%%%%%%%%% Plotting Settings %%%%%%%%%%%%%%%%%%%%%%%%%%%%%
\usepgfplotslibrary{colorbrewer}
\pgfplotsset{width=8cm,compat=1.9}
%%%%%%%%%%%%%%%%%%%%%%%%%%%%%%%%%%%%%%%%%%%%%%%%%%%%%%%%%%%%%%%%%%%%%%%%%%%%%%%

%%%%%%%%%%%%%%%%%%%%%%%%%%%%%%% Title & Author %%%%%%%%%%%%%%%%%%%%%%%%%%%%%%%%
\title{Lecture notes}
\author{Jiaqi Wang}
%%%%%%%%%%%%%%%%%%%%%%%%%%%%%%%%%%%%%%%%%%%%%%%%%%%%%%%%%%%%%%%%%%%%%%%%%%%%%%%

\begin{document}
    \maketitle

    \tableofcontents
    \section{Logic}
\subsection{Statements}
\begin{definition}[Statement]
    A \emph{statement} is a sentence that is either true or false but never both.
    A \emph{proposition}, \emph{logical statement} or \emph{assertion} can also be used to refer to a statement.
\end{definition}

\subsection{Logical operations}
\begin{itemize}
    \item Logical and: $\lor$
    \item Logical or: $\land$
    \item Logical not: $\lnot$
\end{itemize}
\begin{definition}[Implication]
    If A and B are assertions, then the assertion if A then B ($A \Rightarrow B$)
    is true if and only if one of the following occurs:
    \begin{itemize}
        \item A is true and B is true
        \item A is false and B is true
        \item A is false and B is false
    \end{itemize}
\end{definition}

\begin{definition}[Biimplication (if and only if)]
    $A \Leftrightarrow B \equiv (A \Rightarrow B) \land (B \Rightarrow A)$
\end{definition}

\subsection{Proposition Calculus}
Using logical operators and assertions $P_1,P_2,...,P_k$ to form new assertions and analyze them.

\begin{theorem}[Some true assertions]
    Suppose P,Q, and R are assertions. Then the following assertions are true:
    \begin{enumerate}[label=(\alph*)]
        \item $P \lor \lnot P$
        \item $P \Leftrightarrow \lnot (\lnot P)$
        \item $\lnot (P \land \lnot P)$
        \item $P \Rightarrow Q \Leftrightarrow \lnot P \lor Q$
        \item $\lnot (P \lor Q) \Leftrightarrow \lnot P \land \lnot Q$
        \item $\lnot (P \land Q) \Leftrightarrow \lnot P \lor \lnot Q$
        \item $P \Rightarrow Q \Leftrightarrow \lnot Q \Rightarrow \lnot P$
        \item $(P \lor Q) \land R \Leftrightarrow (P \land R) \lor (Q \land R)$
        \item $(P \land Q) \lor R \Leftrightarrow (P \lor R) \land (Q \lor R)$
        \item $(P \lor Q) \Rightarrow R \Leftrightarrow (P \Rightarrow R) \land (Q \Rightarrow R)$
    \end{enumerate}
\end{theorem}

\subsection{Methods of proof}
If the statement is of the form
\begin{center}
    If P then Q.
\end{center}
\subsubsection{Direct proof}
We only need to consider the case where P is true and deduce the truth of Q. \\
A direct proof of $P \Rightarrow Q$ looks like:
\begin{center}
    Assume that P is true.
\end{center}
Then we use arguements that imply that Q is also true and end the proof with:
\begin{center}
    Hence Q is true.
\end{center}

\subsubsection{Proof by contraposition}
In instead of proving the statement $P \Rightarrow Q$ we prove its contrapositive
($\lnot Q \Rightarrow \lnot P$).
\subsubsection{Proof by contradiction}
In order to prove P we assume the opposite $\lnot P$ to be true and deduce a
condradiction with some obviously true statement Q.
\par Thus, we prove that $\lnot Q \Rightarrow \lnot P$. But then the contrapositive
$Q \Rightarrow P$ must also be true. And the obvious truth of Q implies P to be true.
    \section{Logic}
\subsection{Statements}
\begin{definition}[Statement]
    A \emph{statement} is a sentence that is either true or false but never both.
    A \emph{proposition}, \emph{logical statement} or \emph{assertion} can also be used to refer to a statement.
\end{definition}

\subsection{Logical operations}
\begin{itemize}
    \item Logical and: $\lor$
    \item Logical or: $\land$
    \item Logical not: $\lnot$
\end{itemize}
\begin{definition}[Implication]
    If A and B are assertions, then the assertion if A then B ($A \Rightarrow B$)
    is true if and only if one of the following occurs:
    \begin{itemize}
        \item A is true and B is true
        \item A is false and B is true
        \item A is false and B is false
    \end{itemize}
\end{definition}

\begin{definition}[Biimplication (if and only if)]
    $A \Leftrightarrow B \equiv (A \Rightarrow B) \land (B \Rightarrow A)$
\end{definition}

\subsection{Proposition Calculus}
Using logical operators and assertions $P_1,P_2,...,P_k$ to form new assertions and analyze them.

\begin{theorem}[Some true assertions]
    Suppose P,Q, and R are assertions. Then the following assertions are true:
    \begin{enumerate}[label=(\alph*)]
        \item $P \lor \lnot P$
        \item $P \Leftrightarrow \lnot (\lnot P)$
        \item $\lnot (P \land \lnot P)$
        \item $P \Rightarrow Q \Leftrightarrow \lnot P \lor Q$
        \item $\lnot (P \lor Q) \Leftrightarrow \lnot P \land \lnot Q$
        \item $\lnot (P \land Q) \Leftrightarrow \lnot P \lor \lnot Q$
        \item $P \Rightarrow Q \Leftrightarrow \lnot Q \Rightarrow \lnot P$
        \item $(P \lor Q) \land R \Leftrightarrow (P \land R) \lor (Q \land R)$
        \item $(P \land Q) \lor R \Leftrightarrow (P \lor R) \land (Q \lor R)$
        \item $(P \lor Q) \Rightarrow R \Leftrightarrow (P \Rightarrow R) \land (Q \Rightarrow R)$
    \end{enumerate}
\end{theorem}

\subsection{Methods of proof}
If the statement is of the form
\begin{center}
    If P then Q.
\end{center}
\subsubsection{Direct proof}
We only need to consider the case where P is true and deduce the truth of Q. \\
A direct proof of $P \Rightarrow Q$ looks like:
\begin{center}
    Assume that P is true.
\end{center}
Then we use arguements that imply that Q is also true and end the proof with:
\begin{center}
    Hence Q is true.
\end{center}

\subsubsection{Proof by contraposition}
In instead of proving the statement $P \Rightarrow Q$ we prove its contrapositive
($\lnot Q \Rightarrow \lnot P$).
\subsubsection{Proof by contradiction}
In order to prove P we assume the opposite $\lnot P$ to be true and deduce a
condradiction with some obviously true statement Q.
\par Thus, we prove that $\lnot Q \Rightarrow \lnot P$. But then the contrapositive
$Q \Rightarrow P$ must also be true. And the obvious truth of Q implies P to be true.

    \section{Sets}
\subsection{Sets and subsets}
\begin{definition}[Set]
    A is set any collection of "things" or "objects
\end{definition}
\begin{definition}[subset]
    Suppose $A$ and $B$ are sets. The $A$ is called a \emph{subset} of $B$,
    if for every element $ a \in A $ we also have that $ a \in B $. \par
    If $A$ is a subset of $B$,then we write $ A \subset B $ or $ A \subseteq B $. We also say that $B$ conatins $A$. \par
    By $ B \supset A $ or $ B \supseteq A $ we mean $ A \subset B $ or $ A \subseteq B $.
\end{definition}
\begin{example}
    It is true that $ 1 \in \{1,2,3\} $ and $ \{1\} \subseteq \{1,2,3\} $, but \emph{not} that $ 1\subseteq \{1\} \in \{1,2,3\} $ or $ \{1\} \in \{1,2,3\} $
\end{example}
\begin{example}
    Notice that $ \emptyset \in \{\emptyset\} $ and $ \emptyset \subseteq \{\emptyset\} $
\end{example}
\begin{example}
    To following inclusions are proper $$ \mathbb{N} \subsetneq \mathbb{Z} \subsetneq \mathbb{Q} \subsetneq \mathbb{R} \subsetneq \mathbb{C} $$
\end{example}

\begin{definition}[Power set]
    If $B$ is a set, then by $ \mathcal{P}(B) $ we denote the set of all subsets $A$ of $B$. The set $ \mathcal{P}(B) $ is called the \emph{power set} of B.
    \par
    {\color{red} !} The power set of a set is never empty.
\end{definition}

\begin{example}
    Suppose $ A = \{x,y,z\} $m then $ \mathcal{P}(A) $ consists of 8 subsets of $A$.
\end{example}

\begin{proposition}
    Let $ A $ be a set with $n$ elements. Then its power set $ \mathcal{P}(A) $ contains $ 2^n $ elements.
\end{proposition}

\begin{proposition}
    Suppose $ A,B\text{ and } C $ are sets. Then the following holds:
    \begin{enumerate}
        \item If $ A \subseteq B $ and $ B \subseteq C $, then $ A \subseteq C $.
        \item If $ A \subseteq B $ and $ B \subseteq A $, then $ A = B $
    \end{enumerate}
\end{proposition}
\begin{proof}[Proof: Statement 1]
    Suppose $ A \subseteq B $ and $ B \subseteq C $. Let $ a \in A $. Since $ A \subseteq B $, $ a \in B $. Now since $ B \subseteq C $, $ a \in C $.
    Since for every $ a \in A : a \in C $, $ A \subseteq C $
\end{proof}

\subsection{How to describe a set}
\begin{definition}[Set description]
    Let $ P $ be a predicate with reference set $ X $, then $$ \{x \in X \mid P(x)\} $$
    denotes the subset of $ X $ consisting of all elements $ x \in X $ for which the statement $ P(x) $ is true.
\end{definition}
\begin{example}
    The set $ \{x \in \mathbb{R} \mid x > 0\} $ consists of all posistive real numbers.
\end{example}

\subsection{Operations on sets}
\begin{definition}
    Let $A, B$ be sets.
    \begin{enumerate}
        \item \emph{intersection}: $ A \cap B $ - the set of all elements contained in both $A$ and $B$.
        \item \emph{union}: $ A \cup B $ - the set of elements that are in at least on of $A$ or $B$.
        \item Two sets $A$ and $B$ are called \emph{disjoint}, if their intersection $ A \cap B $ is the empty set.
    \end{enumerate}
\end{definition}

\begin{proposition}
    Let $A$,$B$ and $C$ be sets. Then the following holds:
    \begin{enumerate}[label=(\alph*)]
        \item $ A \cup B = B \cup A $
        \item $ A \cup \emptyset = A $
        \item $ A \subseteq (A \cup B) $
        \item If $ A \subseteq B $, then $ A \cup B = B $
        \item $ (A \cup B) \cup C = A \cup (B \cup C)$
        \item $ A \cap B = B \cap A $
        \item $ A \cap \emptyset = \emptyset $
        \item $ A \cap B \subseteq A $
        \item If $ A \subseteq B $, then $ A \cap B = A $
        \item $ (A \cap B) \cap B = A \cap (B \cap C) $
    \end{enumerate}
\end{proposition}

\begin{definition}[Big Unions and Intersections of sets]
    Suppose $I$ is a set and for each element $i$ there exists a set $A_i$, then
    $$ \bigcup_{i\in I}A_i := \{x \mid \text{there is an } i \in I \text{ with } x \in A_i\} $$
    and
    $$ \bigcap_{i \in I} A_i := \{x \mid \text{for all } i \in I \text{ we have } x \in A_i\} $$
    (the set $I$ is called the index set) \par
    If $ \mathscr{C} $ is a set/collection of sets, then we can define
    $$ \bigcup_{A \in \mathscr{C}}A := \{x \mid \text{there is an } A \in \mathscr{C}\} $$
    and
    $$ \bigcap_{A \in \mathscr{C}}A := \{x \mid \text{for all } A \in \mathscr{C} \text{ we have } x \in A\} $$
\end{definition}

\begin{example}
    Suppose for each $ i \in \mathbb{N} $ the set $A_i$ is defined as $\{x \in \mathbb{R} \mid 0 \leq x \leq i\}$. Then
    $$ \bigcap_{i \in \mathbb{N}}A_i = \{0\} $$
    (here we assume that $ 0 \in \mathbb{N} $) and
    $$ \bigcup_{i \in \mathbb{N}}A_i = \mathbb{R}_{\geq 0} = \{x \in \mathbb{R} \mid x \geq 0\} $$
\end{example}

\begin{definition}[Setminus and symmetric difference]
    Let $A$ and $B$ be sets. The \emph{difference} of $A$ and $B$, notation $A \setminus B$, is the set of all elements from $A$ that are \emph{not} in $B$. \par
    The \emph{symmetric difference} of $A$ and $B$, notation $A \triangle B$, is the set of all elements in \emph{exactly one} of $A$ or $B$.
\end{definition}

\begin{proposition}
    Let $A$,$B$ and $C$ be sets. Then the following holds:
    \begin{enumerate}
        \item $ A \setminus B \subseteq A $
        \item If $ A \subseteq B $, then $ A \setminus B = 0 $
        \item $ A = (A \setminus B) \cup (A \cap B) $
        \item $ A \triangle B = (A \setminus B) \cup (B \setminus A) $
        \item $ A \triangle B = B \triangle A $
        \item If $ A \subseteq B $, then $ A \triangle B = B \setminus A $
        \item $ A \triangle (B \triangle C) = (A \triangle B) \triangle C $
    \end{enumerate}
\end{proposition}

\begin{proposition}
    Let $A$,$B$ and $C$ be sets. Then the following hold:
    \begin{enumerate}
        \item $ (A \cup B) \cap C = (A \cap C) \cup (B \cap C) $
        \item $ (A \cap B) \cup C = (A \cup C) \cap (B \cup C) $
        \item $ A \setminus (B \cup C) = (A \setminus B) \cap (A \setminus C) $
        \item $ A \setminus (B \cap C) = (A \setminus B) \cup (A \setminus C) $
    \end{enumerate}
\end{proposition}

\begin{definition}[Set Complement]
    If one is working inside a fixed set $U$ and only cnsidering subsets of $U$, then the difference $U \setminus A$ is also called the \emph{complement} of $A$ in $U$.
    We write $A^*$ or $A^c$ for the complement of $A$ in $U$. In this case the set $U$ is also called the \emph{universe}.
\end{definition}

\begin{proposition}
    For subsets $A$,$B$ and $C$ of the universe $U$ we have:
    \begin{enumerate}
        \item $ A \cup A^* = U $
        \item $ B \setminus C = B \cap C^* $
        \item $ (A^*)^* = A $
        \item If $ A \subseteq B $ then $ B^* \subseteq A^* $
        \item $ (A \cup B)^* = A^* \cap B^* $
        \item $ (A \cap B)^* = A^* \cup B^* $
    \end{enumerate}
\end{proposition}

\subsection{Cartesian product}
\begin{definition}[Cartesian Product]
    The Cartesian product $ A_1 \times A_2 \times \dots \times A_k $ of sets $ A_1,\dots,A_k $ is the set of all ordered k-tuples $ (a_1,a_2,\dots,a_k) $ where $ a_i \in A_i $ for $ 1 \leq i \leq k $. \par
    In particular, if $A$ and $B$ are sets, then
    $$ A \times B = {(a,b) \mid a \in A \text{ and } b \in B} $$
\end{definition}

\subsection{Partitions}
\begin{definition}[Partition]
    Let $S$ be a none-empty set. A collection $ \Pi $ of subsets of $S$ is called a \emph{partition} if and only if
    \begin{enumerate}
        \item $ \emptyset \notin \Pi $
        \item $ \bigcup_{X \in \Pi}X = S $
        \item for all $ X \ne Y \in \Pi $ we have $ X \cap Y = \emptyset $
    \end{enumerate}
\end{definition}

\begin{example}
    The set $ \{1,2,\dots,10\} $A can be partiioned into the sets $ \{1,2,3\}, \{4,5\}, \{6,7,8,9,10\} $
\end{example}
\begin{example}
    Suppose $ \mathcal{L} $ is the set of all lines in $\mathbb{R}^2$ parallel to a fixed line $\ell$. Then $ \mathcal{L} $ partitions $\mathbb{R}^2$
\end{example}
\begin{example}
    Let $ n > 1 $ be an integer. Then the set $ \mathbb{Z} $ can be partitioned into the following subsets:
    \begin{align*}
        \{z \in \mathbb{Z} &\mid z = 0 + nx \text{ for some } x \in \mathbb{Z}\} \\
        \{z \in \mathbb{Z} &\mid z = 1 + nx \text{ for some } x \in \mathbb{Z}\} \\
                           &\vdots \\
        \{z \in \mathbb{Z} &\mid z = (n-1) + nx \text{ for some } x \in \mathbb{Z}\} \\
    \end{align*}
\end{example}

\subsection{Quantifiers}
\begin{definition}[Quantifiers]
    Let $ P $ be a predicate on a reference set $X$. Then by
    $$ \forall x \in X \left[P(x)\right] $$
    we denote the assertion "For all $ x \in X $ the assertion $ P(x) $ is true".

    $ \forall $ is called the \emph{for all-}quantifier or \emph{universal quantifier}.

    By
    $$ \exists x \in X \left[P(x)\right] $$
    we denote the assertion "There exists an $ x \in X $ with $ P(x) $ true".

    $ \exists $ is called the \emph{existential quantifier}.
\end{definition}

\begin{example}
    The following statements are true:
    $$ \forall x \in \mathbb{R} \left[x \ge 0 \implies |x| = x\right], $$
    $$ \exists x \in \mathbb{R} \left[|x| = x\right] $$
    $$ \forall x \in \mathbb{Q} \left[-1 < \sin(x) < 1\right] $$

    Here a few statements that are false:
    $$ \forall x \in \mathbb{R} \left[|x| = x\right] $$
    $$ \forall x \in \mathbb{R} \left[-1 < \sin(x) < 1\right] $$
\end{example}

\begin{example}
    We can make combinations of quantifiers to create various assertions. For example
    $$ \forall x \in \mathbb{Z} \exists y \in \mathbb{Z} \left[x + y = 0\right] $$
\end{example}

\begin{proposition}[DeMorgan's rule]
    $$ \lnot (\forall x \in X \left[P(x)\right]) \iff \exists x \in X \left[\lnot P(x)\right]$$
    $$ \lnot (\exists x \in X \left[P(x)\right]) \iff \forall x \in x \left[\lnot P(x)\right] $$
\end{proposition}

\begin{example}
    Let $ X = \left\{1,2,\dots,9\right\} $ and consider the following statements.
    $$ P = \forall x \in X \exists y \in X \left[x+y = 10\right] $$
    $$ Q = \exists x \in X \forall y \in X \left[x+y = 10\right] $$

    The assertion $P$ is clearly true.

    The assertion $Q$ is false. We prove $ \lnot Q $. By DeMorgan's rule the assertion $ \lnot Q $ is equivalent with
    $$ R = \forall x \in X \exists y \in X \left[x+y \ne 10\right] $$
\end{example}
    \section{Sets}
\subsection{Sets and subsets}
\begin{definition}[Set]
    A is set any collection of "things" or "objects
\end{definition}
\begin{definition}[subset]
    Suppose $A$ and $B$ are sets. The $A$ is called a \emph{subset} of $B$,
    if for every element $ a \in A $ we also have that $ a \in B $. \par
    If $A$ is a subset of $B$,then we write $ A \subset B $ or $ A \subseteq B $. We also say that $B$ conatins $A$. \par
    By $ B \supset A $ or $ B \supseteq A $ we mean $ A \subset B $ or $ A \subseteq B $.
\end{definition}
\begin{example}
    It is true that $ 1 \in \{1,2,3\} $ and $ \{1\} \subseteq \{1,2,3\} $, but \emph{not} that $ 1\subseteq \{1\} \in \{1,2,3\} $ or $ \{1\} \in \{1,2,3\} $
\end{example}
\begin{example}
    Notice that $ \emptyset \in \{\emptyset\} $ and $ \emptyset \subseteq \{\emptyset\} $
\end{example}
\begin{example}
    To following inclusions are proper $$ \mathbb{N} \subsetneq \mathbb{Z} \subsetneq \mathbb{Q} \subsetneq \mathbb{R} \subsetneq \mathbb{C} $$
\end{example}

\begin{definition}[Power set]
    If $B$ is a set, then by $ \mathcal{P}(B) $ we denote the set of all subsets $A$ of $B$. The set $ \mathcal{P}(B) $ is called the \emph{power set} of B.
    \par
    {\color{red} !} The power set of a set is never empty.
\end{definition}

\begin{example}
    Suppose $ A = \{x,y,z\} $m then $ \mathcal{P}(A) $ consists of 8 subsets of $A$.
\end{example}

\begin{proposition}
    Let $ A $ be a set with $n$ elements. Then its power set $ \mathcal{P}(A) $ contains $ 2^n $ elements.
\end{proposition}

\begin{proposition}
    Suppose $ A,B\text{ and } C $ are sets. Then the following holds:
    \begin{enumerate}
        \item If $ A \subseteq B $ and $ B \subseteq C $, then $ A \subseteq C $.
        \item If $ A \subseteq B $ and $ B \subseteq A $, then $ A = B $
    \end{enumerate}
\end{proposition}
\begin{proof}[Proof: Statement 1]
    Suppose $ A \subseteq B $ and $ B \subseteq C $. Let $ a \in A $. Since $ A \subseteq B $, $ a \in B $. Now since $ B \subseteq C $, $ a \in C $.
    Since for every $ a \in A : a \in C $, $ A \subseteq C $
\end{proof}

\subsection{How to describe a set}
\begin{definition}[Set description]
    Let $ P $ be a predicate with reference set $ X $, then $$ \{x \in X \mid P(x)\} $$
    denotes the subset of $ X $ consisting of all elements $ x \in X $ for which the statement $ P(x) $ is true.
\end{definition}
\begin{example}
    The set $ \{x \in \mathbb{R} \mid x > 0\} $ consists of all posistive real numbers.
\end{example}

\subsection{Operations on sets}
\begin{definition}
    Let $A, B$ be sets.
    \begin{enumerate}
        \item \emph{intersection}: $ A \cap B $ - the set of all elements contained in both $A$ and $B$.
        \item \emph{union}: $ A \cup B $ - the set of elements that are in at least on of $A$ or $B$.
        \item Two sets $A$ and $B$ are called \emph{disjoint}, if their intersection $ A \cap B $ is the empty set.
    \end{enumerate}
\end{definition}

\begin{proposition}
    Let $A$,$B$ and $C$ be sets. Then the following holds:
    \begin{enumerate}[label=(\alph*)]
        \item $ A \cup B = B \cup A $
        \item $ A \cup \emptyset = A $
        \item $ A \subseteq (A \cup B) $
        \item If $ A \subseteq B $, then $ A \cup B = B $
        \item $ (A \cup B) \cup C = A \cup (B \cup C)$
        \item $ A \cap B = B \cap A $
        \item $ A \cap \emptyset = \emptyset $
        \item $ A \cap B \subseteq A $
        \item If $ A \subseteq B $, then $ A \cap B = A $
        \item $ (A \cap B) \cap B = A \cap (B \cap C) $
    \end{enumerate}
\end{proposition}

\begin{definition}[Big Unions and Intersections of sets]
    Suppose $I$ is a set and for each element $i$ there exists a set $A_i$, then
    $$ \bigcup_{i\in I}A_i := \{x \mid \text{there is an } i \in I \text{ with } x \in A_i\} $$
    and
    $$ \bigcap_{i \in I} A_i := \{x \mid \text{for all } i \in I \text{ we have } x \in A_i\} $$
    (the set $I$ is called the index set) \par
    If $ \mathscr{C} $ is a set/collection of sets, then we can define
    $$ \bigcup_{A \in \mathscr{C}}A := \{x \mid \text{there is an } A \in \mathscr{C}\} $$
    and
    $$ \bigcap_{A \in \mathscr{C}}A := \{x \mid \text{for all } A \in \mathscr{C} \text{ we have } x \in A\} $$
\end{definition}

\begin{example}
    Suppose for each $ i \in \mathbb{N} $ the set $A_i$ is defined as $\{x \in \mathbb{R} \mid 0 \leq x \leq i\}$. Then
    $$ \bigcap_{i \in \mathbb{N}}A_i = \{0\} $$
    (here we assume that $ 0 \in \mathbb{N} $) and
    $$ \bigcup_{i \in \mathbb{N}}A_i = \mathbb{R}_{\geq 0} = \{x \in \mathbb{R} \mid x \geq 0\} $$
\end{example}

\begin{definition}[Setminus and symmetric difference]
    Let $A$ and $B$ be sets. The \emph{difference} of $A$ and $B$, notation $A \setminus B$, is the set of all elements from $A$ that are \emph{not} in $B$. \par
    The \emph{symmetric difference} of $A$ and $B$, notation $A \triangle B$, is the set of all elements in \emph{exactly one} of $A$ or $B$.
\end{definition}

\begin{proposition}
    Let $A$,$B$ and $C$ be sets. Then the following holds:
    \begin{enumerate}
        \item $ A \setminus B \subseteq A $
        \item If $ A \subseteq B $, then $ A \setminus B = 0 $
        \item $ A = (A \setminus B) \cup (A \cap B) $
        \item $ A \triangle B = (A \setminus B) \cup (B \setminus A) $
        \item $ A \triangle B = B \triangle A $
        \item If $ A \subseteq B $, then $ A \triangle B = B \setminus A $
        \item $ A \triangle (B \triangle C) = (A \triangle B) \triangle C $
    \end{enumerate}
\end{proposition}

\begin{proposition}
    Let $A$,$B$ and $C$ be sets. Then the following hold:
    \begin{enumerate}
        \item $ (A \cup B) \cap C = (A \cap C) \cup (B \cap C) $
        \item $ (A \cap B) \cup C = (A \cup C) \cap (B \cup C) $
        \item $ A \setminus (B \cup C) = (A \setminus B) \cap (A \setminus C) $
        \item $ A \setminus (B \cap C) = (A \setminus B) \cup (A \setminus C) $
    \end{enumerate}
\end{proposition}

\begin{definition}[Set Complement]
    If one is working inside a fixed set $U$ and only cnsidering subsets of $U$, then the difference $U \setminus A$ is also called the \emph{complement} of $A$ in $U$.
    We write $A^*$ or $A^c$ for the complement of $A$ in $U$. In this case the set $U$ is also called the \emph{universe}.
\end{definition}

\begin{proposition}
    For subsets $A$,$B$ and $C$ of the universe $U$ we have:
    \begin{enumerate}
        \item $ A \cup A^* = U $
        \item $ B \setminus C = B \cap C^* $
        \item $ (A^*)^* = A $
        \item If $ A \subseteq B $ then $ B^* \subseteq A^* $
        \item $ (A \cup B)^* = A^* \cap B^* $
        \item $ (A \cap B)^* = A^* \cup B^* $
    \end{enumerate}
\end{proposition}

\subsection{Cartesian product}
\begin{definition}[Cartesian Product]
    The Cartesian product $ A_1 \times A_2 \times \dots \times A_k $ of sets $ A_1,\dots,A_k $ is the set of all ordered k-tuples $ (a_1,a_2,\dots,a_k) $ where $ a_i \in A_i $ for $ 1 \leq i \leq k $. \par
    In particular, if $A$ and $B$ are sets, then
    $$ A \times B = {(a,b) \mid a \in A \text{ and } b \in B} $$
\end{definition}

\subsection{Partitions}
\begin{definition}[Partition]
    Let $S$ be a none-empty set. A collection $ \Pi $ of subsets of $S$ is called a \emph{partition} if and only if
    \begin{enumerate}
        \item $ \emptyset \notin \Pi $
        \item $ \bigcup_{X \in \Pi}X = S $
        \item for all $ X \ne Y \in \Pi $ we have $ X \cap Y = \emptyset $
    \end{enumerate}
\end{definition}

\begin{example}
    The set $ \{1,2,\dots,10\} $A can be partiioned into the sets $ \{1,2,3\}, \{4,5\}, \{6,7,8,9,10\} $
\end{example}
\begin{example}
    Suppose $ \mathcal{L} $ is the set of all lines in $\mathbb{R}^2$ parallel to a fixed line $\ell$. Then $ \mathcal{L} $ partitions $\mathbb{R}^2$
\end{example}
\begin{example}
    Let $ n > 1 $ be an integer. Then the set $ \mathbb{Z} $ can be partitioned into the following subsets:
    \begin{align*}
        \{z \in \mathbb{Z} &\mid z = 0 + nx \text{ for some } x \in \mathbb{Z}\} \\
        \{z \in \mathbb{Z} &\mid z = 1 + nx \text{ for some } x \in \mathbb{Z}\} \\
                           &\vdots \\
        \{z \in \mathbb{Z} &\mid z = (n-1) + nx \text{ for some } x \in \mathbb{Z}\} \\
    \end{align*}
\end{example}

\subsection{Quantifiers}
\begin{definition}[Quantifiers]
    Let $ P $ be a predicate on a reference set $X$. Then by
    $$ \forall x \in X \left[P(x)\right] $$
    we denote the assertion "For all $ x \in X $ the assertion $ P(x) $ is true".

    $ \forall $ is called the \emph{for all-}quantifier or \emph{universal quantifier}.

    By
    $$ \exists x \in X \left[P(x)\right] $$
    we denote the assertion "There exists an $ x \in X $ with $ P(x) $ true".

    $ \exists $ is called the \emph{existential quantifier}.
\end{definition}

\begin{example}
    The following statements are true:
    $$ \forall x \in \mathbb{R} \left[x \ge 0 \implies |x| = x\right], $$
    $$ \exists x \in \mathbb{R} \left[|x| = x\right] $$
    $$ \forall x \in \mathbb{Q} \left[-1 < \sin(x) < 1\right] $$

    Here a few statements that are false:
    $$ \forall x \in \mathbb{R} \left[|x| = x\right] $$
    $$ \forall x \in \mathbb{R} \left[-1 < \sin(x) < 1\right] $$
\end{example}

\begin{example}
    We can make combinations of quantifiers to create various assertions. For example
    $$ \forall x \in \mathbb{Z} \exists y \in \mathbb{Z} \left[x + y = 0\right] $$
\end{example}

\begin{proposition}[DeMorgan's rule]
    $$ \lnot (\forall x \in X \left[P(x)\right]) \iff \exists x \in X \left[\lnot P(x)\right]$$
    $$ \lnot (\exists x \in X \left[P(x)\right]) \iff \forall x \in x \left[\lnot P(x)\right] $$
\end{proposition}

\begin{example}
    Let $ X = \left\{1,2,\dots,9\right\} $ and consider the following statements.
    $$ P = \forall x \in X \exists y \in X \left[x+y = 10\right] $$
    $$ Q = \exists x \in X \forall y \in X \left[x+y = 10\right] $$

    The assertion $P$ is clearly true.

    The assertion $Q$ is false. We prove $ \lnot Q $. By DeMorgan's rule the assertion $ \lnot Q $ is equivalent with
    $$ R = \forall x \in X \exists y \in X \left[x+y \ne 10\right] $$
\end{example}

    \section{Relations}
\subsection{Binary relations}
\begin{definition}[Ralation]
    A relation $R$ between the sets $S$ and $T$ is a subset of the Cartesian product $S \times T$.  \par
    Suppose $R$ is a relation between $S$ and $T$. If $(a,b) \in R$, we say $a$ is in relation $R$ to $b$ ($aRb$).  \par
    $S$ is called the domain, while $S$ - \emph{codomain}. \par
    If $S=T$ we say $R$ is a relation on $S$.
\end{definition}
\begin{example}
    We give some examples:
    \begin{enumerate}
        \item $ R \{(0,0), (1,0), (2,1)\} $ is a relation between sets $ S = \{0,1,2\} $ and $ T = \{0,1\} $
        \item $ R = \{(x,y) \in \mathbb{R}^2 \mid y = x^2 \} $ is a relation on $ \mathbb{R} $
        \item Let $ \Omega $ be a set, then "is a subset of" $ \subseteq $ is a relation on the set $S = \mathcal{P}(\Omega)$ of all subsets of $ \Omega $
    \end{enumerate}
\end{example}

\begin{definition}[Image]
    Let $R$ be a relation from a set $S$ to a set $T$. Then for each element $a \in S$ we define $ \left[a\right]_R $ to be the set
    $$ \left[a\right]_R := \left\{b \in T \mid aRb\right\}$$
    (Sometimes this set is also denoted by $ R(a) $) This set is called the $(R-)$ image of $a$. \par
    For $b \in T$ the set
    $$ _R\left[b\right] := \left\{a \in S \mid aRb\right\} $$
\end{definition}

Relations between finite sets can be described using matrices.
\begin{definition}[Adjecency Matrix]
    If $ S = \left\{s_1, s_2, \dots, s_n\right\} $ and $ T = \left\{t_1, t_2, \dots, t_m\right\} $ are finite sets
    and $ R \subseteq S \times T $ is a binary relation, then the \emph{adjecency} matrix $ A_R $ of the relation $ R $ is the $ n \times m $ matrix
    whose rows are indexted by $S$ and columns by $T$ defined by:
    $$ A_{s,t} = \begin{cases}
        1 &\text{ if } (s,t) \in R \\
        0 &\text{ otherwise }
    \end{cases} $$
\end{definition}

\begin{example}
    \begin{enumerate}
        \item The adjecency matrix of the relation $ R = \left\{\left(0,0\right), \left(1,0\right),\left(2,1\right)\right\} $ between the sets $ S = \left\{0,1,2\right\} $ and $ T = \left\{0,1\right\} $ equals
        $$ \begin{pmatrix}
            1 & 0 \\
            1 & 0 \\
            0 & 1
        \end{pmatrix} $$
        \item The adjecency matrix of the identity relation on a set $S$ of size n:
        $$ I_n = \begin{pmatrix}
            1 & 0 & \dots & 0 & 0 \\
            0 & 1 & \dots & 0 & 0 \\
            \vdots & & \ddots & & \vdots \\
            0 & 0 & \dots & 1 & 0 \\
            0 & 0 & \dots & 0 & 1
        \end{pmatrix} $$
        \item The adjecency matrix of relation $ \leq $ on the set $ \left\{1,2,3,4,5\right\} $ is the upper triangular matirx:
        $$ \begin{pmatrix}
            1 & 1 & 1 & 1 & 1\\
            0 & 1 & 1 & 1 & 1\\
            0 & 0 & 1 & 1 & 1\\
            0 & 0 & 0 & 1 & 1\\
            0 & 0 & 0 & 0 & 1\\

        \end{pmatrix} $$
    \end{enumerate}
\end{example}

Some relations have special properties:
\begin{definition}[Special relation properties]
    Let $ R $ be a relation on set $S$. Then $R$ is called
    \begin{itemize}
        \item \emph{Reflexive} if for all $ x \in S $ we have $ (x,x) \in R $
        \item \emph{Irreflexive} if for all $ x \in S $ we have $ (x,x) \notin R $
        \item \emph{Symmetric} if for all $ x,y \in S $ we have $ xRy \implies yRx $
        \item \emph{Antisymmetric} if for all $ x,y \in S $ we have that $ xRy \text{ and } yRx \implies x = y$
        \item \emph{Transitive} if for all $ x,y,z \in S $ we have that $ xRy $ nd $ yRz \implies xRz$
    \end{itemize}
\end{definition}

\subsection{Relations and Directed Graphs}

\begin{definition}[Directed graph]
    A \emph{directed edge} of a set $V$ is an element of $V\times V$. If $ e (v,w) $ is a directed edge of $V$, then
    $v$ is called its \emph{tail} and $w$ its \emph{head}. Both $v$ and $w$ are called \emph{end points} of the edge $e$.
    The \emph{reverse} of the edge $e$ is the edge $ (w,v) $. A \emph{loop} is an edge from a vertex to itself. \par
A \emph{directed graph} (also called \emph(digraph)) $\Gamma = (V,E)$ consists of a set of \emph{vertices} and a subset $E$ of $V\times V$ of (directed) \emph{edges}.
The elements of $V$ are called teh vertices of $\Gamma$ and the elements of $E$ the \emph{edges} of $\Gamma$
\end{definition}

\subsubsection{Some graph theoretical language}
Suppose $ \Gamma = (V,E) $ is a digraph. A \emph{walk} from $v$ to $w$, where $v,w \in V$, is a sequence $ v_0,v_1,\dots,v_k $ of vertices with $ v_0 = v, v_k = w $ and $ (v_i, v_{i+1}) \in E $ for all $ 0 \leq i \le k $.
A \emph{path} from $v$ to $w$ is a walk from $v$ to $w$ in which all vertices, except possibly the first vertex $v$ and the last vertex $w$ are different. \par
An \emph{undericted walk} from $v$ to $w$ is a sequence $ v_0,v_1,\dots,v_k $ of vertices with $ (v_i,v_{i+1}) \in E $ or $ (v_{i+1}, v_i) \in E $ for all $ 0 \leq i \le k $, while an \emph{undirected path} from $v$ to $w$ is an undirected walk in which all vertices except possibly the first and last are different.
The \emph{length} of the (directed or undirected) walk or path is $k$. A \emph{cycle} is a path from $v$ to $v$ of length at least 1. \par
If $v,w \in V$ are vertices of the digraph $\Gamma$, then the \emph{distance} from $v$ to $w$ is the minimum of the lengths of
the paths from $v$ to $w$. The distance is set to $\infty$ (infinity) if there is no path from $v$ to $w$. \par
The digraph is called \emph{weakly connected} if for any two vertices $v$ and $w$ there is an undirected path between
$v$ and $w$. It is \emph{called strongly} connected if there exist paths in both directions. \par
% If W is a subset of V, then the induced subgraph of Γ on W is the digraph (W,E ∩(W ×W)). A (weakly)
% connected component C of Γ is a maximal subset of V such that the induced subgraph is (weakly) connected.
% If we define the relation R on V by vRw if and only if there is an undirected path from v to w, then R is an
% equivalence relation. The weakly connected components of Γ are then the R-equivalence classes.
% A strongly connected component is a maximal subset of V such that the induced subgraph is strongly
% connected.
% Let S be the relation on the set V where vSw if there is an directed path from v to w and from w to v.
% Then S is an equivalence relation. (Proof this!) The strongly connected components are then the Sequivalence classes.
% As we can identify a directed graph with the corresponding relation, we can of course represent the graph
% by its adjacency matrix as well as by a diagram as in 3.1.6. However, as a directed graph Γ = (V,E) defines a
% relation on a set V, there is no need to draw this set twice. The usual way that we draw the graph Γ is to draw
% the vertices of V with arrows from a vertex v ∈ V to a vertex w ∈ V if and only if (v,w) ∈ E.

\begin{proposition}
    Let $ (V,E) $ be a directed graph. Then we have the following.
    \begin{enumerate}
        \item $E$ is reflexive if and only if every vertex $ v \in V $ is in a loop.
        \item $E$ is symmetric if and only if for every edge $ e \in E $, also its reverse is in $E$.
        \item $E$ is transitive if and only if for each walk of length at least 1 starting from $x$ and ending in $y$ we have that $ (x,y) \in E $.
    \end{enumerate}
\end{proposition}

\begin{example}
    The complete directed graph on a vertex set $V$ is the graph in which all vertices are adjacent to each other and tehmselves. This graph is clearly strongly connected. \par
    So, te corresponding relation is reflexive, symmetric and transitive.
\end{example}

\begin{proposition}
    Let $R$ be a relation on the set $V$ which is reflexive, symmetric and transitive. Then all (weakly) connected components of the graph $ \Gamma = (V,R)$ are complete graphs.
\end{proposition}

\begin{definition}[Indegree / Outdegree]
    Let $ \Gamma = (V,E) $ be a digraph and $ v \in V $ a vertex. The \emph{indegree} of $v$ is the number of edges with $v$ as head.
    The \emph{outdegree} of $v$ is teh number of edges with $v$ as tail.
\end{definition}

\subsection{Equivalence relations}
\begin{definition}[Equivalence Relation]
    A relation $R$ on a set $S$ is called an \emph{equivalence relation} on $S$ if and only if it is relfexive, symmetric and transitive.
\end{definition}
\begin{example}
    Consider the plane $ \mathbb{R}^2 $ and in it the set $S$ of straight lines. We call two lines parallel in S if and only if they are equal or do not intersect.
    Notice that two lines in S are parallel if and only if thir slopes are equal. Being parallel defines an equivalence relation on the set $S$.
\end{example}
\begin{example}
    Fix $n \in \mathbb{Z}$, and consider the relation $R$ on $\mathbb{Z}$ by $aRb$ if an only if $a-b$ is divisible by $n$. We also write $ a = b \pmod{n} $. \par
    The relation $R$ is an equivalence realtion. Indeed, suppose $a,b,c \in \mathbb{Z}$. Then
    \begin{enumerate}
        \item $ aRa $ as $ a-a=0 $ is divisible by n.
        \item If $ aRb $, then $ a-b $ is divisible by $n$ and hence also $ b - a $. Thus $ bRa $.
        \item If $ aRb, bRc $, then $n$ divides both $ a-b $ and $ b-c $ and then also $ (a-b)+(b-c) = a-c $. So $ aRc $
    \end{enumerate}
\end{example}

\begin{example}
    Let $\Pi$ be a partition of the set $S$. We define the relation $R_\Pi$ as follows: $a,b \in S$ are in relation $R_\Pi$
    if and only if there is a subset $X$ of $S$ in $\Pi$ containing both $a$ and $b$. We check that the relation $R_\Pi$ is an equivalence relation on $S$.
    \begin{itemize}
        \item Reflexivity: Let $a \in S$. Then there is an $X \in \Pi$ containing a. Hence, $a,a \in X$ and $a R_\Pi a$
        \item Symmetry: Let $a R_\Pi b$. Then there is an $X \in \Pi$ with $a,b \in X$. But then also $b,a \in X$ and $b R_\Pi a$
        \item Transitivity: If $a, b, c \in S$ with $a R_\Pi b$ and $b R_\Pi c$, then there are $X,Y \in \Pi$ with $a,b \in X$ and $b,c \in Y$.
        However, then $b$ is in both $X$ and $Y$. But then, as $\Pi$ partitions $S$, we have $X = Y$. So $a,c \in X$ and $a R_\Pi c$
    \end{itemize}
\end{example}

\begin{lemma}
    Let $R$ be an equivalence relation on a set $S$. If $b\in \left[a\right]_R$, then $ \left[b\right]_R = \left[a\right]_R$
\end{lemma}
\begin{proof}[Proof]
    Suppose $ b \in \left[a\right]_R $. Thus $aRb$. If $ c \in \left[b\right]_R $, then $bRc$ and, as $aRb$, we have by transitivity $aRc$.
    In particular, $ [b]_R \subseteq [a]_R $.\par
    Since, by symmetry of $R$, $aRb$ implies $bRa$ and hence $a \in [b]_R$, we similarly get $ [a]_R \subseteq [b]_R $.
\end{proof}

\begin{definition}[Equivalence classes]
    Let $R$ be an equivalence relation on a aset $S$. Then the sets $ [s]_R $, where $ s \in S $ are called the $R$-\emph{equivalence} calsses on S. \par
    We denote the set of $R$-equivalence classes by $ S/R $
\end{definition}
% \begin{definition}[name of the definition]

% \end{definition}

\begin{theorem}
    Let $R$ be an equivalence relation on a set $S$. Then the set $S/R$ of R-equivalence classes partions the set $S$.
\end{theorem}
\begin{proof}[Proof]
    Let $ \Pi_R $ be the set of $R$-equivalence classes. Then by reflexivity of $R$ we find that each element $ a \in S $
    is inside the class $ \left[a\right]_R \Pi_R $.

    If an element $ a \in S $ is in the classes $ \left[b\right]_R and \left[c\right]_R $ of $ \Pi $, then by the previous lemma
    we find $ \left[b\right]_R = \left[a\right]_R $ and $ \left[c\right]_R = \left[a\right]_R $. In particular, $ \left[b\right]_R = \left[c\right]_R $.
    Thus each element $ a \in S $ is inside an unique member of $ \Pi_R $, which therefore is a partition of $S$.
\end{proof}

\begin{example}[Construction of $\mathbb{Q}$]
    The rational numbers can be constructed from integers with the help of an equivalence relation.

    We consider the set $V = Z \times Z \setminus \{0\}$. On $V$ we define the relation $\equiv$ by
    $$ (a,b) \equiv (c,d) \iff a \cdot d = b \cdot c $$
    for all $ (a,b) $ and $ (c,d) $ in $V$.

    Now we denote the $ \equiv $-equivalence class of a pair ($a,b$) by $ \frac{a}{b} $.
\end{example}

\subsection{Composition of relations}
If $R_1$ and $R_2$ are relations between a set $S$ and a set $T$, then we can form new relations
by taking the intersection $R_1 \cap R_2$ or the union $R_1 \cup R_2$. Also the complement of $R_1$
in $R_2$, $R_1 \setminus R_2$ is a new relation. Furhtermore, we can consider the relation $R^{\top}$
(sometimes also denoted by $R^{-1}, R^{\sim} \text{ or } R^{\vee}$) from $T$ to $S$ as the relation
$ \{(t,s) \in T \times S \mid (s,t) \in R\} $

Another way of making new relations out of old ones is the following. If $R_1$ is a relation between $S$ and
$T$ and $R_2$ is a relation between $T$ and $U$, then the \emph{composition} or product $R=R_1;R_2$ (sometimes
denoted by $R_2 \circ R_1$ or $R_1 * R_2$) is the relation between $S$ and $U$ defined by $sRu$ for $s \in S$ and $u \in U$, if and
only if there is a $t \in T$ with $sR_1t$ and $tR_2u$.

\begin{example}
    $ R_1 = \left\{(1,2),(2,3),(3,3),(2,4)\right\} $ and $R_2 = \left\{(1,a), (2,b), (3,c), (3,d)\right\}$.
    Then $ R_1;R_2 = \left\{(1,b), (2,c), (3,c), (2,d)\right\} $.
\end{example}

We get the adjecency matrix of a composition by multiplying the respective adjecency matrices and then replacing
all non-zero entries with 1.

\begin{example}
    Suppose $R_1 = $ $\{(1,2),(2,3),(3,3), (2,4), (3,1)\}$ and $R_2$ is the relation \{\(\left(1,1\right)\),
    \(\left(2,3\right)\), \(\left(3,1\right)\), \(\left(3,3\right)\), \(\left(4,2\right)\)\}
    Then the adjecency matrices $A_1$ and $A_2$ for $R_1$ and $R_2$ are
    \[
        A_1 = \begin{pmatrix}
            0 & 1& 0& 0 \\
            0 & 0 & 1 & 1 \\
            1 & 0 & 1 & 0
        \end{pmatrix},
        A_2 = \begin{pmatrix}
            1 & 0 & 0 \\
            0 & 0 & 1 \\
            1 & 0 & 1 \\
            0 & 1 & 0
        \end{pmatrix}
    \]
    The product of these matrices equals
    $$ M = \begin{pmatrix}
        0&0&1 \\
        1&1&1 \\
        2&0&1
    \end{pmatrix} $$
    So the adjecency matrix of $R_1;R_2$ is
    $$\begin{pmatrix}
        0&0&1 \\
        1&1&1 \\
        1&0&1
    \end{pmatrix}$$
\end{example}

\begin{proposition}
    Suppose $R_1$ si a relation from $S$ to $T$, $R_2$ a relation from $T$ to $U$ and $R_3$ a realtion from $U$ to $V$.
    Then $R_1;(R_2;R_3) = (R_1;R_2);R_3$.

    Composing relations is associative.
\end{proposition}

\subsection{Transitive Closure}

\begin{lemma}\label{lemma:3.5.1}
    Let $ \mathscr{C} $ be a collection of relations $R$ on a set $S$. If all relations $R$ in $\mathscr{C}$ are transitive (symmetric or reflexive),
    then the relation $ \bigcap_{R \in \mathscr{C}} R$ is also transitive (symmetric or transitive, respectively).
\end{lemma}

\begin{proof}[Proof]
    Let $ \bar{R} = \bigcap_{R \in \mathscr{C}}R $. Suppose all memebers of $ \mathscr{C} $ are transitive. Then for all $ a,b,c \in S $ with $ a\bar{R}b $
    and $ b\bar{R}c $ we have $aRb$ and $bRc$ for all $ R \in \mathscr{C} $. Thus by transitivity of each $ R \in \mathscr{C} $ we also have $ aRc $ for each $R \in \mathscr{C}$.
    Thus we find $ a \bar{R}c $. Hence $ \bar{R} $ is transitive.
\end{proof}

The above lemma makes it possible to define the \emph{reflexive, symmetric, or transitive closure}
of a relation $R$ on a set $S$. It is the smallest refexive, symmetric or transitive
relation containing $R$. This means, as follows from \cref{lemma:3.5.1}, it is the
intersection $ \bigcap_{R^\prime \in \mathscr{C}}R^\prime $, where $ \mathscr{C} $
is the collection of all reflexive, symmetric, or transitive relations containing $R$.

\begin{proposition}
    $ \bigcup_{n>0}R^n $ is the transitive closure of the relation $R$.
\end{proposition}

\begin{proof}[Proof]
    Define $ \bar{R} = \bigcup_{n>0}R^n $. We prove transitivity of $ \bar{R} $. Let $ a\bar{R}b $
    and $ b\bar{R}c $, then there are sequence $ a = a_1,\dots,a_k = b $ and $ b = b_1,\dots,b_l = c $
    with $ a_iRa_{i+1} $ and $ b_iRb_{i+1} $. But then the sequence $ a=a_1=c_1,\dots,c_k=a_k=b_1,\dots,c_{k+l-1}=b_l=c$
    is a sequence from $a$ to $c$ with $ c_iRc_{i+1} $. Hence $ aR^{k+l-2}c $ and $ a\bar{R}c $.
\end{proof}

The transitive, symmetric and reflexive closure of a relation $R$ is an equivalence relation.
In terms of the graph $ \Gamma_R $, the equivalence classes are the strongly connected
componenets of $ \Gamma_R $.

\begin{algorithm}[H]
    Warhall's Algorithm
\end{algorithm}

    \section{Relations}
\subsection{Binary relations}
\begin{definition}[Ralation]
    A relation $R$ between the sets $S$ and $T$ is a subset of the Cartesian product $S \times T$.  \par
    Suppose $R$ is a relation between $S$ and $T$. If $(a,b) \in R$, we say $a$ is in relation $R$ to $b$ ($aRb$).  \par
    $S$ is called the domain, while $S$ - \emph{codomain}. \par
    If $S=T$ we say $R$ is a relation on $S$.
\end{definition}
\begin{example}
    We give some examples:
    \begin{enumerate}
        \item $ R \{(0,0), (1,0), (2,1)\} $ is a relation between sets $ S = \{0,1,2\} $ and $ T = \{0,1\} $
        \item $ R = \{(x,y) \in \mathbb{R}^2 \mid y = x^2 \} $ is a relation on $ \mathbb{R} $
        \item Let $ \Omega $ be a set, then "is a subset of" $ \subseteq $ is a relation on the set $S = \mathcal{P}(\Omega)$ of all subsets of $ \Omega $
    \end{enumerate}
\end{example}

\begin{definition}[Image]
    Let $R$ be a relation from a set $S$ to a set $T$. Then for each element $a \in S$ we define $ \left[a\right]_R $ to be the set
    $$ \left[a\right]_R := \left\{b \in T \mid aRb\right\}$$
    (Sometimes this set is also denoted by $ R(a) $) This set is called the $(R-)$ image of $a$. \par
    For $b \in T$ the set
    $$ _R\left[b\right] := \left\{a \in S \mid aRb\right\} $$
\end{definition}

Relations between finite sets can be described using matrices.
\begin{definition}[Adjecency Matrix]
    If $ S = \left\{s_1, s_2, \dots, s_n\right\} $ and $ T = \left\{t_1, t_2, \dots, t_m\right\} $ are finite sets
    and $ R \subseteq S \times T $ is a binary relation, then the \emph{adjecency} matrix $ A_R $ of the relation $ R $ is the $ n \times m $ matrix
    whose rows are indexted by $S$ and columns by $T$ defined by:
    $$ A_{s,t} = \begin{cases}
        1 &\text{ if } (s,t) \in R \\
        0 &\text{ otherwise }
    \end{cases} $$
\end{definition}

\begin{example}
    \begin{enumerate}
        \item The adjecency matrix of the relation $ R = \left\{\left(0,0\right), \left(1,0\right),\left(2,1\right)\right\} $ between the sets $ S = \left\{0,1,2\right\} $ and $ T = \left\{0,1\right\} $ equals
        $$ \begin{pmatrix}
            1 & 0 \\
            1 & 0 \\
            0 & 1
        \end{pmatrix} $$
        \item The adjecency matrix of the identity relation on a set $S$ of size n:
        $$ I_n = \begin{pmatrix}
            1 & 0 & \dots & 0 & 0 \\
            0 & 1 & \dots & 0 & 0 \\
            \vdots & & \ddots & & \vdots \\
            0 & 0 & \dots & 1 & 0 \\
            0 & 0 & \dots & 0 & 1
        \end{pmatrix} $$
        \item The adjecency matrix of relation $ \leq $ on the set $ \left\{1,2,3,4,5\right\} $ is the upper triangular matirx:
        $$ \begin{pmatrix}
            1 & 1 & 1 & 1 & 1\\
            0 & 1 & 1 & 1 & 1\\
            0 & 0 & 1 & 1 & 1\\
            0 & 0 & 0 & 1 & 1\\
            0 & 0 & 0 & 0 & 1\\

        \end{pmatrix} $$
    \end{enumerate}
\end{example}

Some relations have special properties:
\begin{definition}[Special relation properties]
    Let $ R $ be a relation on set $S$. Then $R$ is called
    \begin{itemize}
        \item \emph{Reflexive} if for all $ x \in S $ we have $ (x,x) \in R $
        \item \emph{Irreflexive} if for all $ x \in S $ we have $ (x,x) \notin R $
        \item \emph{Symmetric} if for all $ x,y \in S $ we have $ xRy \implies yRx $
        \item \emph{Antisymmetric} if for all $ x,y \in S $ we have that $ xRy \text{ and } yRx \implies x = y$
        \item \emph{Transitive} if for all $ x,y,z \in S $ we have that $ xRy $ nd $ yRz \implies xRz$
    \end{itemize}
\end{definition}

\subsection{Relations and Directed Graphs}

\begin{definition}[Directed graph]
    A \emph{directed edge} of a set $V$ is an element of $V\times V$. If $ e (v,w) $ is a directed edge of $V$, then
    $v$ is called its \emph{tail} and $w$ its \emph{head}. Both $v$ and $w$ are called \emph{end points} of the edge $e$.
    The \emph{reverse} of the edge $e$ is the edge $ (w,v) $. A \emph{loop} is an edge from a vertex to itself. \par
A \emph{directed graph} (also called \emph(digraph)) $\Gamma = (V,E)$ consists of a set of \emph{vertices} and a subset $E$ of $V\times V$ of (directed) \emph{edges}.
The elements of $V$ are called teh vertices of $\Gamma$ and the elements of $E$ the \emph{edges} of $\Gamma$
\end{definition}

\subsubsection{Some graph theoretical language}
Suppose $ \Gamma = (V,E) $ is a digraph. A \emph{walk} from $v$ to $w$, where $v,w \in V$, is a sequence $ v_0,v_1,\dots,v_k $ of vertices with $ v_0 = v, v_k = w $ and $ (v_i, v_{i+1}) \in E $ for all $ 0 \leq i \le k $.
A \emph{path} from $v$ to $w$ is a walk from $v$ to $w$ in which all vertices, except possibly the first vertex $v$ and the last vertex $w$ are different. \par
An \emph{undericted walk} from $v$ to $w$ is a sequence $ v_0,v_1,\dots,v_k $ of vertices with $ (v_i,v_{i+1}) \in E $ or $ (v_{i+1}, v_i) \in E $ for all $ 0 \leq i \le k $, while an \emph{undirected path} from $v$ to $w$ is an undirected walk in which all vertices except possibly the first and last are different.
The \emph{length} of the (directed or undirected) walk or path is $k$. A \emph{cycle} is a path from $v$ to $v$ of length at least 1. \par
If $v,w \in V$ are vertices of the digraph $\Gamma$, then the \emph{distance} from $v$ to $w$ is the minimum of the lengths of
the paths from $v$ to $w$. The distance is set to $\infty$ (infinity) if there is no path from $v$ to $w$. \par
The digraph is called \emph{weakly connected} if for any two vertices $v$ and $w$ there is an undirected path between
$v$ and $w$. It is \emph{called strongly} connected if there exist paths in both directions. \par
% If W is a subset of V, then the induced subgraph of Γ on W is the digraph (W,E ∩(W ×W)). A (weakly)
% connected component C of Γ is a maximal subset of V such that the induced subgraph is (weakly) connected.
% If we define the relation R on V by vRw if and only if there is an undirected path from v to w, then R is an
% equivalence relation. The weakly connected components of Γ are then the R-equivalence classes.
% A strongly connected component is a maximal subset of V such that the induced subgraph is strongly
% connected.
% Let S be the relation on the set V where vSw if there is an directed path from v to w and from w to v.
% Then S is an equivalence relation. (Proof this!) The strongly connected components are then the Sequivalence classes.
% As we can identify a directed graph with the corresponding relation, we can of course represent the graph
% by its adjacency matrix as well as by a diagram as in 3.1.6. However, as a directed graph Γ = (V,E) defines a
% relation on a set V, there is no need to draw this set twice. The usual way that we draw the graph Γ is to draw
% the vertices of V with arrows from a vertex v ∈ V to a vertex w ∈ V if and only if (v,w) ∈ E.

\begin{proposition}
    Let $ (V,E) $ be a directed graph. Then we have the following.
    \begin{enumerate}
        \item $E$ is reflexive if and only if every vertex $ v \in V $ is in a loop.
        \item $E$ is symmetric if and only if for every edge $ e \in E $, also its reverse is in $E$.
        \item $E$ is transitive if and only if for each walk of length at least 1 starting from $x$ and ending in $y$ we have that $ (x,y) \in E $.
    \end{enumerate}
\end{proposition}

\begin{example}
    The complete directed graph on a vertex set $V$ is the graph in which all vertices are adjacent to each other and tehmselves. This graph is clearly strongly connected. \par
    So, te corresponding relation is reflexive, symmetric and transitive.
\end{example}

\begin{proposition}
    Let $R$ be a relation on the set $V$ which is reflexive, symmetric and transitive. Then all (weakly) connected components of the graph $ \Gamma = (V,R)$ are complete graphs.
\end{proposition}

\begin{definition}[Indegree / Outdegree]
    Let $ \Gamma = (V,E) $ be a digraph and $ v \in V $ a vertex. The \emph{indegree} of $v$ is the number of edges with $v$ as head.
    The \emph{outdegree} of $v$ is teh number of edges with $v$ as tail.
\end{definition}

\subsection{Equivalence relations}
\begin{definition}[Equivalence Relation]
    A relation $R$ on a set $S$ is called an \emph{equivalence relation} on $S$ if and only if it is relfexive, symmetric and transitive.
\end{definition}
\begin{example}
    Consider the plane $ \mathbb{R}^2 $ and in it the set $S$ of straight lines. We call two lines parallel in S if and only if they are equal or do not intersect.
    Notice that two lines in S are parallel if and only if thir slopes are equal. Being parallel defines an equivalence relation on the set $S$.
\end{example}
\begin{example}
    Fix $n \in \mathbb{Z}$, and consider the relation $R$ on $\mathbb{Z}$ by $aRb$ if an only if $a-b$ is divisible by $n$. We also write $ a = b \pmod{n} $. \par
    The relation $R$ is an equivalence realtion. Indeed, suppose $a,b,c \in \mathbb{Z}$. Then
    \begin{enumerate}
        \item $ aRa $ as $ a-a=0 $ is divisible by n.
        \item If $ aRb $, then $ a-b $ is divisible by $n$ and hence also $ b - a $. Thus $ bRa $.
        \item If $ aRb, bRc $, then $n$ divides both $ a-b $ and $ b-c $ and then also $ (a-b)+(b-c) = a-c $. So $ aRc $
    \end{enumerate}
\end{example}

\begin{example}
    Let $\Pi$ be a partition of the set $S$. We define the relation $R_\Pi$ as follows: $a,b \in S$ are in relation $R_\Pi$
    if and only if there is a subset $X$ of $S$ in $\Pi$ containing both $a$ and $b$. We check that the relation $R_\Pi$ is an equivalence relation on $S$.
    \begin{itemize}
        \item Reflexivity: Let $a \in S$. Then there is an $X \in \Pi$ containing a. Hence, $a,a \in X$ and $a R_\Pi a$
        \item Symmetry: Let $a R_\Pi b$. Then there is an $X \in \Pi$ with $a,b \in X$. But then also $b,a \in X$ and $b R_\Pi a$
        \item Transitivity: If $a, b, c \in S$ with $a R_\Pi b$ and $b R_\Pi c$, then there are $X,Y \in \Pi$ with $a,b \in X$ and $b,c \in Y$.
        However, then $b$ is in both $X$ and $Y$. But then, as $\Pi$ partitions $S$, we have $X = Y$. So $a,c \in X$ and $a R_\Pi c$
    \end{itemize}
\end{example}

\begin{lemma}
    Let $R$ be an equivalence relation on a set $S$. If $b\in \left[a\right]_R$, then $ \left[b\right]_R = \left[a\right]_R$
\end{lemma}
\begin{proof}[Proof]
    Suppose $ b \in \left[a\right]_R $. Thus $aRb$. If $ c \in \left[b\right]_R $, then $bRc$ and, as $aRb$, we have by transitivity $aRc$.
    In particular, $ [b]_R \subseteq [a]_R $.\par
    Since, by symmetry of $R$, $aRb$ implies $bRa$ and hence $a \in [b]_R$, we similarly get $ [a]_R \subseteq [b]_R $.
\end{proof}

\begin{definition}[Equivalence classes]
    Let $R$ be an equivalence relation on a aset $S$. Then the sets $ [s]_R $, where $ s \in S $ are called the $R$-\emph{equivalence} calsses on S. \par
    We denote the set of $R$-equivalence classes by $ S/R $
\end{definition}
% \begin{definition}[name of the definition]

% \end{definition}

\begin{theorem}
    Let $R$ be an equivalence relation on a set $S$. Then the set $S/R$ of R-equivalence classes partions the set $S$.
\end{theorem}
\begin{proof}[Proof]
    Let $ \Pi_R $ be the set of $R$-equivalence classes. Then by reflexivity of $R$ we find that each element $ a \in S $
    is inside the class $ \left[a\right]_R \Pi_R $.

    If an element $ a \in S $ is in the classes $ \left[b\right]_R and \left[c\right]_R $ of $ \Pi $, then by the previous lemma
    we find $ \left[b\right]_R = \left[a\right]_R $ and $ \left[c\right]_R = \left[a\right]_R $. In particular, $ \left[b\right]_R = \left[c\right]_R $.
    Thus each element $ a \in S $ is inside an unique member of $ \Pi_R $, which therefore is a partition of $S$.
\end{proof}

\begin{example}[Construction of $\mathbb{Q}$]
    The rational numbers can be constructed from integers with the help of an equivalence relation.

    We consider the set $V = Z \times Z \setminus \{0\}$. On $V$ we define the relation $\equiv$ by
    $$ (a,b) \equiv (c,d) \iff a \cdot d = b \cdot c $$
    for all $ (a,b) $ and $ (c,d) $ in $V$.

    Now we denote the $ \equiv $-equivalence class of a pair ($a,b$) by $ \frac{a}{b} $.
\end{example}

\subsection{Composition of relations}
If $R_1$ and $R_2$ are relations between a set $S$ and a set $T$, then we can form new relations
by taking the intersection $R_1 \cap R_2$ or the union $R_1 \cup R_2$. Also the complement of $R_1$
in $R_2$, $R_1 \setminus R_2$ is a new relation. Furhtermore, we can consider the relation $R^{\top}$
(sometimes also denoted by $R^{-1}, R^{\sim} \text{ or } R^{\vee}$) from $T$ to $S$ as the relation
$ \{(t,s) \in T \times S \mid (s,t) \in R\} $

Another way of making new relations out of old ones is the following. If $R_1$ is a relation between $S$ and
$T$ and $R_2$ is a relation between $T$ and $U$, then the \emph{composition} or product $R=R_1;R_2$ (sometimes
denoted by $R_2 \circ R_1$ or $R_1 * R_2$) is the relation between $S$ and $U$ defined by $sRu$ for $s \in S$ and $u \in U$, if and
only if there is a $t \in T$ with $sR_1t$ and $tR_2u$.

\begin{example}
    $ R_1 = \left\{(1,2),(2,3),(3,3),(2,4)\right\} $ and $R_2 = \left\{(1,a), (2,b), (3,c), (3,d)\right\}$.
    Then $ R_1;R_2 = \left\{(1,b), (2,c), (3,c), (2,d)\right\} $.
\end{example}

We get the adjecency matrix of a composition by multiplying the respective adjecency matrices and then replacing
all non-zero entries with 1.

\begin{example}
    Suppose $R_1 = $ $\{(1,2),(2,3),(3,3), (2,4), (3,1)\}$ and $R_2$ is the relation \{\(\left(1,1\right)\),
    \(\left(2,3\right)\), \(\left(3,1\right)\), \(\left(3,3\right)\), \(\left(4,2\right)\)\}
    Then the adjecency matrices $A_1$ and $A_2$ for $R_1$ and $R_2$ are
    \[
        A_1 = \begin{pmatrix}
            0 & 1& 0& 0 \\
            0 & 0 & 1 & 1 \\
            1 & 0 & 1 & 0
        \end{pmatrix},
        A_2 = \begin{pmatrix}
            1 & 0 & 0 \\
            0 & 0 & 1 \\
            1 & 0 & 1 \\
            0 & 1 & 0
        \end{pmatrix}
    \]
    The product of these matrices equals
    $$ M = \begin{pmatrix}
        0&0&1 \\
        1&1&1 \\
        2&0&1
    \end{pmatrix} $$
    So the adjecency matrix of $R_1;R_2$ is
    $$\begin{pmatrix}
        0&0&1 \\
        1&1&1 \\
        1&0&1
    \end{pmatrix}$$
\end{example}

\begin{proposition}
    Suppose $R_1$ si a relation from $S$ to $T$, $R_2$ a relation from $T$ to $U$ and $R_3$ a realtion from $U$ to $V$.
    Then $R_1;(R_2;R_3) = (R_1;R_2);R_3$.

    Composing relations is associative.
\end{proposition}

\subsection{Transitive Closure}

\begin{lemma}\label{lemma:3.5.1}
    Let $ \mathscr{C} $ be a collection of relations $R$ on a set $S$. If all relations $R$ in $\mathscr{C}$ are transitive (symmetric or reflexive),
    then the relation $ \bigcap_{R \in \mathscr{C}} R$ is also transitive (symmetric or transitive, respectively).
\end{lemma}

\begin{proof}[Proof]
    Let $ \bar{R} = \bigcap_{R \in \mathscr{C}}R $. Suppose all memebers of $ \mathscr{C} $ are transitive. Then for all $ a,b,c \in S $ with $ a\bar{R}b $
    and $ b\bar{R}c $ we have $aRb$ and $bRc$ for all $ R \in \mathscr{C} $. Thus by transitivity of each $ R \in \mathscr{C} $ we also have $ aRc $ for each $R \in \mathscr{C}$.
    Thus we find $ a \bar{R}c $. Hence $ \bar{R} $ is transitive.
\end{proof}

The above lemma makes it possible to define the \emph{reflexive, symmetric, or transitive closure}
of a relation $R$ on a set $S$. It is the smallest refexive, symmetric or transitive
relation containing $R$. This means, as follows from \cref{lemma:3.5.1}, it is the
intersection $ \bigcap_{R^\prime \in \mathscr{C}}R^\prime $, where $ \mathscr{C} $
is the collection of all reflexive, symmetric, or transitive relations containing $R$.

\begin{proposition}
    $ \bigcup_{n>0}R^n $ is the transitive closure of the relation $R$.
\end{proposition}

\begin{proof}[Proof]
    Define $ \bar{R} = \bigcup_{n>0}R^n $. We prove transitivity of $ \bar{R} $. Let $ a\bar{R}b $
    and $ b\bar{R}c $, then there are sequence $ a = a_1,\dots,a_k = b $ and $ b = b_1,\dots,b_l = c $
    with $ a_iRa_{i+1} $ and $ b_iRb_{i+1} $. But then the sequence $ a=a_1=c_1,\dots,c_k=a_k=b_1,\dots,c_{k+l-1}=b_l=c$
    is a sequence from $a$ to $c$ with $ c_iRc_{i+1} $. Hence $ aR^{k+l-2}c $ and $ a\bar{R}c $.
\end{proof}

The transitive, symmetric and reflexive closure of a relation $R$ is an equivalence relation.
In terms of the graph $ \Gamma_R $, the equivalence classes are the strongly connected
componenets of $ \Gamma_R $.

\begin{algorithm}[H]
    Warhall's Algorithm
\end{algorithm}


    \section{Maps}
\subsection{Definition}
\begin{definition}
    A relation $F$ from a set $A$ to a set $B$ is called a map or function from $A$ to $B$
    if for each $a \in A$ there is one and only one $b \in B$ with $aFb$
    \par If $F$ is a map from $A$ to $B$, we write $F: A \rightarrow B$
    \par The set of all maps from $A$ to $B$ if denoted by $B^A$
    \par A \emph{partial map} $F$ from $A$ to $B$ is a relation with the property
    that for each $a \in A$ there is at most one $b \in B$ with $aFb$.
\end{definition}

\begin{example}
    \begin{enumerate}
        \item polynomial functions like $ f:\mathbb{R} \to \mathbb{R} $,
         with $ f(x) = x^3 $ for all $x$
        \item functions like $ \cos, \sin, \tan $
        \item $ \sqrt{}:\mathbb{R^+} \to \mathbb{R} $, taking square roots
        \item $ \ln:\mathbb{R^+} \to \mathbb{R} $, the natural logarithm
    \end{enumerate}
\end{example}

\begin{proposition}
    Let $ f: A \to B $ and $ g: B \to C $ be maps, then the composition
    $ g \circ f = f;g $ is a map from $A$ to $C$.
\end{proposition}

Let $A$ and $B$ be two sets and $ f:A \to B $. The set $A$ is called the
\emph{domain} of $f$, the set $B$ the \emph{codomain}. If $ a \in A $, then
the element $ b = f(a) $ is called the \emph{image} of $a$ under $f$.
The subset of $B$ consisting of the images of the elements of $A$ under $f$
is called the \emph{image} or \emph{range} of $f$ and is denoted by $\text{Im}(f)$. So
$$ \text{Im}(f) = \{b \in B \mid \text{ there is an } a \in A \text{ with } b = f(a)\} $$
If $ A^\prime $ is a subset of $A$, then the image of $A^\prime$ under $f$ is the set
$ f(A^\prime) =\{f(a) \mid a \in A^\prime\}$
If $ A^\prime $ is a subset of $A$, then the image of $ A^\prime $ under the set
$ f(A^\prime) =\{f(a) \mid a \in A^\prime\} $. So, $ \text{Im}(f) = f(A) $.

If $ a \in A $ and $ b = f(a) $, then the element $a$ is called a \emph{pre-image}
of $b$. Notice that $b$ can have more than one pre-image. The set of all pre-images
of $b$ is denoted by $ f^{-1}(b) $. So
$$ f^{-1}(b) =  \{a \in A \mid f(a) = b\}$$
If $B^\prime$ is a subset of $B$, then the pre-image of $B^\prime$, denoted by
$ f^{-1}(B^\prime) $ is the set of elements $a$ from $A$ tjhat are mapped to
an element $b$ of $B^\prime$. In particular
$$ f^{-1}(B^\prime) = \{a \in A \mid f(a) \in B^\prime\} $$

\begin{example}
    \begin{enumerate}
        \item Let $ f:\mathbb{R} \to \mathbb{R} $ with $ f(x) = x^2 $
        for all $ x \in \mathbb{R} $. Then $ f^{-1}(\left[0,4\right]) = \left[-2,2\right] $
        \item Consider the map from $ \mathbb{Z} $ to $ \mathbb{Z} $, which
        maps an integer $a$ to the unique element $b$ in {0,$\dots$,7} with
        $ a=b \pmod{8} $. The inverse image of 3 is the set $\left\{\dots, -5,3, 11, \dots\right\} $.
        The inverse image of 11, however, is the emptyset.
    \end{enumerate}
\end{example}

\newpage
\begin{theorem}
    Let $ f: A \to B $ be a map.
    \begin{itemize}
        \item If $ A^\prime \subseteq A $, then $ f^{-1}(f(A^\prime)) \supseteq A^\prime $
        \item If $ B^\prime \subseteq B $, then $ f(f^{-1}(B^\prime)) \subseteq B^\prime $
    \end{itemize}
\end{theorem}

\begin{proof}[Proof]
    Let $ a^\prime \in A^\prime $, then $ f(a^\prime) \in f(A^\prime) $ and hence
    $ a^\prime \in f^{-1}(f(A^\prime)) $. Thus $ A^\prime \subseteq f^{-1}(f(A^\prime)) $

    Let $ a \in f^{-1}(B^\prime) $, then $ f(a) \in B^\prime $. Thus
    $ f(f^\prime(B^\prime)) \subseteq B^\prime $
\end{proof}

\begin{theorem}
    Let $ f: A \to B $ and $ g: B \to C $ be maps. Then
    $ \text{Im}(g \circ f) = g(f(A)) \subseteq \text{Im}(g) $
\end{theorem}

\subsection{Special maps}
\begin{definition}[Surjective, injective and bijective maps]
    A map $ f: A \to B $ is called \emph{surjective}, if for every $b \in B$ there is
    an $ a \in A $ with $ b = f(a) $. In other words if $ \Im(f) = B $.

    The map $f$ is called \emph{injective}, if for each $b \in B$, there is at most
    one $a$ with $ f(a) = b $. So the pre-image of $b$ is either empty or consists
    of a unique element. In other words, $f$ is injective if for any elements $a$
    and $b$ from $A$ we find that $ f(a) = f(b) $ implies $ a = b $.

    The map $f$ is \emph{bijective} if it is both surjective and injective. So, if
    for each $b \in B$ there is a unique $a \in A$ with $f(a) = b$.
\end{definition}

\begin{example}
    \begin{enumerate}[label=(\alph*)]
        \item The map $ \sin:\mathbb{R} \to \mathbb{R} $ is not surjective,
        nor injective
        \item The map $ \sin:\left[-\pi/2,\pi/2\right] \to \mathbb{R} $ is
        injective, but not surjective
        \item The map $ \sin:\mathbb{R} \to \left[-1,1\right] $ is a surjective,
        but not injective map
        \item The map $ \sin:\left[-\pi/2,\pi/2\right] \to \left[-1,1\right] $
        is a bijective map
    \end{enumerate}
\end{example}

\begin{theorem}[Pigeonhole Principle]
    Let $A$ be a set of size $n$ and $B$ be a set of size $m$. Let $ f:A \to B $
    be a map between sets $A$ and $B$.
    \begin{enumerate}[label=(\alph*)]
        \item If $ n < m $, then $f$ cannot be surjective.
        \item If $ n > m $, then $f$ cannot be injective.
        \item If $ n = m $, then $f$ is injective if and only if $f$ is surjective.
    \end{enumerate}
\end{theorem}

\begin{remark}
    The above result is called the pigeonhole principle because of the following.
    If one has $n$ pigeons (the set $A$) and the same number of holes (the set $B$),
    then one pigeonhole is empty if and only if one of the other holes contains at
    least two pigeons.
\end{remark}

\begin{example}
    Suppose you have to pick seven distinct numbers of the set
    $\left\{1,2,\dots,11\right\}$. Then among these seven numbers there is a pair
    that adds up to 12.

    Suppose $S$ is the set of 7 numbers picked. Now consider the following six
    subsets $$ \{1, 11\},\{2, 10\},\{3, 9\},\{4, 8\},\{5, 7\},\{6\} $$ partitioning
    $ \{1,\dots,11\} $. The map that assigns to each of the seven elements of $S$
    the unique part of this partition to which it belongs can not be injective.
    So, there is a pair of this partition that is contained in S providing us
    with two numbers in S adding up to 12.
\end{example}

\begin{proposition}
    Let $ f:A \to B $ be a bijection. Then for all $ a \in A $ and $ b \in B $
    we have $ f\inv(f(a)) = a $ and $ f(f\inv(b)) = b $. In particular, $f$ is the
    inverse of $ f\inv $.
\end{proposition}

\begin{theorem}
    Let $ f: A \to B $ and $ g: B \to C $ be two maps.
    \begin{enumerate}[label=(\alph*)]
        \item If $f$ and $g$ are surjective, then so is $ g \circ f $
        \item If $f$ and $g$ are injective, then so is $ g \circ f $
        \item If $f$ and $g$ are bijective, then so is $ g \circ f $
    \end{enumerate}
\end{theorem}

\begin{proposition}
    If $ f:A \to B $ and $ g:B \to A $ are maps with $ f \circ g = I_B $ and
    $ g \circ f = I_A $ where $I_A$ and $I_B$ denote the identity maps on $A$
    and $B$, respectively. Then $f$ and $g$ are bijections. Moreover, $ f\inv = g $
    and $ g\inv = f $.
\end{proposition}

\begin{lemma}
    Suppose $ f:A \to B $ and $ g:B \to C $ are bijective maps. Then the inverse
    of the map $ g \circ f $ equals $ f\inv \circ g\inv $.
\end{lemma}

\subsection{Permutations and Symmetric groups}
\begin{definition}[Permutations and Symmetric groups]
Let $X$ be a set.
\begin{itemize}
    \item A bijection on $X$ to itself is also called a \emph{permutation} of X.
    The set of all permutations of $X$ is denoted by $ \sym(X) $. It is called the
    \emph{symmetric group} on $X$.

    \item The product $ g \cdot h $ of two permutations $ g,h \in \sym(X) $ is
    defined as the composition $ g \circ h $ of $g$ and $h$. Thus for all $x \in X$
    we have $ g \cdot h(x) = g(h(x)) $.

    \item If $ X = \{1,\dots, n\} $, we also write $ \sym_n $ instead of $ \sym(X) $.
    Furthermore, a permutation $f$ of $X$ is often given by $ \left[f(1),f(2),\dots,f(n)\right] $.
\end{itemize}
\end{definition}

\begin{theorem}
    $ \sym_n $ has exactly $ n! $ elements.
\end{theorem}

\begin{definition}[Order of a permutation]
    The order of a permutation $g$ is the smallest positive integer $m$ such that
    $ g^m = e $.
\end{definition}

\subsection{Cycles}
\begin{definition}[Fix points and Support]
    The \emph{fixed points} of $g$ in $X$ are the elements of $x$ in $X$ for which
    $ g(x) = x $ holds. The set of all fix points is $ \fix(g) = \{x \in X \mid g(x) = x\} $.

    The \emph{support} of $g$ is the complement in $X$ of $ \fix(g) $. It is denoted
    by $ \support(x) $
\end{definition}

\begin{example}
    Consider the permutation $g = [1, 3, 2, 5, 4, 6] \in \sym_6$.
    The fixed points of $g$ are 1 and 6. So $\fix(g) = \{1, 6\}$.
    Thus the points moved by $g$ form the set $\support(g) = \{2, 3, 4, 5\}$.
\end{example}

    Cycles are elements in $Sym_n$ of special importance.

\begin{definition}[Cycles]
    Let $ g \in \sym_n $ be a permutation with $ \support(g) = \{a_1,\dots,a_m\} $,
    where the $ a_i $ are pairwise distict. We say $g$ is an $m$-cycle if
    $ g(a_i) = g(a_{i+1}) $ for all $ i \in \{1,\dots,m-1\} $ and $ g(a_m) = a_1 $.
    For such a cycle $g$ we also use the cycle notation $ \left(a_1,\dots,a_m\right) $.

    2-cycles are called \emph{transpositions}.
\end{definition}

\begin{theorem}
    Every permutation in $ \sym_n $ is a product of disjoint cycles. This product
    is unique up to rearrangement of the factors.
\end{theorem}

\begin{definition}[Cycle structure]
    The cycle structure of a permutation is the unordered sequence of the cycle
    lenghts in an expression of $g$ as a product of disjoint cycles.
\end{definition}

\subsection{Alternating groups}
\begin{theorem}
    If a permutation is written in two ways as a product of transpositions, then
    both products have even length or both have odd length.
\end{theorem}

\begin{definition}
    Let $g$ be an element of $ \sym_n $. the sign of $g$, denoted by $ \sign(g) $,
    is defined as
    \begin{itemize}
        \item 1 if $g$ can be written as a product of an even number of 2-cycles, and
        \item -1 if $g$ can be writeen as a product of an odd number of 2-cycles.
    \end{itemize}
    We say that $g$ is even $ \sign(g) = 1 $ and odd if $ \sign(g) = -1 $.
\end{definition}

\begin{theorem}[Multiplicative property of $ \sign $]
    For all permutations $ g,h $ in $ \sym_n $, we have
    $$ \sign(g \cdot h) = \sign(g) \cdot \sign(h) $$
\end{theorem}

\begin{corollary}
    If a permutation $g$ is written as a product of cycles, then $ \sign(g) = (-1)^w $,
    where $w$ is the number of cycles of even length.
\end{corollary}

\begin{definition}[Alternating group]
    By $ \alt_n $ we denote the set of even permutations in $ \sym_n $. We call
    $ \alt_n $ the \emph{alternating group} on $n$ letters.

    The alternating group is closed with respect to taking products and inverse
    elements.
\end{definition}

There are exactly as many even as odd permutations in $ Sym_n $.

\begin{theorem}[Size of $ \alt_n $]
    For $ n > 1 $, the alternating group $ \alt_n $ contains precisely
    $ \frac{n!}{2} $ elements.
\end{theorem}

\begin{theorem}
    Every even permutation is a product of 3-cycles.
\end{theorem}
    \section{Maps}
\subsection{Definition}
\begin{definition}
    A relation $F$ from a set $A$ to a set $B$ is called a map or function from $A$ to $B$
    if for each $a \in A$ there is one and only one $b \in B$ with $aFb$
    \par If $F$ is a map from $A$ to $B$, we write $F: A \rightarrow B$
    \par The set of all maps from $A$ to $B$ if denoted by $B^A$
    \par A \emph{partial map} $F$ from $A$ to $B$ is a relation with the property
    that for each $a \in A$ there is at most one $b \in B$ with $aFb$.
\end{definition}

\begin{example}
    \begin{enumerate}
        \item polynomial functions like $ f:\mathbb{R} \to \mathbb{R} $,
         with $ f(x) = x^3 $ for all $x$
        \item functions like $ \cos, \sin, \tan $
        \item $ \sqrt{}:\mathbb{R^+} \to \mathbb{R} $, taking square roots
        \item $ \ln:\mathbb{R^+} \to \mathbb{R} $, the natural logarithm
    \end{enumerate}
\end{example}

\begin{proposition}
    Let $ f: A \to B $ and $ g: B \to C $ be maps, then the composition
    $ g \circ f = f;g $ is a map from $A$ to $C$.
\end{proposition}

Let $A$ and $B$ be two sets and $ f:A \to B $. The set $A$ is called the
\emph{domain} of $f$, the set $B$ the \emph{codomain}. If $ a \in A $, then
the element $ b = f(a) $ is called the \emph{image} of $a$ under $f$.
The subset of $B$ consisting of the images of the elements of $A$ under $f$
is called the \emph{image} or \emph{range} of $f$ and is denoted by $\text{Im}(f)$. So
$$ \text{Im}(f) = \{b \in B \mid \text{ there is an } a \in A \text{ with } b = f(a)\} $$
If $ A^\prime $ is a subset of $A$, then the image of $A^\prime$ under $f$ is the set
$ f(A^\prime) =\{f(a) \mid a \in A^\prime\}$
If $ A^\prime $ is a subset of $A$, then the image of $ A^\prime $ under the set
$ f(A^\prime) =\{f(a) \mid a \in A^\prime\} $. So, $ \text{Im}(f) = f(A) $.

If $ a \in A $ and $ b = f(a) $, then the element $a$ is called a \emph{pre-image}
of $b$. Notice that $b$ can have more than one pre-image. The set of all pre-images
of $b$ is denoted by $ f^{-1}(b) $. So
$$ f^{-1}(b) =  \{a \in A \mid f(a) = b\}$$
If $B^\prime$ is a subset of $B$, then the pre-image of $B^\prime$, denoted by
$ f^{-1}(B^\prime) $ is the set of elements $a$ from $A$ tjhat are mapped to
an element $b$ of $B^\prime$. In particular
$$ f^{-1}(B^\prime) = \{a \in A \mid f(a) \in B^\prime\} $$

\begin{example}
    \begin{enumerate}
        \item Let $ f:\mathbb{R} \to \mathbb{R} $ with $ f(x) = x^2 $
        for all $ x \in \mathbb{R} $. Then $ f^{-1}(\left[0,4\right]) = \left[-2,2\right] $
        \item Consider the map from $ \mathbb{Z} $ to $ \mathbb{Z} $, which
        maps an integer $a$ to the unique element $b$ in {0,$\dots$,7} with
        $ a=b \pmod{8} $. The inverse image of 3 is the set $\left\{\dots, -5,3, 11, \dots\right\} $.
        The inverse image of 11, however, is the emptyset.
    \end{enumerate}
\end{example}

\newpage
\begin{theorem}
    Let $ f: A \to B $ be a map.
    \begin{itemize}
        \item If $ A^\prime \subseteq A $, then $ f^{-1}(f(A^\prime)) \supseteq A^\prime $
        \item If $ B^\prime \subseteq B $, then $ f(f^{-1}(B^\prime)) \subseteq B^\prime $
    \end{itemize}
\end{theorem}

\begin{proof}[Proof]
    Let $ a^\prime \in A^\prime $, then $ f(a^\prime) \in f(A^\prime) $ and hence
    $ a^\prime \in f^{-1}(f(A^\prime)) $. Thus $ A^\prime \subseteq f^{-1}(f(A^\prime)) $

    Let $ a \in f^{-1}(B^\prime) $, then $ f(a) \in B^\prime $. Thus
    $ f(f^\prime(B^\prime)) \subseteq B^\prime $
\end{proof}

\begin{theorem}
    Let $ f: A \to B $ and $ g: B \to C $ be maps. Then
    $ \text{Im}(g \circ f) = g(f(A)) \subseteq \text{Im}(g) $
\end{theorem}

\subsection{Special maps}
\begin{definition}[Surjective, injective and bijective maps]
    A map $ f: A \to B $ is called \emph{surjective}, if for every $b \in B$ there is
    an $ a \in A $ with $ b = f(a) $. In other words if $ \Im(f) = B $.

    The map $f$ is called \emph{injective}, if for each $b \in B$, there is at most
    one $a$ with $ f(a) = b $. So the pre-image of $b$ is either empty or consists
    of a unique element. In other words, $f$ is injective if for any elements $a$
    and $b$ from $A$ we find that $ f(a) = f(b) $ implies $ a = b $.

    The map $f$ is \emph{bijective} if it is both surjective and injective. So, if
    for each $b \in B$ there is a unique $a \in A$ with $f(a) = b$.
\end{definition}

\begin{example}
    \begin{enumerate}[label=(\alph*)]
        \item The map $ \sin:\mathbb{R} \to \mathbb{R} $ is not surjective,
        nor injective
        \item The map $ \sin:\left[-\pi/2,\pi/2\right] \to \mathbb{R} $ is
        injective, but not surjective
        \item The map $ \sin:\mathbb{R} \to \left[-1,1\right] $ is a surjective,
        but not injective map
        \item The map $ \sin:\left[-\pi/2,\pi/2\right] \to \left[-1,1\right] $
        is a bijective map
    \end{enumerate}
\end{example}

\begin{theorem}[Pigeonhole Principle]
    Let $A$ be a set of size $n$ and $B$ be a set of size $m$. Let $ f:A \to B $
    be a map between sets $A$ and $B$.
    \begin{enumerate}[label=(\alph*)]
        \item If $ n < m $, then $f$ cannot be surjective.
        \item If $ n > m $, then $f$ cannot be injective.
        \item If $ n = m $, then $f$ is injective if and only if $f$ is surjective.
    \end{enumerate}
\end{theorem}

\begin{remark}
    The above result is called the pigeonhole principle because of the following.
    If one has $n$ pigeons (the set $A$) and the same number of holes (the set $B$),
    then one pigeonhole is empty if and only if one of the other holes contains at
    least two pigeons.
\end{remark}

\begin{example}
    Suppose you have to pick seven distinct numbers of the set
    $\left\{1,2,\dots,11\right\}$. Then among these seven numbers there is a pair
    that adds up to 12.

    Suppose $S$ is the set of 7 numbers picked. Now consider the following six
    subsets $$ \{1, 11\},\{2, 10\},\{3, 9\},\{4, 8\},\{5, 7\},\{6\} $$ partitioning
    $ \{1,\dots,11\} $. The map that assigns to each of the seven elements of $S$
    the unique part of this partition to which it belongs can not be injective.
    So, there is a pair of this partition that is contained in S providing us
    with two numbers in S adding up to 12.
\end{example}

\begin{proposition}
    Let $ f:A \to B $ be a bijection. Then for all $ a \in A $ and $ b \in B $
    we have $ f\inv(f(a)) = a $ and $ f(f\inv(b)) = b $. In particular, $f$ is the
    inverse of $ f\inv $.
\end{proposition}

\begin{theorem}
    Let $ f: A \to B $ and $ g: B \to C $ be two maps.
    \begin{enumerate}[label=(\alph*)]
        \item If $f$ and $g$ are surjective, then so is $ g \circ f $
        \item If $f$ and $g$ are injective, then so is $ g \circ f $
        \item If $f$ and $g$ are bijective, then so is $ g \circ f $
    \end{enumerate}
\end{theorem}

\begin{proposition}
    If $ f:A \to B $ and $ g:B \to A $ are maps with $ f \circ g = I_B $ and
    $ g \circ f = I_A $ where $I_A$ and $I_B$ denote the identity maps on $A$
    and $B$, respectively. Then $f$ and $g$ are bijections. Moreover, $ f\inv = g $
    and $ g\inv = f $.
\end{proposition}

\begin{lemma}
    Suppose $ f:A \to B $ and $ g:B \to C $ are bijective maps. Then the inverse
    of the map $ g \circ f $ equals $ f\inv \circ g\inv $.
\end{lemma}

\subsection{Permutations and Symmetric groups}
\begin{definition}[Permutations and Symmetric groups]
Let $X$ be a set.
\begin{itemize}
    \item A bijection on $X$ to itself is also called a \emph{permutation} of X.
    The set of all permutations of $X$ is denoted by $ \sym(X) $. It is called the
    \emph{symmetric group} on $X$.

    \item The product $ g \cdot h $ of two permutations $ g,h \in \sym(X) $ is
    defined as the composition $ g \circ h $ of $g$ and $h$. Thus for all $x \in X$
    we have $ g \cdot h(x) = g(h(x)) $.

    \item If $ X = \{1,\dots, n\} $, we also write $ \sym_n $ instead of $ \sym(X) $.
    Furthermore, a permutation $f$ of $X$ is often given by $ \left[f(1),f(2),\dots,f(n)\right] $.
\end{itemize}
\end{definition}

\begin{theorem}
    $ \sym_n $ has exactly $ n! $ elements.
\end{theorem}

\begin{definition}[Order of a permutation]
    The order of a permutation $g$ is the smallest positive integer $m$ such that
    $ g^m = e $.
\end{definition}

\subsection{Cycles}
\begin{definition}[Fix points and Support]
    The \emph{fixed points} of $g$ in $X$ are the elements of $x$ in $X$ for which
    $ g(x) = x $ holds. The set of all fix points is $ \fix(g) = \{x \in X \mid g(x) = x\} $.

    The \emph{support} of $g$ is the complement in $X$ of $ \fix(g) $. It is denoted
    by $ \support(x) $
\end{definition}

\begin{example}
    Consider the permutation $g = [1, 3, 2, 5, 4, 6] \in \sym_6$.
    The fixed points of $g$ are 1 and 6. So $\fix(g) = \{1, 6\}$.
    Thus the points moved by $g$ form the set $\support(g) = \{2, 3, 4, 5\}$.
\end{example}

    Cycles are elements in $Sym_n$ of special importance.

\begin{definition}[Cycles]
    Let $ g \in \sym_n $ be a permutation with $ \support(g) = \{a_1,\dots,a_m\} $,
    where the $ a_i $ are pairwise distict. We say $g$ is an $m$-cycle if
    $ g(a_i) = g(a_{i+1}) $ for all $ i \in \{1,\dots,m-1\} $ and $ g(a_m) = a_1 $.
    For such a cycle $g$ we also use the cycle notation $ \left(a_1,\dots,a_m\right) $.

    2-cycles are called \emph{transpositions}.
\end{definition}

\begin{theorem}
    Every permutation in $ \sym_n $ is a product of disjoint cycles. This product
    is unique up to rearrangement of the factors.
\end{theorem}

\begin{definition}[Cycle structure]
    The cycle structure of a permutation is the unordered sequence of the cycle
    lenghts in an expression of $g$ as a product of disjoint cycles.
\end{definition}

\subsection{Alternating groups}
\begin{theorem}
    If a permutation is written in two ways as a product of transpositions, then
    both products have even length or both have odd length.
\end{theorem}

\begin{definition}
    Let $g$ be an element of $ \sym_n $. the sign of $g$, denoted by $ \sign(g) $,
    is defined as
    \begin{itemize}
        \item 1 if $g$ can be written as a product of an even number of 2-cycles, and
        \item -1 if $g$ can be writeen as a product of an odd number of 2-cycles.
    \end{itemize}
    We say that $g$ is even $ \sign(g) = 1 $ and odd if $ \sign(g) = -1 $.
\end{definition}

\begin{theorem}[Multiplicative property of $ \sign $]
    For all permutations $ g,h $ in $ \sym_n $, we have
    $$ \sign(g \cdot h) = \sign(g) \cdot \sign(h) $$
\end{theorem}

\begin{corollary}
    If a permutation $g$ is written as a product of cycles, then $ \sign(g) = (-1)^w $,
    where $w$ is the number of cycles of even length.
\end{corollary}

\begin{definition}[Alternating group]
    By $ \alt_n $ we denote the set of even permutations in $ \sym_n $. We call
    $ \alt_n $ the \emph{alternating group} on $n$ letters.

    The alternating group is closed with respect to taking products and inverse
    elements.
\end{definition}

There are exactly as many even as odd permutations in $ Sym_n $.

\begin{theorem}[Size of $ \alt_n $]
    For $ n > 1 $, the alternating group $ \alt_n $ contains precisely
    $ \frac{n!}{2} $ elements.
\end{theorem}

\begin{theorem}
    Every even permutation is a product of 3-cycles.
\end{theorem}

    
\subsection{Exercises}
    
\subsection{Exercises}

    \subsection{Exercises}
    \subsection{Exercises}

    \section{Cardinalities}

\subsection{Cardinality}
\begin{definition}[Cardinality]
    Two sets $A$ and $B$ have the same \emph{cardinality} if there exists a
    bijection from $A$ to $B$.
\end{definition}

\begin{example}
    Two finite sets have the same cardinality if and only if thery have the same
    number of elements.
\end{example}

\begin{example}
    The sets $\mathbb{N}$ and $ \mathbb{Z} $ have the same cardinality. Indeed,
    consider the map $ f:\mathbb{N} \rightarrow \mathbb{Z} $ defined by
    $ f(2n) = n $ and $ f(2n+1) = -n $ where $ n \in \mathbb{N} $.
    This map is clearly a bijection
\end{example}

\begin{theorem}[Cardinality as equivalence relation]
    Having the same cardinality is an equivalence relation.
\end{theorem}

\subsection{Countable sets}
\begin{definition}[Finite/Inifinite sets]
    A set is called \emph{finite} if it is empty or has the same cardinality as the set $ \mathbb{N}_n := \left\{1,2,\dots,n\right\} $ and \emph{infinite} otherwise.
\end{definition}

\begin{definition}[Countable/Uncountable sets]
    A set is called \emph{countable} if it is finite or has the same cardinality as the set $ \mathbb{N} $.

    An infinite set that is not countable is called \emph{uncountable}.
\end{definition}

\begin{theorem}[Countable sets in infinite sets]
    Every infinite set contains an infinite countable subset.
\end{theorem}
\begin{proof}[Proof]
    Suppose $A$ is an infinite set. Since $A$ is infinite, we can start enumerating the elements $ a_1,a_2,\dots $ such that all the elements are distinct. This yields a sequence of elements in $A$.
    The set of all the elements in this sequence form a countable subset of $A$.
\end{proof}

\begin{theorem}
    Let $A$ be a set. If there is a surjective map from $ \mathbb{N} $ to $A$, then $A$ is countable.
\end{theorem}
\begin{proof}[Proof]
    Let $ f:\mathbb{N} \rightarrow A $ be a surjection. Then consider the sequence $ f(1), f(2), \dots $. Remove from this sequence (going from left to right) each element that you have seen before. The result is either a finite sequence,
    or an infinite sequence $ f(n_1), f(n_2), \dots $ of which all elements are distinct. In the latter case, consider the map
    $ g: \mathbb{N} \rightarrow A $ with $ g(i) = f(n_i) $. This map is a bijection, which proves $A$ to be countable.
\end{proof}
\begin{corollary}
    Let $A$ be countable and $ f : A \rightarrow B $ surjective, then B is countable.
\end{corollary}

\begin{proof}[Proof]
    Suppose A is a countable set and $ f: A \to B $ a surjective map. If $A$ is finite, then so is B.
    Thus assume that $A$ has infintely many elements. Since $A$ is countable, there is a bijection $ g: \mathbb{N} \to A $.
    But then $ f \circ g $ is a surjection from $ \mathbb{N} $ to B. Hence we can apply the previous result and find a bijection from $ \mathbb{N} $ to $B$. This proves B to be countable.
\end{proof}

\begin{theorem}
    Any subset of a countable set is countable.
\end{theorem}
\begin{proof}[Proof]
    Suppose $A$ is an infinite subset of a countable set $B$. Let $ f : \mathbb{N} \to B $ be bijective and fix an element $ a \in A $.
    Now consider the map $ g : \mathbb{N} \to A $ defined by $ g(x) = f(x) $ if $ f(x) \in A $ and $ g(x) = a $ if $ f(x) \in B \setminus A $.
    Then $g$ is surjective, as $f$ is surjective. Thus A is countable.
\end{proof}

\begin{proposition}
    $ \mathbb{N} \times \mathbb{N} $ is countable.
\end{proposition}
\begin{proof}[Proof]
    Let $ n \in \mathbb{N} $. Let $m$ be maximal with $ \sum_{i=0}^m i < n $. Now let $k= n - \sum_{i=0}^m i$So, $ 1 \leq k \leq m+1 $.
    \par We define $ f : \mathbb{N} \to \mathbb{N} $ in the following way:
    $$ f(n) = (k, m+2-k). $$
    So, in a table this looks as follows:
    \begin{center}
        \begin{tabular}{| c | c | c | c | c |}
            \hline
            $ f(1) = (1,1) $ & $ f(2) = (1,2) $ & $ f(4) = (1,3) $ & $ f(7) = (1,4) $ \\
            \hline
            $ f(3) = (2,1) $ & $ f(5) = (2,2) $ & $ f(8) = (2,3) $ & \dots & \\
            \hline
            $ f(6) = (3,1) $ & $ f(9) = (3,2) $ & \dots & & \\
            \hline
            \vdots & \vdots & & & \\
            \hline
        \end{tabular}
    \end{center}
    By construction, $f$ is injective. Indeed, the $m$ and $k$ are uniquely defined by n. \par
    So it only remains to prove surjectivity. Let $ (k,l) \in \mathbb{N} \times \mathbb{N} $. Set $ m=k+l-2 $. Hence $ (k,l) = (k,m+2-k) $ and $ (k,l) = f(n) $ for $n$ equal to $ \sum_{i=0}^m i+k $.
\end{proof}

\begin{theorem}
    Let $A$ and $B$ be countable sets. Then $ A \times B $ is countable.
\end{theorem}
\begin{proof}[Proof]
    Suppose $ f: \mathbb{N} \to A $ and $ g: \mathbb{N} \to B $ are surjections. The map $ h: \mathbb{N} \times \mathbb{N} \to A \times B $ defined by $ h(i,j) = (f(i), h(i)) $ is surjective.
    So, since $ \mathbb{N} \times \mathbb{N} $ is countable, also $ A \times B $ is countable.
\end{proof}

\begin{proposition}
    The sets $ \mathbb{Z} $ and $ \mathbb{Q} $ are countable.
\end{proposition}

\begin{proof}[Proof]
    The map $ g: \{-1,1\} \times \mathbb{N} \to \mathbb{Z} $ given by $ g(x,y) = xy $ is surjective. Since $ \{-1,1\} \times \mathbb{N} $ is countable, hence $\mathbb{Z}$ is also countable.

    Now let $ f: \mathbb{Z} \times \mathbb{N} \to \mathbb{Q} $ be defined by $ f(i,j) = \frac{i}{j}$ for $ (i,j) \in \mathbb{Z} \times \mathbb{N} $. This is clearly a surjective map.
    Since $\mathbb{Z}$ and $\mathbb{N}$ are countable so is $\mathbb{Z} \times \mathbb{N}$. Hence $\mathbb{Q}$ is also countable.
\end{proof}

\begin{theorem}
    Let $\mathscr{C}$ be a countable collection of countable sets. Then
    $ \bigcup_{A \in \mathscr{C}}A $ is countable.
\end{theorem}
\begin{proof}[Proof]
    For each $ A \in \mathscr{C} $ there exists a bijection
    $ f_A : \mathbb{N} \to A $. Moreover, as $\mathscr{C}$ is countable, there
    exists also a bijection $ g: \mathbb{N} \to \mathscr{C} $. We write
    $ A_i = g(i) $.

    Now consider the map $ f: \mathbb{N} \times \mathbb{N} \to \bigcup_{A\in\mathscr{C}}A $
    defined by $ f(i,j) = f_{A_i}(j) $. This is a surjection. Thus
    $ \bigcup_{A\in\mathscr{C}}A $ is countable.
\end{proof}
\begin{example}
Let $S$ be the set of all finite subsets of $\mathbb{N}$. Then $S = \bigcup_{i\in\mathbb{N}S_i}$, where $S_i$ is the set of subsets of size at most $i$ of $\mathbb{N}$. \par
We already showed that $ \mathbb{N}^i $ is countable. But the map $ (a_1,\dots,a_i) \in \mathbb{N}^i \mapsto \{a_1,\dots, a_i\} \in S_i $ is clearly surjective.
Thus $S_i$ is also countable. Hence $ S = \bigcup_{i\in\mathbb{N}S_i} $ is also countable.
\end{example}

\begin{proposition}
    If $A$ is infinite and $B$ is finite, then $A$ and $A \cup B$ have the same cardinality.
\end{proposition}

\begin{proof}[Proof]
    Assume that $A$ is infinite and, withouth loss of generality, that $A$ and
    $B$ are disjunct. Let $A_0$ be a countable subset of $A$. Then $A_0 \cup B$
    is also countable. Then there exists a bijection $ g: A_0 \cup B \to A_0 $.
    Now define $ f: A \cup B \to A $ by
    $$ f(x) \begin{cases}
        g(x) &\text{if } x \in A_0 \cup B \\
        x &\text{if } x \notin A_0 \cup B,
    \end{cases} $$

    Then clearly $f$ is a bijection between $A \cup B$ and $A$.
\end{proof}

\subsection{Some uncountable sets}
\begin{proposition}
    The set $ \{0,1\}^\mathbb{N} $ is uncountable.
\end{proposition}
\begin{proof}[Proof]
    Let $ F: \mathbb{N} \to \{0,1\}^\mathbb{N} $. By $ f_i $ we denote the
    function $F(i)$ from $\mathbb{N}$ to $ \{0,1\} $.

    We will show that $F$ is not surjective by constructing a function
    $ f \in \{0,1\}^\mathbb{N} $ which is different from all the function
    $f_i$ with $i \in \mathbb{N}$.

    For each $ i \in \mathbb{N} $ let
    $$ f(i) = 0 \text{ if } f_i(i) = 1 \text{ and}$$
    $$ f(i) = 1 \text{ if } f_i(i) = 0 $$

    Clearly, for all $ i \in \mathbb{N} $ we have $ f(i) \ne f_i(i) $ and
    hence $ f \ne f_i $. So $F$ is not surjective. This shows that there is
    no surjection from $\mathbb{N}$ to $\{0,1\}^\mathbb{N}$. In particular,
    $ \{0,1\}^\mathbb{N} $ is not countable.
\end{proof}

\begin{remark}[Cantor's diagonal argument]

\end{remark}

If $A$ is a set, then for each subset $B$ of $A$ we define the
\emph{characteristic function} $ \chi_B: A \to \{0,1\}^\mathbb{N} $ to be the
function that takes the value 1 on all elements in $B$ and the value 0 on all
elements in $A \setminus B$.

Clearly, every element $ f \in \{0,1\}^\mathbb{N} $ is the characteristic
fucntion of the set $ \{a \in A \mid f(a) = 1\} $. So, we find the map
$ B \in \mathscr{A} \mapsto \chi_B $ to be a bijection between
$ \mathscr{P}(A) $ to $ \{0,1\}^\mathbb{N} $.

\begin{corollary}
    The set $ \mathscr{P}(A) $ has the same cardinality as $ \{0,1\}^\mathbb{N} $
    and hence is uncountable.
\end{corollary}

\begin{proposition}
    The interval $ [0,1) $ is uncountable.
\end{proposition}
\begin{proof}[Proof]
    Consider the map $ f \in \{0,1\}^\mathbb{N} \mapsto
    \displaystyle\sum_{i=1}^{\infty}\frac{f(i)}{10^i} $. This map is injective.
    So, if $ [0,1) $ is countable, then so is $ \{0,1\}^\mathbb{N} $, which is
    a contradiction.

    This proves that $ [0,1) $ is uncountable.
\end{proof}

\begin{corollary}
    $ \mathbb{R} $ is uncountable.
\end{corollary}
\begin{proof}[Proof]
    As $ \mathbb{R} $ contains the uncountable subset $ [0,1) $, it is also
    uncountable.
\end{proof}

\begin{theorem}
    If $A$ and $B$ are sets with the same cardinality, then $ \mathscr{P}(A) $
    and $ \mathscr{P}(B) $ also have the same cardinality.
\end{theorem}
\begin{proof}[Proof]
    Suppose $A$ and $B$ have the same cardinality. Let $f : A \to B$ be a
    bijection. Consider the map $\hat{f}:P(A) \to P(B)$ given by $\hat{f}(S)
    = \{f(s) | s \in S\}$. This map is a bijection.
\end{proof}

\begin{corollary}
    If $A$ is an infinite set, then $ \mathscr{P}(A) $ is an uncountable set.
\end{corollary}

\begin{theorem}
    Let $X$ be a set, then $ \mathscr{P}(X) $ does not have the same
    cardinality as $X$.
\end{theorem}

\begin{remark}
    The above theorem shows us that we can get bigger and bigger sets in the following way:
    \begin{align}
        X_1 &:= \mathbb{N} \\
        \text{for } n > 1, X_n &:= \mathscr{P}(X_{n-1})
    \end{align}
\end{remark}

\subsection{Cantor-Schr\"oder-Bernstein Theorem}
\begin{theorem}[Contor-Schr\"oder-Bernstein Tehorem]
    Let $A$ and $B$ be sets and assume that there are two maps $ f:A \to B $ and $ g:B \to A $ which are injective.
    Then there exists a bijection $ h:A \to B $.

    In particular, $A$ and $B$ have the same cardinality.
\end{theorem}

\begin{corollary}
    Let $A$ be a set and assume $B \subseteq A$ has the same cardinality as $A$.
    Then each subset $C$ of $A$ with $B \subseteq C \subseteq A$ has the same cardinality as $A$.
\end{corollary}

\begin{proposition}
    The sets $ \{0,1\}^\mathbb{N} $ and $ [0,1) $ have the same cardinality.
\end{proposition}

\begin{theorem}
    The sets $ \mathbb{R}, \{0,1\}^\mathbb{N}, \mathscr{P}(N) $ have the same cardinality.
\end{theorem}

\begin{theorem}
    The sets $ \mathbb{R}^n $ with $n>0$, and $ \mathbb{R} $ have the same cardinality.
\end{theorem}

\subsection{Additional axioms of set theory}
\begin{principle}[Axiom of Choice]
    Let $ \mathscr{C} $ be a collection of nonempty sets. Then there exists a map
    $$ f: \mathscr{C} \to \bigcup_{A \in \mathscr{C}}A $$
    with $ f(A) \in A $.

    The image of $f$ is a subset of $ \displaystyle\bigcup_{A \in \mathbb{C}}A $.

    The function $f$ is called a \emph{choice function}.
\end{principle}

\begin{principle}
    The following statements are equivalent to the Axiom of Choice.
    \begin{itemize}
        \item For any two sets $A$ and $B$ ther edoes exist a surjective map from $A$ to $B$ or from $B$ to $A$.
        \item The cardinality of an infinte set $A$ is equal to the cardinality of $A \times A$.
        \item Every vector space has a basis.
        \item For every surjective map $ f:A \to B $ there is a map $ g:B \to A $ with $ f(g(b)) = b $ for all $b \in B$.
    \end{itemize}
\end{principle}

\begin{principle}[Axiom of Regularity]
    Let $X$ be a nonempty set of sets. Then $X$ contains an element $Y$ with $ X \cap Y = \emptyset $.
\end{principle}
    \section{Cardinalities}

\subsection{Cardinality}
\begin{definition}[Cardinality]
    Two sets $A$ and $B$ have the same \emph{cardinality} if there exists a
    bijection from $A$ to $B$.
\end{definition}

\begin{example}
    Two finite sets have the same cardinality if and only if thery have the same
    number of elements.
\end{example}

\begin{example}
    The sets $\mathbb{N}$ and $ \mathbb{Z} $ have the same cardinality. Indeed,
    consider the map $ f:\mathbb{N} \rightarrow \mathbb{Z} $ defined by
    $ f(2n) = n $ and $ f(2n+1) = -n $ where $ n \in \mathbb{N} $.
    This map is clearly a bijection
\end{example}

\begin{theorem}[Cardinality as equivalence relation]
    Having the same cardinality is an equivalence relation.
\end{theorem}

\subsection{Countable sets}
\begin{definition}[Finite/Inifinite sets]
    A set is called \emph{finite} if it is empty or has the same cardinality as the set $ \mathbb{N}_n := \left\{1,2,\dots,n\right\} $ and \emph{infinite} otherwise.
\end{definition}

\begin{definition}[Countable/Uncountable sets]
    A set is called \emph{countable} if it is finite or has the same cardinality as the set $ \mathbb{N} $.

    An infinite set that is not countable is called \emph{uncountable}.
\end{definition}

\begin{theorem}[Countable sets in infinite sets]
    Every infinite set contains an infinite countable subset.
\end{theorem}
\begin{proof}[Proof]
    Suppose $A$ is an infinite set. Since $A$ is infinite, we can start enumerating the elements $ a_1,a_2,\dots $ such that all the elements are distinct. This yields a sequence of elements in $A$.
    The set of all the elements in this sequence form a countable subset of $A$.
\end{proof}

\begin{theorem}
    Let $A$ be a set. If there is a surjective map from $ \mathbb{N} $ to $A$, then $A$ is countable.
\end{theorem}
\begin{proof}[Proof]
    Let $ f:\mathbb{N} \rightarrow A $ be a surjection. Then consider the sequence $ f(1), f(2), \dots $. Remove from this sequence (going from left to right) each element that you have seen before. The result is either a finite sequence,
    or an infinite sequence $ f(n_1), f(n_2), \dots $ of which all elements are distinct. In the latter case, consider the map
    $ g: \mathbb{N} \rightarrow A $ with $ g(i) = f(n_i) $. This map is a bijection, which proves $A$ to be countable.
\end{proof}
\begin{corollary}
    Let $A$ be countable and $ f : A \rightarrow B $ surjective, then B is countable.
\end{corollary}

\begin{proof}[Proof]
    Suppose A is a countable set and $ f: A \to B $ a surjective map. If $A$ is finite, then so is B.
    Thus assume that $A$ has infintely many elements. Since $A$ is countable, there is a bijection $ g: \mathbb{N} \to A $.
    But then $ f \circ g $ is a surjection from $ \mathbb{N} $ to B. Hence we can apply the previous result and find a bijection from $ \mathbb{N} $ to $B$. This proves B to be countable.
\end{proof}

\begin{theorem}
    Any subset of a countable set is countable.
\end{theorem}
\begin{proof}[Proof]
    Suppose $A$ is an infinite subset of a countable set $B$. Let $ f : \mathbb{N} \to B $ be bijective and fix an element $ a \in A $.
    Now consider the map $ g : \mathbb{N} \to A $ defined by $ g(x) = f(x) $ if $ f(x) \in A $ and $ g(x) = a $ if $ f(x) \in B \setminus A $.
    Then $g$ is surjective, as $f$ is surjective. Thus A is countable.
\end{proof}

\begin{proposition}
    $ \mathbb{N} \times \mathbb{N} $ is countable.
\end{proposition}
\begin{proof}[Proof]
    Let $ n \in \mathbb{N} $. Let $m$ be maximal with $ \sum_{i=0}^m i < n $. Now let $k= n - \sum_{i=0}^m i$So, $ 1 \leq k \leq m+1 $.
    \par We define $ f : \mathbb{N} \to \mathbb{N} $ in the following way:
    $$ f(n) = (k, m+2-k). $$
    So, in a table this looks as follows:
    \begin{center}
        \begin{tabular}{| c | c | c | c | c |}
            \hline
            $ f(1) = (1,1) $ & $ f(2) = (1,2) $ & $ f(4) = (1,3) $ & $ f(7) = (1,4) $ \\
            \hline
            $ f(3) = (2,1) $ & $ f(5) = (2,2) $ & $ f(8) = (2,3) $ & \dots & \\
            \hline
            $ f(6) = (3,1) $ & $ f(9) = (3,2) $ & \dots & & \\
            \hline
            \vdots & \vdots & & & \\
            \hline
        \end{tabular}
    \end{center}
    By construction, $f$ is injective. Indeed, the $m$ and $k$ are uniquely defined by n. \par
    So it only remains to prove surjectivity. Let $ (k,l) \in \mathbb{N} \times \mathbb{N} $. Set $ m=k+l-2 $. Hence $ (k,l) = (k,m+2-k) $ and $ (k,l) = f(n) $ for $n$ equal to $ \sum_{i=0}^m i+k $.
\end{proof}

\begin{theorem}
    Let $A$ and $B$ be countable sets. Then $ A \times B $ is countable.
\end{theorem}
\begin{proof}[Proof]
    Suppose $ f: \mathbb{N} \to A $ and $ g: \mathbb{N} \to B $ are surjections. The map $ h: \mathbb{N} \times \mathbb{N} \to A \times B $ defined by $ h(i,j) = (f(i), h(i)) $ is surjective.
    So, since $ \mathbb{N} \times \mathbb{N} $ is countable, also $ A \times B $ is countable.
\end{proof}

\begin{proposition}
    The sets $ \mathbb{Z} $ and $ \mathbb{Q} $ are countable.
\end{proposition}

\begin{proof}[Proof]
    The map $ g: \{-1,1\} \times \mathbb{N} \to \mathbb{Z} $ given by $ g(x,y) = xy $ is surjective. Since $ \{-1,1\} \times \mathbb{N} $ is countable, hence $\mathbb{Z}$ is also countable.

    Now let $ f: \mathbb{Z} \times \mathbb{N} \to \mathbb{Q} $ be defined by $ f(i,j) = \frac{i}{j}$ for $ (i,j) \in \mathbb{Z} \times \mathbb{N} $. This is clearly a surjective map.
    Since $\mathbb{Z}$ and $\mathbb{N}$ are countable so is $\mathbb{Z} \times \mathbb{N}$. Hence $\mathbb{Q}$ is also countable.
\end{proof}

\begin{theorem}
    Let $\mathscr{C}$ be a countable collection of countable sets. Then
    $ \bigcup_{A \in \mathscr{C}}A $ is countable.
\end{theorem}
\begin{proof}[Proof]
    For each $ A \in \mathscr{C} $ there exists a bijection
    $ f_A : \mathbb{N} \to A $. Moreover, as $\mathscr{C}$ is countable, there
    exists also a bijection $ g: \mathbb{N} \to \mathscr{C} $. We write
    $ A_i = g(i) $.

    Now consider the map $ f: \mathbb{N} \times \mathbb{N} \to \bigcup_{A\in\mathscr{C}}A $
    defined by $ f(i,j) = f_{A_i}(j) $. This is a surjection. Thus
    $ \bigcup_{A\in\mathscr{C}}A $ is countable.
\end{proof}
\begin{example}
Let $S$ be the set of all finite subsets of $\mathbb{N}$. Then $S = \bigcup_{i\in\mathbb{N}S_i}$, where $S_i$ is the set of subsets of size at most $i$ of $\mathbb{N}$. \par
We already showed that $ \mathbb{N}^i $ is countable. But the map $ (a_1,\dots,a_i) \in \mathbb{N}^i \mapsto \{a_1,\dots, a_i\} \in S_i $ is clearly surjective.
Thus $S_i$ is also countable. Hence $ S = \bigcup_{i\in\mathbb{N}S_i} $ is also countable.
\end{example}

\begin{proposition}
    If $A$ is infinite and $B$ is finite, then $A$ and $A \cup B$ have the same cardinality.
\end{proposition}

\begin{proof}[Proof]
    Assume that $A$ is infinite and, withouth loss of generality, that $A$ and
    $B$ are disjunct. Let $A_0$ be a countable subset of $A$. Then $A_0 \cup B$
    is also countable. Then there exists a bijection $ g: A_0 \cup B \to A_0 $.
    Now define $ f: A \cup B \to A $ by
    $$ f(x) \begin{cases}
        g(x) &\text{if } x \in A_0 \cup B \\
        x &\text{if } x \notin A_0 \cup B,
    \end{cases} $$

    Then clearly $f$ is a bijection between $A \cup B$ and $A$.
\end{proof}

\subsection{Some uncountable sets}
\begin{proposition}
    The set $ \{0,1\}^\mathbb{N} $ is uncountable.
\end{proposition}
\begin{proof}[Proof]
    Let $ F: \mathbb{N} \to \{0,1\}^\mathbb{N} $. By $ f_i $ we denote the
    function $F(i)$ from $\mathbb{N}$ to $ \{0,1\} $.

    We will show that $F$ is not surjective by constructing a function
    $ f \in \{0,1\}^\mathbb{N} $ which is different from all the function
    $f_i$ with $i \in \mathbb{N}$.

    For each $ i \in \mathbb{N} $ let
    $$ f(i) = 0 \text{ if } f_i(i) = 1 \text{ and}$$
    $$ f(i) = 1 \text{ if } f_i(i) = 0 $$

    Clearly, for all $ i \in \mathbb{N} $ we have $ f(i) \ne f_i(i) $ and
    hence $ f \ne f_i $. So $F$ is not surjective. This shows that there is
    no surjection from $\mathbb{N}$ to $\{0,1\}^\mathbb{N}$. In particular,
    $ \{0,1\}^\mathbb{N} $ is not countable.
\end{proof}

\begin{remark}[Cantor's diagonal argument]

\end{remark}

If $A$ is a set, then for each subset $B$ of $A$ we define the
\emph{characteristic function} $ \chi_B: A \to \{0,1\}^\mathbb{N} $ to be the
function that takes the value 1 on all elements in $B$ and the value 0 on all
elements in $A \setminus B$.

Clearly, every element $ f \in \{0,1\}^\mathbb{N} $ is the characteristic
fucntion of the set $ \{a \in A \mid f(a) = 1\} $. So, we find the map
$ B \in \mathscr{A} \mapsto \chi_B $ to be a bijection between
$ \mathscr{P}(A) $ to $ \{0,1\}^\mathbb{N} $.

\begin{corollary}
    The set $ \mathscr{P}(A) $ has the same cardinality as $ \{0,1\}^\mathbb{N} $
    and hence is uncountable.
\end{corollary}

\begin{proposition}
    The interval $ [0,1) $ is uncountable.
\end{proposition}
\begin{proof}[Proof]
    Consider the map $ f \in \{0,1\}^\mathbb{N} \mapsto
    \displaystyle\sum_{i=1}^{\infty}\frac{f(i)}{10^i} $. This map is injective.
    So, if $ [0,1) $ is countable, then so is $ \{0,1\}^\mathbb{N} $, which is
    a contradiction.

    This proves that $ [0,1) $ is uncountable.
\end{proof}

\begin{corollary}
    $ \mathbb{R} $ is uncountable.
\end{corollary}
\begin{proof}[Proof]
    As $ \mathbb{R} $ contains the uncountable subset $ [0,1) $, it is also
    uncountable.
\end{proof}

\begin{theorem}
    If $A$ and $B$ are sets with the same cardinality, then $ \mathscr{P}(A) $
    and $ \mathscr{P}(B) $ also have the same cardinality.
\end{theorem}
\begin{proof}[Proof]
    Suppose $A$ and $B$ have the same cardinality. Let $f : A \to B$ be a
    bijection. Consider the map $\hat{f}:P(A) \to P(B)$ given by $\hat{f}(S)
    = \{f(s) | s \in S\}$. This map is a bijection.
\end{proof}

\begin{corollary}
    If $A$ is an infinite set, then $ \mathscr{P}(A) $ is an uncountable set.
\end{corollary}

\begin{theorem}
    Let $X$ be a set, then $ \mathscr{P}(X) $ does not have the same
    cardinality as $X$.
\end{theorem}

\begin{remark}
    The above theorem shows us that we can get bigger and bigger sets in the following way:
    \begin{align}
        X_1 &:= \mathbb{N} \\
        \text{for } n > 1, X_n &:= \mathscr{P}(X_{n-1})
    \end{align}
\end{remark}

\subsection{Cantor-Schr\"oder-Bernstein Theorem}
\begin{theorem}[Contor-Schr\"oder-Bernstein Tehorem]
    Let $A$ and $B$ be sets and assume that there are two maps $ f:A \to B $ and $ g:B \to A $ which are injective.
    Then there exists a bijection $ h:A \to B $.

    In particular, $A$ and $B$ have the same cardinality.
\end{theorem}

\begin{corollary}
    Let $A$ be a set and assume $B \subseteq A$ has the same cardinality as $A$.
    Then each subset $C$ of $A$ with $B \subseteq C \subseteq A$ has the same cardinality as $A$.
\end{corollary}

\begin{proposition}
    The sets $ \{0,1\}^\mathbb{N} $ and $ [0,1) $ have the same cardinality.
\end{proposition}

\begin{theorem}
    The sets $ \mathbb{R}, \{0,1\}^\mathbb{N}, \mathscr{P}(N) $ have the same cardinality.
\end{theorem}

\begin{theorem}
    The sets $ \mathbb{R}^n $ with $n>0$, and $ \mathbb{R} $ have the same cardinality.
\end{theorem}

\subsection{Additional axioms of set theory}
\begin{principle}[Axiom of Choice]
    Let $ \mathscr{C} $ be a collection of nonempty sets. Then there exists a map
    $$ f: \mathscr{C} \to \bigcup_{A \in \mathscr{C}}A $$
    with $ f(A) \in A $.

    The image of $f$ is a subset of $ \displaystyle\bigcup_{A \in \mathbb{C}}A $.

    The function $f$ is called a \emph{choice function}.
\end{principle}

\begin{principle}
    The following statements are equivalent to the Axiom of Choice.
    \begin{itemize}
        \item For any two sets $A$ and $B$ ther edoes exist a surjective map from $A$ to $B$ or from $B$ to $A$.
        \item The cardinality of an infinte set $A$ is equal to the cardinality of $A \times A$.
        \item Every vector space has a basis.
        \item For every surjective map $ f:A \to B $ there is a map $ g:B \to A $ with $ f(g(b)) = b $ for all $b \in B$.
    \end{itemize}
\end{principle}

\begin{principle}[Axiom of Regularity]
    Let $X$ be a nonempty set of sets. Then $X$ contains an element $Y$ with $ X \cap Y = \emptyset $.
\end{principle}

    \subsection{Exercises}
    \subsection{Exercises}

    \section{Modular Arithmetic}

\subsection{Arthimetic modulo n}
\begin{definition}
    Let $n$ be an integer. On the set $ \mathbb{Z} $ of integers we define
    the relation \emph{congruence modulo n} as follows: $a$ and $b$ are
    \emph{congruent modulo n} if and only if $ n \mid a - b $.

    We write $ a \equiv b \pmod{n} $ to denote that $a$ and $b$ are congruent
    modulo $n$.
\end{definition}

\begin{example}
    If $a =342, b=241$, and $n=17$, then $a$ is not congruent to $b$ modulo $n$.
\end{example}

\begin{proposition}
    Let $n$ be an integer. The relation congruence modulo $n$ is an equivalence
    relation.

    For nonzero $n$, there are exactly $n$ distinct equivalence classes

    The set of equivalence classes of $ \mathbb{Z} $ modulo $n$ is denoted
    by $ \mathbb{Z}/n\mathbb{Z} $.
\end{proposition}

\begin{example}
    The relation modulo 2 partitions the integers into two clases, the even
    numbers and the odd numbers.
\end{example}

\begin{theorem}[Addition and Multiplication]
    On $ \modset{n} $ we define two so-called binary operations, an
    \emph{addition} and a \emph{multiplication}, by:
    \begin{itemize}
        \item Addition: $ x \pmod{n} + y \pmod{n} = x + y \pmod{n} $
        \item Multiplication: $ x \pmod{n} \cdot y \pmod{n} = x \cdot y \pmod{n} $
    \end{itemize}

    Both operations are well defined.
\end{theorem}

\begin{proposition}[Propoerties of Modular Arithmetic]
    Let $n$ be an integer bnigger than 1. For all integers $a,b,c$ we have
    the following equalities.
    \begin{itemize}
        \item Commutativity of addition:
        $$ a + b = b + a \pmod{n} $$
        \item Commutativity of multiplication:
        $$ a \cdot b = b \cdot a \pmod{n} $$
        \item Associativity of additiono:
        $$ (a + b) + c = a + (b + c) \pmod{n} $$
        \item Associativity of multiplication:
        $$ (a \cdot b) \cdot c = a \cdot (b \cdot c) \pmod{n} $$
        \item Distributivity of multiplication over addition:
        $$ a \cdot (b + c) = a \cdot b + a \cdot c \pmod{n} $$
    \end{itemize}
\end{proposition}

\subsection{Invertible elements and zero divisors}
\begin{definition}
    An element $ a \in \modset{n} $ is called \emph{invertible} if there is
    an element $b$, called \emph{inverse} of $a$, such that $ a \cdot b = 1 $.

    Of $a$ is invertible, its inverse will be denoted by $ a\inv $.

    The set of all invertible elements in $\modset{n}$ will be denoted by
    $ \modset{n}^\times $. This set is also called the \emph{multiplicative
    group} of $ \modset{n} $.
\end{definition}

\begin{proposition}[Uniqueness of the Inverse]
    Let $ n > 1 $. If an element $a \in \modset{n}$ is invertible, then its
    inverse is unique.
\end{proposition}

In $ \mathbb{Z} $ division is not always possible. Some nonzero elemetns do
have an inverse, others don't. The following theorem tells us precisely
which elements of $ \modset{n} $ have an inverse.

\begin{theorem}[Characterization of Modular Invertibility]
    Let $ n>1 $ and $ a \in \mathbb{Z} $
    \begin{enumerate}[label=(\alph*)]
        \item The class $a \pmod{n}$ in $ \mathbb{Z}/n\mathbb{Z} $ has a multiplicative inverse if and only if $ \gcd(a,n) = 1 $
        \item If $a$ and $n$ are relatively prime, then the inverse of $ a \pmod{n} $ is the class $ \egcd(a,n)_2 \pmod{n} $
        \item In $ \mathbb{Z}/n\mathbb{Z} $, every class distinct from 0 has an inverse if and only if $n$ is prime.
    \end{enumerate}
\end{theorem}

\begin{example}
    The invertible elements in $ \modset{2^n} $ are the classes $ x \pmod{2^n} $
    for which $x$ is an ood integer.
\end{example}

An arithmetical system such as $ \modset{n} $ with $p$ prime, in which every
element not equal to 0 has a multiplicative inverse, is called a \emph{field},
just like $ \mathbb{Q}, \mathbb{R}, \mathbb{C} $.

Besides invertible elements in $ \modset{n} $, which can be viewed as divisors
of 1, one can also consider the divisors of 0.

\begin{definition}[Zero Divisor]
    An element $ a \in \modset{n} $ not equal to 0 is called a \emph{zero
    divisor} if there is a nonzero element $b$ such that $a \cdot b = 0$.
\end{definition}

The following theorem shows which elements of $ \modset{n} $ are zero divisors.
They turn out to be those nonzero elements that are not invertible.

\begin{theorem}[Zero Divisor Characterization]
    Let $ n>1 $ and $ n \in \mathbb{Z} $
    \begin{enumerate}
        \item The class $ a \pmod{n} $ in $ \mathbb{Z}/n\mathbb{Z} $ is a zero divisor if and only if $ \gcd(a,n) > 1 $ and $ a \pmod{n} $ is nonzero.
        \item The residue ring $ \mathbb{Z}/n\mathbb{Z} $ has no zero divisors if and only if $n$ is prime.
    \end{enumerate}
\end{theorem}

Let $n$ be an integer. Inside $ \modset{n} $, we can distinguish the set of
invertible elements and the set of zero divisors. The set of invertible
elements is closed under multiplication, the set of zero divisors together
with 0 is even closed under multiplication by arbitrary elements.

\begin{lemma}
    Let $n$ be an integer with $n > 1$.
    \begin{enumerate}
        \item If $a$ and $b$ are elements in $\modset{n}^\times$, then their
        product $a \cdot b$ is invertible and therefore also in
        $ \modset{n}^\times $. The inverse of $a\cdot b$ is given by
        $ b\inv\cdot a\inv$.
        \item If $a$ is a zero divisor in $ \modset{n} $ and $b$ is an item
        arbitrary element, then $a \cdot b$ is either 0 or a zero divisor.
    \end{enumerate}
\end{lemma}

\subsection{Linear congruence}

\begin{algorithm}[Linear Congruence]
    % \KwIn{integers}
    % \KwOut{the set }
    % \begin{itemize}
    %     \item Input:
    %     \item Output:
    % \end{itemize}
\end{algorithm}

\begin{remark}
    There are exactly $ \gcd(a,n) $ distict solutions.
\end{remark}

\begin{example}
In order to find all solutions to the congruence $ 24x \equiv 12 \pmod{15} $
we first compute the $ \gcd(24,15) $.Using the Extended Euclidean Algorithm
we find
$$ \gcd(24,15) = 3 = 2 \cdot 24 - 3 \cdot 15 $$

Now 3 divide 12, so the solution set is
$$ \{2 \cdot 12 + k \cdot 15 \mid k \in \mathbb{Z}\} $$
\end{example}

Instead of using the algorithm, we can also use the expression of the gcd as
a linear combination of 24 and 15 to argue what the solution is. To this end,
multiply both sides of the equality $ 3 = 2 \cdot 24 - 3 \cdot 15 $ by 4. This
gives $ 12 = 8 \cdot 24 - 12 \cdot 15 $.

So, a solution of the confgurence is $ x = 8 \pmod{15} $.

We extend the study of a single congruence to a method for solvin special
systems of congruences.

\begin{theorem}[Chinese Remainder Theorem]
    Suppose that $ n_1,\dots,n_k $ are pairwise coprime integers. Then for
    all integers $ a_1,\dots,a_k $ the system of linear congruences
    $$ x \equiv a_1 \pmod{n_i} $$
    with $ i \in  \{1,\dots,k\} $ has solution.

    Indeed, the integer
    $$ x = \sum_{i=1}^{k}a_i \cdot y_i \cdot \frac{n}{n_i} $$
    where
    $$ n = \prod_{i=1}^kn_i $$
    and for each $i$ we have
    $$ y_i = \egcd\left(\frac{n}{n_i},n_i\right)_3 $$
    satisfies all congruences.

    Any two solutions to the system of congruences are congruent modulo the
    product $ \displaystyle\prod_{i=1}^{k}n_i $.
\end{theorem}


\subsection{The theorems of Fermat and Euler}
Let $p$ be a prime. Consider $ \modset{p} $, the set of equivalence classes
of $ \mathbb{Z} $ modulo $p$. In $ \modset{p} $ we can add, subtract, multiply
and divide by elemnts which are not 0. Moreover, it contains no zero divisors.

\begin{theorem}[Fermat's Little Theorem]
    Let $p$ be a prime. For every integer $a$ we have
    $$ a^p \equiv a \pmod{p} $$
    In particular, if $a$ is not in $ 0 \pmod{p} $ then
    $$ a^{p-1} \equiv 1 \pmod{p} $$
\end{theorem}

\begin{example}
    The integer $ 1234^1234 - 2 $ is divisible by 7.

    Indeed, if we compute modulo 7, then we find that $ 1234 \equiv 2 \pmod{7} $.
    Moreover, by Fermat's Little Theorem we have $ 2^6 \equiv 1 \pmod{7} $, so
    $$ 1324^1234 = 2^1234 = 2^{6 \cdot 205 + 4} = 2^4 = 2 \pmod{7} $$
\end{example}

Fermat's Little Theorem states that the multiplicative group
$ \modset{p}^\times $, where $p$ is a prime, contains precisely $p-1$
elements. For arbitrary positive $n$, the number of elements in the multiplicative
group $ \modset{n}^\times $ is given by the so-called \emph{Euler totient function}.

\begin{definition}[Euler totient function]
    The Euler totient function $ \Phi: \mathbb{N} \to \mathbb{N} $ is defined
    by $$ \Phi(n) = \left\vert\modset{n}^\times\right\vert $$
    for all $ n \in \mathbb{N} $ with $ n > 1 $, and by $ \Phi(1) = 1 $.
\end{definition}

\begin{theorem}[Euler Totient]
    The Euler totient function satisfies the following properties.
    \begin{enumerate}
        \item Suppose that $n$ and $m$ are positive integers. If $ \gcd(n,m) = 1 $,
        then $$ \Phi(n \cdot m) = \Phi(n) \cdot \Phi(m) $$

        \item If $p$ is a prime and $n$ is a positive integer, then
        $$ \Phi(p^n) = p^n - p^n-1 $$
    \end{enumerate}
\end{theorem}

\begin{theorem}[Euler's Theorem]
    Suppose $n$ is an integer with $n \ge 2$. Let $a$ be an element of
    $ \modset{n}^\times $. Then $$ a^{\Phi(n)} = 1 $$
\end{theorem}

Let $n$ be an integer. The \emph{order} of an element $a$ in $\modset{n}^\times$
is the smallest positive integer $m$ such that $a^m = 1$. By Euler's Theorem
the order of $a$ exists and is at most $ \Phi(n) $. More precise statements
on the order of elements in $ \modset{n}^\times $ can be found in the
following result.

\begin{theorem}[Orders]
    Let $n$ be an integer greater than 1.
    \begin{enumerate}
        \item If $a \in \modset{n}$ satisfies $a^m=1$ for some positive integer
        $m$, then $a$ is invertible and its order divides $m$.

        \item For all elements $a \in \modset{n}^\times$ the order of $a$ is
        a divisor of $ \Phi(n) $

        \item If $ \modset{n} $ contains an element $a$ of order $n-1$, then
        $n$ is prime.
    \end{enumerate}
\end{theorem}

\begin{definition}
    An element $a$ from $ \modset{p} $ is called a \emph{primitive element}
    of $ \modset{p} $ if every element of $ \modset{p}^\times $ is a power
    of $a$.
\end{definition}

\begin{theorem}
    For each prime $p$ there exists a primitive element in $ \modset{p} $.
\end{theorem}

\subsection{The RSA cryptosystem}

    \section{Modular Arithmetic}

\subsection{Arthimetic modulo n}
\begin{definition}
    Let $n$ be an integer. On the set $ \mathbb{Z} $ of integers we define
    the relation \emph{congruence modulo n} as follows: $a$ and $b$ are
    \emph{congruent modulo n} if and only if $ n \mid a - b $.

    We write $ a \equiv b \pmod{n} $ to denote that $a$ and $b$ are congruent
    modulo $n$.
\end{definition}

\begin{example}
    If $a =342, b=241$, and $n=17$, then $a$ is not congruent to $b$ modulo $n$.
\end{example}

\begin{proposition}
    Let $n$ be an integer. The relation congruence modulo $n$ is an equivalence
    relation.

    For nonzero $n$, there are exactly $n$ distinct equivalence classes

    The set of equivalence classes of $ \mathbb{Z} $ modulo $n$ is denoted
    by $ \mathbb{Z}/n\mathbb{Z} $.
\end{proposition}

\begin{example}
    The relation modulo 2 partitions the integers into two clases, the even
    numbers and the odd numbers.
\end{example}

\begin{theorem}[Addition and Multiplication]
    On $ \modset{n} $ we define two so-called binary operations, an
    \emph{addition} and a \emph{multiplication}, by:
    \begin{itemize}
        \item Addition: $ x \pmod{n} + y \pmod{n} = x + y \pmod{n} $
        \item Multiplication: $ x \pmod{n} \cdot y \pmod{n} = x \cdot y \pmod{n} $
    \end{itemize}

    Both operations are well defined.
\end{theorem}

\begin{proposition}[Propoerties of Modular Arithmetic]
    Let $n$ be an integer bnigger than 1. For all integers $a,b,c$ we have
    the following equalities.
    \begin{itemize}
        \item Commutativity of addition:
        $$ a + b = b + a \pmod{n} $$
        \item Commutativity of multiplication:
        $$ a \cdot b = b \cdot a \pmod{n} $$
        \item Associativity of additiono:
        $$ (a + b) + c = a + (b + c) \pmod{n} $$
        \item Associativity of multiplication:
        $$ (a \cdot b) \cdot c = a \cdot (b \cdot c) \pmod{n} $$
        \item Distributivity of multiplication over addition:
        $$ a \cdot (b + c) = a \cdot b + a \cdot c \pmod{n} $$
    \end{itemize}
\end{proposition}

\subsection{Invertible elements and zero divisors}
\begin{definition}
    An element $ a \in \modset{n} $ is called \emph{invertible} if there is
    an element $b$, called \emph{inverse} of $a$, such that $ a \cdot b = 1 $.

    Of $a$ is invertible, its inverse will be denoted by $ a\inv $.

    The set of all invertible elements in $\modset{n}$ will be denoted by
    $ \modset{n}^\times $. This set is also called the \emph{multiplicative
    group} of $ \modset{n} $.
\end{definition}

\begin{proposition}[Uniqueness of the Inverse]
    Let $ n > 1 $. If an element $a \in \modset{n}$ is invertible, then its
    inverse is unique.
\end{proposition}

In $ \mathbb{Z} $ division is not always possible. Some nonzero elemetns do
have an inverse, others don't. The following theorem tells us precisely
which elements of $ \modset{n} $ have an inverse.

\begin{theorem}[Characterization of Modular Invertibility]
    Let $ n>1 $ and $ a \in \mathbb{Z} $
    \begin{enumerate}[label=(\alph*)]
        \item The class $a \pmod{n}$ in $ \mathbb{Z}/n\mathbb{Z} $ has a multiplicative inverse if and only if $ \gcd(a,n) = 1 $
        \item If $a$ and $n$ are relatively prime, then the inverse of $ a \pmod{n} $ is the class $ \egcd(a,n)_2 \pmod{n} $
        \item In $ \mathbb{Z}/n\mathbb{Z} $, every class distinct from 0 has an inverse if and only if $n$ is prime.
    \end{enumerate}
\end{theorem}

\begin{example}
    The invertible elements in $ \modset{2^n} $ are the classes $ x \pmod{2^n} $
    for which $x$ is an ood integer.
\end{example}

An arithmetical system such as $ \modset{n} $ with $p$ prime, in which every
element not equal to 0 has a multiplicative inverse, is called a \emph{field},
just like $ \mathbb{Q}, \mathbb{R}, \mathbb{C} $.

Besides invertible elements in $ \modset{n} $, which can be viewed as divisors
of 1, one can also consider the divisors of 0.

\begin{definition}[Zero Divisor]
    An element $ a \in \modset{n} $ not equal to 0 is called a \emph{zero
    divisor} if there is a nonzero element $b$ such that $a \cdot b = 0$.
\end{definition}

The following theorem shows which elements of $ \modset{n} $ are zero divisors.
They turn out to be those nonzero elements that are not invertible.

\begin{theorem}[Zero Divisor Characterization]
    Let $ n>1 $ and $ n \in \mathbb{Z} $
    \begin{enumerate}
        \item The class $ a \pmod{n} $ in $ \mathbb{Z}/n\mathbb{Z} $ is a zero divisor if and only if $ \gcd(a,n) > 1 $ and $ a \pmod{n} $ is nonzero.
        \item The residue ring $ \mathbb{Z}/n\mathbb{Z} $ has no zero divisors if and only if $n$ is prime.
    \end{enumerate}
\end{theorem}

Let $n$ be an integer. Inside $ \modset{n} $, we can distinguish the set of
invertible elements and the set of zero divisors. The set of invertible
elements is closed under multiplication, the set of zero divisors together
with 0 is even closed under multiplication by arbitrary elements.

\begin{lemma}
    Let $n$ be an integer with $n > 1$.
    \begin{enumerate}
        \item If $a$ and $b$ are elements in $\modset{n}^\times$, then their
        product $a \cdot b$ is invertible and therefore also in
        $ \modset{n}^\times $. The inverse of $a\cdot b$ is given by
        $ b\inv\cdot a\inv$.
        \item If $a$ is a zero divisor in $ \modset{n} $ and $b$ is an item
        arbitrary element, then $a \cdot b$ is either 0 or a zero divisor.
    \end{enumerate}
\end{lemma}

\subsection{Linear congruence}

\begin{algorithm}[Linear Congruence]
    % \KwIn{integers}
    % \KwOut{the set }
    % \begin{itemize}
    %     \item Input:
    %     \item Output:
    % \end{itemize}
\end{algorithm}

\begin{remark}
    There are exactly $ \gcd(a,n) $ distict solutions.
\end{remark}

\begin{example}
In order to find all solutions to the congruence $ 24x \equiv 12 \pmod{15} $
we first compute the $ \gcd(24,15) $.Using the Extended Euclidean Algorithm
we find
$$ \gcd(24,15) = 3 = 2 \cdot 24 - 3 \cdot 15 $$

Now 3 divide 12, so the solution set is
$$ \{2 \cdot 12 + k \cdot 15 \mid k \in \mathbb{Z}\} $$
\end{example}

Instead of using the algorithm, we can also use the expression of the gcd as
a linear combination of 24 and 15 to argue what the solution is. To this end,
multiply both sides of the equality $ 3 = 2 \cdot 24 - 3 \cdot 15 $ by 4. This
gives $ 12 = 8 \cdot 24 - 12 \cdot 15 $.

So, a solution of the confgurence is $ x = 8 \pmod{15} $.

We extend the study of a single congruence to a method for solvin special
systems of congruences.

\begin{theorem}[Chinese Remainder Theorem]
    Suppose that $ n_1,\dots,n_k $ are pairwise coprime integers. Then for
    all integers $ a_1,\dots,a_k $ the system of linear congruences
    $$ x \equiv a_1 \pmod{n_i} $$
    with $ i \in  \{1,\dots,k\} $ has solution.

    Indeed, the integer
    $$ x = \sum_{i=1}^{k}a_i \cdot y_i \cdot \frac{n}{n_i} $$
    where
    $$ n = \prod_{i=1}^kn_i $$
    and for each $i$ we have
    $$ y_i = \egcd\left(\frac{n}{n_i},n_i\right)_3 $$
    satisfies all congruences.

    Any two solutions to the system of congruences are congruent modulo the
    product $ \displaystyle\prod_{i=1}^{k}n_i $.
\end{theorem}


\subsection{The theorems of Fermat and Euler}
Let $p$ be a prime. Consider $ \modset{p} $, the set of equivalence classes
of $ \mathbb{Z} $ modulo $p$. In $ \modset{p} $ we can add, subtract, multiply
and divide by elemnts which are not 0. Moreover, it contains no zero divisors.

\begin{theorem}[Fermat's Little Theorem]
    Let $p$ be a prime. For every integer $a$ we have
    $$ a^p \equiv a \pmod{p} $$
    In particular, if $a$ is not in $ 0 \pmod{p} $ then
    $$ a^{p-1} \equiv 1 \pmod{p} $$
\end{theorem}

\begin{example}
    The integer $ 1234^1234 - 2 $ is divisible by 7.

    Indeed, if we compute modulo 7, then we find that $ 1234 \equiv 2 \pmod{7} $.
    Moreover, by Fermat's Little Theorem we have $ 2^6 \equiv 1 \pmod{7} $, so
    $$ 1324^1234 = 2^1234 = 2^{6 \cdot 205 + 4} = 2^4 = 2 \pmod{7} $$
\end{example}

Fermat's Little Theorem states that the multiplicative group
$ \modset{p}^\times $, where $p$ is a prime, contains precisely $p-1$
elements. For arbitrary positive $n$, the number of elements in the multiplicative
group $ \modset{n}^\times $ is given by the so-called \emph{Euler totient function}.

\begin{definition}[Euler totient function]
    The Euler totient function $ \Phi: \mathbb{N} \to \mathbb{N} $ is defined
    by $$ \Phi(n) = \left\vert\modset{n}^\times\right\vert $$
    for all $ n \in \mathbb{N} $ with $ n > 1 $, and by $ \Phi(1) = 1 $.
\end{definition}

\begin{theorem}[Euler Totient]
    The Euler totient function satisfies the following properties.
    \begin{enumerate}
        \item Suppose that $n$ and $m$ are positive integers. If $ \gcd(n,m) = 1 $,
        then $$ \Phi(n \cdot m) = \Phi(n) \cdot \Phi(m) $$

        \item If $p$ is a prime and $n$ is a positive integer, then
        $$ \Phi(p^n) = p^n - p^n-1 $$
    \end{enumerate}
\end{theorem}

\begin{theorem}[Euler's Theorem]
    Suppose $n$ is an integer with $n \ge 2$. Let $a$ be an element of
    $ \modset{n}^\times $. Then $$ a^{\Phi(n)} = 1 $$
\end{theorem}

Let $n$ be an integer. The \emph{order} of an element $a$ in $\modset{n}^\times$
is the smallest positive integer $m$ such that $a^m = 1$. By Euler's Theorem
the order of $a$ exists and is at most $ \Phi(n) $. More precise statements
on the order of elements in $ \modset{n}^\times $ can be found in the
following result.

\begin{theorem}[Orders]
    Let $n$ be an integer greater than 1.
    \begin{enumerate}
        \item If $a \in \modset{n}$ satisfies $a^m=1$ for some positive integer
        $m$, then $a$ is invertible and its order divides $m$.

        \item For all elements $a \in \modset{n}^\times$ the order of $a$ is
        a divisor of $ \Phi(n) $

        \item If $ \modset{n} $ contains an element $a$ of order $n-1$, then
        $n$ is prime.
    \end{enumerate}
\end{theorem}

\begin{definition}
    An element $a$ from $ \modset{p} $ is called a \emph{primitive element}
    of $ \modset{p} $ if every element of $ \modset{p}^\times $ is a power
    of $a$.
\end{definition}

\begin{theorem}
    For each prime $p$ there exists a primitive element in $ \modset{p} $.
\end{theorem}

\subsection{The RSA cryptosystem}


    \listoftheorems[show={definition,theorem, proposition, corrolary}]

    \section{Exercises for exam}

\subsection{Logic}

\begin{exercise}
    The statements $P$ and $Q$ can be true or false.  When is the statement
    \par when is the statement $$ R = (P \land Q) \lor ((\lnot P \lor Q) \land (P \lor \lnot Q)) $$true?
\end{exercise}
\begin{enumerate}
    \item If $P$ and $Q$ are true, then $ P \land Q $ hence $R$ is true
    \item If $P$ is true and $Q$ is false, then $ P \land Q $ and $ \lnot P \lor Q $ are false, hence $R$ is false
    \item If $P$ is false and $Q$ is true, then $ P \land Q $ and $ P \lor \lnot Q $ are false, hence $R$ is false
    \item If $P$ and $Q$ are false, then $ \lnot P \lor Q $ and $ P \lor \lnot Q $ are true, hence $R$ is true
\end{enumerate}

\begin{exercise}
    Prove or disprove the following statement: \par
    For all statements $ p,q,r $ we have $ ((p \lor q) \land r) \iff ((p \land r) \lor (q \land r)) $
\end{exercise}
Using the distribive property of the $ \land \text{ over } \lor $ we get:
$$ (p \lor q) \land r = (p \land r) \lor (q \land r) $$
Thus we see that $ ((p \lor q) \land r) \iff ((p \land r) \lor (q \land r)) $

\begin{exercise}
    Prove or disprove the following statement:
    for all statements $P,Q$ and $R$ it holds that:
    $$ \left[(P \implies R) \lor (P \implies Q)\right] \iff \left[P \implies (Q \lor R) \right] $$
\end{exercise}
When $P$ is true we get $ \text{true} \iff \text{true} $ which is true.
When $P$ is false we get $ R \lor Q \iff Q \lor R $ which is also true.
Hence for all $P,Q,R$ the statement is true.

\subsubsection{Sets}
\begin{exercise}
    Prove or disprove
    $$ \forall x \in U \left[x \in (A \cap B) \implies (x \in A \lor x \in B)\right] \iff A=B $$
\end{exercise}
The statement is false and a counter example is $A = \left\{1,2\right\}, B = \left\{1\right\}$

\begin{exercise}
    Prove or disprove:
    For all sets $A,B$ and $C$ we have: $A$
\end{exercise}


\subsection{Final Exam 2022/2023}
\begin{exercise}
    Which statement is true
    \begin{enumerate}
        \item For all sets $A$ and $B$ we have if $ \mathscr{P}(A) = \mathscr{P}(B) $, then $ A = B $
        \item For all sets $A,B,C$ we have $ (A \cup B = A \cup C \land B \subseteq C) \implies A = B $
    \end{enumerate}
\end{exercise}



\end{document}
